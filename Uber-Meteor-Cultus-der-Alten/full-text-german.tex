\documentclass[a4paper, 11pt, oneside, polutonikogreek, german]{article}
\usepackage{gfsbaskerville}
% Load encoding definitions (after font package)
\usepackage[LGR,T1]{fontenc}
\usepackage{textalpha}
\usepackage{graphicx}
\graphicspath{ {./} }
\usepackage[figurename=]{caption}
\usepackage{listings}
\usepackage{subcaption}
\lstset{basicstyle=\ttfamily}

% Babel package:
\usepackage{babel}
\usepackage{cjhebrew}
% With XeTeX/LuaTeX, load fontspec after babel to use Unicode
% fonts for Latin script and LGR for Greek:
\ifdefined\luatexversion \usepackage{fontspec}\fi
\ifdefined\XeTeXrevision \usepackage{fontspec}\fi

% "`Lipsiakos" italic font `cbleipzig`:
\newcommand*{\lishape}{\fontencoding{LGR}\fontfamily{cmr}%
		       \fontshape{li}\selectfont}
\DeclareTextFontCommand{\textli}{\lishape}

\usepackage{booktabs}
\setlength{\emergencystretch}{15pt}
\usepackage{fancyhdr}
\usepackage{microtype}
\begin{document}
\begin{titlepage} % Suppresses headers and footers on the title page
	\centering % Centre everything on the title page
	%\scshape % Use small caps for all text on the title page

	%------------------------------------------------
	%	Title
	%------------------------------------------------
	
	\rule{\textwidth}{1.6pt}\vspace*{-\baselineskip}\vspace*{2pt} % Thick horizontal rule
	\rule{\textwidth}{0.4pt} % Thin horizontal rule
	
	{\scshape\LARGE Über Meteor-Cultus der Alten,\\[1.25pt] vorzüglich in Bezug auf Steine,\\[1.25pt] die vom Himmel gefallen.\\[1.25pt]}
	
	\rule{\textwidth}{0.4pt}\vspace*{-\baselineskip}\vspace{3.2pt} % Thin horizontal rule
	\rule{\textwidth}{1.6pt} % Thick horizontal rule

	%------------------------------------------------
	%	Subtitle
	%------------------------------------------------
	
	{\scshape Ein Beitrag zur Altertumskunde von Fr. v. Dalberg.} % Subtitle or further description
	
    {\scshape\scriptsize Mit einer Kupfertafel.} % Subtitle or further description
    
    %------------------------------------------------
	%	Cover photo
	%------------------------------------------------
	
	\includegraphics[width=\textwidth,height=\textheight,keepaspectratio]{cover.png}
	
	%------------------------------------------------
	%	Editor(s)
	%------------------------------------------------
    \vspace*{\fill}

	\vspace{1\baselineskip}

	{\small\scshape Heidelberg, bei Mohr und Zimmer.}
	
	{\small\scshape{1811.}}
	
	\vspace{0.5\baselineskip} % Whitespace after the title block

    \scshape Internet Archive Online Edition  % Publication year
	
	{\scshape\small Namensnennung Nicht-kommerziell Weitergabe unter gleichen Bedingungen 4.0 International} % Publisher
\end{titlepage}
\setlength{\parskip}{1mm plus1mm minus1mm}
\clearpage
\vspace*{\fill}
\begin{center}
"`Die Urzeit hat keine andere Geschichte hinter sich, als Naturgeschichte, in ihr ruht sicher auch die Mythe."''
\end{center}
\begin{center}
Görres Mythengeschichte der asiatischen Welt.
\end{center}
\vspace*{\fill}
\clearpage
\section*{Inhaltsverzeichnis.}
\begin{enumerate}
    \item[] Inhalt.
    \item[] Einleitung.
    \item Ursprung des Elementen- und Meteor-Dienstes.
    \begin{enumerate}
        \item Überhaupt.
        \item Insbesondere.
    \end{enumerate}
    \begin{enumerate}
        \item Bei den Indiern,
        \item Tibetanern,
        \item Chinesen, Japanern und übrigen südöstlichen Völkern,
        \item Im nördlichen Asien,
        \item Bei den Persern und Chaldäern,
        \item Arabern,
        \item Ägyptern,
        \item Phöniziern,
        \item Griechen.
    \end{enumerate}
    \item Himmel-Steine und deren Verehrung.
    \begin{enumerate}
        \item Ursprung des Steindienstes.
        \item Gebrauch heiliger Steine als Merkzeichen.
        \item Als Altäre.
        \item Viereckige Steine, Hermen.
        \item Symbole der Volks-Einheit.
        \item Bild der Zeugung Phallus, Lingam.
        \item Rechensteine zu Bestimmung der Zeit.
        \item Verträge, Bündnisse, Eide an Steinen geheiligt.
    \end{enumerate}
    \item Stoff und Bestandteile der Bätylien oder heiligen Steine.
    \begin{enumerate}
        \item Viele derselben ächte Aerolithen.
        \item Analyse der Meteorsteine nach den neuesten Erfahrungen.
        \item Einzelne im Altertum bekannt gewordene Aerolithen und Steinregen.
        \item Verschiedene Nahmen, welche die Alten den heiligen Steinen gaben: Jakob- oder Gilead-Stein, Abadir, Alassovid, Pater magnus, magna Mater, der schwarze Stein der Kaaba, Alagabal oder Helagabalos. Dessen Tempel und Dienst als Sonnengott zu Emesa.
        \item Meteorische Luft-, Feuer- und Wasser-Erscheinungen; darauf deutende Mythen: Dioskuren, Patäcken, Cabiren, die zwölf großen phönizischen Götter.
        \item Feuer im Steine verschlossen.
        \item Frühe Entdeckung, Feuer aus Steinen zu ziehen. Gebrauch des Feuers; dahin deutende Mythen.
        \item Ursprung der Steine und Metalle nach Hesiods Theogonie und dem Buche Hiob.
        \item Verglichen mit Theophrast, Plinius und andern alten Mineralogen.
        \item Mythische Sagen, die auf Natur und Ausbildung der Metalle und Gemmen Bezug haben.
        \item Von Talismanen, Abraxas, Amuletten, persischen Zaubersteinen; Erklärung ihrer Zeichen.
        \item Ihr Gebrauch, ihre vorzügliche Heimat, Handel mit denselben, allgemeine Verbreitung dieser Steine.
        \item Rückblick auf die analoge Natur der alten Bätylien und der Meteor-Steine.
        \item Resultate.
    \end{enumerate}
\end{enumerate}
\clearpage
\paragraph{}
Wenn, wie ein trefflicher Schriftsteller\footnote{Herders Ideen zur Gesch. der Menschheit I. Th.} sagt: die Philosophie der Geschichte unseres Geschlechts, um diesen Namen zu verdienen, vom Himmel anfangen muss, soll auch gegenseitig, ihren höchsten Zweck zu erreichen, sie von der Erde sich zum Himmel erheben. In die Atmosphäre, den Ursitz der Stoffe, der Gebärerin aller Organisationen, sollen wir aufblicken; denn Luft, dieser allbelebende durchdringende Hauch, aus dem alles hervor, zu dem alles zurück geht, ist das umfassende Band, die letzte schöne Verwandlung, zu welcher die Stoffe, von ihrer schweren Basis befreit, sich erheben, wie das Insekt, das seiner Raupenhülle entwunden, sich als entfesselte Psyche emporschwingt, zum Äther hinauf, wo der anscheinende Tod selbst Stoffe und Fettige zu neuen Umwandlungen findet, wende der Mensch sein Auge als zu seiner Heimat, und die Atmosphäre, die alle Weltkörper trägt und bewegt, wird ihm Aufschlüsse geben über so manche Phänomene, die er von der Erde allein, welche gleichfalls Leben und Erhaltung von ihr empfängt, und wahrscheinlich selbst ein Erzeugnis der Atmosphäre ist, nie zu erhalten vermag. Denn die Meteore, woher nehmen sie ihren Ursprung anders, als aus dem uns umgebenden Luftkreis, diesem Ursitz der Elemente, worin elektrische und magnetische Ströme, brennbare Luftsäulen, erkaltete Salze, Lichtteile, und andere Bildungsstoffe enthalten sind.

Wenn unter den organischen Wesen nun der Mensch allein zu dieser höheren Ansicht geeignet ist, was besonders aus dem vollkommeneren Bau seines Hauptes, und seiner aufrechten Gestalt, nach dem Zeugnis der vorzüglichsten Anatomen,\footnote{Monro, Kamper, Sömmering, auch Herder in der Geschichte der Menschheit, I. S. 150 u. 170.} hervor geht, so wird er hierdurch vor allen Tieren auch geeignet, von der niederen Erde hinauf zu blicken in den Sitz alles Lebens, wo Sonne, Mond, das Heer der Gestirne, und die wechselnden Phänomene, die durch ihren Einfluss bewirkt werden, und auf die Erde zurück wirken, ihm bald die Ahnung einer geheimen höheren Ursache dieser Erscheinungen geben; geboren wird mit ihm diese Ahnung, daher unter allen Erdgeschöpfen der Mensch allein ein religiöses Wesen ist; denn auch die anorganische Schöpfung, der kalte tote Stein selbst ist dem Einfluss der Atmosphäre unterworfen, ihre belebenden sowohl, als zerstörenden Wirkungen fühlt jedes Tier, aber nur der Mensch --- „nach Gottes Ebenbild geschaffen“ --- hat eine erhöhte Vernunft, die ihn vermögend macht, die Ursache der Dinge zu erforschen, und ein Gemüht, das, in sich selbst Gottes Ebenbild findend, ihn zur Anbetung jenes Wesens leitet, dessen Macht, Weisheit und Güte alles schuf und erhält.

Die Natur des Unerschaffenen zu erkennen, muss dieser selbst sich dem geschaffenen Wesen durch Offenbarung mitteilen; denn, wenn gleich Himmel und Erde seine Größe verkünden, so gewährt gleichwohl die Erkenntnis, die der Mensch aus der Natur (dem Inbegriff alles Gebildeten) zieht, so sehr dessen Betrachtung seine Bewunderung erregen, und sein Herz zur Andacht erhöhen mag, nur unvollkommene Erkenntnis des einen und ewigen Gottes.

Wenn der rohe Mensch, das Kind der Natur, in seinem hilflosen Zustande kämpfend mit den Elementen, den mächtigen Einfluss der waltenden Naturkräfte fühlt, wenn alles um ihn in regem Leben und stetem Wechsel ist, so führt ihn die kindliche Fantasie von selbst dahin, diesen Erscheinungen ein inneres Leben zu leihen, die Phänomene der Natur werden ihm ebenso viele Lebenszeichen, und die wirkenden Ursachen derselben die Elemente, höhere über ihn gebietende Wesen.

Von dieser Erkenntnis zu ihrer Verehrung, ihrem Dienste, ist nur ein Schritt, aber der Cultus, den er ihnen weiht, ist so einfach und roh, als seine Begriffe. Wenn die Wasser sich empor heben, dass die Meerestiefe erschüttert wird,\footnote{Psalm 76.} Regengüsse aus Wolken strömen, Hagel die Felder verheeren, die oberen Lüfte donnern, und Blitze wie Pfeile umher fahren, wenn des Donners Stimme brüllt im Wirbelwinde, und die Blitze das Land erleuchten, dass die Vögel in den Lüften, und die Tiere des Waldes, und des Meeres Bewohner sich verbergen --- dann erbebt der schwache hilflose Mensch; im Kampf der Elemente sieht er Tod und Vernichtung, drohende Geister, die um Schonung anzuflehen, und (indem er ihnen seiner rohen Denkweise gemäß seine eigene Natur leiht) durch Gaben und Opfer zu versöhnen, sein erstes dringendstes Geschäft ist; erschreckt durch die furchtbaren Meteore, die den friedlichen Genuss des Erdenlebens stören, und die schönsten (seine Bewunderung anziehenden) Werke der Schöpfung zernichten, fühlt er den Drang, die Wiederkehr dieser verheerenden Erscheinungen, wo möglich, zu vermeiden, oder wenigstens sich vor ihren Wirkungen zu sichern. Vögel, die in frühester Zeit schon als Deuter und Propheten der Zukunft galten, werden befragt, und ist seine Furcht auf die verheerenden Geister, welche die Elemente beherrschen, gerichtet, so bildet seine Fantasie ihm auch versöhnende --- das Böse bekämpfende Wesen, die, wenn er sie anfleht, ihn hilf leistend retten. In Träumen besonders glaubt er sie zu vernehmen, in dem rauchenden Opfer der Tiere, oder aus Dämpfen und Dünsten der Erde; der tote Stein selbst, Bäume, Berge, Flüsse, die er von geistigen Naturen belebt glaubt, werden als Orakel von ihm befragt, und diejenige, die durch Alter oder Weisheit ein näheres Recht zu deren Deutung sich erwarben, mit besonderer Achtung von ihm geehrt, als Handhaber des Opfergeschäfts, als Priester und Traumdeuter eingesetzt. Blos als schadende oder wohltätige Wesen (Dämonen) verehrten, wie die ältesten Geschichtsurkunden lehren, die ersten Menschen die Elemente, ohne ihnen noch bestimmte Namen zu geben, oder bildlich sie durch eigentümliche Attribute zu unterscheiden. Vor anderem Götterdienste verehrten die Indier im Budajagna-Opfer die guten und bösen Genien\footnote{F. Paulo a Bartolomaeo Darstellung der Brahmanischen Götterlehre. S. 34.} wie durch das Jagam und Homam oder Feueropfer, Sonne, Mond und die Planeten. Dies bezeigt auch der schöne Hymnus in Sakontala an die Elemente:
\vspace{9pt}
\\
Wasser war des Schöpfers erstes Werk,\\
Feuer empfängt die Gaben\\
Anbefohlen im Gesetz:\\
Heilig ist die Opferweihe!\\
\\
\hspace*{1cm} Zeiten misst das Himmelslichterpaar\footnote{Sonne und Mond.}\\
\hspace*{1cm} Und des Schalles Führer\\
\hspace*{1cm} Zarter Äther, füllt das All!\\
\hspace*{1cm} Erd' ist des Gebührens Mutter;\\
\\
\hspace*{2cm} Leben alles atmenden ist Luft:\\
\hspace*{2cm} So in acht Gestalten,\\
\hspace*{2cm} Sichtbar, nähr' und segn' euch Gott,\\
\hspace*{2cm} Issa der Natur Verwandler! ---\\

In Menus Gesetzbuche brachte Brama zuerst zehn Herrn der erschaffenen Wesen, diese aber erzeugten sieben andere Menus --- dann wohlwollende Genien und wütende Riesen (wohltätige und zerstörende Naturkräfte), himmlische Gänger, Nymphen und Dämonen, endlich Blitze und Donnerkeile, Wolken, und farbige Bogen des Indra, fallende Meteore, die Erde zerreißende Dünste, Kometen und Lichtkörper verschiedener Grade.\footnote{Menus Gesetzbuch nach Hüttners Übers. S. 9.}

Ein auf Elemente deutendes Symbol war gleichfalls jene am Berge Meru stehende Feuersäule ohne Anfang und Ende, deren Höhe Brahma in hunderttausend Jahren nicht ersteigen konnte, indessen ihr Fuß im Abgrund stand; so der tibetanische Berg Righiel Lumbo aus vier Elementen zusammengesetzt. Ostwärts bei den Chinesen finden wir in frühester Zeit die beiden Urmächte Jang und In. Jang des Vollkommenere vorstellend, daher Himmel, Sonne, Wärme, männliche Kraft, Urfeuer; In im Gegensatze das Unvollkommenere, daher Erde, Mond, weibliche Kraft, Kälte, Nacht, Urfeuchte; beide Mächte erzeugten vier Bilder Su-siang, und diese: acht nach verschiedenen Kombinationen verbundenen Gestalten Kua genannt. Südöstlicher hin nach Correa, Japan, Siam, und auf Ceylan zeigen die Mythen überall Spuren des frühen Elementen-Dienstes, nur nach Klima und Lokalität verschieden. Nordwärts in Hinterindien bei den skythischen und skandinavischen Völkern treffen wir gleichfalls nebst der Verehrung von Sonne, Mond, und anderer Himmelslichter, als Untergottheiten: Wolken, Regenbogen, Blitze, Gewitter, Sturm, Hagel, Feuer, Wasser, Erde, Berge, Flüsse.\footnote{S. Pallas Reisen und Georgis Russland, verglichen mit Herodot und Strabo.} So glaubten die Völker finnischen, slavischen, gotischen und germanischen Stammes, ihre Helden und Väter würden nach dem Tode in die Wolken versetzt, und erschienen öfters als Meteore.\footnote{Häufige Beispiele hievon finden sich in der Edda und in Ossians Gesängen.} Von den skythischen Scoloten sagt Herodot Buch IV. K. 5., dass sie dem Himmel und seiner Gattin Erde, der Sonne, dem Monde, dem Eisen (oder Mars) und dem Herkules opferten; die sogenannten Königs-Skythen aber dem Wasser; sie hatten weder Bilder, Tempel, noch Altäre, aber den Mars stellten sie durch ein bloßes Schwert und einen großen Haufen Reißig vor. Eine merkwürdige Sage fügt er hinzu, dass nämlich ein gewisser Targitaus Sohn des Himmels und des Flusses Boristhenes der erste Bewohner dieser Gegend gewesen, und daselbst drei Söhne erzeugt habe, während welcher Regierung vier Heiligtümer vom Himmel fielen, eine Pflugschaar, ein Joch, eine Axt und ein Goldstück; letzteres habe der älteste Bruder zuerst fallen sehen, als er sich ihm aber genähert, sei es glühend geworden, bei Ankunft des zweiten noch heißer, nur als der jüngste kam, sei die Gluth erloschen, und habe ihn in Stand gesetzt, das Goldstück nach Hause zu nehmen, wonach die beiden Eltern ihm Land und Regierung abtraten; weshalb (fährt Herodot fort) die Könige der Skythen gewisse Goldstücke als Heiligtümer verwahren, und ihnen Opfer bringen.\footnote{Sollte dieser Mythe nicht die Tradition eines vor alter Zeit in jener Gegend gefallenen Meteor-Steins zum Grunde liegen? Glühend fiel das Metallstück herab, und das Glänzende seiner Feuergestalt verwandelte die Sage in Gold, womit sich noch die Idee vom frühen Gebrauch dieses Metalls und gleich früher Benutzung des Pflugs verband, die Sage scheint demnach einer Zeit anzugehören, wo die Skythen schon ein ackerbauendes Volk waren.}

Die Massageten verehrten vorzüglich die Sonne; mit gezogenem Schwerte schwuren die Avaren und Hungaren dem Himmel und dem Weltfeuer oder dem Gott Isten (vielleicht Vesta?). So verehrten die alten Türken und die Mongolen gleichfalls die Elemente. Dass die Anhänger der uralten Lichtlehre, die Zerduscht, später nur verbesserte, nebst dem Hauptdienste des Feuers\footnote{Und zwar in Herm-aphroditer-Gestalt als Symbol des männlich tätigen, und weiblich leidenden Princips. "`Persae et Magi omnes (sagt Firmicus de Errore prof. Rel. p. 10.) qui Persiae Regionis incolunt fines, ingnem praeferunt, et omnibus Elementis putant debere praeponi. Hi itaque Jovem in duos dividunt potestatem, naturam ejus ad utriusque Sexus transferentes, et viri et foeminae simulacra ignis substantiam deputantes, et mulierem quidem triformi vultu constituunt, monstrosis eam serpentibus illigantes. --- Virum vero ab actorem boum colentes, sacra ejus ad ignis transferunt potestatem."'} auch den anderen Elementen huldigten, zeigen häufige Stellen im Zendavesta, besonders in den Büchern Izeschnè und Siruzè; denn die Amschaspands, Fervers, und Izeds, welche die Parsis durch Gebete und Opfer sich günstig zu machen suchen, was sind sie anders, als Genien der Naturkräfte und Elemente? wie dies aus dem persischen Weltsysteme deutlich hervor geht. Wir sehen hier Albordi, das große Urgebirgs in des Himmels Mitte, auf ihm ruhend die große Säule, die den Weltbau stützt, bis in die Region des reinen Lichts reichend. Da thront Hougner, Herrscher der Höhen, denen die Quellen entströmen; auf Albordis Gipfel ruht vor allen der erste der Amschaspand, die Sonne, die wie Wasser in den Höhen die Erde umkreist; ihr zunächst der Mond, der seinen Lichtglanz über die geschaffene Erde ausgießt. Tiefer stehen die Fixsterne und die Wandelgestirne, in ihren Bahnen unter die Weltgegenden verteilt, jeder Planet, worunter Taschter, der helle Oststern, der erste ist, an diesen höheren Himmel schließt sich unmittelbar jener der Meteore; aber nicht Feuer allein, auch heiliges himmlisches Wasser, auch der Regen entquillt den Sternen, und wenn die Divs (böse Genien) die Welt zerrütten, dann fährt von Albordis Höhen ein Stern herab, und befruchtende Wasserströme ergießen sich über die Erde. Wie es nun, nach Zerduschts Lehre\footnote{S. Bundehesch.} sieben Arten Feuers gibt: Berezeseng, das vor Ormurzd und den Königen brennt, Voh-Freuin in Menschen und Tierkörpern, Oruazescht, in Gewächsen, Vazescht, vor und aus dem Berg Sapojequier (wahrscheinlich ein alter Vulkan), Sprenescht, Küchenfeuer; so zählte man auch sieben oder vielmehr vierzehn Arten Wassers: nämlich Thau, oder Wasser auf Pflanzen, Quellwasser, Regenwasser, Brunnenwasser, Flüssigkeiten von Tieren und Menschen, Schweiß, Mark, Exkremente, Speichel, Oelteile, Dauungssaft, die Flüssigkeiten im Inneren der Pflanzen, endlich Milch.\footnote{Kleukers Zend à Vesta im Kleinen II. 174.}

Wenden wir uns westwärts, so finden wir gleich frühzeitig diesen Elementen-Dienst in Verbindung mit Sabäism bei den Arabern und Äthiopiern. Die Phönizier, sagt Philo von Biblus aus Sanchuniaton, legten den Namen ihrer Könige den Welt-Elementen, und verschiedenen ihrer vermeinten Götter bei, Sonne, Mond, Sterne, und die Elemente waren ihre einzigen Götter. Von den Ägyptern erzählt Diodor,\footnote{Diodor Bibl. der Gesch. I. 12.} dass sie nebst Isis und Osiris --- Sonne und Mond --- zuerst das Feuer (Phanes oder Dionysus) und die übrigen Elemente verehrten. --- Dasselbe zeigt uns der griechische Cultus; denn auch hier wurden vor anderen Göttern die Naturkräfte in ihren Urprincipien und Phänomenen oder Meteoren verehrt,\footnote{Von den Telchinen, den ersten Bewohnern der Insel Rhodus, erzählt Diodor Bibl. der Gesch. 5. Buch Kap. 55: "`dass sie die ersten gewesen, welche Bildsäulen der Götter gemacht haben, und verschiedene alte geweihte Bilder [wahrscheinlich Hermen oder Hausgötter] führten von ihnen den Namen. Zugleich waren sie Zauberer, die ebenso wie die Magier, wenn sie gewollt, Wolken, Regen, Hagel und Schnee heraufbrachten, auch ihre eigene Gestalt verwandelten, in ihrem Unterricht in den Künsten aber sehr zurückhaltend waren."'} so hieß es in einem alten Gesang, den Pausanias\footnote{Buch X., Kap. 12. Sie lebte zur Zeit der Peleaden, die, wie Pausanias hinzusetzt, älter waren, als Phemonon.} der Phänis, einer der ältesten Sibyllen, zuschreibt, von Zeus: "`- Jupiter, der war, ist, und sein wird, durch deine Hilfe gibt die Erde ihre Früchte, wir nennen sie daher unsre Mutter. --- Einer merkwürdigen Vorstellung des Zeus erwähnt\footnote{Buch II. K. 24.} Pausanias; mit drei Augen nämlich, davon das eine mitten auf der Stirne ruhte, deutend auf die obere, Mittel- und Unterregion des Weltalls, die er beherrscht; ward Zeus nun als das Symbol des Himmels vor anderen Göttern verehrt, so hatte das Element des Wassers oder Neptun in anderen Gegenden gleichfalls frühen Dienst und Tempel, wie jener uralte, den Pausanias zu Teraphne sah.\footnote{Buch III. Kap. 20.} Dahin gehören die Hydrophorien, besonders jenes uralte Wasserfest, das nach Pausanias Buch I. K. 18. nahe am Tempel des Olympus, an einer Öffnung, durch welche, der Sage nach, die deukalionische Flut sich verlaufen hatte, zur Erinnerung ihrer durch Wasser vertilgten Voreltern gefeiert wurde; wobei man die interirdischen Götter anrief, und durch Opfer zu versöhnen suchte. Ein ähnliches Fest seierten die Egineten zu Ehren Apolls, vielleicht weil er der Gott ist, der die vom Schlamm der Gewässer erzeugte Schlange überwand und tötete. In den Festen, welche zu Hierapolis der syrischen Göttin gefeiert wurden, waren nach Lucians Erzählung (de Dea Syria) mehrere Gebräuche, die mit dem athenischen Wasserfeste Ähnlichkeit hatten, und sich, wie jenes, auf den Elementen Dienst des Wassers bezogen. Die meisten alten Völker feierten solche Hydrophorien, nicht bloß wie Boulanger in Antiquité dévolée meint, zum Andenken der allgemeinen Flut, oder einzelner Überschwemmungen; vielmehr um durch Opfer und Gebete die Elementar-Geister der Meere, Seen, Flüsse zu versöhnen, und sich geneigt zu machen. Auch der Luft oder den Winden war zu Titane im korinthischen Gebiete ein Altar geweiht,\footnote{Buch II. Kap. 12.} und die Erde hatte zu Sparta einen Tempel, Gasepton genannt.\footnote{Buch III. Kap. 12.} Von den alten Pelasgern sagt Herodot,\footnote{Herodot II. 52.} dass sie anfänglich den Göttern, weil diese alles in Wohlordnung gesetzt und in Einteilung gebracht, unter Gebeten zwar mancherlei opferten, aber ihnen noch keine bestimmte Namen gaben, vielmehr erst nach Befragung des Bodonischen Orakels durch dessen Aussage bestimmt wurden, diese höhere Wesen Himmel und Erde zu nennen, wozu sie später (gleichfalls auf Geheiß des Orakels) den Dionysos (obschon von ägyptischer Herkunft) gesellten. Aber da der menschliche Geist im Fortgang seiner Entwickelung sich nicht genügt an diesem einfachen Dienste, und da seine rege Einbildung überhaupt gern in Bildern lebt, suchte er die durch Meteore zwar fühlbaren, jedoch in ihren einfachen Bestandteilen nicht anschaulichen Urkräfte bildlich und symbolisch darzustellen; daher der Ursprung der Hermen, und der durch Einwirkung von Zeit und Lokalität in unendliche Formen verwandelte Polytheismus, dessen Bilder und Schemen gleichwohl nichts als Attribute des einigen Gottes sind, dessen reine geläuterte Verehrung durch ausgeartete, beschränkte und kindische Begriffe entstellt wurde, wie gegenseitig die Idee seiner bloß geistigen Natur nur das Resultat eines reinen, über Sinnlichkeit sich erhebenden Gemüts zu sein vermag. Zwar hat diese reinere, durch Offenbarung dem Menschen bei seiner Bildung mitgeteilte Vorstellung, sich im Geschlechte selbst mitten unter aller geistigen und religiösen Entartung in einem kleinen Haufen erhalten. Aus Ur (oder Chaldäa), dem Lichtlande, ward durch den Stamm der Abrahamiden die Verehrung eines einigen Gottes bewahrt und bei den Hebräern fortgepflanzt, bis durch des Gottmenschen Sendung das Licht reiner Wahrheit in hehrem Glanz schimmernd sich allgemein verbreitete, und dem Polytheismus ein Ende machte. --- So lange hatte derselbe in immer rascherem Fortgange den größten Teil der Völker ergriffen; im Beginnen ihres gesellschaftlichen Zustandes hatten dieselben ihren rohen Begriffen und regen Phantasie gemäß alles belebt, was in der Natur sie umgab, daher ihre äußere Götterlehre im wahren Sinne pantheistisch ist, und je tiefer wir ins Altertum zurück blicken, je mehr sehen wir die Idee eines einigen Gottes, nach den verschiedenen Stämmen der Völker geteilt in ebenso viele Lokalgötter, wie die Kunst oder bildliche Vorstellung dieser Wesen je älter, je mehr mit vervielfachten Teilen und Attributen überhäuft.\footnote{Man sehe z. B. in Fr. Paulo Barthol. Brahminenlehre die Abbildungen der drei ersten Verwandlungen Vischnus, und jene des Shiva, welche die Bildung der Erde aus dem Wasser und den Kampf des Feuers mit den andern Elementen darstellt.}

So roh diese Vorstellungen sein mögen, haben sie gleichwohl eine merkwürdige Deutung, indem sie die ersten Blätter in der Geschichtsurkunde der Erd- und Menschenbildung sind; denn die in allen Mythologien erscheinenden Bilder des Chaos, und der aus dessen Gärung entstehenden Feuer- und Wasser-Verheerungen, die Riesenkämpfe, die verschiedenen aufeinander folgenden, sich immer zerstörenden Göttergeschlechter, was sind sie anders, als bildliche, vieldeutende Orakel vom ersten Ursprung der Dinge? --- Aber noch frühere anschaulichere Beweise und Zeugnisse der ersten Urzeit hat unsere Erde aufzuweisen: fürs erste jene Urgebirge und höchsten Felsspitzen, die am frühesten aus dem immer mehr niedersinkenden Gewässer hervortraten, und längst vor der belebten Schöpfung als einzelne Inseln hervorragten, daher die den ältesten Völkern eigne Verehrung der Berge und Flüsse. Ferner alle Metalle und Gemmen (edle Steine), deren Ursprung und Bildung durch Einwirkung der mächtigeren Elemente, Feuer und Wasser, gleichfalls Zeugnis geben von der Urzeit und der Bildung unseres Erdkörpers. Endlich --- im Kreis der Meteore, deren Erscheinung der rohe Naturmensch stets einer höheren Ursache, einem geistigen Wesen beimisst, jene Feuermassen, die teils als elektrische Flammen in der Atmosphäre schimmern,\footnote{Z. B. die als Sternschnuppen, St. Elms-Feuer, Dioskuren, und unter anderen Namen bekannten Meteore, wozu auch die Nordscheine gehören und der als Friedensbild des versöhnten Himmels mit der Erde (nach der Mosaischen Sage) so bedeutende Regenbogen; wie gegenseitig Kometen als Boten des Brandes und der Zerstörung immer die furchtbarsten Meteore waren.} teils als größere oder kleinere Steinmassen zur Erde fallen, und, ihrem äußeren sowohl als inneren Gehalt nach, Spuren eines fremdartigen Ursprungs, einer fernen Heimat tragen. Vom Himmel oder aus höherer Atmosphäre fallen sie herab, ein Feuer-Meteor ist ihr Begleiter, sie selbst im Augenblick ihres Sinkens glühend und lichtstrahlend; kein Wunder daher, dass man sie himmlischen Ursprungs und der Vergötterung wert hielt; denn eben der Glaube, der die Gestirne für belebte geistige, die niedere Welt beherrschende Wesen halten machte, erzeugte auch die Idee, dass diese Feuermassen untergehen, und im Augenblick ihres Erlöschens oft zur Erde sänken, welches in der alten Mythik umso gegründeter ist, als ihr gemäß, die Himmlischen oft unter den Sterblichen wandelten, sich zur Erde herab ließen, und dass nicht ungeformte Steine allein, sondern selbst Bilder der Götter und andere Heiligtümer, die mit größter Ehrfurcht in Tempeln verehrt wurden, vom Himmel fielen.\footnote{Ein Beispiel ist das in Ephesos, der Sage nach, vom Himmel gefallene aus Holz geformte Bild der Diana, von dem wir später reden werden.}

Die Zeugnisse der Alten von öfters sich ereigneten Steinregen, und dem Herabfalle einzelner Aeroliten hat man lange teils übersehen, teils für märchenhafte Sagen gehalten, bis neuere Physiker aufmerksam geworden auf die in mehreren Gegenden sich ereignete Erscheinung von Feuerkugeln und sogenannten Meteor-Steinen, den inneren Gehalt, die Bestandteile dieser Massen chemisch untersuchten,\footnote{Chladni, Proust, Reuß, Klapproth u. a.} und, indem man zugleich die Beschreibung einiger im Altertums bemerkten Steine dieser Art damit verglich, kamen vorlängst schon einige Altertumsforscher\footnote{Vorzüglich Falconnet in mehreren Abhandlungen der Mém. de l'acad. des Inscript. et belles lettres.} auf die Mutmaßung, dass die von den Alten so religiös verehrten Bätylien und heiligen Steine, größtenteils Aerolithen waren, oder mindestens für Steine himmlischen, d. i. außertellurischen Ursprungs gehalten wurden.

Ein neuerer Forscher war vorzüglich bemühet, einen aus Quellen geschöpften Vergleich dieser Bätylien mit den Meteor-Steinen zu machen. Seine Abhandlung kam mir zu Handen, als ich längst, angelockt durch Lesung der in Gilberts Journal der Physik und anderen Schriften befindlichen Analysen dieser Steine bemüht war, Materialien zu einer Untersuchung über den Steindienst, und den ebenso merkwürdigen Meteor-Cultus (wahrscheinlich die früheste Verehrung) zu sammeln. Seine fleißigen Forschungen\footnote{D. Münters Schrift über die Bätylien der Alten, in den Verhandlungen der gelehrten Gesellschaft von Kopenhagen.} gaben mir wichtige Fingerzeige in manchem, auch bin ich seinem Pfade gefolgt, doch schien mir diese Schrift gleichsam nur Vorarbeit, indem sie den physisch-chemischen Teil der Meteor-Steine kaum berührt, auch über die höhere mythische Ansicht, die der Verehrung derselben zum Grunde liegt, und die, wie mir däucht, in so genauem Zusammenhange mit der ältesten Theurgie steht, dass ohne Beihilfe derselben sie nicht erklärt werden kann, nur wenige Winke gibt.

Dies nun zu verfolgen, und den Zusammenhang der Steinverehrung mit dem Dienst der Elemente, oder ihren Erscheinungen in den Meteoren in ein helleres Licht zu setzen, ist der Zweck gegenwärtiger Abhandlung.
\clearpage
\paragraph{}
Welchen Ursprung jene merkwürdigen Massen auch haben mögen, die aus höheren Luftregionen teils einzeln, teils in zahlreicher Menge als Steinregen auf die Erde fallen; ob sie im Wasser oder Feuer entstehen? ob, nach Chladni, sie als Teile zertrümmerter Weltkörper anzusehen sind, oder ob sie nach La Place dem Monde entfallen; nach Proust\footnote{Gilbert Annalen der Physik, Bd. 24. S. 261.} und anderen hingegen in der Atmosphäre sich bilden, und wenn gleich sie sich in den uns bekannten Gegenden der Erde nicht finden, noch in ihnen finden können, doch Regionen unseres Erbkörpers, und zwar den unermesslichen noch unbekannten Polargegenden ans gehören, von diesen losgerissen, und aufwärts geschleudert, in unsern südlichen Gegenden niederfallen; diese Erforschungen seien dem Physiker überlassen; Tatsache ist indessen, dass von den ältesten Zeiten her Meteor-Steine zur Erde fielen, denen man, wie die Geschichte lehrt, göttliche Verehrung bezeigte.

Woher nun, fragt sich, entstand dieser Cultus?

Folgende Hauptursachen lassen sich, glaube ich, hievon angeben:

1. Wenn wir Humes und Boulangers Idee\footnote{Origine of Religion, in Humes Works. Boulangers Antiquité dévoilée.}: dass Furcht der Ursprung aller Religion sei, auch nicht unbedingt, beistimmen können, so zeigt die Geschichte unseres Geschlechts doch, wie Schrecken vor ungewöhnlichen Erscheinungen bei rohen sinnlichen Menschen den so natürlichen Glauben erzeugen konnte, dass Meteore und außerordentliche Phänomene von unsichtbaren höheren Wesen herrühren, die, indem sie ihren Wirkungen nach mehr zerstörender als milder Natur scheinen, man durch Gebete, Sühnopfer u. d. gl. sich geneigt machen müsse, ein Glaube, woraus der erste rohe Fetischismus hervorging. Da nun öfters Meteor-Steine teile einzeln, teils in größerer Zahl aus höheren Regionen herabfielen, war es natürlich, in ihnen die Kraft eines sie belebenden in Tätigkeit setzenden höheren Wesens zu ahnen.

Eine 2te Ursache ist in der Physik und Kosmologie der alten Völker zu suchen.

Sterne wurden in der frühesten Zeit, ehe der Polytheismus noch tiefer herabsank, als göttliche Wesen verehrt, und ihrem Einflüsse war alles, was irdisch ist, unterworfen. Diese Verehrung aber gründete sich auf die Meinung: vom Dasein gewisser Mittelwesen, die (vermöge des Systems der Emanation) die ganze Kette der Intelligenzen, Genien, Dämonen bildeten, die ein Ausstrahle des unendlichen Lichtquells oder eine fortgehende Progression von Potenzen aus der Einheit sind; eine Lehre, die allen Völkern gemein war, und welche man in allen Mythologien wieder findet.

In den frühesten Zeiten nahmen die Indier (wie die Purana lehren) gute und böse, himmlische, irdische und unterirdische Mittelwesen an\footnote{Trefflich entwickelt und zusammengestellt findet man die indische Emanations-Lehre in v. Polliers Mythologie des Indous. T. II. Ch. 12. 13. 14.}; diese Genien (Deiotas) verändern nach Willkür ihre Form, find als Vorsteher und Leiter über die Elemente und alle Wesen von den größten zu den kleinsten gesetzt, wie ihr Einfluss sich auch über alle Wesen verbreitet, weshalb die Schöpfung in 15 Regionen (Sourg) geteilt ward, die alle unter ihrer Gewalt stehen, und ihrem guten oder bösen Einfluss (da dieser Wesen es gute und schlimme gibt\footnote{Daher weise und schwarze Magie.}) unterworfen sind. Sie bilden unter sich eine Hierarchie, an deren Spitze sieben Haupt Deiotas stehen, die die höheren Regionen leiten --- andere beherrschen die Erde, Meere, Flüsse, Quellen, Berge und Wälder; den sieben unteren Regionen (Lock genannt) sind wieder ebenso viele Genien (böse, verderbende Wesen, die die reinen Geister betrügen, ihnen jedoch gewissermaßen untergeordnet sind) vorgesetzt. --- Diesen Deiotas vollkommen ähnlich sind die sieben Amschaspands nebst den oberen und untergeordneten Izeds und Fervers des persischen Magismus; der Chaldäersieben Fünten der oberen Welt; die sieben göttlichen Throngeister der Juden und Zephiren der Kabbala, die sieben heiligen Laute der Ägypter,\footnote{Hievon Jablonsky Panth. Aegypt. Proleg. p. 53. -- Zend à Vesta. -- Die heiligen Bücher und rabbinischen Schriften.} Orphiker und Pythagoräer; die Äonen der Gnostiker, wie endlich überhaupt die aus älteren orientalischen Quesen fließende Dämonen-Lehre der Griechen,\footnote{Mit welchen auch die nordischen Mythen übereinstimmen.} "`Die Götter, sagt Plutarch, mischen sich, und betreiben nicht selbst die Wahrsagungen, Beschwörungen u. d. gl., sondern die Dämonen als ihre Diener und Geschäftsträger; so sind einige Aufseher der Opfer, Vorsteher der Feste und Mysterien; andere gehen als Rächer des Übermuts und der Ungerechtigkeit auf Erden umher, andere sind gute wohltätige Geister."'

Wie man nun glaubte,\footnote{Plutarch über den Verfall der Orakel.} dass diese Mittelwesen der Planeten Gestirnen und Elementen vorstünden, hielt man auch dafür, dass nicht allein Untergötter, sondern selbst ausgezeichnet Menschen (Heroen) in Sterne verwandelt würden, und als solche am Himmel glänzten, aber da sie nicht unfehlbar wären, wegen vergangenes Verbrechen ihren Glanz auch wieder verlieren und herabsinken könnten, daher der Glaube, dass Sterne belebt seien, und zuweilen auf die Erde herabfielen. Der alten Physik gemäß hielt man Sterne für Feuermassen,\footnote{Nach Anaxagoras, Demokrit und Metradorus war die Sonne ein feuriger Klumren, oder ein glühender Stein: so der Mond eine feste glühende Masse, und die Sterne, nach Diogenes, glühende Steine, die oft zur Erde herabfielen und da verloschen. Plutarch. de Placit. Philos.} und da ihre eigentliche Größe in jener früheren Zeit noch nicht hinlänglich berechnet war, schien nichts weniger als ungereimt, dass, sobald ihr Glanz (d. h. der sie regierende Geist) erlosch, sie zur Erde sinken konnten. Der gefallene Stern war ein verwandelter entflohener Daimon, und der glühend herab gefallene Stein ein erloschener Stern.\footnote{Les orientaux croyoient, que les Anges sont des Esprits ignées, opinion, qui passa depuis chez les Chrètiens, et qui, si je neme trompe, s'étoit communiquée aux juifs longtems auparavant. Beausobre Hist. du Manichéisme. T. I. p. 323.}

Eine 3te Ursache der Verehrung, die man sogenannten heiligen Steinen bezeigte, ist in dem Gebrauch und der Anwendung, den man von denselben zu öffentlichen Denkmalen und Merkzeichen machte, gegründet. Bevor wir nun zur näheren Bettachtung der Aerolithen-Verehrung übergehen, wird es nicht überflüssig sein, einen näheren Bück auf. Entstehung des Stein-Cultus überhaupt zu werfen.

In jenem Weltalter, wo Sinnbilder, Schriftzüge, Buchstaben noch nicht erfunden waren, und rohe Menschen keine anderen als gleichfalls rohe Werkzeuge hatten, ihre Ideen aufzuzeichnen, mussten Steine als Merkzeichen vorzüglicher Ereignisse, die zugleich als Vereinigungspunkt bei feierlichen Handlungen und Verträgen, oder als Grenzbezeichnungen dienen konnten, eine Achtung gewinnen, die allmählich bis zur Vergötterung erhöht ward, und durch den geheimen Sinn, den spätere Mysterien, Priester, Hierophanten diesen Gegenständen beilegten, eine noch höhere Verehrung erhalten.

Aus dem häuslichen Feuerherde, der bei jedem Volke im nomadischen Zeitalter in der Mitte des Zeltes stehend, die Familien zu gemeinsamen Verrichtungen, Gebeten, traulichen Gesprächen sammelte, und ihnen darum so heilig war, als ihre Laren, entstand der Vesta-Dienst, und mit ihm die Bewahrung des reinen Feuers, auf dem allgemeinen Heerde oder Steinaltar, als dem Mittelpunkte des ganzen Staates. So ging aus dem frühen Gebrauch roher, zuerst unförmlicher, dann gehauener viereckiger, endlich mit Kopf und menschlichen Gliedmaßen versehenen Steinen (woraus allmählich die Bildsäulen entstanden) der Dienst des Jupiter lapis oder horcus,\footnote{Auch des ΖΕΥΣ ΚΕΡΑΥNIOΣ oder Fulminatoris, dessen Dienst in den ältesten Zeiten schon herrschend war, und dem Seleucus Nicator, als er die Stadt Seleucia am Meer erbaute, unter dem Symbol des Blitzes Tempel und Altäre errichtete, wie besonders eine Münze dieses Königs zeigt, auf deren Rückseite sich ein beflügelter Donnerkeil befindet. S. Spanheim de Praestantia et Usu Numm. antiq. p. 393. Zwei andere Münzen in demselben Werke zeigen den beflügelten Blitz auf einem Tische oder Altare liegend; s. Kupfertafel No. 6, 7, 8.} des Terminus, Pales, der Äcker, Gärten und Weg- Genien, Hauslaren und Hermen hervor."' Die Griechen, sagt Pausanias Buch VII. verehrten anfänglich rohe Steine statt Götter."' So stellte, gleichfalls nach Pausanias Buch IX. K. 24. 27. ein bloßer Stein in Böotien den Herkules, zu Thespis den Cupido, zu Orchomenos die Grazien, zu Theben den Bacchus, und nach Herodian zu Paphos die Venus in der Gestalt eines Ecksteins oder einer Pyramide vor.\footnote{Von diesem letzten sind die Nachrichten zu unsicher, um bestimmt zu sagen, dass er ein Meteor-Stein gewesen, so von mehreren anderen, z. B. dem Stein des Jupiter Caseos, der Diana im Tempel zu Laodicäa --- (von dem Eckhel in Mus. Caesar. Abbildung auf Münzen anführt) vom Stein im Tempel zu Perga, zu Calchis in Syrien, zu Flavia Neapolis (dem alten Sichem) s. gleichfalls Eckhel und Pellerin Recueil de Medailles, und andere mehr; aber von jenem zu Orchomenas sagt Pausanias XI. C. 38. und IX. 25. bestimmt, dass er vor dem trojanischen Kriege zu Zeiten des Königs Eteokles vom Himmel gefallen.} Ein Beispiel zeigt die Münze Iaus Vaillaut Num. Graec. Imp. Tab. IV. No. 14., die ein Konisches Idol mit 2 Tauben und Leuchtern aus der Insel Cypern (wo der Venus-Dienst herrschte) vorstellt. Auch bei den Mexikanern findet sich nebst dem Elementen-Dienste die Verehrung der Bätylien oder heiligen Steine; merkwürdig ist die Nachricht, welche Alexander von Humbold im 2ten Hefte der Vues des Cordilieres; Seite 94, von der Gottheit der Tolteques gibt: "`ihr vornehmster Gott hieß Tlalocteuctli, er war zugleich Gott des Wassers, der Berge, und des Gewitters. Diesem Bergvolke waren die hohen in stete Nebel gehüllten Gipfel der Gebirge der geheimnisvolle Ort, auf dem der Donner erzeugt wird. Dahin versetzten sie den Thron des großen Geistes Teotl, jenes unsichtbare Wesen Ipalnemoani, und Tloque Nahuaque genannt, weil er nur durch sich selbsten ist, und sich allein umschließt. Von dieser kaum ersteiglichen Höhe herab kommt der Orkan sowohl, der die friedlichen Hütten zerstört, als der wohltätige Regen, der die Felder erquickt. Auf einem der höchsten Berge hatten die Tolteques dem Gotte Tlalocteuctli eine Bildfäule errichtet, aus einem wissen Steine, den sie für göttlich hielten (teotetl), nur roh war sie aasgehauen, denn dieses Volk, ähnlich darin den Orientalen, hatte abergläubige Verehrung für gewisse Farben der Steine. Vorgestellt ward diese Gottheit mit dem Blitz und Donnerkeil in der Hand, auf einem cubusförmigen Stein sitzend, eine Vase vor sich gestellt, wo rinne man ihm Caoutchoue und verschiedene Erdsamen opferte. Derselbe Cultus findet sich bei den Aztequen, die ihn bis zum Jahr 1317 der christlichen Zeitrechnung beibehielten, wo der Krieg, den sie mit den Einwohnern der Stadt Xochimileo führten, die erste Veranlassung zur Einführung der Menschenopfer ward."'

Bemerkenswert ist die Analogie dieses mexikanischen Stein-Gottes mit dem Alagabal der- Syrer, dem schwarzen Stein der Kaaba, dem Alasovid, Abadir, Dusares der arabischen Stämme, von denen wir später reden werden. Wenn es überhaupt scheint (wie Hr. v. Humbold aus wahrscheinlichen Gründen zeigt), das die Nationen, welche Amerika bevölkern, sich frühzeitig vom ersten Wohnsitze des Menschengeschlechts entfernt, und ihrer eigenen Leitung überlassen, lange herumirrend, allen Übeln und Unbequemlichkeiten eines herumirrenden Lebens unterworfen, sich später entwickeln, und zu einem höheren Grade von Kultur erheben konnten so finden sich bei ihnen dagegen d Spuren des ursprünglichen Natur-Cultus, wie früher bei den Völkern des alten Kontinents in seiner primitiven Form wieder. Wir dürfen z. B. nur die Zwergenähnliche Basalt-Büste der mexikanischen Priesterin im I. Hefte des Humboldischen Atlas pittoresque (die doch eher eine mexikanische Haus- oder Schutz-Gottheit zu sein scheint), wie überhaupt die meisten monströsen Abbildungen der Aztequischen Gottheiten anblicken, so zeigen sie eine auffallende Ähnlichkeit mit jenen phönizischen und alt-ägyptischen zwergartigen Götterchen, welche Herodot unter dem Namen Patäcken erwähnt, und die als Schutzgötter in Tempel und Vorhallen gestellt wurden.

Überhaupt war im Altertume der Glaube herrschend, dass die Gottheit ihr Bildnis selbst oft vom Himmel zur Erde gesandt habe, es mag nun dieses aus Stein oder einem andern Stoffe bestanden haben, so jenes der Diana von Ephesos, welches nach Plinius Zeugnis\footnote{Nat. Gesch. Buch XVI. Kap. 77.} aus Weinstock geschnitzt, und täglich mit Rürden getränkt wurde, um die Gliederfügungen beweglich zu erhalten, dass es für ein vom Himmel gefallenes Bild galt, älter sogar als Bacchus und Minerva, wie Plinius versichert, zeigen die Worte in der Apostelgeschichte\footnote{Kap. XIX.}: "`Ihr Bürger von Ephesus! ist denn irgend ein Mensch auf der Welt, der nichtweiß, dass die Stadt Ephesos die Dienerin des Tempels der Diana, und des vom Himmel herab gefallenen Bildes ist?"'

Jedes Volk hatte in der Mitte seines Versammlungsortes einen solchen Stein als Symbol seiner Einheit (την πολιν εστηαν), einen Lokal-Schutzgott (Δαιμον οιχοδεσποτην), und der Gott einen ihm dienenden Priester, nebst einem Hause (das später zum Tempel ward). Mit diesem rohen uralten Fetischismus verband sich frühzeitig ein geheimer Sinn, den die Priester und Mysterien dem Stein als Symbol beilegten,\footnote{Auch den Juden war der heilige Stein ein Symbol der Gottheit; so sagt Jesaias VIII. 14.\\
"`Jehovah Zebaoth -- den seht für heilig an,\\
Nur der sei eure Furcht, nur der euer\\
\hspace*{1cm} Schrecken,\\
Dann dient er euch zur Sicherheit, gleich\\
\hspace*{1cm} dem heil'gen Stein."'\\
Nach Justis Übersetzung in den Blumen althebräischer Dichtkunst.} und indessen das rohe Volk im exoterischen Dienste seinen Fetisch, den rohen Holz- oder Stein-Block vergötterte, deutete die geheimere Einweihung denselben als Darstellung und Symbol einer mächtigen Naturkraft, nämlich als Bild der Zeugung und der sich stets erneuernden Zeit, woher im frühesten Zeitalter der Lingam-Stein entstand, der mit der Tradition durch alle Cultus gehet.\footnote{Einen dahin sich beziehenden Gebrauch zweier indischen Völker (der Zechien und Albarachen), deren Sitz und Ursprung jetzt schwer anzugeben ist, führt ein wenig bekannter Schriftsteller, Vincentius Bollovancensis in Speculo histor. Cap. IV. an; die Stelle, deren Ouseli in seinen Noten zum Min. Felix. Lugd. Batav. 1672. p. 17. erwähnt, ist zu merkwürdig, um hier nicht ausführlich zu stehen:\\
\hspace*{0.5cm} Duarum Indiae gentium, quae vocantur Zechiam et Albarachuma, antiqua consuetudo fuit, nudos et decalvatos, magnisque ululatibus personantes Simulachra Daemonum circumire, angulos quoque osculari, et projicere lapides in acervum, qui quasi pro honore Diis exstruebatur. Inde est, quod in libro Salomonis dicitur; qui projicit lapidem in honorem Mercurii. Faciebant autem hoc bis in anno, sole scilicet existente in primo gradu Arietis, et rursus, cum esset in primo gradu Librae: hoc est, initio Veris et Autumni. Haec ergo consuetudo cum ab Indis ad Arabes descendisset; eamque suo tempore apud Mecham in honorem Veneris Mahumed celebrari reperisset; sic illam manere praecepit, cum tamen cetera idololatriae vestigia removisset. Illud vero soli veneri in illa celebratione dicitur exhiberi solitum, ut lapilli retro, id est, sub genitalibus membris projicerentur, eo quod Venus maxime partibus illis dominetur. Unde id adhuc hodie fit in domo Dei illicita quam vocant.}

Dieser Lingam oder Phallus-Dienst war jedoch in seinem Ursprunge nicht so unrein und obscön als Zoega\footnote{De orig. et usu obelisc.} und Friedr. Schlegel\footnote{Von der Weisheit der Indier.} ihn schildern; durch Entartung, besonders Einführung der Bacchischen Orgien mag er es später geworden sein; seinem Ursprunge aber, und dem inneren Sinne der alten Religion gemäß, war diese Idee gewiss heilig und rein. Denn die Indier, (und wie sie alle alten Völker) glaubten: sowohl Holz als Stein enthielte das Elementar-Feuer, und mit ihm das Zeugungs-Prinzip. Die Sonne nebst den. übrigen Sternen seien Steine. Im Steine ließen sie die Gottheit wohnen, und sich in Stein verwandeln. --- So Krischna am Ende seines Lebens, nach der griechischen Tradition der Stein des Chronos, den in Windeln gelegt, Rhea ihrem Gemahl statt des verfolgten Zeus zu verschlingen gibt.\footnote{S. das Titelkupfer nach einer antiken Ara im Museo Capitolino.} Pausanias traf ihn noch im Tempel, und er ist offenbar ein Bild der Zeit, so Niobe mit ihren 12 Monat-Kindern; des Sisyphus Rad; der Mythos des aus Steinen Menschen schaffenden Deukalions u. s. f. Der-Stein-Cultus schien demnach nicht sowohl aus einem rohen Fetischismus, wie Zoega meint, ausgegangen zu sein, als: eine durch Verwilderung der in der frühesten Zeit aus dem Ursitze des Menschengeschlechts in verschiedene Weltgegenden auswandernden Stämme entstandene Ausartung der ursprünglich reinen. Idee, welche die erste Natur-Religion unter diesen Symbole verstand.

Von der ursprünglichen Deutung des Steins auf Zeit kommen die Mahl- oder sogenannten Rechensteine; im Hermes (von dem alle Zeitrechnung kam) wurden sie in ganzen Haufen zusammengelesen, die ερμαχες ober ερμειοι φιλοι hießen,\footnote{Kanne allgemeine Mythologie.} der Cultus vergaß den Ursprung der heiligen Sitte, und jeder Wanderer legte zu dem Haufen noch einen neuen Stein. In Arabien erhielt sich die Sitte lange nach Muhameds Zeiten und die Geschichte der Erzväter erwähnt diesen Gebrauch. Dahin können wir auch die 360 Steinegefäße der ägyptischen Priester auf der Nil Insel nächst Philoe rechnen, mittelst welchen sie, indem sie selbe bei jedesmaligen Wiederbeginnen des bürgerlichen Jahres anfüllten, die Zeit- oder Zahl ihrer Jahrestage anzeigten.\footnote{S. Phamenophis. S. 96.}

Bündnis, Verträge und Eide wurden daher stets an solchen Steinen geheiligt. So hatte der Libanon einen Hügel, der wie Jacobs Steinhause \<gl`d> Gilead hieß. Dem Ursprung den Sitten, bei Steinen zu schwören, deutet selbst die Wurzel der Sprache an. So heißt: \<`bdy> häufen, und \<`bd> ein harter Stein. \<`yd> und Zeugnis geben, \<`d> Zeuge, welches in der Grundbedeutung Zeit, Zeugung und Feuer heißte und in der Zusammensetzung Gilead \<gl`d> sowohl. Steinhaufe des Zeugnisses, als der Zeit bedeutet.

Diese Einleitung führt uns auf nähere Erforschung des Stoffes und inneren Gehalts jener Steine, die man im Altertume zu religiösen oder dem gemeinen Wesen dienlichen Zwecken gebrauchte, und darum für besonders heilighielt. Hier sprechen. nun alle Zeugnisse, dass, wo nicht alle, doch gewiss eine große Zahl derselben wahre Aerolithen gewesen, deren Beschreibung ganz mit jener übereinstimmt, die uns die Physiker von den in späteren Zeiten gefallenen Meteor-Steinen geben. Die Rinden dieser Steine (sagt Reuß\footnote{Über den Steinregen bei Lissa, im Journal der Chemie, Physik und Mineralogie, 8. Band 2. Heft, S. 457.}) sind dunkelschwarz, stellenweis ins Braune ziehender Farbe, teils matt, teils schwach schimmernd, und an den sammetschwarzen Stellen non Pech-, an andern von schwachem Metallglanze; sie zeigen zahlreiche größere und kleinere Eindrücke und Erhabenheit, wie sie ein weicher dehnbarer Körper annimmt, wenn man ihn mit dem Finger dehnt oder kratzt. Sie fühlen sich im Ganzen ziemlich glatt, nur hie und da etwas rau an. In Hinsicht ihrer schwarzen Kruste bemerkt man zwar Verschiedenheit, einige sind dunkelschwarz, andere haben ein pechartig metallisches schwach schimmerndes Ansehen, und sind etwas mürbe; dieser Verschiedenheit ungeachtet ist doch ihre Gleichartigkeit nicht zu verkennen, und man bemerkt beim ersten Anblick derselben ihre Abstammung aus höheren Regionen. Ihre vorzüglichen Bestandteile sind, nach Klaproths neuester Untersuchung\footnote{S. Analyse des Meteor-Steins von Lissa in demselben Journal.}: Eisen, Nickel, zuweilen Chromium, Mangan, Kiesel-Erde, Bittersalz-Erde, Alaun-Erde, Kalk, Schwefel, wovon (wie er hinzusetzt) das Eisen als gediegen anzunehmen. Dieses metallischen Eisens wegen wirken auch die Aerolithen stärker oder schwächer auf die Magnetnadel, und werden von ihr angezogen.

Wie verschieden dieselben in ihrem Gewichte seien, ist daraus abzunehmen, dass man Steine von der keinen Dimension eines Zolls bis auf Massen von 3 bis 400 Pfund, ja selbst von 14 Zentnern (wie jener berühmte Eisenstein, der in Sibirien niederfiel) kennt.\footnote{S. die erwähnte Abhandlung, S. 457.} Unbestimmt ist ihre Form und Zahl; bald zugerundet-eiförmig, mit Ecken und Kanten; in der Mitte eingedrückt, mehrseitig, pyramidalförmig, an den Seiten abgestumpft, oft ganz kugelrund u. s. f. Gleichverschieden sind sie an Zahl, und in den ihren Fall begleitenden Nebenumständen, indem sie teils einzeln, teils in geringer Zahl, teils als förmlicher Regen (der Steine auf Meilenwegs umher streut, welche Hügel und kleine Berge bilden, wie kürzlich zu L’Aigle in der Normandie\footnote{S. Journal der Chemie am angeführten Orte.}; bald fallen sie bei ganz heiteren Tagen, bald bei stürmischem wolkigtem Himmel, aber immer (sagt Reuß) ist beim Herabfallen ein außerordentliches Getöse zu hören, das mit einem Knalle aus Kanonen, Pelotonfeuer, Wirbeln auf Trommeln, türkischer Musik, einem orgelähnlichen Pfeifen und Sausen in den höheren Luftregionen verglichen wurde; und in vielen, ja den meisten Fällen bemerkte man dabei eine Richtung von Südwest nach Nordost, so dass die nördlichsten Steine zuerst, die südlichen zuletzt niederfielen.\footnote{Journal für Chemie und Physik. S. 455.} Neuere Berichte erzählen die außerordentliche Wirkung solcher Phänomene auf die Gemüter der Augenzeugen. Ein Meteor dieser Art kam im I. 1789 über eine Gegend unweit Worms, woselbst ich mich auf einem, meiner Familie gehörigen Schlosse befand. Im Sommer bei heiterem ganz wolkenlosem Himmel entstand gegen Abendzeit ein immer zunehmendes Rauschen in der Luft, dass nicht sowohl dem Rollen des Donners, als dem lärmenden Zug eines Kriegsheeres- zu vergleichen war, und immer zunahm, je näher es unserem Wohnsitze kam. Die erschrockenen Landleute, die eben noch auf dem Felde waren, liefen, ein nahes Unglück ahnend, ihren Häusern zu, indessen das unsichtbare Meteor (ohne Schaden zuzufügen) über unsere Häupter hinrollte; alle Bewohner des Schlosses vernahmen dessen wunderbare Laute; aber unter den Anwesenden befand sich keiner im Falle, dies Phänomen gleich nach seiner Erscheinung zu verfolgen; erst nach einiger Zeit erfuhr man, dass eine Feuerkugel nicht fern des Ortes niedergefallen sei, nach der man sich gleichfalls nicht weiter umsah, die aber, den seither über Aerolithen bekannten Tatsachen nach, unbezweifelt ein Meteor-Stein war.

Beispiele der Art lassen mutmaßen, welche tiefe Eindrücke von Furcht und Erstaunen solche Phänomene in früherer Zeit auf rohe Menschen, die jede außerordentliche Erscheinung als züchtigende oder Schaden bringende Wirkung einer Gottheit ansahen, erregen mußten.\footnote{Noch gegen Ende des 17ten Jahrhunderts wurde ein in der Ortenau gefallener Meteor-Stein für ein sichtbares Zornzeichen des Himmels gehalten In einem seltenen Büchelchen (gedruckt 1671) wird diese Erscheinung erzählt. (Herr Gilbert hat sie im 1ten Stück der physischen Annalen für 1809 mitgeteilt.) Der Berichterstatter sagt unter andern: "`dass dieser Stein wie die Donnerfeile in der Luft generiert worden, werde ich mich schwerlich überreden können, weil er ein mineralisch Erz zu haben scheint, und nicht, wie andere dergleichen Steine, die frisch bekommen werden, nachdem sie herunter gefallen, nach Schwefel gerochen, oder heiß gewesen; sondern will viel ebender zugeben, dass diese Steine, weil man sie an unterschiedenen Orten so weit von einander gehört, aus Verhängnuss Gottes vom bösen Geist und seinem Anhang auf Erden gesammelt, in die Luft geführt, und von da wieder zerstreut worden."' --- Dieser Meinung gemäß, hält der Verfasser sie demnach für ein Prognostikon der steinern Türken Herzen und grimmigen Hundes Art, die sie gegen das teure Christenblut zu verüben pflegen! In der Beschreibung dieses Luft-Steines verdient besonders bemerkt zu werden, dass einer der Anwesenden erzählt: Er habe etwas über sich hinausfahren sehen, wie eine glühende Kugel, davon er niedergesunken; ein anderer sah etwas vom Grunde über sich spritzen, fand daselbst ein Loch, und den darin liegenden Stein anderthalb Fuß tief. Worauf der Verfasser jener Beschreibung die Hypothese begründet, dass dieser Stein aus der Erde in die Luft geschoben, und dann herabgefallen sei. Eine Meinung, die viel Ähnlichkeit mit jener neuen des Herrn Patrin (Annalen der Physik, 10. Stück für 180. S. 1891 hat, der die Bildung der vulkanischen Materien überhaupt aus einer chemischen Verbindung der gasförmigen im Innern der Erde zirkulierenden Flüssigkeiten erklärt, welche durch die mineralische Assimilation zu Steinen und Metallen werden; denen ähnlich, von welchen man annimmt, dass sie auf nassem Wege gebildet worden find. --- Der angeführte Meteor-Stein wog 10 Pfund, war, wie alle übrigen, an Farbe äußerlich schwarz, von Innen grau, etwas löchericht, wie mit dem Finger eingedrückt, seine Form beinah die eines Hundskopfs ohne Ohren u. s. f.} Selbst Blitze, sagt Plinius,\footnote{Nat. Gesch. Buch II. Kap. 53.} glaubten die Römer, würden von neun Göttern geworfen, und es gäbe derselben elf Arten, davon drei dem Jupiter zugehörten; aber nur zwei blitzende Gottheiten gäbe es, der Zeus, welcher am Tag, und der Summanus, der bei Nacht die Blitze wirft. --- Auch Steinregen (sagt derselbe Naturkundiger) sendet die Gottheit öfters herab; ein Beispiel liefert er in der Erzählung des bekannten Steinregens, der sich während dem Konsulat des Marius ereignete\footnote{S. Nat. Gesch. Buch II. Kap. 58. "`Man bat uns erzählt, dass zu Zeit des zimbrischen Kriegs, und noch öfter, sowohl vor- als nachher, ein Geräusch der Waffen und der Schall einer Trompete gehört worden sei. Im 3ten Konsulat des Marius sahen die Ameriner und Tudertiner Waffen am Himmel, die vom Morgen und Abend her so lange gegeneinander fuhren, dass die auf der Abendseite zurückgetrieben wurden."' --- Im folgenden Kapitel erwähnt Plinius dreier, ihm bekannt gewordener Meteor-Steine: einen, den Anaxagoras im 2. Jahr der 78. Olymp. vorhergesagt, dass er aus der Sonne fallen würde, und wirklich fiel er zu Aegospotamos (desselben erwähnt Aristoteles in Meteorologia C. VII. Δ) in Thrazien nieder; er war von der Größe einer fahrbaren Last, schwarz an Farbe, und an seiner Kruste angebrannt, weil, sagt Plinius, in der Nacht, da er fiel, eben ein Komet brannte; --- ein anderer im Gymnase zu Abydos; ein dritter zu Cassandria im Makedonischen; endlich einen, den er selbst kurz nach seinem Falle gesehen, im Vocontischen Gebiete. Übrigens hat von Steinregen, die in früherer und späterer Zeit fielen, Chladni ein, soviel möglich, vollständiges Verzeichnis gegeben in Gilberts Annalen der Physik XV. 310., worauf ich, um bekannte Dinge nicht zu wiederholen, verweise.}; und am Schlusse dieser Schilderung, Kap. 59, wo er von jenem Meteor-Steine spricht, den Anaxagoras, wie die Sage erzählt, längst vorherberechnet hatte, sagt er bestimmt, dieser Aerolithe werde im Gymnas zu Abydos in besonderer Ehre gehalten. --- Colitur (ist sein Ausdruck), welches Große, sein deutscher Übersetzer, wie mich dünkt, unrichtig mit aufbewahrt ausdrückt, da in der nächstfolgenden Periode es heißt: und die Kolonie, welche man Potidäa nennt, des Steines (nämlich der Verehrung wegen), so diese Kolonie dem Stein bezeigte, der wahrscheinlich ihr Lokal- und Schutzgott war, hierher geführt ward.\footnote{Von Steinregen, die zu verschiedenen Zeiten sich ereigneten, verdient besonders jener im Jahr Christi 769 gefallene erwähnt zu werden, dessen Aboulfaradi oder Gregorius-barhebraeus in seiner syrischen Chronik gedenkt, und der aus schwarzen Steinen bestand. Von einem andern im Jahr Christi 893 gefallenen spricht derselbe Schriftsteller. Beide bestätigen die Mutmaßung, dass die schwarzen Steine auf dem sogenannten Gileads-Hügel in Syrien (einer Gegend, wo sich gewöhnlich nur weiße Kalksteine vorfinden) gleichfalls Aerolithen oder Produkte eines Steinregens sind. S. Silvestre de Sacy Noten zu seiner französischen Übersetzung von Abd-Alatifs (eines arabischen Arztes aus Bagdad) Beschreibung Ägyptens. Paris 1810. 4. p. 505.}

Aber früher noch, als diese Epoche, wurden heilige Steine, Διιπετρα, verehrt, die der Beschreibung nach alle Eigenschaften wahrer Aerolithen hatten. Aus Sanchuniaton, den Eusebius\footnote{Praepar. Evang. L. I. C. 10.} anführt, lernen wir, dass die Göttin Astarte einen Stein fand, der als Stern vom Himmel gefallen war, und nachdem sie ihn aufgehoben hatte, denselben der Stadt Tyrus weihte.\footnote{Man erinnere sich hierbei, dass Astarte oder Alilat die Göttin war, die nach Herodot B. III. K. S. die Araber unter dem Bild eines Steines verehrten.} Man liest ferner in einem alten, dem Orpheus zugeschriebenen Gedichte: Von den Steinen, die Beschreibung eines Steines, Ophites genannt,\footnote{Ille enim Phoebus Apollo lapidem vocalem habendum Sideriten verum dedit, quem hominibus aliis placuit vocare anima carentem Ophiten funestum, subasperum, durum, nigrum, spissum circa ipsum vero circulo ab omni parte undique fibrae rugis similes, insculptae extenduntur.\\
Orph. λιθιχα v. 16-21.\\
Edit. Gesneri. Lipsiae 1764.} der (wie es darin heißt) der wahre Siderit sei. Apollon (so erzählt das Gedicht) gab diesen Siderit (Stern-Stein) --- anders nennen ihn Ophites --- dem Trojaner Helenos; der Stein, der die Gabe zu reden hatte, ist hart, schwer an Gewicht, und schwarz an Farbe. Viele im Kreis laufende Streifen sind auf seiner äußeren Rinde zu sehen. Helenos gebrauchte denselben mittelst verschiedener bei Beschwörungen üblicher Gebräuche zu Bezauberungen und Wahrsagerkünsten, wodurch er, wie es heißt, den Untergang Trojas vorgesagt habe.\footnote{S. Falconnet sur les Baetylos in den Mém. de l'acad. des Inscr. T. VI. p. 514.} Ein anderer Stein, den Plinius Astroides\footnote{Plin. Hist. nat. I. 37. C. 9.} nennt, und dessen sich, wie er hinzusetzt, Zoroaster zu seinen magischen Künsten bediente, ist unbezweifelt ein Aerolith, dem man magische Kräfte zuschrieb. In denen von ihm übrig gebliebenen Orakeln wird vorgeschrieben, einen solchen dem Himmel entfallenen und Gott geheiligten Stein zu opfern, so oft sich ein böser Daimon der Erde nähere; und im Leben dieses Weltweisen sagt Porphyr: Als der weise Perser sich in der Insel Kreta befunden, sei er vom Priester Morgos, einem der idäischen Daktylen mit einem Donnersteine zur Einweihung bereitet worden, eine Sage, die umso möglicher ist, da ihr ein naturhistorisches Factum zum Grunde liegt, und es scheint, dass in frühesten Zeiten auf dem Ida ein Steinregen gefallen sein müsse, da man auf dessen Gipfel nach Plin. Nat. Gesch. 37. B. K. 61. idäische Daktylen oder fingerförmige Steine von weißblaulichter Farbe in großer Menge findet, woraus sich leicht der später entstandene Mythos von Zeus Geburt am Ida erklären lässt. Denn in Kreta (nach Dikäarch die älteste Insel) wohnten die Eteo-Creter oder Idäi Daktili, eingewanderte Kolonisten aus Phrygien, die sich besonders um den höchsten Berg der Insel ansiedelten, und ihm den Namen des phrygischen Ida gaben. Von Cres, ihrem Könige, nahm die Insel den Namen. Die Idäi Daktili aber brachten die erste Kultur sowohl als den Dienst der großen Göttin dahin, und der Cabiren; die ersten Waffen wurden von ihnen erfunden; dieser Eisen-, vielleicht auch Magnetreiche Berg, der eine Wetterscheide der Gegend war, woran sich öfters meteorologische Phänomene zeigten, musste daher der Schauplatz großer Naturerscheinungen sein, und der Himmel (dem Mythos gemäß), als erzeugender Vater, kam -- sich verbindend --- zur mütterlichen Erde; der junge Zeus fand hier seine Wiege, und die Korybanten, die ersten Priester der Insel, pflegten seiner Kindheit; was Uranos aber aus seiner Höhe durch Meteore herabsandte, wurde als Erzeugnis von ihm angesehen; so waren auch die daktylischen Steine von ihm gesandt worden. Ob sie nun den ersten Bewohnern, oder diese jenen den Namen gegeben, bestimmen die Nachrichten nicht, aber das sagen die ältesten Geschichtsschreiber: Scepsius, Pherecides,\footnote{S. Hermanns Handbuch der Mythologie III. Th. S. 160 u. f.} Herodot,\footnote{Buch III. 37.} und späterhin Diodorus,\footnote{Buch V.} dass diese idäischen Finger (Daktylen), die für Zauberer galten, und sich auf Magie, Ordens-Einweihungen, geheime Wissenschaften legten, wodurch sie die wilden Bewohner aller Länder, zu denen sie sich begaben, in Erstaunen setzten, zuerst den Gebrauch des Feuers, des Eisens und die Bearbeitung desselben erfanden, weshalb sie für Söhne des Vulkans galten, und verdienten als Götter verehrt zu werden. Am Ida, den sie bewohnten, waren auch die ersten Erzgruben, und des Berges Schoos barg einen reichen Vorrat des ergiebigsten Eisens. --- Von des Zeus Geburt, Jugend und zarter Pflege durch die Nymphen in einer Höhle des Berges liefert die Sage die lieblichsten Mythen, denn das Kind, gefüttert mit Milch und Honig, an der Brust der Ziege Amalthea säugend, ward von freundlichen Bienen genährt, die, wie Diodor (Gesch. Buch V. Kap. 70.) erzählt, der Gott, um das Andenken seiner Geneigtheit gegen dieselben zu bewahren, in ihrer Farbe einem goldgelben Erze ähnlich machte, und sie so verwandelte. -- Also nicht in Gold selbst, wie die Worte deutlich sagen, sondern in ein dem Golde an Farbe ähnliches Erz, worunter, wie mir scheint, kein anderes zu verstehen ist, als der bei allen Erzgängen so gewöhnliche Schwefelkies, Markasit, den Agricola, seiner Analogie mit dem Magnet wegen, mica Magnetis nennt, und Langius in Hist. Lap. Helv. p. 21: μεταλλολιθος. Da die Biene aber das Bild der Arbeit ist, so scheint die Mythe in dieser Allegorie die fleißige Ausförderung des Eisens, eines Geschenkes, das der Gott dem Berge gab, symbolisiert zu haben. Vielleicht hat es auch Bezug auf das Feuer selbst, das dem Himmel oder Zeus heilig war, denn die Alten bedienten sich des Markasits zum Feuerzeug, gleich anderen Kieselarten, weshalb er den Namen Pyriten (Feuerstein) erhielt.\footnote{Versuch einer Lithurgik oder ökonomischen Mineralogie von Schmieder. Leipzig 1804. II. Teil, S. 524.}

Mehrere Zeugnisse über dem Himmel entfallene und deshalb für heilig gehaltene Steine ließen sich aus alten Schriftstellern anführen.\footnote{Man lese vorzüglich Bochard Canaan L. 2. Selden de Dissyris, Zoega de Usu et Orig. Obelisc. Falconnet in der angeführten Schrift.} Wir eilen jedoch unserem Zwecke näher, indem wir den eigentlichen Namen berühren, den sie im Altertume trugen.

Bätylien hießen sie, ein Name, der nicht ursprünglich griechisch, sondern, wie Eusebius\footnote{Praepar. Evangel. L. I. C. 10. S. auch Hesychius, Priscian, das Etymologicon.} nach Philo von Biblos, dem Übersetzer des Sanchuniaton, zeigt, phönizisch ist; belebte Steine nennt sie dieser Geschichtsschreiber, und sagt, der Gott Cölus habe sie mit kühner Kunst gebildet; aber Betül war einer der vier Kinder des Himmels (Uranos) und der Gea (Erde), deren drei andere Saturn, Dago und Atlas hießen.

Diesen Bätylos (Sohn des Himmels), der Phönizier Gott, muss man daher sorgfältig von den kleineren Bätylien unterscheiden, die, wie der Aberglaube wähnte, von kleinen Gottheiten bewohnt werden, welche ihnen magische Kräfte mitteilten.

Den Namen Bätylos leitet Bochard\footnote{Canaan L. II. Cap. 2. p. 75.} vom Steine Jacobs, den er auf seiner Flucht nach Mesopotamien an einem Orte des Libanons fand, und nachdem er ihm in der Nacht, wo er den bekannten Traum hatte, zum Hauptküssen gedient, ihn\footnote{Genes. 28. 18. von diesem Gebrauch heilige dem Götterdienste geweihte Steine zu salben. S. Mnuc. Felix Edit. Lugduni 1672. p. 15.} (einem alten ursprünglichen Gebrauche nach) mit Oel salbte, zugleich ausrufend: Der Herr ist wahrhaft an diesem Orte, und ich wusste es nicht, und den Ort, der vorher Luz hieß, Bethel nannte, das ist: Haus des Herrn, wovon der Stein den Namen erhielt, und bei den Nachkommen in besondere Verehrung kam, die aber bald bei den nomadischen Kananiten in Abgötterei ausartete, weshalb er auch, wie die jüdische Tradition erzählt,\footnote{Quanquam ille Cippus amatus fuit a Deo temporibus Patriarcharum, postea tamen edit eum, propterea quod Chananaei deduxerunt illum in ritum Idolatriae. Bochard Canaan p. 785.} als Gott zuwider durchs Gesetz verworfen ward.

Nicht unfern von diesem Orte zu Gilead,\footnote{Nach den mosaischen Nachrichten die eigentlichen Vor-Alpen des Libanons.} dem Steinhügel oder Steinberge, hatte früher schon Jacob, eh' er von seinem Schwiegervater mit Söhnen und Frauen schied, Steine zu einem Haufen gesammelt, die der eine Sahadutta, der andere Gilead nannte, aber Jacob hatte zuvor einen einzigen Mahlstein errichtet. Genes. 21 --- 45. Dieser Stein muss von beträchtlicher Größe gewesen sein, indem er ihn als Säule aufrichtete, die unbeweglich stand, und zu nichts anderem, als zum Altar dienen konnte; so sagt die Tradition, und setzt hinzu: seine Farbe sei schwarz gewesen.

Wenn hieraus auch nicht zu erweisen ist, dass es derselbe Stein sei, der, wie die Moslemin vorgeben, noch heutzutage sich in der Kaaba zu Mekka befindet, so ist doch so viel sicher, dass die Verehrung, die man ihm seit den Zeiten dieses Patriarchen bezeigte, dem Aberglauben Veranlassung gab, die großen schwarzen Steine, die man unter verschiedenen Benennungen im Orient verehrte, für diesen Stein auszugeben, dessen Heimat Syrien und der Libanon ist, dem er gleichwohl als eigentümliche Steinart nicht zugehören kann; denn wie ein unterrichteter Reisender,\footnote{} dem wir in aller Hinsicht glauben dürfen, berichtet, besteht die Grundlage dieses Gebirges, wie überhaupt von ganz Syrien, aus einem harten, weißlichten kieselartigen Kalksteine, so dass die schwarzen größeren und kleineren Steine, die zu verschiedenen Epochen sich dort vorfanden, dieser Gegend fremd waren, und außer-Tellurisch durch irgend ein Meteor dahin gebracht sein müssen. Steinregen und dergleichen Phänomene scheinen überhaupt zu jener Zeit in Phönizien und Syrien sich öfters (ob aus atmosphärischen*) oder anderen Ursachen) ereignet zu haben. So jener zu Josua Zeiten: "`Gott ließ bis gen Aseka große Steine über sie regnen, so dass durch diesen Steinhagel eine größere Zahl Kanaaniten umkam, als durch das Schwert der Israeliton,\footnote{} und der Steinregen bei Mose 28. a4., womit der Herr sein Volk bedrohet: "`der Herr wird deinem Volke Staub und Asche für Regen geben vom Himmel, bis du vertilget werdest."`

*) Volney Reise nach Surien und Ägypten, deutsche Übersetzung Th. I. S. 2 n. s. f.

Jener Jacobs-Stein, der wahrscheinlich doch nicht mehr wog, als die 14 Zentner schwere Eisenmasse, welche in Sibirien niederfiel, ist den vorerwähnten Umständen gemäß unbezweifelt ein Aerolithe gewesen, und die Verehrung, die man ihm (seines Ursprungs wegen) bezeigte, artete in späteren Zeiten durch Aberglauben in Abgötterei aus.

*) Was dadurch begreislich wird, dass diese Bergkette als Hauptwetterscheide jener Gegend, wie Volney sagt, in meteorologischer Hinsicht äußerst merkwürdig ist.

*) Josua 10. 11.

Wie nun diese Steinmasse als Symbol einer Gottheit auf den Spitzen des Libanon stand, so verehrte man östlich an Indiens Grenzen zu. Nepal\footnote{} (unfern Benares) den schwarzen Stein als Bild des Mahadeo (den Gott der Liebe und Zeugung), auch in Cachemir verehrte man einen vom Himmel gefallenen Stein, einen anderen als Lingam in der Pagode von Perwuttum; so wie westwärts in Griechenland den viereckigten schwarzen Stein Saturns, der zu Pausanias Zeiten noch im Apollo-Tempel zu Delphos bewahrt,\footnote{} täglich mit Oel bestrichen und roher Wolle umwickelt wurde. Dahin kann man gleichfalls jenen Stein von Pessinunt zählen, welcher der Cybele heilig war, und im zweiten punischen Krieg nach Rom gebracht wurbe.

*) In Dappers Asia steht p. 111 nach Dèlla Vallès Beschreibung eine treue Abbildung des Mahadeva (oder Gott der Zeugung) als Lingam. S. ferner Account of the Kingdom of Nepal im 2. Th. der Asiatik Researches, 8te Ausgabe S. 307, von einem im nördlichen Europa niedergefallenen Aerolithen liefert Bartholin eine Beschreibung in ist. Anatom. Centur. IIl. et IV. p. 337.

Es zeigten sich (sagt Appian vom Hannibalischen Kriege, Kap. 56.) zu jener Zeit zu Rom schreckliche Wunderzeichen am Himmel, weswegen die zehn Männer die Sibyllinischen Bücher nachschlagen mussten, und aus denselben antworteten:

"`Es werde in jenen Tagen zu Pessinus in Phrygien, wo die Mutter, der Götter verehrt wird, etwas vom Himmel fallen, dass man nach Rom bringen müsse;"' nicht lange darnach sei die Nachricht gekommen, es sei wirklich herabgefallen, worauf denn das Bildnis der Göttin nach Rom geholt wurde, an welchem Tage die Römer (setzt Apian hinzu) wirklich noch das Fest der Mutter der Götter feiern. --- Die Größe, Form und Farbe des Steins beschreibt Arnobius advers. Gentes L. VI. et VII. allatum ex Phrygia --- - --- nihil quidem aliud nisi lapis quidam non magnus, ferri manu hominis sine ulla impressione, qui posset, Coloris fulvi atque atri, angulis prominentibus inaequalis, at quem omnes hodie ipso illo videmus in signo Oris loco positum indolatum, et asperum et simulacro faciem minus expressam simulatione praebentem. --- Die Worte: in Signo Oris, zeigen, dass der Stein einem Munde glich, und man ihn deswegen an des Mundes Sselle ins Antlitz der Göttin einfasste, wodurch, dem geheimen Sinn nach, die Bildsäule diejenige Gottheit wurde, die man im Stein verborgen glaubte die Orakel gingen aus diesem Munde (dem geheiligten Steine) hervor, welches Prudentius in folgenden Versen beschreibt: Lapis nigellus evehendus essedo Muliebris Oris clausus argento sedet. Diesen Woxten Muliebris Oris gemäß, auch des Umstandes wegen, dass der- Stein von sehr unbeträchtlicher Größe\footnote{} gewesen sein müsse, indem bei seiner Ankunft in Rom im Jahre 548 die römischen Damen ihn, wie Livius versichert, wechselseitig von Hand zu Hand bis zum Tempel der Victoria trugen,\footnote{} hält Falconnet (in den Mém. de l'acad. des Inscr. T. XXIII. Dissert. de la Mere des Dieu) diesen Stein für einen Hysteriolithen der Ähnlichkeit wegen mit einem Munde, welche Form dem frühen Aberglauben Anlass gegeben habe, ihn als den Mund einer Göttin zu verehren, der esoterische Cultus aber habe darunter die Natur, als Urquelle aller Wesen symbolisiert, in welchem Sinne auch Irenäus von den Kainiten, einer christlichen Sekte der früheren Jahrhunderte, sagt: Cainiti Hysteram fabricatorem Coeli et Terrae vocant. L. I. contr. Haereses C. 35. wie dasselbe Wort zuweilen auch für die Mutter aller Wesen galt. Ähnliche Bätylien (wie Salmasius zu Lambridii Helo gabal. Edit. Lugd. Battav. 1671. T. 1. p. 801 zeigt) waren auch anderen Gottheiten, dem Jupiter, Saturn, der Sonne geweiht, und von derselben Gattung war jener Stein, der im Eingang des Tempels der Diana zu Laodicäa stand.

*) Pausan. Griechenland B. X., Kap. 24.

*) Welches jedoch Banier in einer Abhandlung über die Mutter der Götter im V. Band der Mém. de l'acad. des Inscr. p. 244. leugnet, hauptsächlich aus dem Grunde, weil, wäre der zu Pessinunt gefallene Stein von unbeträchtlicher Größe gewesen, er nicht so leicht bemerkt, und in fernen Gegenden bekannt worden wäre. Vielleicht aber fielen mit demselben noch mehrere kleine, und einer dieser minder großen Steine konnte ja nach Rom gebracht worden sein. -- Zugleich bemerkt Banier, dass die pessinuntischen Priester der großen Göttin auf dem Gürtel, der sich um den Leib schloss, kleine geweihte Bätylien trugen; ob diese gleichfalls vom Himmel gefallen waren, wird nicht angegeben. --- Aber von jenem Hauptsteine der Göttin sagt auch Arnobius, er sei von beträchtlicher Größe gewesen, und habe (der alten Sage nach) für jene Felsmasse gegolten, von der Deukalion die Steine abschlug, aus denen er Menschen bildete; der nach Rom gebrachte muss also doch wohl von kleinerer Art gewesen sein.

*) In terram elatam tradidit (scipio Nasica) ferendam Matronis --- - --- eae per manus, succedentes aliae aliis, in aedem viotoriae pertulere. Liv. XVII. v. 16.

Abadir,\footnote{} der große mächtige Herrscher, -- Pater magnus, der Gott des Berges, hieß auch dieser Steingott, und die Nabatäer (ein arabischer Stamm) verehrten ihn unter dem Namen Dusares, Teusares,\footnote{} dessen Cultus später sich bis nach Großgriechenland verbreitete, unter dem Namen Abadad in Persien, Alassovid bei den Arabern, kommt er noch nach, Mahameds Zeiten vor, und in vielen Stämmen des südlichen Arabiens bei den Patrinsern, Adränern, Bostrenen, Dacherauern wurden, wie bei den Nabatäern, Bätylien oder Steingötter verehrt. In der alten Coelo- Syrien-Stadt Emesa,\footnote{} wo nach Pococks Bericht in der Nähe sich sine Menge schwarzer Steine finden, woraus sich allmählich ein Hügel formte,\footnote{} bildete sich, nach Strabo, sehr frühe schon ein Sammelplatz der Verehrung mehrerer arabischen Stämme, welche die Sonne unter dem Bilde eines an jenem Orte aufgefundenen schwarzen runden, spitzig zulaufenden Steines anbeteten, dem in der Folge ein prächtiger mit Gold und Silber ausgeschmückter Tempel erbaut wurde.\footnote{} Heliogabal, Gabal, Alagabal, auch Malach Bilos, und zwar wie Selden vermutet, von Moloch-Bel oder Baal, hieß dieser Gott; Selden, der gleich Bochard\footnote{} denselben mit Recht als phönizisch annimmt, leitet den Namen von Ahgol-Baal\footnote{} oder Agalibal, dem runden zirkelförmigen schnellen Gott\footnote{} her, und ist der Meinung, nicht die Sonne, sondern Zeus oder Jupiter sei unter diesem Symbol verehrt worden. Andere setzen den Uranos, andere Bel an die Stelle, es sei nun dieses, oder wie es wahrscheinlicher und allgemein angenommen ist, die Sonne selbst der Emeser Gott gewesen, immer werden wir auf das Urelement Feuer zurück geführt, und der Meteor-Stein, dem höheren Äther entfallen, in seiner sphärischen Pyramidalform ist ein schickliches Symbol der Gottheit; indem, wie wir aus Platon wissen, die Pyramide die erste ursprüngliche Form, als der Urstoff des Feuers angesehen ward (s. Plutarch über den Verfall der Orakel) und die Kugel oder Sphäre eines der ältesten Symbole der Gottheit war,\footnote{} wozu in Hinsicht der Steinmassen, die auf unsere Erde herab fallen, noch in Erwägung kommt, dass Diogenes die Sterne für bimsteinartige und glühende Steine hält,\footnote{} die oft auf die Erde herab fallen und da verlöschen; und Anaxagoras (der uralten orientalischen Tradition gemäß) behauptete, der die Erde umgebende Äther sei seiner Natur nach feurig, und reiße durch die Heftigkeit seines Umschwungs Felsenstücke von der Erde mit sich fort, die er mittelst der Entzündung in Sterne verwandle. Zu dieser Meinung ward Anaxagoras wahrscheinlich durch jene im zweiten Jahr der 75. Olympiade zu Aegos Potamos in Thrazien niedergefallene Steinmasse, deren Fall er, einer alten Sage nach, berechnet und vorhergesagt, bewogen. Mehrere Schriftsteller versichern einstimmig, nach Plinius Worten: \emph{Praedixisse Coelestium Litterarum scientia quibus diebus saxum casurum esse e sole, idque factum inter diu} --- sollte man denken, es sei dabei eine Sonnenfinsternis, oder sonst ein Himmels-Phänomen vorgefallen; da aber die Geschichte hierüber schweigt, ist diese Vorhersagung wohl zu jenen Dichtungen und Traditionen zu zählen, die im Altertume so häufig sind. Von diesem Steine gibt auch Diogenes Laert. II. 10. Zeugnis: φασι δ'αυτον προειπειν την περἰ Αιγος ποταμον γενομενιν τοῦ λιθοῦ πτωσιν, ὀν ειπεν εκ του ηλιου πεσεισθαι. So merkwürdig war derselbe dem Altertume, dass selbst der Parische Marmor seiner erwähnt:
αφ' ὁυ εν Αιγος ποταμοις
ὁ Λιθος επεσε --- -

*) Bochard Chanaan L. II. C. 2. p. 786. leitet den Namen vom Phönizischen: Eben-Dir oder: Aban-Dir Lapis sphaericus, Talis enim (setzt er hinzu) Boetyli forma.

*) Suidas sagt hievon: "`Theusares --- Dusares, id est Deus Mars qui Petrae in Arabia maxime colitur. Simulacrum ejus est Lapis niger, quadratus, informis, altus pedes sex, latus duo, et aurea basi impositus."' Weitläufiger hierüber Zoega de usu et orig. Obelisc. p. 205.

*) Heut zu Tage Hems oder Hims. S. Mannert Geogr. der Griechen und Römer. VI. Th. I. Heft, S. 458.

*) Gleich jenem großen Steine zu Bethel, umgehen von einem Hügel kleinerer Bätylien.

*) Herodian V. C. V.

*) Chanaan L. II. C. 2. p. 797.

*) Selden de Dis Syriis Syntagma 2. p. 220 et seq.

*) Deus rotundus, circularis aut volubilis, ut dicebant ii, qui sphaeram mundi Deum sentiebant apud Ciceronem de Nat. Deor. II.

*) Von der sphärischen Form des Alls, oder der Welt, und dem Kreise als Bild der Gottheit s. Aristoteles de mundo L. II. C. 14. D. Meteorologia C. 7. item L. de Zeone et Gorgia. --- Plutarch de Plac. Phil. C. 6 --- Auch den Ägyptern war die Kugel das Symbol des Universums. S. Kircheri Sphynx p. 25. --- Dass noch jetzt in Indien die Gottheit unter der Gestalt einer Kugel verehrt wird, lesen wir in Haafners Reise nach der Küste Koromandel:
"Am Meeres Ufer sahen wir einen alten verlassenen Tempel, der wahrscheinlich einst dem Ischuren oder allerhöchsten Wesen geweiht war, denn man sah an dem Gebäude keine Gottheiten abgemalt oder ausgehauen. Die indischen Pundits sagen, das höchste Wesen, das sich in zahllosen Werken offenbart, sei so erhaben, dass es nicht durch Figuren dargestellt werden könne. Da diese Figur keine und doch alle Gestalten hat, so wird sie unter dem Bilde einer steinernen Kugel auf einem Fußgestell in der Mitte des Tempels vorgestellt, nie wird dieses Symbol in Prozessionen umhergetragen, die Tempel haben keine Tänzerinnen, und öffentliche Feierlichkeiten werden in ihnen nicht angestellt; man opfert diesem Wesen nichts als Feldfrüchte, und der Dienst, den die Brahminen in seinem Tempel versehen, besteht bloß in Lobgesängen und Gebeten. --- Dies ist das einzige allerhöchste Wesen der Indier, das Breem Brrm, Gott, oder das höchste Wesen, der Tausendnamige heißt.

*) Plutarch de Placit. Philos. C. 13.

Aus allen Zeugnissen, die wir im Aristoteles,\footnote{} Plutarch\footnote{} und Stobäus\footnote{} gesammelt finden, erhellt der allgemeine Glaube der alten Welt an die Feuer-Natur der Gestirne, und die Meinung, dass teils erloschen, teils als Lichtbothen sie zuweilen zur Erde sinken, überhaupt ihrer Natur nach freundlich leuchtende, leitende Wesen sind, günstig dem Wanderer und Schiffenden auf unbekanntem Meere. Das waren in den ältesten Zeiten schon die Plejaden, die hellen Morgen- und Abendgestirne, besonders die himmlischen Dioskuren, deren Feuer-Gewalt in der Atmosphäre so mächtig wirkte, dass ein alter Mythos sie mit den Cabiren vermischend (wahrscheinlich ihrer Feuer-Natur wegen), für Söhne des Vulkans und der Cabeira, die selbst eine Tochter des Proteus war,\footnote{} ausgab. --- Wie nun herabgefallene Steine für erloschene leblose Sterne galten, wurden auch die Feuer- und Lufterscheinungen, die in der Atmosphäre teils auf der Erde, teils im Meer sich zeigten, als meteorische, den Gestirnen zugehörige oder entsunkene Teile angesehen, die der Mythos bald in Sterne, und, da Sterne göttlicher Natur waren, in Götter verwandelt, so bewirkten, nach Diodors Erzählung, die wohltätigen Zwillingsbrüder bei drohendem Sturme die Rettung ihrer Gefährten der Argonauten durch zwei Sterne, die auf ihre Köpfe fielen, und dies, fügt Diodor\footnote{} hinzu, gab die Veranlassung, dass Seefahrende, die Sturm litten, den samothrazischen Gottheiten (in deren Geheimnisse Castor und Pollux eingeweiht waren) Gelübde taten, und die Erscheinung der Sterne als ein Sichtbarwerden der Dioskuren. betrachteten. Am samothrazischen Seehafen standen ihre Bildsäulen als Schutzgötter, mit deren Errichtung es folgende Bewandtnis hat: vor ihnen, in den frühesten Zeiten standen an derselben Stelle zwei Statuen des Himmels und der Erde, die großen cabirischen oder phönizischen Gottheiten, deren Varro de Ling. lat. Lib. IV.\footnote{} erwähnt. Dieser Gottesdienst, ursprünglich aus Phrygien und Thrazien, blieb unverändert bis zu Ankunft der Pelasger, wozu (vermutlich noch vor deren Ankunft) die Hecate, eine thrazische Landesgottheit, kam, die in Samothrazien eine ihr eigne furchtbare Höhle ober. Grotte, Mysterien und Opfer erhielt. Zu diesen Landesgottheiten brachten die Pelasger bei ihrer Einwanderung ihre eigene Gottheiten, die Ceres, Proserpina, und die drei Dioskuren mit (denn vor den zwei späteren kannte die ältere Mythe drei derselben). Diese 5 auf Samothraze eingeführten Gottheiten behielten jedoch aus Achtung an dem Altertum den Nahmen der älteren bei, und wurden wie jene Cabiren (Καβαιροι) benahmt. Den Ursprung dieser Cabiren, wie schon der Nähme gibt, müssen wir in Phönizien suchen, denn Cabir \<kbyr> gilt bei den Hebräern und Arabern für Groß, daher bei den Sarazenen, wie wir aus Cedrenii Chronicon\footnote{} lernen, Cabar, Alla, die große Cubar oder Göttin, so viel als Venus, Astarte, der Abendstern hieß. --- Gatterer, in einer Abhandlung in den Göttingischen gelehrten Akten, der alles aus Ägypten herleitet, und auf den Kalender beziehet, hat die meisten Götter, auch die Cabiren, vom Nile hergeleitet; aus der Ursache hauptsächlich, weil im Tempel zu Memphis ihre Bildsäulen, die Cambyses zerstörte, verehrt wurden; allein Herodot, der Buch III. Kap. 37 die Sache erzählt, sagt bloß: "`Cambyses trat in den Tempel der Cabiren, dessen Eingang das Gesetz nur dem Priester gestattete, und ließ alle darin befindliche Bildsäulen verbrennen."' Sie glichen jener des Vulcan, dessen Söhne, wie man sagt, die Cabiren waren. Was aber den Griechen Vulcan war, nannten die Ägypter Phthas, den belebenden Weltgeist; und die vier Untergötter, seine Söhne, mögen sonach entweder die Elemente, oder die fünf Sinnlichkeiten, die in der indischen und persischen Kosmologie gleichfalls erscheinen, vorgestellt haben. --- Vor dieser Stelle noch sagt er: "`auch in den Tempel des Vulcan trat Cambyses, und auf tausenderlei Weise verspottete er die Bildsäule dieses Gottes. Sie glichen jenen Patäken, die die Phönizier an das Hinterteil ihrer dreirudrigen Schiffe befestigen, und um denen, so noch keine gesehen, einen. Begriff davon zu geben, genügt zu sagen, dass sie Pygmäen glichen; es waren aber, so viel sich von einem Gegenstande sagen lässt, über den, außer Homer, alle alten\footnote{} Schriftsteller schweigen, diese Patäken kleine ungestaltete Figuren mit runden dicken Köpfen und Bäuchen, wahrscheinlich Laren und Hausgötter der Phönizier, die sie als Schutzgeister auf ihren Seereisen mit sich nahmen. Gatterer (a. a. Orte) indem er sie aus Ägypten herleitet, hält sie für Symbole der 5 ägyptischen Schalttage\footnote{} z es mag sein, dass die Ägypter diese Tage ebenfalls unter solchen kleinen Bildern verehrten, oder ihre Steingötterchen an die Stelle jener phönizischen Pygmäen setzten, ihr Ursprung, wie der Nähme bleibt, was auch Herodot bezeigt, nichtsdestoweniger phönizisch. Es waren mit der Bildsäule Vulcans, ihres Vaters, der, wie Herodot sagt, ebenso vorgestellt wurde, dieser Götterchen 5; aber nach einem Scholiasten des Appollonius, den Bochard Chanaan Buch I. Kap. 12. S. 427. anführt, waren der cabirischen Gottheiten nur vier, namentlich: Arieros, Ariokersa, Ariokursos und Casmilus, die der Scholiast also deutet: den ersten auf die Demeter, den zweiten auf die Kore oder Proserpina, den dritten auf den Hades, den vierten auf Hermes; die drei ersten stimmen ganz mit der Idee der Kybele, oder großen Göttin überein, der vierte, Casmilus, oder Camilus (Mercur, Thot), war eigentlich ein die andern bedienender Gott. Diese Cabiren nun sind dieselbe, welche durch Phönizier oder Phrygier nach Samothrazien kamen. Wie gesellten sich aber zu diesen das Zwillingspaar Castor und Pollux? -- Aus Pausanias (III. -12.) wissen wir, dass sie 40 Jahre nach ihrem Gefechte mit dem Idas und Lynäus vergöttert, und (wie Clem. Alex. Strom. I. S. 382. sagt) drei und fünfzig Jahre nach Herkules unter die Götter versetzt worden. Ferner berichtet Pausanias III. 26., dass die Dioskuren auf der kleinen an der Küste Lakoniens gelegenen Insel Pephnos, (worauf, einer alten Sage nach, sie geboren waren), durch zwei kleine erzne Säulen, die längst ohne Bedeutung dastanden, und wahrscheinlich von Phöniziern oder Ägyptern aus Dankbarkeit für eine glücklich vollbrachte Reise ans Ufer gestellt waren, abgebildet wurden. Diese Säulchen waren zugleich Schutzgötter der Schifffahrt; und da, durch diese Idee verleitet, man sie zu den Cabiren gesellte, ward man umso leichter bewogen, sie mit jenen am samothrazischen Seehafen stehenden zu verwechseln, als alle Cabiren wie Kinder, oder Pygmäen, das heißt: Bilderchen mit dicken Bäuchen, großen Munden, Augen und Ohren (Zwergen) vorgestellt wurden, wozu noch kommt, dass in Attika beide, die Cabiren und Dioskuren, den Ehren-Nahmen Anaktes\footnote{} gemeinschaftlich trugen. Bald wurden beide zu Sparta, in Attika, Samothrazien, Lemnos, mit einander verwechselt, ja das Ansehen der älteren bald durch den Dienst der jüngeren Dioskuren, Castor und Pollux verdrängt, so, dass wenn von Cabiren und Dioskuren die Rede war, man immer an diese als die bekanntesten dachte, und da in Ermangelung des Kompasses die Alten bei ihren Seefahrten sich bloß nach dem Laufe der Sterne richteten, war es natürlich, dass die verstirnten Heroen Beschützer der Schifffahrt wurden, und die leuchtenden Schiffe umgebenden Meteore wurden angesehen als unmittelbar von den göttlichen Dioskuren gesandt.

*) In den Büchern der Physik, Meteorologie, de mundo.

*) de Placit. Philos.

*) Stobaei Ecclog. physicae C. XV. I.

*) Handb. der Mythologie von Herrmann III. Th. S. 172.

*) Im 4ten Buch C. XLIII. der Geschichte. Diese Sterne aber waren nichts als elektrische Funken, die bei Meeres Stürmen öfters vorkommen, und heutzutage unter dem Namen St. Elms-Feuer bekannt sind.

*) Principes Dei Coelum et terra, und hins zufügt: sunt Tautes et Astarte apud Phoenicos, ut idem principes in latio Saturnus et Obs. Terra enim et Coelum ut Samothracum initia docent, sunt Dei magni et hi, quos dixi multeis nominibus --- nam neque ut vulgus putat, hi Samothraces Dii, qui Castor et Pollux. Sed hi Mas et foemina --- - Divi potes, et sunt pro illeis, qui iis Samothrace haec sunt Coelum et Terra.

*) S. Is. Vossius de Orig. Idolatriae Lib. 2. p. 467. --- Dasselbe bestätigt der Orubische Hymus 37, und Tertullian de Spectac. C. 8.

*) S. hierüber Larchers Roten in seiner Überhebung des Herodot. Tome 3. p. 303.

*) Das für und wider diese Meinung s. in Hermanns Handb. III. Th. 15.

Diesen Cabiren-Cyclus, den wir, alten Sagen gemäß, hier in gedrängter Kürze sammelten, haben Mythologen und Dichter so vermischt, dass das widersprechendste Ganze daraus entstand. Klärer finden wir die Sache in den Historikern, besonders den älteren vorgestellt, und nach kosmologischer Ansicht ist, je weiter wir ins Altertum dringen, die früheste Vorstellung gewiss die der Natur-Weisheit gemäßeste, nämlich: dass die zwei früheren Bildsäulen auf Samothraze das Symbol des Himmels und der Erde, des männlich und weiblichen, oder (wie Varro sagt) des trockenen und feuchten Prinzips sind. Dann die drei ältesten Dioskuren, Tritopatreos, Kabuleos und Dionysios, Symbole des Einflusses der Gestirne oder des höheren Himmels auf die Erde; endlich die vier Cabiren der Phönizier [zu denen sich jene fünf ägyptischen Götter (gleich falls Söhne des Hephaistos) anreihten, Symbole der Elemente: Luft, Feuer, Wasser, Erde] und wollte man das fünfte hinzusetzen, des Äthers. Wir finden demnach, um auf den Stein-Cultus zurückzukehren, in diesen alten missgestalteten Götterchen die erste Veranlassung zur Anwendung der Bätylien, in denen man die Elemente und Naturkräfte, so wie in den Patäken die kugelförmigen leuchtenden Himmelskörper verehrte. Wenn man nun zu den vier ursprünglichen Cabiren die fünf ägyptischen Tagesgötter reiht, und mit diesen die drei älteren Dioskuren verbindet, so geben diese zusammen die zwölf phönizischen Hauptgötter, gleichfalls auf die Urelemente deutend, deren älteste Abbildung in rohen Bätylien teils die Abraxas, andern Teils die Hermen veranlassten, aus denen späterhin die geformteren Bildsäulen hervorgingen.

*) Cicero de Nat. Deorum III. 21.

Syrien und Phönizien ist, wie wir gesehen haben, die vorzügliche Heimat der Bätylien, und wie Emesa ein Hauptort religiöser Vereinung, zugleich aber auch Stapel- und Sammelplatz des Handels vieler umher liegenden Stämme und Völkerschaften geworden, so verbreitete sich auf demselben Caravanen-Wege und in den Tälern des Libanons der Sonnendienst vom Haupttempel in mehrere benachbarte Städte, worunter die beträchtlichsten Heliopolis (oder Balbeck) und die Palmenstadt Palmira waren, deren Ruinen noch heut zu Tag von ihrer ehemaligen Pracht zeugen, wo aber überall der schwarze viereckige Stein den Gott abbildete. Als die Römer Syrien erobert hatten, ging dieser Sonnendienst nach Rom, ein Emmesenet Priester des Helagabal-Tempels (der falsche Antonin) der, wie es üblich war, als Oberpriester den Namen des Gottes annahm,\footnote{} jener reitzende Jüngling (wie Julian in den Cäsaren ihn nennt), gleich berühmt durch seine Schönheit und seine Ausschweifungen, bestieg den Thron des Reichs, und auf Münzen sowohl\footnote{} als Denkmalen\footnote{} wird das Bild des Gottes unter der Gestalt eines großen Steins oder Hügels auf einem Wagen ruhend, vorgestellt. Dass die Steinmasse, die den emmesener Gott darstellte, ein wahrer Aerolithe gewesen, zeugen Herodians Worte: er sei schwarz von Farbe gewesen, und, wie man versichere, vom Himmel gefallen. Den Stein ließ Helagabal nach Rom führen, von wo er jedoch nach dessen Tode wieder nach Emesa zurückgebracht wurde. Einen prachtvollen Tempel erbaute er ihm in der Vorstadt, und führte den Sonnendienst ein, der mitten im Sommer gefeiert wurde, er selbst blieb, was er vorher gewesen, der oberste Priester. Die glänzendsten Fest wurden dem Gotte gefeiert, und Herodian gibt in der Beschreibung eines ihm zu Ehren angestellten öffentlichen Umganges, eine merkwürdige Schilderung der dabei üblichen Gebräuchen. Den Gott selbst, heißt es,\footnote{} ließ er auf einen goldenen mit kostbaren Steinen besetzten Wagen setzen, und ihn darauf aus der Stadt in die Vorstadt fahren; der Wagen war mit einem Zug der schönsten weißen Pferde bespannt, mit dem reichsten Geschirr geschmückt. Der Gott selbst hielt die Zügel, denn kein Mensch durfte den Wagen besteigen; alle nur standen herum, als wenn der Gott selbst führe. Helogabal ging vor dem Wagen, lief zuweilen aber zurück, sah die Gottheit an, zog die Zügel rückwärts, und sah während dem ganzen Weg die Gottheit beständig an. Damit er nicht anstoßen oder fallen möge, ließ er die Straßen mit Goldstaube bestreuen, die Soldaten hielten ihn zu beiden Seiten, und sorgten, dass er im Fahren sicher sein möchte; das Volk lief zu beiden Seiten des Wagens mit brennenden Fackeln, streute Blumen und Kränze. Die Statuen der übrigen Götter, nebst den kostbarsten, die in Tempeln verwahrt wurden, die kaiserlichen Insignien und prächtigsten Hausgeräte, wurden vorgetragen; seinen Gott noch mehr zu ehren, ließ er die Statue der Urania oder der phönizischen Astarte, die eigentlich den Mond vorstellte (Dido soll sie in Karthago haben errichten lassen, und vielleicht ist sie eben der pessinuntische Stein, der im zweiten punischen Kriege nach Rom kam?), in den Sonnen-Tempel zur Statue des Gottes stellen, um sich mit ihm als der Sonne zu vermählen, wozu er beträchtliche Schätze und Kostbarkeiten als Heiratsgut gab. Auch müssten Rom und ganz Italien das Hochzeitsfest feiern. Mehrere Münzen von Elagabal, Caracalla, Alexander Severus, welche die emeser Gottheit unter dem Symbol eines Steines gewöhnlich mit einem oder mehreren darüber schwebenden. Sternen) wahrscheinlich den Ursprung des Steins anzudeuten, zuweilen auch mit einem halben Monde (auf die Astarte deutend) vorstellen, finden sich in Vaillant, Eckels Münzsammlungen, und in Spanheims Noten zu Julians Cäsarn S. 87. -- und 47. der Proben. Aus letzterem sind die dieser Abhandlung beigefügten Münzen Nr. 2. 3. 4. genommen, davon die vordere von Caracalla, die andere von Alexander Severus ist. Nr. 4. ist eine goldene Münze, worauf drei Sterne als Symbole des Steingottes erscheinen. --- Andere nicht seltene Münzen Helagabals selbsten haben die Umschrift: Sancto Deo Soli Helagabalo.

*) Sacerdos Dei Solis elagabali.

*) Les Caesars de l'empereur Julien traduits du grec par Spanheim avec des remarques et preuves, p. 46. de preuves.

*) Mehrere derselben in Rom und Neapel befindlich führt Selden an de Diis Syris p. 220 et seq.

*) Herodian Buch V. Kap. 6. verglichen mit Lampridii vita Antonii. Heliogabali in script. rei aug. mit Casaubons, Salmasius und Gruters Noten.

Gleiche Bewandtnis hat es mit Dusares von dem Maxim. Tyr. Diss. 8. Cap. 8. sagt: rabes, quem colunt non novi, at Simulaerum vidi, lapis erat quadrangulis ). Desgleichen die Stelle im Porphyr: die Dumatenier (ein arabischer Stamm) pflegen jährlich einen Knaben, den sie vorher geopfert haben, zu begraben, und zwar an einem Steinaltar, der ihnen zur Abbildung der Gottheit dient.\footnote{} Ob der in der Kaaba noch jetzt befindliche Stein (wie Sage und frommer Aberglaube wähnt), jener Betel-Stein, Dusares, oder Alagabal sei, lässt sich bezweifeln; wahrscheinlicher ists, dass da Mekka und Medina, jetzt die Hauptsitze des Islamisms sind, so wie sie es in früheren Zeiten vom Sterndienste waren (denn in Mekka stand ein Tempel\footnote{} des Mondos), die arabischen Coraischiten, die im Besitze dieser beiden Oerter wären, und dies Gestirn unter dem Bild des Steins verehrten, eine in der Gegend aufgefundene Steinmasse dort aufrichteten, und als eine vom. Himmel herab gefallene Masse für ein schickliches Bild des Gestirns ansahen.

*) S. auch Elem. Alexandr. C. 4. item Arnob. contra Gentes. L. 6.

*) Porph. de Abstinentia C. 2.

*) S. Mohsen Fan Dabistan deutsche Übersetzung. Aschaffenburg 1809.

So fabelhaft die arabischen Märchen über diesen Stein immer sein mögen, sind sie doch darum nicht zu übergehen, weil sie einiges Licht auf die meteorische Natur desselben werfen.

Das Merkwürdigste an diesem Haus (sogt Niebuhr, Beschreibung Arabiens, S. 312. u. f. Koppenhagener Ausgabe) ist der schwarze Stein Hhadjar-el-assouad genannt, der in der Wand auf der südöstl. Seite, nur wenig von der Erde erhoben, sich eingemauert findet. Die Araber behaupten, der Engel Gabriel habe ihn vom Himmel zur Erbauung der Ka'abah gebracht; eine Mythe, die den himmlischen Ursprung des Steins ganz nach der orientalischen Theurgie, vermöge welcher die Naturkräfte unter dem Bilde geistiger Mittelwesen dargestellt wurden, bezeichnet. Der Sage nach soll er anfänglich weiß und schimmernd gewesen sein (vielleicht weil er als ein glühender Stein herabfiel), nachher aber wäre er der Tränen willen, die er für die Sünden der Menschen vergoss, ganz schwarz geworden, und habe seinen ersten Glanz verloren. Nichts in der Welt ward mehr verehrt, als dieser Stein, der in Silber eingefasst ist, und von jedem Muselmann berührt werden muss, so oft er die Kaaba umgeht. -- Nächst den vier kleineren Häusern, den vier verschiedenen Sekten der Sunniten gehörig, und dem Magam-Hhasaret-Ibrahim, dem Orte, wo Abraham gebetet haben soll, während man die Kaaba erbauet, ist noch ein anderer Stein daselbst, der vielleicht eher der älteste arabische Stein-Gott, Bethilos, Jacobs-Stein, oder Dusares sein mögt, indem sie ihn so wenig als einen andern (Ismael-Stein genannt) in Ehren achten, aus der Ursache vielleicht, weil der Prophet gegen die abgöttischen Bilder der alten Sternanbether (deren der Koran unter dem Namen Alassavid erwähnt) so streng eifert.*

*) Bernh. von Breitenbach, der im 15ten Jahrhundert Jerusalem und das Morgenland bereiset, gibt im Itiner. Hierosolymit. Mogunt. 1486 folgende Nachricht von der Verehrung des weißen und schwarzen Steines in der Kaaba: "`Duo filii Loth, Ammon scilicet et Moab, hanc domum honorabant, ibique duo colebant idola, unum ex albo factum lapide, quod Mercurium; alterum ex nigro, quod Camos appellabant. Et istud quidem ex nigro lapide, in honorem Saturni, alterum ex albo in Martis honorem, venerabantur. Et bis in anno ad haec idola adoranda eorum ascendebant cultores. Ad Martem quidem quando sol primum intrat arietis gradum, quoniam aries honor est Martis, im cujus discessione, ut mos erat, lapides jaciebantur. Ad Saturmum vero, quando sol primum gradum librae ingrediebatur, quia libra honor erat Saturni, sicque nudi ac tonsis capitibus thurificabant. Arabes quoque cum Ammonitis et Moabitis haec idola adorabant: longissima post tempore veniens Mahumet, pristinam gentis consuetudinem nolens tollere, quasi mutato quodammodo more, inconsutis opertos tegumentis domum circumire permisit. Sed ne videretur idolis sacrificare praecipere Saturni simulacrum in pariete in angulo domus constituit, cujus ne appareret facies, dorsum tantum extra posuit. Idolum vero Martis, quod undique erat sculptum, subtus terram misit, lapidesque supposuit. Hominibus autem, qui ibi ad adorandum conveniunt, lapides istos osculari praecepit, et humiliatis tonsisque capitibus intra crura lapides retro jactare, qui et dorsa denudabant, quod est signum pristinae legis, et ad effugandum daemones, se hoc modo lapides jacere dicunt, quos clam in eo ritu potius venerantur. Et haec est illa praeclara Muhammeti industria, imo malitia, ut cum a ceterorum cultu idolormm suos inhihuerit, istud tamen in honorem Veueris apud Meccham suam fieri permisit."'

Früher haben wir aus Sanchuniathon gelernt, dass man diese einzeln größeren Steinmassen von den kleineren Bätylien unterscheiden müsse, denen die Alten vermöge eines in ihnen wohnenden Gettes oder Daimons magische Kräfte zuschrieben, und desfalls sich ihrer als Talismane, Amulette zu ihren Beschwörungen bedienten, ein Gebrauch, der von den ältesten Zeiten her sich bis spät in die christliche Epoche\footnote{} erhielt, ja durch die Gnostiker, Valentinianer und Schüler des Basilides noch lebhafter erneuert warb.

*) S. hierüber Fouchers Abhandlung von der Religion der Perser mit Kleukers Anmerkungen in dem deutschen 8end-vi-Vesta III. Th. S. 104. 170. 190.

Ein verlässiger Schriftsteller Photius\footnote{} liefert im Auszug von Damascius Leben des Isidors die Beschreibung einer dem Arzte Asklepiades widerfahrenen Erscheinung, welche auffallend mit den Umständen übereintrist, die nach den neuesten Erfahrungen das Herabfallen der Aerolithen begleiten.

"Nächst Heliopolis in Syrien (erzählt er) habe er, als einsmals er den Libanon erstiegen, daselbst eine Menge herabgefallener Bätylien gesehen, von denen der Aberglaube viele Wunder erzählt, und davon ihm selbst und dem Isidorus folgendes wiederfahren: ich sähe nämlich (fährt er fort) einen in der Lust schwebenden Bätylos, bald mit einem Gewand bedeckt; bald in den Händen dessen, der ihm diente, er hieß Eusebius und erzählte zugleich folgendes: zur Nachtszeit sei er einst von Emesa nicht allzufern gegen jene Berghohe gewandelt, worauf der a prächtige Pallas-Tempel erbauet ist; als er Zeit lang dort am Fuß des Berges vom Wage seitwärts gesessen, habe er eine feurige Kugel vom Himmel herabfallen, und einen mächtigen Löwen dabei stehen sehen; der Löwe sei sogleich verschwunden, und er, als das Feuer schnell verloschen war, zur Kugel gelaufen, die er für einen Bätylos erkannt; er habe den Stein sogleich aufgehoben, und befragt, welchem Gott er zugehöre? worauf er ihm geantwortet: dem Gennäo, einer Gottheit, so die Einwohner von Heliopolis im Tempel des Jupiters unter der Gestalt eines Löwen verehren. Eusebius(fährt die Erzählung fort) war aber nicht Herr dar Bewegung des Steines, er rief ihn nur durch Gebete an, und erhielt dann vom Steine durch einen dem Zischen ähnlichen Laut Antwort auf seine Fragen). Der Stein war, wie die übrigen dieser Gattung, rund, von mäßiger Größe, an Farbe schwarzbläulich, hie und da mit Linien und Figuren, die der Aberglaube für Zauberzeichen und Zählen ansah. Damascius nennt sie Buchstaben, γραμματα εκ τω λιθω γεγραμμενα bezeichnet. Nach Isidors Meinung werden die Steine durch einen in ihnen wohnendest Dämon beherrscht, nicht zwar von einem bösen, der Materie anhängenden, noch auch einem ganz reinen, sondern von einem jener Mittelwesen, von denen der ganze Äther erfüllt ist. Mehreren Gottheiten, dem Saturn, Jupiter, Mars, der Sonne, und anderen sind diese Steine geweiht.

*) Photii Biblioth. p. w. ac calcem; Zoega de Orig. Obelisc. p. 20.

*) Ähnliche Vorschrift gibt das Orphische Gedicht Aeza, Edit. Gesneri, p. 324. "`et tu quandoquidem vocem Deorum vis audire, sic facias, ut miraculum animo tuo, intelligas: quado enim valde laboraveris illum manibus jactare et commovere, subito edit vocem recens nati infantis, nutricis in sinu plorando lac efflagitantis. Oportet vero te constanti animo curare eum semper, ne forte infirmo timore solutus, e manibus in terram abjiciens iram difficilem excites immortalium. Tum aude de vaticiniis interrogare umnia enim tibi vera. Eumque postea proprius ad oculos admovens, quando laveris, intuere divinitus enim exspirantem intelliges.

Abgerechnet alles Wunderbare, was in der Erzählung liegt, und gnostischen Schwärmern, wie Eusebius und dessen Biograph waren, ganz eigen ist, liefert die Beschreibung dieser Erscheinung uns ein anschauliches Beispiel eines Meteors, und die herab gefalleng glühende Kugel ist nichts anders, als ein Aerolithe.

Bätylien, sagt zudem Plinius nach Sotakos, sind Steine, die auf dem Libanon bei Heliopolis gefunden werden. An einer anderen Stelle\footnote{} erwähnt dieser Naturkundige, nachdem er die verschiedenen Asteriden oder Sternsteine durchgegangen ist, einer Gattung derselben, die er Ceraunia nennt; Sothacus (sagt er) nimmt zwei Gattungen Ceraunien an; nämlich einen schwarzen und einen rötlichen, und sagt, dass beide einer Art ähnlich seien. Mit denen, welche schwarz und zugleich rund sind, könne man Städte einnehmen, und Flotten erobern, und sie hießen Betuli; --- Die länglichten, kristallischen und himmelblauen an Farbe, die besonders in Karamanien wüchsen, heißen Cerauniae. Sie nehmen auch noch einen sehr seltenen Stein dieses Namens an, der von den Parthischen Magiern sehr gesucht wird, weil er an Stellen gefunden wird, die vom Blitz getroffen sind. Wirklich finden sich\footnote{} solche Steine auf Inseln des roten Meeres, und, wie Reisende berichten, in den südlichen Gegenden Persiens, aber nicht dort allein, auf der ganzen Erde find sie zerstreut, und der Glaube, dass sie Erzeugnisse des Blitzes feien, findet sich bei allen Völkern. Sie bedienten sich ihrer zu gottesdienstlichen Gebräuchen, oder als Streitäxte. Häufig findet man sie (im Norden am meisten) teils frei in der Erde liegend, teils in alten Grabmählern verschlossen. Natürlich war es, dass Steinen, die man durch Himmelsfeuer gebildet glaubte, höhere Kraft zugeschrieben wurde. Dem Naturmenschen aller Gegenden wird der Gebrauch des Feuers erst durch Reibung zweier Steine oder Hölzer aneinander bekannt; die Erkenntnis aber, dass dem Stein diese Feuerkraft innewohne, setzt, wenn nicht zufällig der Wilde zu dieser Entdeckung gelangte, früheres Anschauen, äußere Veranlassung zum voraus. Zu der Ahnung, dass Feuer aus dem Steine zu ziehen\footnote{} sei, können ihn hauptsächlich drei Ursachen bringen:

*) Hist. nt. Lib. 37. C. 51.

*) Plin. Nat. Gesch. 37. 52.

1. Der Blitz, der im Augenblick des Herabfallens Bäume spaltet und entzündet,\footnote{}

2. die durch unterirdische, oder andere Ursachen auf der Oberfläche der Erde entstehenden Brände, und die furchtbaren Vulkane, deren Feuerschlünde nicht bloß glühende Steine in großen Massen, sondern selbst Feuerströme auswerfen, die erkältet sich wieder zu felsharten Massen bilden; endlich

3. die überall und in den frühesten Zeiten entfallenen Himmels-Steine, deren Ankunft meist ein feuriges Meteor begleitet. Feuer ist das furchtbarste und wirksamste Element, das in den ältesten Dichtungen, die uns Nachricht von der Welt-Schöpfung geben, zuerst aus dem Schoos der Nacht erschien. In Hesiods Theogonie, dieser ehrwürdigen Urkunde, die gewiss aus Traditionen des Orients entstand, wird, nachdem Äther und Hemera aus dem Schoos des Erebos entstiegen waren,\footnote{} von Uranos und Gea (Himmel und Erde) Hyperion geboren, der mit Thia, seiner Schwester, den Helios, (Sonne) Selene (Mond) und Eos die Tags- und Morgengöttin erzeugt, von welcher (Asträos Gattin) nebst Boreas, Zephyr und Nothos, Phosphorus oder Hesper und die Gestirne geboren wurden. Hyperion, eine feurigleuchtende Masse, brachte der kalten Natur die erste Wärme, er umfasste Sonne und Mond, die später erst aus einer Masse geschieden wurden.\footnote{} Der Tiefe entstieg er gleich der Sonne, von der Erde zum Himmel, und erwärmte von dort die Natur, wie er die größeren Feuermassen, Sonne und Mond, erzeugte, so Kojos sein Bruder die größeren Sterne; aber Asteria und Letho (von ihnen erzeugt) waren Göttinnen der kleineren Sterne,\footnote{} die aus jener Feuermasse sich zum ersten male schieden, und in Gesellschaft der Letho die erste sternhelle Nacht brachten, denn ohne Dunkel der Nacht erscheinen (nach dem alten Glauben) keine Sterne, Letho das Dunkel begleitete daher gleich von Geburt an ihre Schwester Asteria, weshalb auch die Orphiker ihnen die Hekate beigesellten, und (der Magie wegen, weil Sterne und Zauberei im engsten Verhältnisse stehen) zur Tochter derselben machten. Wie nun Theogonie uns die Entstehung aus dem Chaos und die ersten Uranfänge zeigt, so finden wir in ihr auch das Bild des Elementen-Kampfs zur Vollendung der Wesen, und wie die drei aufeinander folgenden Himmels-Reiche des Uranos, Chronos und Zeus, in physischer Hinsicht nichts anders, als die früheren Revolutionen im Weltall darstellen, so enthalten die Riesen und Titanen Kämpfe, die bei Veränderung eines jeden dieser Himmels-Regierungen vorfielen (gleich jenen indischen Sagen bei Erscheinung eines neuen Jug oder Zeitalters), ein sinnlich anschauliches Bild der mittelst gärender Elemente durch die Zeugungskraft des jüngeren Gottes gebildeten Wesen, sowohl Lebenden, als Pflanzen, Erden und Metallen. Die Stelle, worin Hesiod die Tiefe des Tartaros schildert, und das Bild vom Falle des ehrnen Ambosses, dessen er sich dabei bedient, bezeigt nicht allein das hohe Alter der Metallurgie, auch die Tiefe, worin die Metalle im Schoos der Erde verborgen liegen, wird darin bezeichnet; denn alles, Fußboden, Pforten, Mauern, ist hier von Erz.

*) Ignis ubique latet, naturam complectitur omnem.

*) Der Blitz war nach Proclus (in Timaeum p. 34.) ein Symbol der Demiurgischen Kraft, welche die Welt schafft und belebt.

*) Theogonie nach Voß Übersetzung Vers 123 u. f.

*) Die spätere Erscheinung des Mondes bezeugen nicht allein Plutarch, Lucian und andere Schriftsteller, auch die Überlieferung mehrerer Völker, der Arkadier besonders, die sich älter als der Mond, oder vorhanden, ehe der Mond die Erde beschien, angeben, stimmen damit überein, und im Alexis macht Hemsterhuys (s. dessen Werke 3. Teil) es mehr als wahrscheinlich, dass der von allen Sternkundigen angezeigte Komet (der erste seit der Weltschöpfung, der sich der Erde näherte, im Zeichen der Fische ums Jahr 2312 vor der gewöhnlichen Zeitrechnung) eben derselbe gewesen, der durch seine Annäherung an die Sonne in einen bimsteinartigen, anagebrannten, verglasten Körper verwandelt, und (in seiner ferneren Laufbahn gehemmt) nun bestimmt worden sei, bei der Erde zu verbleiben, wodurch unbezweifelt alle Atmosphär- und Gestalteränderungen unserer Erde geschahen, durch die ihr erster besserer Zustand im Physischen sowohl als Geistigen (das sogenannte goldene Weltalter) verschwand; darum auch hat das Andenken dieses ersten furchtbarsten aller Meteore sich so lange im Andenken aller Völker erhalten.

*) S. Kanne Mythologie der Griechen, S. 32.

"- --- Gleich fern von der Erde ist des Tartaros finsterer Abgrund. Wenn neun Tag' und Nächte dereinst ein eherner Amboss Fiele vom Himmel herab, am zehenten käm' er zur Erde; Wenn neun Tag' und Nächte sodann ein eherner Amboss Fiele hinab vom der Erd', am zehenten käm' er zum Abgrund. Ehrnes Geheg' umläuft den Tartaros; aber umher ruht Dreifach gelagerte Nacht an dem Eingang; oben herab dann Wachsen die Wurzeln der Erd' und des ungebändigten Meeres. Allda sind die Titanen im nachtenden Schlunde des Dünkels Eingehemmt, nach dem Nathe des schwarz umwölkten Kronion, Tief in der dumpfigen Kluft, am Rand der unendlichen Erde. Keiner vermag zu entfliehn; denn es schloss Poseidon den Ausgang Fest mit eherner Pfort', und rings umschränkt sie die Mauer."'

Nachdem Hesiod uns in den Abgrund geführt hat, schildert er mit der alten Dichtungen eigenen Kraft die Beschaffenheit der tiefen Erdschkünde; und im Bild des furchtbaren Typhäeos, des vielköpfigen Drachen, Sohn der Gata, steht eine Szene vor uns schauderhast erhaben, vom inneren Lehen, Tosen, Sausen, und Gären entgegen gesetzter Elemente --- Feuer, Luft und Wasser in den verschlossenen Bergschluchten:

"So aus den Häuptern gesamt, wenn er schauete, brannt' es wie Feuer. Auch war hallende Stimm' in allen entsetzlichen Häuptern, Von vielartigem Wundergetön: denn in häufigem Wechsel Lautete jetzt für die Götter verständliches; jetzo hinwieder Scholl es, wie dumpfes Gebrüll des in Wut anrasenden Stieres; Jetzo gleich, wie des. Löwen von unaufhaltsamer Kühnheit, Jetzo gleich dem Gebelfer der Hundeleine tönet es seltsam, Jetzo wie gellendes Pfeifen, dass rings nachhallten die Berghohen."'

Aber des Berges Höhlen entsteigen Dünste in die höhere Atmosphäre, Wolken bilden sich, und aus ihnen die zerstörenden Wetter.

"Ernst nun schwang er die Donner, und donnerte; rings in dem Aufruhe Toste das Land grauenvoll, und der wölbende Himmel von oben, Auch des Okeanos Strom, Meerflut und tartarischer Abgrund. Ja dem unsterblichen Fuß erbebten die Höhn des Olympos, Als sich der Herrscher erhub; und tiefauf dröhnte das Erdreich. Beiden entloderte Brand, um das finstere Meer sich verbreitend, Hier von dem Donner und Blitz, und dort von der Flamme des Scheusals, Von glutwirbelndem Sturm und zuckendem Strale der Wetter. Auf nun brauste die Erd', und der Himmel umher, und die Meerflut; Und die Gestad' umtobt' unermessliches Wogengetümmel, Durch der Unsterblichen Schwung; und es schwankte das All in Erschütterung."'

Und wie das Feuer aus der wetterschwangeren Wolke sich zur Erde senkt, begleitet von Regan und Hagel, also entströmen aus den tiefsten Schlünden verheerende Feuergluthen; --- das Meer erbraußt, und durch die Kraft der Wasser, Luft- und Feuer-Meteore treten neue Gestalten mannigfacher Gestirne, Erde, Metalle, aus dieser großen Werkstatt hervor.

"Lodernde Glut entströmte dem niedergedonnerten Herrscher, In des Gebirgs Waldthalen, von Felsabhängen umdunkelt, Wo er erlag; weit brannte die mächtige Erd' in des Wetters Stürmischer Loh', und zerfloss, dem schmelzenden Zinne vergleichbar, Welches der Jünglinge Kunst im wohlgehöhleten Tiegel, Glühet; oder wie Eisen, das stark vor allem Metall ist, In des Gebirgs Waldthalen von flammender Hitze gebändigt, Schmilzt in dem heiligen Grund, durch künstliche Hand des Hefästos: Also zerschmolz auch die Erd' in stralender Lohe des Feners."'

So reichen Physik und Fabel sich die Hand; und wie letztere im Mythos das Bild der Entstehung der Wesen durch Gärung und Verbindung der Elemente zeigt, so sagt der älteste uns bekannte Mineraloge schon: "`die Anhäufung (Bildung) der Steine wird teils durch Hitze, teils durch Kälte bewirkt, auch können einige durch beide entstehen, überhaupt scheinen die Erdarten durch Feuer zu erhärten. Übrigens werden alle Mineralien durch jene entgegengesetzten Kräfte bald gebildet, bald aufgelößt."\footnote{}

Dass diesem großen Naturkenner das Gesetz der Kristallisation nicht unbekannt geblieben, zeigt folgende Stelle:\footnote{}

*) Theophrast von den Steinen, §. 3. Sein deutscher Übersetzer, D. Schmieder, bemerkt dabei der Gedanke ist gut chemisch, nur wird er durch Mangel der Kunstwörter entstellt; genau genommen gibt es wirklich keine andere Umbildungskräfte fürs Mineralwich, als Hitz. und Kälke. Alle nasse Auflösungen, Schmelzungen, Verflüchtigungen, Verwitterungen haben doch nur Ortsänderungen des warmen Stoffs zum Grunde.

*) Schade, dass sein größeres Werk von den Metallen verloren gegangen.

"Man muss sich im Allgemeinen vorstellen, dass die Steinarten aus reinen und gleichförmigen Flüssigkeiten entweder für sich, oder beim Durchseihen einer fremden Masse entstanden sind. Davon sind denn die Glätte, Dichtigkeit, der Glanz, die Durchsichtigkeit, und andere Eigenschaften der Steine herzuleiten. Je gleichförmiger und reiner der Urstoff war, je regelmäßiger die Bildung erfolgte, desto mehr sind auch den Produkten jene Merkmale eigen."'

Dies Entstehen zusammengesetzter Wesen aus einfachen Elementen nun ward nebst anderen Gegenständen, nach dem Zeugnis alter Schriftsteller, in den Mysterien gelehrt, und symbolisch dargestellt. Nach der indischen Lehre (um nur die vorzüglichsten anzuführen) ward Porsch als das Bild der ganzen Welt, und Archetyp der Schöpfung angesehen. Genau nach der Lehre der Vedas beschreibt der Brahman Sandales, Zeitgenosse des Bardesanes, dessen Abbildung. In einer Höhle des höchsten Berges auf der Erdmitte sei eine Bildsäule aufgerichtet, zehn Cubitus hoch, die Arme gekreuzt, die rechte Hälfte des Gesichts, Arm, Fuß und die ganze Seite männlich, die linke weiblich, beide kunstreich und geschickt verbunden. Auf der rechten Brust gebildet sei die Sonne, auf der linken der Mond, auf beiden Armen aber seien durch die Kunst des Bildners viele Engel und alles Übrige, was die Welt befasst, Himmel, Berge, Meer, der Strom des Ozeans, Pflanzen und Tiere, und was sonst in der Natur existiert, vorgestellt. Dies Bild habe Gott seinem Sohne gegeben, als er die Welt gründete, um an ihm ein sichtbar Vorbild seines Werks zu haben. Es sei nicht von Silber noch Gold, Erz oder Stein, am ähnlichsten noch einem harten Holze, und doch nicht Holz. Auf dem Haupte der Gestalt sitze das Bild Gottes wie auf einem Throne. Hinter dem Bilde sei tiefes Dunkel durch die Höhle, man gehe mit Fackeln hinein, und dem Reinen offene sie sich weit, Unreine aber könnten nicht hindurch. Zur Zeit des Sommers und im Herbste versammelten sich die Brachmanen in ihrer Nähe, um das Bild zu sehen, und sich selbst zu prüfen; sie unterredeten sich dann miteinander über die Gestalten auf dem Bilde, die nicht leicht verständlich seien, teils ihrer Menge wegen, teils weil nicht jedes Land alle Pflanzen und Tiere trägt. -- So wurden auch in den Mithras Mysterien den Eingeweihten die Urgeschichte der Schöpfung gelehrt: die dieser Gottheit (der Sonne) geweihte Höhle auf dem höchsten Berge Persiens durch unterirdische Blumengärten, Quellen und Flüsse verschönert, stellte das Universum dar, und die in gleicher Entfernung aufgestellten Bildsäulen waren Symbole der Elemente und Himmelszeichen.\footnote{} Sterne, Konstellationen, Tiere, Pflanzen, Metalle, wurden nebst dem Hauptbilde des Mithras, auf dem Stier reitend, mit der Aufschrift: Gott dem unüberwindlichen Mithras, darin vorgestellt, die Priester sowohl, als Eingeweihten, erhielten jedes den Rahmen eines Tieres, und die Neophyten wunden nicht anders, als nach den furchtbarsten Prüfungen aufgenommen. Gleiche Einweihungen begleiteten in Lemnos den Dienst des Vulkan, des Zeus auf Kreta, der göttlichen Mutter in Syrien, der Cabiren auf Samothraze, ja die Orgien, und die Mysterien des Adonis waren nichts, als geheimer Naturdienst. In ihnen wurde vorzüglich gelehrt und durch Anschauungen versinnlicht die allmählige Entstehung aller Wesen, vorzüglich der Steine und Metalle, denen man (ihrer verwandten Natur wegen mit den Gestirnen) geheime und höhere Kräfte zuschrieb, wie auch gegenseitig jedes Metall ein ihm analoges Gestirn hat. Bedürfnis und Luxus machten sie frühzeitig dem Menschen notwendig, und ihre Entdeckung hatte die Natur ihm, ohne dass er selbst desfalls tiefe Gruben zu befahren bedurft hätte, noch zur Hand gelegt, indem in der ersten Zeit, wie noch jetzt, durch Wasserfälle von hohen Bergen herab, durch Strömungen der Flüsse und Regen häufige Metalle in kleinen und größeren Massen an Tag gefördert werden, auch Vulkane, und die in der Urgeschichte so häufigen Erdbrände lehrten die Menschen bald das Schmelzen der Erze. Gold, Silber und Kupfer\footnote{} wurde am frühesten zu häuslichem Gebrauch sowohl als zum Cultus verarbeitet. Zeugnisse hierüber sind aus den heiligen Büchern und profan Geschichtsschreibern häufig; seltsam aber ist, dass, Gerätschaften aus Gold und Silber gearbeitet, die zum bloßen Luxus dienten, sich eben so frühzeitig vorfinden, als jene, die bloße Notwendigkeit erfand; selbst goldene Massen, wie die Bewohner Perus und Mexicos sie hatten ), da die Spanier in diese Länder eindrangen, bedienten sich schon die Ägypter, Lydier, die Bewohner Beticas (des heutigen Protugalls, als die Kartaginenser zum erstenmal dahin kamen) und andere alte Völker.\footnote{} Ist dem Golde, Silber und Kupfer ihr Vorzug und früher Gebrauch nicht abzustreiten (dessen Ursache vorzüglich in ihrer leichteren Behandlung und Schmelzbarmachung, wie gegenseitig in der Härte und Sprödigkeit des Eisens zu suchen ist), so erscheint gleichwohl sein Gebrauch in den frühesten Zeiten.*

*) Stobaeus L. I. p. 141.

*) Porphyr. de Antr. Nymph. p. 234.

*) Tubal, der nach Mose I. C. 4. 22. alles zu hämmern versuchte, und Eisen und Kupferschmied ward, zeigt uns die Kunst, Erz zu verarbeiten, schon vor der allgemeinen Flut.

*) Dass die Peruaner in frühester Zeit schon das Kupfer schmolzen, und zur Bearbeitung ihrer prachtvollen Denkmale aus ungeheuren Porphyr-Massen nicht bloß steinerne Arten, sondern auch Werkzeuge gebrauchten, die aus Kupfer und Zinn vermischt waren (was die Alten Aes χαλκος nannten), zeigt Alex. von Humbold im 2ten Cahier das Vues des Cordellieres p. 117. aus mehreren Tatsachen, vorzüglich durch eine in einem alten Silber-Bergwerke aus der Zeit der Incas vorgefundenem Messer oder Bohrer, welches er nach Europa zurückbrachte, und das nach Vauquelins Analyse 0.94 Th. Kupfer, und 0,06 Th. Zinn enthielt, es war 12 Zentimer lang und 2 breit; an Farbe und Gehalt den Axten und Opfermessern der alten Gallier ähnlich. In einem trefflichen Mémoire über die Bronze der Alten, von Mongez (Tome V. des mémoires de l'institut national des sciences et arts, auch Göttinger gelehrte Anzeigen von 1805. S. 159) zeigt dieser Gelehrte durch chemische Analyse verschiedener alter Schwerdte, Pfeile u. a. Waffen, dass die gewöhnliche Proportion der alten Bronze für Gewehr ein Zehntel Zinn sei, ungefähr die Mischung für unsere Kanonen.

*) Beweise hievon sind mit vielem Fleiße gesammelt in Goguet Orig. des Loix arts et sciences. Liv. II. Ch. 4. Die Fortschritte und Verbesserungen der Schmelzkunst gehören übrigens in die Geschichte der Metallurgie.

Das älteste Buch, Hiob (28.) erwähnt dessen gleichzeitig mit dem Golde. "`- Silber hat der Mensch gefunden, Und den Ort des Goldes, das der Künstler gießt, Eisen nimmt er aus der Erde, Und Steine schmelzet er zu Kupfer, Er macht der Finsternis ein Ende, Alle verwahrten Schätze forscht er auf, Den Stein der acht und der Schatten, Am Fuß des Berges bricht ein Strom aus, Von ihrem Waldbach vergessen, Versiegen die armen Ströme wieder, fern von dem Menschen herumirrend. Ein Erdreich, aus dem oben Speise wächst, Wird unten als vom Feuer umgewühlt; Seine Steine sind der Ort des Lazurs, Der mit goldnem Staube bezeichnet ist, Diesen Fußsteig kannte kein Raubvogel, Das Auge des Falken hat ihn nicht erblickt."'

*) Schon in der Mythe des ägyptischen Welt-Eyes zeigt sich Gold und Silber als die edelsten und ergiebigsten Metalle, denn der Fabel gemäß teilte sich das Ei, als nach Verlauf eines Jahres es zerspaltete in zwei Hälften, die eine war Gold, die andere Silber, und die silberne war die Erde, die goldene die Sonne.

Des Eisens Härte rühmen die mosaischen Bücher auf eine Weise, die dessen frühzeitigen Gebrauch zeigen.\footnote{} Sie reden von eisenhaltigen Berggruben, vom eisernen Bette Ogs, Königs von Basan, von Schwerdtern, Degen, Messern, Degengriffen aus Eisen verfertigt, welches notwendig die Kunst voraussetzt, Eisen zu härten und in Stahl zu verwandeln.*

Eisen ist das allgemein verbreiteste, obschon meist tief vergrabene Metall; auf der ganzen Erde ist es befindlich, aber wo es am vorzüglichsten erscheint, ist Rorden, denn am Kaukasus und in Sibirien sind noch jetzt die ergiebigsten Eisengruben. Dahin auch versetzt der Mythus den Prometheus, den Zeus (weil er den Göttern das Feuer stahl, und den Menschen dessen Gebrauch samt den Künsten, die durch Feuer sich ausbilden, lehrte) auf den Kaukasus verbannte, Hephästos (der göttliche Schmied) musste mit eisernen Banden ihn dort fesseln, und die daher entstandene Sage, dass Prometheus den ersten eisernen Ring getragen, ist ein Dichter-Märchen auf die frühe Bearbeitung des Eisens, und den Gebrauch, Ringe zu tragen.

*) S. ferner Hiob C. 19. 20. 40. 41.

*) Deuter. Cap. 4. V. 20. Cap. 8. V. 9. Auch Wummer 35. 16. Levit. C. 1. 17.

So können Typhon und Horus nach einer andern Mythe gleichfalls als Symbole des Eisens und dessen Ausbeute gelten. Denn Typhäos, der Riese, der von Zeus tief in den Abgrund geworfen ward, bewohnt die eisenhaltigen Höhlen des Tartaros, weshalb auch Eisen für seinen Knochenbau gilt, wie gegenseitig Magnet, weil er das Eisen an sich zieht, die Knochen des Horus genannt wird.*

*) Plin. Nat. Gesch. 33. 3. und 37. 1. Typhon, der später erst in den griechischen Mythos aufgenommen ward, ist syrischen oder ägyptischen Ursprungs. Dort bedeutete er (nach Plutarch über Isis und Osiris --und im Buch von der Mondsscheibe) das dürre trockene auch im Gegensatz des Osiris (der feuchten befruchteten Mondwelt) die Sonne als versengende verödende Kraft; nach einigen Fabeln ward er von Zeus unter den Ätna geschleudert, und dieser Berg auf ihn gewälzet; aber nach Homer liegt er unter den arimischen Gebirgen vergraben. Diese Arimi aber sind (Heyne Excurs I. zum Virgil IX. 3ter Th. S. 313.) in Asien, am ersten in Zilizien zu suchen, welches nach Strabo, nebst Phrygien, Mysien und Lydien von unterirdischen Feuern gebrannt hat; aber Typhon, nach der ägyptischen Mythe, bedeutete auch das Meer, in dem der Nil verschwindet, oder im Allgemeinen das feuchte schlammige Prinzip, aus dem das Weltey erzeugt ward; auch aus Wasser, wie aus Feuer, kamen Wesen hervor (wie neuere Forschungen am Basalt erwiesen haben). Denn nicht Feuer allein, auch Wasser ist Ursache und wirkende Kraft vieler der bedeutendsten Meteore. Am ersten scheinen die Phönizier und Phrygier mit Bearbeitung des. Eisens sich beschäftigt zu haben, und als kunstfertige Bergleute zeigt die früheste Geschichte uns die Daktylen am Ida, aber vorzüglich die Kyklopen berühmt durch Hammer- und Mauer-Arbeiten, jener fleißige, regsame Volksstamm des alten Thrinakias (Sizilien), die nicht dort allein, auch in Griechenland und Latium viele noch bis jetzt unzerstörbaren Denkmale ihrer Kunst erbaut haben ).

*) Denn (sagt Plutarch Isis und Osiris) wie das Eisen sich oft von diesem Steine anziehen lässt, und ihm zu folgen scheint, ebenso pflegt die belebende heilsame vernünftige Bewegung der Welt jene typhonische Hartnäckigkeit gleichsam mit schönen Worten an sich zu ziehen und zu erweichen diese aber reißt sich dann wieder los, kehrt in sich selbst zurück, und überlässt sich einer schrankenlosen Freiheit.

Zeus (der Urbedeutung des Worts nach Ζῆν Leben, und Θεος, das Laufende) war als Naturgeist, dem die ihm untergeordneten Elementargeister (starke Riesen), Donnerkeile und Blitze berieten, die wirkende Kraft im Gewitter, das lebende Prinzip, ein Meteor, dem (als herrschenden Gott der höheren Regionen) Wolken, Blitz und Donner zu Gebot stehen. Aber durch Prometheus kam das Feuer zur Erde, wie gegenseitig durch Typhon (den Flammen sprühenden Drachen). in der Erden Schoos, tief verschlossen in das Innere der Metalle; entlockt jedoch wurdes ihnen wieder durch Vulcan (nach den Orphikern das Symbol der Naturkraft, daher er der Starke, Kraftvolle heißt,\footnote{} das unermüdete Feuer, in flammenden Blitzen glänzend, den Sterblichen Licht bringend, der handfeste, ewig arbeitende Künstler, Teil des Weltalls, das lautere Element, der Allvorzehrende, Allbändigende, dessen Glieder, das Licht, die Sonne, der Mond, die Gestirne sind, der Seelige, der alles, selbst die sterblichen Körper bewohnt, des Feuers rastlose Wut dämpft, und Lebens-Wärme mitteilt). Vulcan mit den Kyklopen, Symbole des Fleißes und der Kunstfertigkeit, sind die Urheber der durch Kunst bewirkten Phänomene, wie oben am Himmel Zeus die natürlichen Meteore erregt; aber dem Eisen, das vorzüglich vom göttlichen Schmiede (Vulkan) bearbeitet wird, d. h. von allen Metallen sich auf Erden am meisten findet, gab die Natur einen ihm verwandten Gefährten zur Seite, den Magnet, der am liebsten auch sich findet, wo Eisen liegt, und mit ihm verbunden so manche Wunder erzeugt. In der frühesten Zeit war er bekannt, denn nach Aristoteles (de Anima L. II. I.) erwähnte schon Thales seiner. Im orphischen Gedichte: von den Steinen, wird er vorgestellt als ein schöner Jüngling im Gefolge des Medea (vermutlich der magischen Kraft wagen, die dem Stein zugeschrieben wurde); nach einer andern Sage ward er durch Zufall, von einem Schäfer auf den Berg Sipylos entdeckt, der ihm seinen Namen gab; dasselbe erzählt Plinius, vom lydischen Steine; doch ists eine bloße Verwechselung; denn dieser hat nichts von der magnetischen Natur. Siderites heißt dieser von seiner Analogie mit dem Eisen; sein bedeutendster Nähme, auf seine Kraft und Wirkung deutend, ist nach Platon im Ion und Timäeos, der des herkulischen Steins ηραχλεια λεθοσ\footnote{} Aristoteles (de Anima L. I. C. II.) nennt ihn vorzugsweise den Stein η λιθος und schrieb nach Diog. Laert. eine eigene (aber verloren gegangene) Abhandlung von demselben. --- Ja er behauptet sogar, Thales habe ihn belebt geglaubt; Plinius (Nat. Gesch. 36. 38.), der fünf Arten derselben angibt, nennt als vorzüglich den äthiopischen und arabischen Androdamos, der schwarz ist, und Härte hat, auch im inneren Afrika gefunden wird, und Silber, Erz und Eisen anzieht. Am stärksten rühmt er den Hämatit, der (wie er sagt) den Magnet selbst anzieht. Theamedes hingegen ist der negative Magnet, der Eisen, statt anzuziehen, von sich stößt; auch diese Eigenschaft war den Alten nicht unbekannt, denn nebst Plinius nennt Marcellus Empiricus des Theodosius Arzt im Buch de Medicam. den Magnet Lapidem antiphyson, qui ferrum trahit et abjicit, welche Eigenschaft des ein-entgegen blasens Claudian epigram. 14., wo er vom martialischen Eisen redet, zu meinen scheint:

*) Nach der früheren Theogonie waren noch vor Zeus die ersten Riesen jene vielarmigen starken Riesen Hesiods, die Urheber des ersten Gewitters, und die drei mächtigen Kyklopen Briareos, Gyges und Kottos (wörtlich: der Gewaltige, der Starke, der Schläger) bereiteten dem Jupiter Blitz und Donner.

*) Ζην ist, wie Kanne Mythologie der Griechen zeigt, die älteste Form von Ζευς aus Ζην wurde Ζεις und nachmalen Ζευς.

*) Orph. Hymne LXV.

*) Eigentlich von der Stadt Heraclea benahmt, in deren Nähe er vorzüglich gefunden wird, allein wahrscheinlich hat die Stadt sowohl als der Stein den Nahmen von Heracles erhalten.

"Ille lacessitus longo spiraminis actu --- -"

und Auson in der Beschreibung von Arsinoes Bildfäule.*

*) In Mosella Idyll. 3.

"Spirat enim tecti testutidine totus Achates
Afflatamque trahit ferrato crine puellam."'

Häufig sind in den Alten die Stellen, die von der Anziehungs-Kraft des Magnets reden. Wie zart ist folgende von Claudian Epigr. 4.

"Pronuba fit Natura Deis ferrumgue maritat.
Aura tenax --- - -
Flagrat anhela silex, et amicam saucia sentit,
Materiem, placidosque Chalybs cognoscit
amores."'

und jene des Lucretius L. VI.

"Exultare etiam Samothracia vidi
Et ramenta simul ferri furere intus ahenis
In scaphiis lapis hic magnes cum sub dictus
esset."'

Von dieser Anziehungskraft des Magnet sagt Plinius*: "`was ist träger, als der starre Stein? und doch gibt Natur dem Magnet Gefühl und Hände. Was widersteht stärker als das starre Eisen, aber hier gibt es nach, und nimmt Sitte an; es wird von Magnet-Stein angezogen, und die Materie, welche alle Dinge zähmt und beherrscht, läuft, ich weiß nicht welchem Nichts entgegen, steht still, wenn sie ihm näherkommt, wird gehalten, und hängt in einer Umarmung fest. Auch Platon gibt\footnote{} in der Beschreibung jener Kette von Eisen, deren Ringe fest gehalten werden durch den obersten, der an einem Magnet hängt, ein schönes Bild davon. Philo, Galenus, Nemesius, erwähnen seiner, und Augustin spricht\footnote{} mit Verwunderung von seinen Eigenschaften.

*) Naturgeschichte 36. 25.

*) Im Ion.

Eine gleich wunderbare ist die Wechselwirkung zwischen ihm und dem Eisen, denn wird dieses vom Magnet angezogen, so erhält gegenseitig der Magnet durch Annäherung und Berührung von Eisenteilchen neues Leben, weshalb Plinius ihn ferrum vivum nennt. Claudian schildert diese Wirkung in folgenden Versen:

"Ex ferro meruit vitam, ferrique rigore
Vescitur; has dulces epulas, haec pabula
novit: Hinc proprias renovat vires."' Epigr. 14.

Wie und durch welches Medium in physischer Hinsicht\footnote{} der Magnet auf das Eisen wirkt, hat Plutarch am ersten gezeigt, und zwar den Äther, oder die in Bewegung gesetzte Luft als das Prinzip davon ausgenommen, seine Worte find*: der Magnet gibt gewisse schwere, windartige Ausflüsse von sich, wodurch die nächste Luft angestoßen wird, so dass sie die vor ihr befindliche verdrängen muss; diese geht nun im Kreise herum, und zieht dann auch das Eisen mit Gewalt an sich, denn das Eisen hat viele raue Stellen, Gänge und Öffnungen, die wegen ihrer Ungleichheiten zum Eindringen der Lust sehr schicklich sind; so dass diese statt abzugleiten, sich leicht festsetzen, und lange genug darin verweilen können, um das Eisen in Bewegung zu setzen, und mit Gewalt nach dem Steine hinzustoßen. Die Meinung der Alten, dass dem Magnet ein belebender Geist innewohne, hat ihn zum Stein der Liebe und Sympathie\footnote{} gemacht, und diese Eigenschaft eignen ihm alle Lithographen vom Verfasser der orphischen Steinschrift bis zu Albertus magnus und Paracelsus zu. Zugleich hat Magie sich seiner bemächtigt, und nicht wenige ihrer Zaubererscheinungen sind durch ihn bewirkt worden.

*) De civit. Dei XXI. 4. Er bemerkt zugleich, dass der Magnetstein zuerst aus Indien gekommen.

*) Thales und andere Weisen hatten nach Aristoteles die Wunderkräfte des Magnets durch eine in ihm wohnende, ihn belebende Seele am besten zu erklären geglaubt.

*) In den Platonischen Fragen.

*) Daher wohl sein französischer Name: Aimant, wie aber Sympathie und Attraktion ihm eigen ist, so auch Antipathie, denn er soll, wie die Alten wähnen, nur bei Tag wirken; manche ihm heterogene Dinge, als Zweifeln, Oel, Mercur schwächen seine Kraft.

Unter den künstlichen Wundern, die ihren Ursprung dem Magnet zu danken haben, verdient zuerst jener Tempel Erwähnung, den nach Plinius (Nat. Gesch. 34. 42.) der Architekt Dinochares auf Geheiß des Ptolemäus erbaute, und darin die eiserne Statue der Arsinoe schwebend anbrachte; Plinius sagt zwar, der Tod des Ptolemäus habe den Bau verhindert, allein Auson beschreibt die Bildsäule als wirklich vorhanden und schwebend:

"Conditor hic forsan fuerit Conditor aulae Dinochares, quadroqui in fastigio cono Surgit, et ipsa suis consumit piramis umbras: Jussus ob incesti qui quondam foedus amoris Arsinoen Pharii suspendit in aere templi."'

Augustin de Civit. Dei Lib. XXI. Cap. 8. bestätigt es, wofern dieser Kirchenlehrer nicht eine Bildsäule der Sonne in Serapis Tempel zu Alexandrien darunter verstand, die gleichfalls schwebend erschien. Allein Ruffin (in der Kirchengeschichte Lib. II. C. 23.) sagt: die Statue sei bloß durch einen oben am Gewölbe angebrachten Magnet (vielleicht war das ganze Gewölbe von Magnet-Stein) festgehalten worden, statt dass jene der Arsinoe zwischen zwei ihr zur Seite stehenden Magneten gleichsam schwebend erschien; mehrere solche Bildsäulen, teils schwebend, teils bei gewissen Feierlichkeiten, wie jene des Apoll durch Magnet künstlich in Bewegung gesetzt, deren Lucian de Dea Syria gedenkt, werden im Altertume erwähnt*), zum Beispiel das Sonnenbild im Tempel des Serapis, der Magnet, der, nach Kircher de Art. Magnet. Lib. I. Cap. I. die ehrnen Kälber des Jeroboams im Tempel, und nach den Rabbinen, jener, der den ehrnen Kranz der Ammoriten festhielt, eine Statue der Sieges-Göttin, die zwischen vier Bildsäulen schwebend hing; ein Cupido, dessen der König Theodorich in einem Brief an Boetius erwähnt, ein berühmter gleichfalls schwebender Bellerophon zu Pferd, u. a. m. --- Von Mahomeds Sarg hatten die Araber gleiche Sage, die sich aber dahin beschränkt, dass nach Zeugnis eines alten Reisenden\footnote{} oberhalb demselben ein großer Magnet-Stein angebracht ist, woran schwebend ein goldener halber Mond hängt.

*) Falconnet Dissert. sur ce que les Anciens ont lu sur laimant in mem. de l'acad des inscript. T. IV.

Besondere Verehrung bezeigten dem Magnet als heiligem Steine die Hebräer; dass er ihnen in der Wüste schon bedeutungsvoll war, lernen wir aus dem Torah, worin unter andern Rabbi Isaac Abarbanel sagt- zur dritten Ordnung gehören drei seltene, aus den Gegenden der Stämme Gad, Asser und Isaschar kommende Steine Lessem, Schebo und Aschlamah,\footnote{} denen aus dem Tierkreis die Zeichen der Waage, des Skorpions und Bogenschützens, und aus den Monaten Thisli, Marschevan, und Choslen, (der 1ste, 2te und 3te Monath) entsprechen. Unter diesen Steinen ist Lessem der Amethyst, Schebo der Magnet, bekannt durch die Eigenschaft, Eisen anzuziehen, Aschlamah aber eine Steinart, die den Schlaf befördert. Steine, besonders die edleren, waren, nach Proklos de Magia, mehr als Pflanzen, in besonderer Verwandtschaft mit Sternen und Geistern; so sollte der Jaspis die Opfer den Göttern angenehm machen, der Achat die Gatten unauflöslich verbinden, so hatte der Lapis Solaris und Lunaris Bezug auf Sonnen- und Mond-Dienst, und wie jedes Mineral einen ihm eigenen Stern hatte, so gehörte dem Polar-Stern der Magnet, der, wie das Trinum Magicum sagt: offenbar die Verwandtschaft der untern mit der oberen Welt zeigt. Talismane wurden daher zu Siegelringen öfters in Magnet-Steine geschnitten. Ein solcher war der Skarabäus, den der Engländer John Gräv im 17ten Jahrhundert aus dem Orient brachte, und der aus dem lebhaftesten kräftigsten Magnete bestand.\footnote{} Wodurch Magnet-Stein aber als Führer und leitendes Wesen unter allen Steinen seine höhere astralische Kraft am meisten bewährt, ist der Compass; eine Entdeckung, die unter die glücklichsten und wichtigsten der Weltgeschichte zu zählen ist. Die allgemeinste auch neuerlich noch behauptete Meinung ist*: Flavio Gioja von Amalfi habe im 16ten Jahrhundert zuerst den Gebrauch der Magnetnadel gefunden; allein D. Hager in einer kürzlich herausgegebenen Schrift\footnote{} hat aus chinesischen Urkunden erwiesen, dass dies Volk sich des Kompasses schon 1100 Jahre vor der christlichen Zeitrechnung bedient, und die Abweichung der Magnetnadel gekannt habe. Von ihnen empfingen ihn durch den Handel die Araber im 9ten Jahrhundert; denn als die Portugiesen auf ihrer ersten Umschiffung Afrikas an die Ostküste dieses Erdteils kamen, waren die Araber längst in Besitz des Kompasses; durch die Kreuzzüge aber waren Amalsitaner bereits mit Syrien, Palestina und Arabien in Verbindung. Auf diese Weise also kam er wahrscheinlich nach Europa.

*) Gabriel Bremond Descript. de l'Egypte 1679.

*) Die Talmudische Tradition hat, wie in so vielen, auch hier alles untereinander geworfen. Der wahren Auslegung gemäß ist Aschlamah der Amethyst, Schebo der Achat, und Lessem der Lynkurer oder Luxstein, der seiner anziehenden Kraft wegen von Theophrast schon für eine Art Bernstein angesehen ward, und deswegen Analogie mit dem Magnet hat. Diese drei gehörten zu den zwölf Steinen, welche im Brustbild des hohen Priesters befindlich waren, und worauf die Namen der zwölf Stämme graviert wurden. Die übrigen hießen: Odem (Sarder), Pithera (Topaz), Barechet (Smaragd), Rophet (Karfunkel), Saphir (Saphir), Jahalom (Jaspis), Tharsis (Chrysolit), Schohem (Onix), Jaspe (Beryll). S. Calmet Bihl Wörterbuch. Nach Hartmanns neuesten scharfsinnigen Untersuchungen in der Hebräerin am Putztische, 1. Th. S. 278 und 3. Th. Anmerk. 137. prangten im priesterlichen Brustschmucke folgende Steine: der Karneol, Smaragd, Karfunkel, Jaspis, Lazurstein, Lyncurier, Amethyst, Chrysolith, Achat, Topaz, Onix, Sardonir. Sprachforscher mögen über die Richtigkeit seiner Bemerkungen und den ächten Sinn der hebräischen Benennungen dieses Steine entscheiden.

*) Kircheri Magnes. pag. 13.

*) Annales des Voyages de l'histoire et de la Geografie par Maltebrun. Cahier XXIX. p. 252.

*) Memoria sulla Bussola orientale. Pavia 1809.

Hat der Magnet durch seine innere Kraft und magische Wirkung sich ein so hohes Ansehen erworben, dass er den Rang vor allen Steinen erhielt, aus dem Grunde, weil unter allen er am meisten Analogie mit der höheren siderischen Region, und desfalls den wichtigsten Einfluss auf die Wesen der niederen Erde haben müsse, so wurden Metalle und Steine, (der Diamant selbst) umso höher geschätzt, als sie die Anziehungs-Kraft gleich dem Magnet besaßen, und dadurch ihm verwandt schienen. Daher der Diamant in älterer Zeit bei Borel und anderen Schriftstellern über Steine oft mit dem Magnet verwechselt, und beiden der Nähme Adamas gegeben wird. -- Die nächste Stelle in dieser Hinsicht nimmt das succinum (Bernstein) ein, den Griechen bekannt unter dem Namen ηλεκτρον von ηλεκτορ, Sonne, seines Glanzes wegen. Von seiner anziehenden Kraft redet schon Theophrast, glaubt den Stein aber ein Mineral.\footnote{} Seiner eigentlichen Natur gemäßer\footnote{} beschreibt ihn Plinius (Naturgeschichte 37. 11.) durch den Mythos der Schwestern des vom Blitz getroffenen Phätons, die durch ihr Heulen in Bäume verwandelt wurden, und von deren Tränen jährlich am Fluss Eridanos das Elektrum entstehe.

Einer Menge anderer Steine schrieben die Alten höhere Kräfte zu, und reihen sie deshalb unter die dem Cultus vorzüglich geeigneten. Sie aufzuzählen, wäre eine ebenso mühsame Arbeit, als ihre Nahmen nach heutiger Nomenklatur bestimmen zu wollen. Was das orphische Gedicht von den Steinen sagt, hat vorzüglich Bezug auf Magie und Heilkunde, die in ihrer Kindheit unter der Herrschaft der Divination stand, und am liebsten durch magische Mittel, Beschwörungen, Zauberformeln, Amulette, heilte, weswegen auch in jener Zeit, wie noch jetzt bei den Wilden, Priesterwürde und das Amt eines Arztes verbunden waren.

*) Theophrasts Lynkür war wahrscheinlich auch eine Art Bernstein.

*) Er ist nämlich ein durch unterirdische Gärung verändertes und verhärtetes Fichtenharz.

Unter den Alten hat Theophrast am meisten Bestimmtheit; allein Plinius, der oft auf bloßes Hörensagen Nachrichten sammelte, und ohne Kritik häufig Steine mit mehreren Nahmen belegte, hat diese Materie so verworren, dass, vermehrt durch die empirischen Nachrichten und Zusätze der Lithographen des Mittelalters, es mit allem Scharfsinne kaum möglich ist, die eigentliche Natur aller bei den Alten vorkommenden Steine zu bestimmen. Gleiwohl hat verbesserte Physik auch in dieses Fach helleres Licht gebracht, und es ist nicht zu bezweifeln, dass die wichtigen Fortschritte der Chemie in der Analyse der Erbsubstanzen, auch über die Natur und inneren Bestandteile der Steinarten uns neue Aufschlüsse geben werde.

Wenn wir auf den frühesten Steingebrauch zurückgehen, der darin bestand, durch Reibung dem Menschen den Gebrauch des Feuers zu lehren, sind wohl jene Steine, die am leichtesten entzündbar sind, und Feuerfunken von sich geben, diejenigen gewesen, die am frühesten hervorgesucht, und weil sie heilige Gluth des Himmels, oder nach der früheren Sprache zu reden, den Feuergeist enthielten, für Bätylien galten. Überhaupt können wir annehmen, dass je mehr die Steine nach inneren und äußeren Bestandteilen, meteorischen Ursprung, astralische Verwandtschaft, magische und höhere Wirkung äußern, umso mehr wurden sie als heilige Steine angesehen; denn nicht der Stein für sich selbst, der Elementar-Geist, dass ihn belebende Princip, ward in ihm verehrt.*

In dieser Hinsicht können wir, um sie einigermaßen zu klassifizieren, sie in nachfolgende Ordnung einteilen: 1) in feuergebende; 2) als solche, die durch Feuer oder sein entgegengesetztes Element Wasser erzeugt werden; 3) Steine, die durch Glanz, Farbe, Durchsichtigkeit, Form, oder andere Eigenschaften sich auszeichnen.

*) Die Alten, unbekannt mit der neueren Chemie, benannten die Steine weniger nach ihrem inneren Gehalte, als nach äußeren Merkmahlen, nach ihrer Anwendung und Nützlichkeit.

Unter den ersten (Feuer gebenden) ist vorzüglich der Schwefelkies, Markasit, zu nennen, das gemeinste Mineral in der Natur, dass darum sowohl, wie seiner Härte wegen, noch vor dem Feuerstein und anderen Kieselarten zum Feuerzeug gebraucht wurde, weshalb es auch den Nahmen Pyrit erhielt. Die härtesten und feuerreichsten nannte man Pyrites vivos, und gebrauchte sie in Feldlagern; da er schöne Politur annimmt, so ward er von den Peruanern, Griechen, Römern, überhaupt dem ganzen Altertume zu Spiegeln gebraucht. Der eigentliche Feuerstein, der oft in Hornstein, zuweilen in Calcedon übergehet, ist unter allen Fossilien, teils wegen seiner Härte, teils weil er am meisten Reibelektrizität besitzt, am dienlichsten zum Feuerzeuge.\footnote{} Über ihn herrschte jedoch unter den Alten viele Dunkelheit, denn Plinius nennt mehrere Steinarten Silex; unser Feuerstein scheint sein Silex globosus zu sein, der späterhin des Gebrauches wegen Pyromachus und Silex creta ceus hieß. Wenn und wo der Gebrauch des Feuerzeugs aufkam, ist nicht zu bestimmen, aber seine Erfindung verliert sich in die Zeit der Mythe; ihr gemäß war, wie Plinius erzählt,\footnote{} Pyrodes der erste, der Feuer aus dem Kieselsteine schlug, Prometheus aber erfand eine Art von Lunte (Ferula), weshalb ihm wahrscheinlich Schuld gegeben ward, dass er heimlich das Feuer vom Himmel entwendet habe. Die ersten Worte sind aber bloß mythisch zu nehmen, denn Pyrodes ist (wie Schmieder in der Lithurgik II. Th. S. 174. bemerkt) dasselbe, was im französischen Feu portatif, im Deutschen Feuerzeug bedeutet, und Cilix ist nichts anders als Silex. Vom Verfahren der Alten beim Feuerschlagen belehrt uns ebenfalls Plinius.*

*) Schmieder Versuch einer Lithurgik oder ökonomischen Mineralogie. Leipzig 1803.

*) Ignem e silice Pyrodes cilicis filius, eundem adservare in ferula Prometheus monstravit. Hist. nat.
2. Hi exploratoribus castrorum maxime necessarü, qui clavo vel altero lapide percussi, scintillas edunt, quae exceptae sulphure aut fungis aridis, vel foliis, dicto celerius ignem trahunt.

Dieses Fossils (das erst nach Erfindung des Schießpulvers und des Feuergewehrs seine wahre Anwendung erhielt*), bediente man sich in älteren Zeiten auch als schneidenden Instruments zu Opfer-Messern und Streit-Äxten; man findet dergleichen häufig noch in Grabmählern.\footnote{} Überhaupt gebrauchte man alle kieselartigen Steine zum häuslichen und religiösen Dienste. Jener Pessinuntische, von dem es immer noch zweifelhaft bleibt, ob er ein Aerolithe gewesen, war vielleicht von dieser Gattung, denn bestimmt wird er religiosa Silex genannt; und warum gab man den Sand, Kalch, Kreide oder Seifenartigen Steinen nicht diesen Nahmen? mir deucht eben, weil sie in Hinsicht des Feuers zu religiösem Zwecke nicht so brauchbar als jene waren.

*) Daher neuerer Name: Flintstein.

*) So bei den Juden und den Priestern der Cybele zur Beschneidung daher er (vielleicht von Sicilex, Sicilis einem schneidenden Instrumente, und scindere schneiden) seine Benennung erhielt.

Hier müssen wie auch des Fossils erwähnen, das unter dem Namen Donner-Steine bekannt ist, und worüber in alten sowohl als neueren Zeiten der Volksglaube allgemein war, dass es Donnerkeile seien, die der Blitz in die Erde versänke. Man findet sie häufig in Norden, besonders zu Streitäxten, Mossern und Waffengeräten verarbeitet, oft in uralten Grabstätten; deutlich erkennt man an ihnen die ursprünglich parallelpipedale Gestalt; an einem Ende schliff man sie spitz zu, am andern wurden sie durchbohrt; andere findet man bloß angebohrt, und zuweilen mit Stielen versehen. Die durchbohrten hing man nicht selten an Baumäste, man hielt es für ein Kennzeichen achter Donnerkeile, wenn sie bei Gewittern zitterten, und wenn ein fest um sie gebundener Faden im Feuer nicht verbrannte, welches, bei jedem andern Steine der Fall ist. Viele derselben sind wahrscheinlich achte Aerolithen, worüber Morhoff, der den Stein Lapis fulminaris nennt, die scharfsinnige Bemerkung äußert: "`manche könnten an der Oberfläche der Erde selbst vom Blitz aus den vorhandenen Materialien auf der Stelle gebildet worden sein. Ob die bekannten idäisch-daktylischen Steine auf Kreta von der meteorischen Art, oder vielleicht Meeresprodukte, nämlich versteinerte Belemniten find? ---. käme auf nähere geologische Untersuchung Viele dieser Donnersteine aber haben nach genauer Analyse, alle Kennzeichen des Lydit oder sogenannten Probiersteins, der ein bloß in Schuttgebirgen umkommender jaspisartiger Kiesel-Schiefer, an Farbe meist schwarz sehr hart, und politurfähig ist.\footnote{} Diejenigen Steinarten, welche die Grundlage der Urgebirge machen, und älter als die organische Schöpfung sind, die Granit, Gneus, Syenit, Graustein, Porphyr, Tonschiefer, Urtrapp, Serpentin, Urkalk, sind ihrer Härte und Unverwitterlichkeit wegen zu Unterlagen, Säulen, Tempeln, Altären, und allen dauerhaften Monumenten geeignet; die Ägypter allein mussten durch die Härte ihrer Arbeitswerkzeuge ihnen bestimmte plastische Formen abzugewinnen; der Marmor hingegen ist seiner weichen zart geschmeidigen Natur und Empfänglichkeit zur schönen Politur wegen, am geeignetsten für plastischen Gebrauch, um die Formen der Götter abzubilden, und dadurch das höchste Ideal der Kunst zu erreichen, was die Griechen ihren Gestalten auch zu geben wussten.

*) S. Schmieders Lithurgik II. 112. und Schröders Lithografisches Lexikon.

Unter den durch Feuer erzeugten Steinen nennen wir zuerst die aus der höheren Region zur Erde herabgefallenen Meteor-Steine; ihr Entstehen sei auch, welches es wolle; immer hatten sie (als dem Himmel entfallen oder mindestens in höherer Atmosphäre gebildet) den nächsten Anspruch auf Verehrung. Gleich schreckhafte Meteore entstehen durch Vulkane, und die Steine; die aus dem inneren Erdenschlund ausgeworfen werden, mussten, da sie in früherer Zeit noch häufiger waren, als jetzt, teils durch ihre mannigfachen Bildungen, teils durch den Nutzen, den man bald aus ihnen zu ziehen lernte, bald die Aufmerksamkeit an sich ziehen. Dergleichen sind Laven und ihre mannigfachen Untergattungen. Der Bimstein \footnote{} z. B. insbesondere der Obsidian, der nach neuesten Forschungen von Humbold und andern ein eigentlich vulkanisches Edukt, und die Mutter des Bimsteins ist, dem er durch Feuer ähnlich wird. --- Andere durch Feuer erzeugte Fossilien sind jene, die ihr Entstehen den in manchen Gegenden so häufigen Erdbränden verdanken. Bergharze, Asphalte, Naphten, wozu auch der Gagat gehört, die verhärtet einer schönen Politur fähig sind, und wie die Kunstgeschichte zeigt, in Babylon besonders und andern Gegenden Vorderasiens, wo diese Brände häufiger waren, vielfach verarbeitet wurden. Manche Steinkohlen, vorzüglich jene, die viel Schwefelkies enthalten, können gleichfalls dahin gezählt werden; die helmartigen schwarzen Steine, die nach Plutarch im Eurotas gefunden werden, und als geweihte Steine häufig im Tempel der Minerva Chol-Kiokos niedergelegt wurden, waren vielleicht von dieser Gattung.*

*) Über den Theophrast K. 14. 19. 20. schon sehr klare Einsicht hatte.

Da häufige Erfahrungen uns belehrt haben, dass nicht durch Feuer allein, sondern durch Wasser auch Steine entstehen, und prismatisch in größeren oder kleineren Massen sich bilden, so können wir vorzüglich den Basalt dahin rechnen, ein Fossil, das im Altertum mehr noch als jetzt, seines schönen Schliffes wegen, den er annimmt, teils zu größeren Werken der Kunst, teils zu Talismanen und Skarabäen benutzt ward. Gleiche Ursache, nämlich Wasser, erzeugte auch jene Überbleibsel der ersten Schöpfung, die Trümmern verloren gegangener Muschel- und Schaaltiere, die als Versteinerungen auf den höchsten Gebirgen oft --- Familienweis gelagert --- sich vorfinden. Die Echiniten besonders, und Belemniten mit ihren Untergattungen, von denen aus mehreren Beweisstellen des Altertums, die Falconnet in der Abhandlung: Sur la Pierre de la mexe des Dieux\footnote{} gesammelt hat, erhellet, dass man ihnen Wunderkräfte zuschrieb, und sie deshalb göttlich verehrte, besonders hatte dieses statt mit den Priapolithen und Hysterolythen, der heutigen Venus-Muschel, die Plinius Diphyes nennt, wobei er sagt; ut concubitu venereo aptum dieris, nisi lapis esset; und zwar ihrer Ähnlichkeit wegen mit dem indischen Ioni und Lingam.

*) Plutarch Abhandl. von den Flüssen.

Wir kommen jetzt auf diejenigen Steine, die durch Glanz, Form, Farbe-Mischung und andere Eigenschaften besonderen Wert erhielten: die Edelsteine nämlich, jene prachtvolle schöne Erzeugnisse der anorganischen Schöpfung, der Schmuck, die schönste Blüthe des Steinreiches, deren Schimmer und Strahlenglanz in den frühesten Zeiten schon aller Augen entzückte, und als höchste Seltenheit aufgesucht wurde. Unsere Schrift würde ihre Grenzen überschreiten, wollten wir sie alle hier aufzählen und nach ihrem inneren Gehalte erforschen, die festere Textur der Teile, ihre Durchsichtigkeit, Härte, Glätte und Zartheit der Politur gibt ihnen zwar hohe Vorzüge vor den gemeinen Metallen; gleichwohl haben sie (den Demant ausgenommen, der seiner inneren Natur nach reiner kristallisierter Kohlenstoff ist) den Metallen und deren Mischungen ihren höchsten Schmuck, die Farbe zu danken.

*) Mém. de l'acad. des Inscr. p. 23.

Asien, die Wiege des Menschen-Geschlechts, besonders jenes an Produkten so gesegnete Land Indien, ist auch die Gegend, aus deren reichhaltigen Gruben im hohen Altertume schon die vorzüglichsten Steine kamen, die teils für Gegenstände des Luxus, teils zu gottesdienstlichen Gebräuchen bestimmt waren, um priesterliche Gewänder, Tempel und Altäre zu schmücken.

Aus diesem Lande der ältesten Mythen kamen zuerst auch jene Sagen der Wunderkräfte und heilbringenden Eigenschaften, die man vielen unter diesen Steinen zuschrieb, und da der weite Handel, der damit getrieben wurde, sie durch viele Hände oft unkundiger leichtgläubiger Steinhändler gehen ließ, vermehrte sich noch der Wunderglaube an magische Wirkungen dieser Steine, wozu unleugbar noch ihr innerer Gehalt und manche Wirkungen, z. B. die magnetische und elektrische Kraft derselben beitrugen. So hingen sie auch als Arzneimittel, innig mit der älteren Heilkunde sowohl, als mit Magie und Astrologie zusammen, davon die Orphische Steinschrift, Theophrast und Plinius im 37ten Buche die überzeugendsten Beweise liefern.

An diesem Glauben der Wirkungen gewisser Steine auf dies oder jenes Uebel, hing das ganze Mittelalter, und noch heut zu Tage ist er im Orient herrschend.

Manche Eigenschaften dieser Steine sind auch wirklich so auffallend, dass Menschen, denen die chemischen Bestandteile derselben nicht bekannt find, diese Phänomene leicht für Wirkungen höherer Kräfte gelten können. Ich will nur einige der merkwürdigsten nennen. Denn alle im Plinius vorkommenden Steine dieser Art lassen sich kaum mehr mit Genauigkeit bestimmen, da er unter die Gemmen so viele rechnet, die nicht dazu gehören; wie z. B. nur er vom Ammons-Horn sagt: Cornu ammonis inter sacratissimaa Aethiopiae gemmes refertur. Besonderer Erwähnung verdient 1) der Turmalin, ein Fossil, von dem zwar die alten Nachrichten schweigen, dessen Dasein ihnen doch höchst wahrscheinlich bekannt sehn musste, da es in Ceylon, dem alten Tabrobane, mit welcher Insel bekanntlich ein sehr alter Handelsverkehr war, zu Hause ist, weshalb dieser Stein wohl auch unter denjenigen Gemmen, deren Plinius erwähnt, ohne sie genau zu bestimmen, leicht verborgen sein konnte. Die außerordentliche Polarität\footnote{} dieses krystallförmigen Schörls, und seine Elektrizität, vermöge welcher er im Sonnenschein, in heißer Asche, auf heißem Eisenblech oder durch Reiben erwärmt wird, machen ihn vorzüglich merkwürdig.

*) Und zwar ist der eine Pol vositiv, der andere negativ elektrisch. Wenn aber zwei erwärmte Turmaline auf Papier über Wasser schwimmen, so verhalten sie sich wie zwei Magnetnadeln, indem sich die gleichnamigen Pole abstoßen, und die ungleichnamigen anziehen.

2. Der Lazur-Stein, Cyanus, den die Alten auch Saphir nannten, der heutige Lapis Lazuli, verdiente besonderer Erwähnung, weil (wie wir gesehen haben) schon Hiob ihn nennt, und Theophrast Kap. 30. 38. 47. 50. ihn beschreibt, sehr verschieden ist er jedoch von unserem Saphir, der weit durchsichtiger ist, und nach der Beschreibung, die Plinius von der Asteria, einem in Indien und Caromanien einheimischen Steine macht, treffen beide ihrer Natur nach aufs genaueste zusammen, indem Lehmann und andere Lithographen gleichfalls an ihm entdeckt haben, was man des Plinius Aussage nach für Fabel hielt, nämlich: dass angeschliffen, er den Schimmer von mehreren übereinander liegenden Sternen\footnote{} zeigt, daher ihn die Neueren Girasol nannten, und ihn bald für Calcedon, Kazzenauge-Opal, bald für Kristall hielten. Überhaupt gaben die Alten den Namen Lapides Stellares allen Edelsteinen, die, wenn sie geschliffen sind, einen sternähnlichen Glanz geben, und denen man ihrer anscheinenden Analogie wegen mit jenen helleren Lichtern der Himmels-Region, auch höhere geheime Kräfte zuschrieb. Plinius\footnote{} gibt als solche vorzüglich an: die Asterie, Astria, Astroides, Astrobol, Akopos, Antipatos, das Belus-Auge (ein babylonischer Stein) Sonnen-, Mond- und Komet-Steine, den persischen Mythras, Heliotropos, Calcophonas, Karfunkel (der heutige Granat), Häphestites, worunter jedoch mehrere eher zu Versteinerungen zu rechnen. Zum Kunstgebrauch wühlten die Steinschneider vorzüglich den Praser,\footnote{} Calzedon, Onix, Jaspis und Achat. Die Steine, aus der die sogenannten Amulette, Abraxas, Talismane, Lapides divi oder vivi bestünden, waren also eben so verschieden, als der Gebrauch, den man von ihnen machte, und man muss sie hiernach ebenso sorgfältig, unterscheiden, als nach der Gegend, woher sie kommen; so sind zum Beispiel Harz- und Asphaltischer Natur die babylonischen mit Keil- oder Pfeilschrift bezeichneten Backsteine, die am wahrscheinlichsten Zauberformeln, oder, (wie Plinius versichert) astronomische Beobachtungen enthielten.\footnote{} Die persischen Zylinder, deren Farbe gewöhnlich weiß oder blaulicht, zuweilen auch schwarz ist,\footnote{} worauf sich gewöhnlich Abbildungen mit oder ohne Schrift befinden, sind ebenfalls zu den Bätylien zu rechnen. Was die ägyptischen betrift, so ergibt sich aus der häufigen Gemeinschaft, die in frühen Zeiten zwischen diesem Lande, Babylonien, Chaldea und Persien war, dass von dort aus häufig geweihte Steine oder Talismane vorkommen, die den persischen auffallend ähnlich sind,\footnote{} wie überhaupt sich immer mehr zeigt, dass der mythische Stoff aller Völker aus einer und derselben Quelle floss; daher auch eine die andere aufschließt und erklärt. Ferner scheint erwiesen, dass von allem Aberglauben, wodurch reine Gottesverehrung in Abgötterei ausartete, der Fetischen-Dienst, und der zugleich damit entstandene Gebrauch der Amulette, Talismane, Zaubersteine der früheste gewesen t), mit allmähliger Zunahme dieses Aberglaubens wurden bei immer vermehrtem Verkehr der entferntesten Völker, diese Steine endlich ein Gegenstand des Handels, der vorzüglich von Chaldäern und Persern, am meisten aber von arabischen Idumäern oder Edomiten in die fernsten Länder getrieben ward, dessen Hauptsitz (wenigstens Stapelplatz) Heliopolis. war, nebst den umliegenden Städten am Libanon (wo der Sonnendienst herrschte, und woselbst, wie wir gesehen haben, sich ein so großer Vorrat Bätylien befand) was umso natürlicher ist, als durch diese Gegend eine noch heut zu Tage von den Caravanen besuchte Haupthandelsstraße geht; aber nicht bloß phönizische Stoffe, arabische Gewürze, Gold, Perlen, Edelgesteine und andere Handelswaren wurden dort vertauscht, auch religiöse Gegenstände teils zum Priester- und Tempeldienst gehörig,\footnote{} teils durch Aberglaube dem Volke unentbehrlich geworden, wurden auf dieser Straße in alle Länder verführt, die ein gemeinsamer Cultus verband. Wir wissen aber, dass Sabaeism oder Verehrung der Gestirne die früheste Religion war, deren Gebräuche und Symbolik man unter mannichfachen Modifikationen von Hinterasien an zum Nil nach Europa herüber bis in den nördlichen Kaukasus verbreitet findet; und da, wie wir früher gezeigt haben, Sterne, dieser uralten Lehre gemäß, für göttliche Wesen angesehen wurden, deren Einflüssen alles Geschaffene unterworfen wäre, so ist der daher entspringende Glaube an geheime, den Pflanzen, Metallen und Steinen in wohnenden (durch Einfluss der Konstellationen) wirkenden Kräfte, wie aller mit solchen Steinen getriebene Missbrauch aus dieser Quelle herzuleiten.

*) Brückmann Abhandlung von Edelsteinen; ach Schmieder Lithurgik II. 260.

*) Nat. Gesch. Buch 37.

*) Ein grüner Quarz, dem Juweliere oft, jedoch uneigentlich, den Namen Smaragd-Mutter geben.

*) S. D. Hagers Abhandlung über die vor kurzem entdeckten babylonischen Inschriften, in Klaproths asiatischem Magazin.

*) Mehrere derselben liefert Cailus Recueil d'antiquités, und Montfaucon ant. expliquée.

*) Beispiele sehe man in Cailus Recueil d'antiquités. Tom. V. pl. 12. 13. 14. 17. Ferner Tome I. planehe 17. Tome II. pl. . Tome III. pl. 1. No. 4. T. IV. pl. 21. 22. T. VI. pl. 19. 20. 21. 22. T. VII. pl. 6. Auch Montfaucon antiquite epl. . II. part. 2.

*) Zoega de orig. et usu Obelisc. p. 241.

*) Von dem Handelsverkehr dieser Stadt mit Phönizien, Afrika, und dem südlichen Asien, s. Heeren Ideen über den Handel der alten Völker I. Th. S. 75.

*) Ein Beispiel finden wir noch im christlichen Zeitalter an jenen silbernen Dianen-Tempelchen, die von dortigen Künstlern verfertigt weit und breit verschickt wurden.

Mehrere Altertumsforscher\footnote{} haben die gnostischen Valentinianer und Basilidianer als Urheber des großer Verkehrs angegeben, der bis ins vierte und fünfte christliche Jahrhundert mit Bätylien getrieben wurde, ja sie selbst als Erfinder er unter dem Namen Talismane, Amuletten, Skarabäen, Abraxas, bekannten Zaubersteine angegeben; allein diesem widerspricht schon die Tatsache; dass die Gnosis (wie gleichfalls die Lehre des Manes) von der die meisten theurgischen Sekten und philosophischen Lehrsysteme jener Zeit entstanden, aus dem Petsismus oder Zerduschts Lichtlehre hervorgegangen,\footnote{} und gleichsam nur ein trüber Spiegel ist, in dem man gleichwohl die obschon entstellten Züge des ersten beinen Bildes erkennt. Christliche Ideen mischten diese Schwärmer mit Chaldäisch-persisch-ägyptischen\footnote{}; und wurden durch den Handel, den sie mit solchen Steinen trieben, in der späteren Zeit das, was für frühere Völker die Chaldäer und Araber waren; überschwemmt haben sie freilich mit ihren geweihten Talismanen das ganze Morgen- und Abendland bis nach Spanien und Gallien; aber Caylus hat an mehreren Stellen seines schätzbaren Recueil d'antiquites gezeigt, dass der Umsprung und Gebrauch solcher Zaubersteine aus den frühesten Zeiten herrühre. Da es nun einmal erwiesen ist, dass Bätylien zu den vom Himmel gefallenen Steinen gehören, wir zugleich aber aus Plinius\footnote{} und anderen Quellen wissen, dass auch andere Körper: Balken, Lanzen, Spiese, leuchtende, wie Ähren geflochtene Kränze, Feuersäulen, Sternschnuppen u. a. zur Erde fielen, so erhellet daraus, dass unter die Klasse der Διιπετρα, die deswegen heilige Steine hießen, weil sie dem Himmel zugehörten, und himmlischen Ursprungs sind, man nicht bloß Aerolithen, sondern: überhaupt alle Steine, denen man übernatürliche Kräfte zuschrieb, und die als Gemmen gebraucht wurden, rechnen könne\footnote{}). Dass ferner aller Aberglaube, den man im Altertume mit solchen Steinen trieb, aus dieser Quelle floss, woher auch die Talismane entstanden. Diese Talismane waren von verschiedener Form und Größe: "`Erant, sagt Kircher in Ödipus T. II. p. 445 duplicis generis. Majora et Minora. Majora, et immobili positae solidata, in publicis locis Urbium; Templorum, coemeteriorum, tum regionum claustris ad hostium arcendum insultus et αντιτεχνιας δαιμονων κακουργων eludendos ponebantur. Minora et portatilia in domibus, in Collo, pectore, manibus hominum, animaliumque ad malorum Averruncationem portata serviebant. Für einen solchen kann auch der ägyptische Canopus gelten, der dem geheimen religiösen Sinne nach, ein Symbol des Wassers als wohltätigen befruchtenden Elements war, wie dies eine Stelle des Abenephius, eines arabischen Schriftstellers zeigt, den Kircher im Ödipus T. I. p. 211 anführt; habent ipsi, sagt er, idolum quoddam Canopis nomine, et est in modum Vasis tumidum, et quando Aquis plenum fuerit, Aqua per Ubera, quae in eo effinxerunt, refunditur, et indicatur efluxu, processus naturae omnia nutrientis. s. auch Suidas. --- Ruffin hist. eccl. I. 11. --- Porphyr apud Euseb. eccl. Hist.

*) Unter andern Montfancon ant. expliquée T. II. art. 3. der zu den Abbildungen in Chifflet-Traktat on Talismanen und Abraxas noch viele andre gesammelt hat.

*) Man lese hierüber die Quellen im Zend-Avesta Upnekhat, und was Kleuker darüber gesammelt hat' im kl. Teil seines deutschen Zend-Avesta an mehreren Stellen. Auch Beausobre Hist. crit. du Manichéisme in mehreren Stellen.

*) Von älteren und späteren ägyptischen Talismanen s. Kircheri Oedipus T. II. p. 459.

*) Nat. Gesch. II. Buch, Kap. 25. 6. 7. Heradian I. Buch, 138. Livius in mehreren Stellen.

*) S. Passeri Gemmae astriferae, Kircheri Oedipus T. II. P. 2. und die Schriften über Talismane und Abraxas, besonders Traite des Talismans ou Figures astrales. Paris 1668.

Ein Talisman von der größeren Gattung wäre jener neun Ellen hohe smaragdene Koloss des Serapis im ägyptischen Labyrinthe, dessen Appian erwähnt, s. Zoega de usu et orig. Obelisc. p. 8 auch andere Bildsäulen von Isis und Osiris, Horus Hermes, die an die Eingänge der Tempel, Grabmäler, an Grenzorte u. s. f. als Schutzgenien gesetzt wurden. Ihr Gebrauch verliert sich in Ägypten, wie in allen Ländern in die frühe Zeit des rohen Fetischen-Dienstes. Solche Schutzgötter von aller Form und Größe finden sich nach dem Zeugnis der Reisenden noch heut zu Tage bei allen wilden Völkern, und wie viele derselben ihren Toten selbst Talismane und Amulette mit ins jenseitige Leben geben, so wickelten die Ägypter aus Furcht vor den Nachstellungen Typhons und anderer bösen schadenden Wesen zwischen die Bandagen ihrer Verstorbenen, Amulette, kleine Osiris Idolen, Skarabäen, Nilpferdchen, kleine Knuphschlangen, geschnittene Steine, u. dgl. m. Hier lässt sich vordersamst die Frage aufwerfen, ob eigentliche Aerolithen, d. h. jene Steine, die man nach der in neuerer Zeit mit ihnen vorgenommenen Analyse, für ächte Meteorsteine erkennt, zu jenen gehören, die zu Gemmen bearbeitet und graviert werden konnten? --- Nachrichten in Theophrast, Plinius und anderen Schriftstellern über Steine\footnote{} sind keine darüber vorhanden, indessen scheint aus ihrer Natur und der lockeren, leicht oft gleichsam durchsichtig verbundenen Textur derselben (welche dem Rade und der Bearbeitung des Künstlers kaum widerstehen würde) die Schwierigkeit zu erhellen, sie zu diesem Zwecke zu gebrauchen, wodurch indessen nicht geleugnet wird, dass härtere, mehr und inniger verbundene Meteorsteine vor Alters nicht sollten bearbeitet worden sein? ) umso mehr, als aus den neuen Untersuchungen der HHr. Scherer und Schreiber über die mährischen Meteorsteine (in Gilberts Annalen der Physik 1809, 1s Stück) sich ergibt, dass diese Aerolithen nicht allein eine schöne Politur annehmen, sondern auch zu Vasen und andere Formen sich leicht bearbeiten lassen; wie denn jener des Arztes Isidorus zu Emesa, nach der Beschreibung, die Photius von ihm gibt, mit zwei Sternen bezeichnet war, und wahrscheinlich gehört der persische Zylinder, den Millin in seine Monuments inedits nouvellement expliqués Tome I aufgenommen hat, wie die meisten, deren Caylus in seiner Sammlung erwähnt, unter die Aerolithen; aber in jenen, die zum Steinschnitte und künstlerischen Gebrauche nicht dienlich waren, hatte man gleichwohl Rücksicht auf die natürlichen Striche und Zeichen, so sich auf ihnen befanden, die man für heilige Zahlen und Zauberzeichen ausgab; woher auch die besondere Achtung entstanden sein mag, so man jenen unter dem Namen Echimten, Hysteriolithen bekannten Muschel und Versteinerungsarten bezeichnete. Was die Erklärung der auf ihnen befindlichen Zeichen betrift, bleiben dieselben nur so lange dunkel, als man nicht auf die Hauptquelle zurück geht, aus der sie entstehen, und worin der einzig wahre Schlüssel zu ihrer Entzifferung zu finden ist, nämlich das System der Emanation, vermöge welchem es ein ewig einziges Wesen gibt, das alle andere schuf, und regiert, aber nicht unmittelbar, sondern durch mehrere ihm untergeordnete Mittelwesen, die den verschiedenen Teilen der Welt vorstehen, und sie als Boten des ewig unerschaffenen leiten; organische und anorganische Schöpfung, Menschen, Pflanzen, Steine und Metalle stehen unter ihrer Gewalt. Diese leitende Mittelwesen aber, die den Gestirnen vorstehen, wurden bald mit ihnen verwechselt, und selbst göttlich verehrt. Sonne und Mond waren ihre Herrscher, ihnen untergeordnet eine Hierarchie von Planeten, Fixsternen, und das ganze Heer der leuchtenden Himmelsschaaren, denen man ihre Bahn, ihre Wohnungen, ihren Einfluss auf Veränderung der Jahreszeiten, Witterung, ja selbst den verschiedenen Konstellationen gemäß, auf das Lebensprinzip das Schicksal, Wohl und Weh aller Wesen zuschrieb, woher auch der Glaube des astralischen Einflusses der höheren auf die niedere Elementarwelt, und der daraus gleich anfänglich damit verbundene Gebrauch der Magie, Divination, und astrologischen Künste. Diesem Glauben an Einfluss der Gestirne und der ihnen vorstehenden Untergötter (eben derselbe, vor dem Moses das erwählte Volk warnte,\footnote{} waren am frühesten die Babylonier und Araber ergeben. Um die reinen Intelligenzen sich geneigt zu machen,\footnote{} verehrten sie die Planeten in ihrem Heiligtume, daher schnitzten und prägten sie dieselben in Bildnisse aus, und wiesen jeder Gegend, jeder Pflanze, jedem Steine ein ihm entsprechendes Gestirn an, teilten unter sie die Jahreszeiten, Monate, Wochen, Tage und Stunden, beobachteten ihren Lauf, ihre Behausung, ihren Standort, ihr Auf- und Niedergehen, ihre Annäherung und Gegensätze, ihre Phasen, Anschauungen, ihr Verschwinden, und was daraus erfolgte.\footnote{} Wollten sie hiernach sich z. B. den Saturn geneigt machen, und durch ihn etwas erwirken, so wählten sie hierzu die erste Stunde des ihm geweihten Samstags; und indem sie eigene mit diesem Planeten sympathisierende Gewänder umtaten, verrichteten sie dem geschnitzten Sinnbild des Gestirns ihre Gebetsformeln, mit vollem Glauben auf die Erfüllung ihrer Wünsche.\footnote{} Dasselbe hofften sie auch von denen eigends dazu geweihten mit dem oder jenem Gestirne in Sympathie stehenden Steinen, dem ein guter oder böser Daimon in wohnte, der jedoch erst durch die bei seinem Gebrauche ausgesprochenen Zauberformeln, womit die auf dem Steine geprägten Zeichen in Bezug stunden, wirksam wurde.

*) Von neuern vorzüglich Paracelsus, Albertus magnus, Lud. Dulcis, von Boot Hist. Gemmarum et Lapidum, nach ihrem Resultat dargestellt in Brückmanns Abhandl. von Edelsteinen.

*) Schon Theophrast von den Steinarten Kap. VI. erwähnt drei verschiedene den Alten bekannte Steinarbeiten, λιθοτομια, lapicidaria, Steinmetzkunst, τορευτικη Steinmetzkunst, und τλυφη Steinschneidekunst; die Lithotomen gruben Inschriften mit eisernen Griffeln in Marmor u. d. gl., Toreuten drehten Gefäße aus Marmor, Alabaster u. s. w., Skalptoren arbeiteten in alle Steinarten, die Eisen nicht angreift, mit diesen gruben sie in die vorhergespaltenen Steine entweder vertiefte Figuren (jetzt Intaglio, Incisura genannt), oder erhabene Figuren, caelatura (jetzt Kameen), oder sie gaben Edelsteinen eine beliebte Form, z. B. Oval. In Kap. 42. sagt er ferner: das Eisen schneidet auch in festere und härtere Körper, weil es mehr Zusammenbang hat. Vergl. damit Plinius Nat. Gesch. 37. 12.

*) Deuteron. 4. v. 16.

*) Pocock Specimen Hist. Arab. p. 139 seqq.

*) Ein arabisches Gedicht, welches Ebn-Khaldoun, ein Schriftsteller des 8ten christlichen Jahrhunderts in seinen historischen Prolegomenen anführt (S. Abd-Alatif Relation de l'Egypte traduit de Silvestre de Sacy pag. 512.) beschreibt folgender Maßen die dabei übliche Zauberformel:
"Toi qui desire apprendre le secret de faire absorber les eaux, écoute les paroles de vérité que t'enseigne un homme bien instruit: laisse là toutes les recettes mensongères et les doctrines trompeuses dont les autres ont rempli leurs livres, et prête l'oreille à mes discours et aux conseils que je te donne, si tu ès du nombre de ceux qui ue suivent point le mensonge. Lors donc que tu voudras faire absorber les eaux d'un puits qui inspire l'effroi à l'imagination embarrassée et incertaine sur les moyens d'executer une telle entreprise, tu auras recours au talisman suivant. Fais la figure d'un homme dont les deux mains tiennent la corde qui sert à tirer le seau du fond du puits. Sur sa poitrine, trace la figure de la lettre ha, comme tu la vois ici; trace la autant de fois, que le divorce peut avoir lieu, et non davantage; qu'il foule aux pieds les figures de la lettre ta, sans cependant les toucher tout-à-fait, imitant la marche d'un homme prudent, fin et adroit. Qu'une ligne entoure tout cela; la forme carrée vaut mieux que la forme circulaire. Immole un oiseau sur ce talisman, que tu frotteras avec le sang de cette victime, apres quoi tu procéderas aux fumigations de sandaraque, d'encens, de stacté et de costus. Ensuite tu le couvriras d'une étoffe de soie, rouge, jaune ou bleue, où il n'y ait ni couleur verte, ni taches. Tu le lieras de deux brins de laine blanche ou rouge, d'un rouge pur. Il faut que cela se fasse quand le signe du lion monte sur l'horizon, ainsi qu'on l'a bien expliqué, dans le temps que la lune de ce mois n'éclaire point; la lune doit être jointe à la Fortune de Mercure, un jour de samedi, à l'heure où tu feras cette opération."'
Was vom Monde erwähnt wird, soll sa viel heißen, dass derselbe in demselben Zeichen mit dem Mercur sich befinden, und dieser in einer günstigen Glückbringenden Station sein müsse; denn die unmittelbaren Einflüsse der Planeten sind nach Bewandtnis ihres Standortes, und der Aspekten gegen andere Planeten großen Änderungen unterworfen. "`Sunt -- sagt Albacit -- ad magisterium judiciorum astrorum isagoge Paris 1521. planetis loca in quibus confortantur, et loca in quibus fiunt fortunae, in quibus fiunt malae."' So deutet die Vereinigung des Monds und Mercurs in derselben Behausung, nach astrologischen Gesetzen, auf glückliche erwünschte Zukunst, und der Haupteinfluß des Mercurs geht nach Ptolomaeus Opus quadri part. Buch I. Kap. 4. 5. auf Dürre und Austrocknung.

*) Kircheri Oedipus, Artikel magia hieroglyphica aeptiorum. T. II. p. .

Gleichfrühzeitig war dieser Astraldienst in Persien einheimisch; unter den geistigen Mittelwesen aber, die gemäß des Zerduschtischen Licht-Systems die Gebote des Allerhöchsten Unerschaffenen Zeruane Akherene in der Schöpfung verrichten, war nach den Amschaspands (den reinsten Intelligenzen) keines in größerem Ansehen, als Mythra.\footnote{} Der erste vornehmste der Ized (Genien des Himmols, und Personifikationen der guten Schöpfung). Mythra der Erhalter und Begleiter aller geschaffenen Wesen, Geber des Lichts, der Wärme, des de fruchtenden Regens und aller Lebenskräfte; dem bösen Einfluss der Dews und Daroudis entgegen gesetzt; der Beschützer alles Reinen, nicht zwar selbst die Sonne (d. h. jene des höheren Himmels) sondern ein Mittler zwischen den zwei Urelementen Feuer und Wasser, oder in elementarischer Beziehung zwischen Sonne unb Mond, dem männlichen und weiblichen Schöpfungs-Prinzip\footnote{}; daher er ursprünglich wie alle Gottheiten hermaphroditisch, spüter aber mit getrennten Geschlechtern, männlich und weiblich vorgestellt ward, woher auch die unterscheidende Benennung von Mythras und Mythra, späterhin μιθρισ Zeus und Astarte, Pater magnus und Dea magna, μιθρα die himmlische Venus genannt, denn Herodot\footnote{} sagt ausdrücklich: die Perser opferten nebst dem Jupiter auf hohen Gebirgen, der uranischen Venus, die sie Müthra, die Assyrer Myllita, die Araber Alitta nennen.

*) Dessen Cultus zwar in Persien eiheimisch ist, der aber, wie wir aus Plutarch in Pompejo und Firmicus de Errore profan. relig. C. 5. lernen, späterhin mit andern Modifikationen sich in Phrygien wieder erneuerte, und von Rom aus, wo er vorzüglich im Jahr 687 herrschend war, sich im ganzen Occident verbreitete, bis er im Jahr 378 nach Christo gänzlich vertilgt wurde.

Auch zu diesem Mythos gab wahrscheinlich ein gefallener Aerolithe Anlass, denn, eine alte Sage, wie wir aus den Kirchenvätern lernen, erzählt,\footnote{} Mythras sei von einem Stein geboren worden: θεοσ εκ πετρασ und setzt die Ursache hinzu; weil man aus Steinen Feuer schlage; ein nicht unwichtiger Umstand, der trefflich den geheimen Sinn aufschließt, welchen der Mythos aller Stein-Gottheiten hatte, nämlich: dass das Element des Feuers als Symbol des Lebens im Stein, wie in jedem Geschaffenen verborgen sei. Die Abbildung eines solchen Mythras findet sich noch in der Justinianischen Sammlung, und zwar nach der ältesten Form, ein aus rohem Felsen hervorgehender Kopf, ihm zur Seite zwei junge Mythras seine Söhne, denn die Fabel lässt ihn aus der Verbindung mit einem andern Steine, zwei Kinder erzeugen, die man deswegen Diorphi hieß.*

*) S. Zend à Vesta T. 2. vendidad. p. 209 seq. nach der französischen Überziehung von Anquetil.

*) Herodot L. I. C. 131.

*) St. Justin Dialog. contra Tripho, p. 296. Julius Firmicus Error profan. relig. Cap. 5. Commodian Intr. 13. St. Hieronymus adv. Jovian L. I. T. 4. p. 2.

*) Diesen Mythos erzählt Plutarch in der Abhandl. von den Flüssen, Art. Araxes.

Aus dieser, wie aus allen ihr ähnlichen Mythen, geht das Resultat hervor: dass Bätylien und heilige Steine als Symbole höherer Kräfte, das ist der Elemente angesehen wurden, indem jedes Wesen des oberen Himmels (Amschaspands nach der persischen Lehre) deren es 7. gab, nämlich: Mensch, Tier, Feuer, Metalle, Erde, Wasser, Bäume, in der niederen Welt --- der Erde, eine ihm entsprechende Form hat, der sie sich freundschaftlich zuwendet, und sie gern bewohnt, um aus und durch sie zu wirken; daher auch Abraxas, Talisman, Amulette, ihre Beziehung auf diese höheren Kräfte haben, und gleichsam deren Orakel sind.

Ihre anfänglich rohe Bezeichnung, wie an jedem Aerolithen, wurde immer zusammengesetzter und gehäufter mit Hieroglyphen, je später man dieselbe von denen aus dem ursprünglichen Licht- und Natur-Kultus ausgegangenen Sekten angewandt findet; indessen zeigt selbst die Signatur der gnostischen Abraxas mit 385. ihre Beziehung auf Zelt- und Jahres-Wechsel. "`Basilides,"' sagt der heilige Hieronymus,\footnote{} "`gab Gott dem Allmächtigen den Nahmen Abraxas und behauptet: dass nach der Bedeutung der griechischen Buchstaben und der Tagszahl des Sonnenlaufs, Abraxas sich in seinem Kreise eingeschlossen befinde."' Diese Stelle wird, wie Macarius bemerkt, durch eine andere des heiligen Augustins\footnote{} erläutert, der von Basilides sagt: er behaupte, es gäbe 365 Götter, weshalb er die Abraxas für heilighalte, weil diese Tagszahl sich im Jahr befinde. Es sind nämlich die griechischen Buchstaben: α, β, ρ, α, ξ, α, σ, analog den Zahlen: 1. 2. 100. 1. 60. 1. 200. die zusammen verbunden die Zahl 365 bilden; daher der Nähme Abraxas gleichbedeutend mit Mithras, (beide die Sonne und ihren Umlauf symbolisierend) genommen ward. Aus welcher Quelle auch der Missbrauch entstand, den die Gnostiker mit diesen Nahmen trieben, die sie selbst mit dem göttlichen Lehrer Christus, als Bild der Sonne, vermischten, dasselbe gilt von der Benennung Οφισ (dem Symbol der Ophiten) und ιαω --- Sabao, Sabazos, Sabaoth, Herr der Heerschaaren, Adonai und anderen Benennungen des Allerhöchsten, zu denen als Untergattungen noch die Nahmen der vollziehenden Himmelsboten, Kräfte, Potenzen zu zählen sind, deren man in Montfaucon\footnote{} mehr als hundert gesammelt findet; ja die zartfühlenden Hindu, die alle geistigen Kräfte symbolisierten, und in der Natur von den Sternen bis zum Grashalm, alles belebten, zählen deren viele tausende.\footnote{} Unbekannt jedoch, wenigstens minder geachtet, blieben ihnen die Abraxas, deren wahre Heimat Chaldäa und Persien ist. Die oben erwähnte nahmen des höchsten Gottes und der Heerschaaren nun, verbunden mit den Zeichen der Gestirne, und der ihnen entsprechenden Konstellationen bildeten die auf diesen magischen Steinen geprägte Zauberformeln, mittelst welcher man in der Gestalt eines umgestürzten Kegels, den man aus den Buchstaben ΑΒΡΑΚΑΔΑΒΡΑ zusammensetzte, Beschwörungen vornahm, und dieselbe teils als Heilmittel bei Krankheiten, teils als Schutz und Rettungs-Werkzeuge wider böse Dämonen gebrauchte.*

*) In dessen Kommentar über Amos.

*) Montfaucon Ant. expl. Tom. II p. 356.

*) Ant. expl. loc. cit.

*) Aoditja, (sagt das Sanskritische Wörterbuch Amarasinha, herausgegeben von P. Paulino, Rom 17s. p. 4.) ist der allgemeine Nahmen dieser Devatas: Aoditja in plurali duodecim Deos sunt, qui praesunt anni, mensibus, ac proinde allegorice duodecim. Stationis puncta, in quibus sol versari videatur, I. P. Ildefonsus Mission. in Cod. ms. de sectis et Relig. Indorum. --- Item: in secundo Choro numerant (Indi) et adorant triginta et tres milliones Deorum, quos vocant Deos coelestes, inter quos numerant deum solem, deam Lunam, Deas planetas, et Deas Stellas; insuper in hocce Choro computant Elementa pro Diis; so sagt auch P. Marcus à Tumba, (von Fra. Bartholomäs im Amarasinha angeführt) gli elementi, li pianeti, li venti sono Die; und zwar nach demselben Amarasinha von männlicher (Pullinga) oder weiblicher (Sri Devata) Natur oder geschlechtlos Clibè. --- s. auch Försters Bemerkungen zur Sokontala p. 256.

Was die auf Talismanen vorkommenden Sinnbilder betrifft, gibt Montfaucon, der eine Menge derselben gesammelt hat, folgende Hauptgattungen an: den Engel mit 4 oder 6 Flügeln, den Mensch-Löwen --- die Schlange mit dem Löwenkopf, den Hahn, den Küfer, den Sphinx und Affen, alles Attribute, die bald so, bald anders modifiziert, sich auf die 7 Hauptkräfte der Schöpfung beziehen, und in den ältesten Religionen unter diesen Symbolen vorgestellt werden. Weitläufiger uns in die mannigfachen Gattungen der Talismane und in die Verschiedenheit ihrer Bezeichnungen einzulassen, wäre außer dem Zwecke dieser Abhandlung.

*) Diesen Gegenstand hat ausführlich erläutert Kreuzer in seinem trefflichen Werke: Symbolik und Mythologie der alten Völker. I. Band. S. 286. -- Im Ganzen können alle Lokal-Götter von Städten, Gegenden und Ländern zu den Schutzgottheiten gerechnet werden; die meisten Götter, die in Ägypten vor den Tempeln und Pyramiden standen, waren solche Talismane, wie dies auch aus dem Zeugnis der Arabischen Schriftsteller hervorgeht, die Langles in Nordens Reise angeführt.

Zur näheren Charakteristik der Bätylien wird es hingegen nicht undienlich sein, die Beschreibung vom Herabfalle eines der neuesten Aerolithen, und die Analyse seiner Bestandteile anzuführen, indem sie uns einigen Aufschluss ihres magischen Gebrauches im Altertum geben kann. Vor anderen erwähnen wir den im Jahr 1773 unfern Sigena in Arragonien herabgefallenen merkwürdigen Stein, von dem Proust, Professor in Madrid, der ihn untersuchte, folgendes sagt*:

Der Stein, 6 Pfund 10 Unzen schwer, war innerlich und äußerlich mit Pünktchen von Rost durchsäet, die höchst wahrscheinlich daher rühren, dass man ihn ins Wasser gelegt, um zu sehen, ob er sich darin verändern werde; er ist unregelmäßig eiförmig, hat so zu sagen nur zwei Seiten, davon die eine abgeplattet, an den Rändern etwas abgestumpft, und in der Mitte etwas eingedrückt; die andere ist eine dreiseitig stumpfe Pyramide von ungleichen Seiten, deren Spitzen und Kanten ebenfalls stark abgerundet sind. Auch ihn umgab eine schwarze glasige Rinde, so dass beim ersten Anblick man ihn mit Pech überzogen glaubte, allein die Zerbrechlichkeit dieser Rinde, die Stöße, welche der Stein ausgehalten hat, die vielen Hände, durch die er gegangen ist, haben den größten Teil desselben dieser Rinde beraubt, so dass sie sich jetzt nur in den Vertiefungen und auf den Seitenflächen der Pyramide zeigt. Die Grundfarbe des Steins, wie aller andern Meteorsteinen, ist ein einförmiges bläuliches Grau, die Farbe eines schwarzen Körpers, welchen ein weißer erhellt, oder vielmehr eine Verbindung von Erden, welche durch Eisen im Minimo der Oxidierung gefärbt ist. --- Die Rinde dieser Steine ist übrigens zufällig, eine fremdartige Ursache hat offenbar ihre Oberstäche verändert, gerade so wie in einem Kalkofen ein Stück Sandstein oder Granit sich mit einer glasigen Kruste umgeben würde. Diese Ursache hat auf den Stein nur eine momentane Wirkung äußern können, wie daraus gewiss ist, dass wenn sie Zeit gehabt hätte, ihre Wirkung über die Kruste hinaus fortzupflanzen, sie ein Aggregat von so schmelzbarer Art als diese Steine, notwendig ganz verglast haben müsste.

*) S. dessen Abhandlungen in Gilberts Annalen, 24. Band, S. 261. verglichen mit der Analyse des Aerolithen, der im Jahre 1806. im ehemaligen Languedoc herabfiel; ebendaselbst p. 189.

Bey genauer Betrachtung dieser Rinde findet sich daher, dass sie die Wirkung eines Feuers gewesen sein muss, welches mit dem Ursprunge des Steins nichts zu tun hat; und die Hitze, die seine äußere Verglasung hervorbrachte, scheint groß genug gewesen zu sein, um seine Oberfläche zu schmelzen, aber nicht lange genug gedauert zu haben, um in das Innere einzudringen. Wenn auch nicht alle Steine dieser Gattung, wie viele Physiker behaupten, glühend auf die Erde fallen, so kommen die meisten doch brennend, d. h. so warm herab, dass sie die Hand verletzen. Der Stein ist übrigens porös wie Sand, der durch kein Zement verbunden ist; mit der größten Leichtigkeit kann man durch ein Stückchen hindurch blasen, wenn man es zwischen den Zähnen hält. Am Stahl schlägt es kein Feuer. Seine Hauptbestandteile sind auf 103 Teile:

Schwefel-Eisen in Minimo zu ... 12 Teilen
Schwarzes Eisen-Oxid ... 5 -
Kieselerde ... 66 -
Magnesia ... 20 -
Manganes und Kalkerde in geringer Menge ... -
103 -

Das darin in ziemlicher Menge befindliche regulinische Eisen ist nur hineingemengt wie die gediegenen Metalle ihrer Gangart ).

*) Bei dem im Jahr 1806. im Dep. du Gad gefallenen Meteorstein fand sich das Verhältnis des Eisens als schwarzes Oxid zu den übrigen Bestandheilen, wie 40 : 100. Siehe Gilberts Annalen 24. Th. S. 202.

Aus dieser Beschreibung, die im Ganzen mit allen anderen über ältere und neue Aerolithen gemachten Beobachtungen übereinstimmt, ergibt sich:

1. dass sie insgesamt mehr oder weniger eisenhaltig sind, und

2. dieser Eigenschaft gemäß, auch mehr oder minder auf die Magnet-Nadel wirken; --- dass

3. ihre Form, besonders die der kleineren Steine, meist sphärisch, teils ganz rund, teils oval ist.

4. Größere Steine hingegen trifte man oft viereckigt, pyramidalisch mit runder Basis, teils polygon, teils ganz unregelmäßig, und ebenso verschieden an Gewichte an.

5. Die äußere Rinde ist zufällig, nur wenige Linien dicht, und für eine leicht abgehende Kruste, die bloß als eine Verglasung gelten kaum, anzusehen.

6. Ihrer inneren Natur nach sind sie alle von hellgraulicher, mehr oder minder ins weiße oder bläuliche schießende Farbe; und

7. das Gewebe, die Verkittung ihres inneren Korns ist so locker, dass sie dadurch eine Art Durchsichtigkeit erlangen, die völlig mit der Beschreibung übereintrifft, welche Plinius von den Astroiden und Sideriten gibt.

8. Davon sind jene Steine doch ausgenommen, deren Masse fast ganz aus Eisen besteht (wie jene in Sibirien) und dadurch die Undurchsichtigkeit und Schwere dieses Metalls erhalten.

Von diesen verschieden, nämlich mit einer geringen Beimischung von Eisen und anderen Mineralteilen, meist aus Tonerde bestehend, sind jene im Jahr 1808. zu Stanneren in Mähren niedergefallene Meteor-Steine, welche die Herrn Scherer und Schreiber in Wien im 1sten Stück der Gilbertischen Annalen der Physik fürs Jahr 1809, beschrieben haben. Ihre über diese Steinmassen gemachten Bemerkungen verdienen umso mehr beachtet zu werden, als sie über den Ursprung und die Natur der Aerolithen ganz neue Aufschlüsse gewähren. Mehrere Physiker sind der Meinung, dass diese Körper beim Eintritt in unsere Atmosphäre in einen glühenden Zustand geraten, und durch ihre Reibung gegen die Luft, darin unterhalten werden; andere glauben, sie kämen durch den freien Wärmestoff, der durch das Zusammenpressen aus der Luft ausgeschieden wird, in Fusion (welche letzte Meinung einige Erfahrung mehr, als jene erste bloß hypothetische hat), Scherer hingegen macht wahrscheinlich, dass diese Massen weder in einem glühenden, noch einem weichen Zustande teichiger Schmelzung herunterfallen, ihre Inkrustierung hingegen nicht während dem Falle des Aerolithen durch die Atmosphäre allmählig, sondern in einem blitzschnellen Momente durch eins elektrische Potenz (obschon nicht mit gleicher Intensität, und allseitig auf ihre Bruchseiten wirkend) erzeugt werde. Gleichen Ursprungs sind nach Scherer die Figuren auf der Rinde, und es findet in dem Akt der Inkrustierung sämtlicher Meteorsteine ein gewisses Maas von Abstufung der Potenz statt, die auf die Steine gewirkt hat. In Hinsicht der äußeren Rinde, haben diese mährischen Aerolithen am meisten Ähnlichkeit mit denen von Siena und Benares, wenig hingegen mit jenen von Eichstädt, Aigle und Tabor. Ihre Bestandteile bestehen nach Vauquelins Analyse aus:

Kieselerde ... 50 Teilen
Kalkerde ... 12 -
Tonerde ... 9 -
Eisen-Oxid ... 29 -
Manganes-Oxid ... 1 -
Nickel-Oxid ... 01 -
Schwefel ... ein Atom
101

Sie enthalten (was die übrigen, bisher untersuchten, doch alle haben) weder Magnesia noch Chromium, sie sind leicht an Gewicht, zerreibbar und wirken nicht auf den Magnet. Obschon alle Meteorsteine darin übereinstimmen, dass sie sich fast immer ovalförmig oder prismatisch (vorzüglich gern vierseitig) finden, so nehmen beide Naturkundiger gleichwohl als Ursache der verschiedenen Form dieser Massen an, dass sie zersprungene und schnell auseinandergerissene Teile eines größeren Meteors aus der höheren Luftregion sind.

An Farbe find diese mährischen Aerolithen äußerlich schwarz, zuweilen ins dunkelbraune ziehend, innerlich aschgrau, wohl auch bläulicht, man sieht darin dichtere dunklere Körper, auch enthalten sie Schwefelkieskörner, doch wenige. Vom Basalt unterscheiden sie sich wesentlich durch den Bruch, die Härte und den Strich. Sie fühlen sich sanft an, ritzen Glas nicht, und geben am Stahl keinen Funken, vor dem Lötrohre schmelzen sie zu einem dunklen Glase, welches der Magnet anziehet.

Wenn mit diesen Erfahrungen wir nun Patrins Hypothese verbinden, dass Luftsteine durch gleiche Ursachen, wie die Laven entstehen, nämlich durch die feinen gasartigen Flüssigkeiten, die von der Atmosphäre in das innere der Erde, und von dieser in die Atmosphäre zirkulieren; dass diese Flüssigkeiten sowohl die Wirkungs-Mittel, als die Elemente zur Erzeugung der mineralischen Körper, der Materie der Meteoren u. s. f. sind, welche durch Verbindung jener Flüssigkeiten mit einander nach den Gesetzen der Assimilation gebildet werden, so öffnet sich hierdurch ein neuer lichtvoller Weg zur Erklärung des Phänomens und des Ursprungs dieser Massen, die auf Meteorologie überhaupt vom wichtigsten Einfluss sein dürften.

Den Ursprung dieser Steine betreffend, haben wir gleich anfangs die verschiedenen Meinungen der Physiker über ihre Entstehung gezeigt, worunter einer der neuesten, Proust, sie aus den unermesslichen noch unbekannten Polar-Gegenden herleitet, woraus sie, seiner Meinung nach, durch irgendeine mächtig wirkende Ursache losgerissen, sich in die höhere Atmosphäre aufschwingen, und in die südlicheren Gegenden niederfallen.

Je mehr man aber die innere Natur dieser Steine untersucht, und je heller von der anderen Seite unsere Einsicht in die ersten einfachen Prinzipen des kosmischen Lebens unserer Erde, und der Weltkörper überhaupt wird, je klarer scheint es: dass diese Steine weder von tellurischen Vulkanen, noch von Vulkanen des Mondes oder eines Himmelskörpers auf unsere Erde geschleudert werden, sondern: von Zeit zu Zeit eintretende starre Ausscheidungen aus dem Luftozean des Himmels sind; denn wenn nach Davy und den vorzüglichsten neuern Chemikern, sich dartun lässt, dass es nur eine einzige wägbare Materie als Repräsentant der Schwere, und Substrat der Schwerkraft in der Natur gibt, so wie gegenseitig zweierlei Licht, davon eines: das freie, merkbare, ungefesselte, unverschlungene; das andere hingegen, das verschlungene mit wägbarem Stoff verbundene, gefesselte, lässt sich aus derselben Ursache annehmen, dass diese ponderable Materie der niederen und höheren Lust vermittelst besonders verstärkter elektrischer Wechselwirkung, die von Zeit zu Zeit zwischen zwei Weltkörpern eintritt, nicht nur zu Wasser, sondern auch zu Stein und zu Metall werden. Zwar lässt sich auch sagen, dass das beiderlei Licht (reines und verlarvtes oder gemeine und magnetische Elektrizität) die wägbaren Grundstoffe von der Erde, der Sonne, dem Monde oder anderen Weltkörpern aus der einen jedem derselben umgebenden Atmosphäre hinweg in den großen Luftozean entführen könne, von woher solche durch eintretende elektrische Wirkungen aus ihrem Luftförmigen Zustande wieder zurück in der Gestalt fester Körper gebracht werden könnten; nicht leicht aber begreift sich dabei, wie diese verflüchtigten Teile dieselbe bleiben sollen, die sie vorher waren, und bei ihrer Ausscheidung wieder als dieselben erscheinen sollten ?\footnote{} Diese Luftsteine haben anfänglich ein inneres elektrisches Leben, und vermögen sich, dem gemäß, so lange schwebend zu erhalten, als sie von neutralelektrischem Ether umgeben sind, indem sie eine polare elektrische Spannung hervorbringen, die der Gravitation entgegen wirkt; sobald sie in den Wirkungskreis eines anderen größeren und stärkeren elektrisierten Himmelskörpers geraten, so wirkt ihr Gravitationsdruck allein. Als leuchtende Kugeln, teils in der höheren Luft, teils beim niederfallen zerplatzend, stürzen sie dem Himmelskörper zu, in dessen elektrische kugelförmige Atmosphäre sie geraten sind.

*) Mehreres hierüber in Dr. Haberles meteorologischem Jahrbuche fürs Jahr 1810. Weimar 1816. ein Buch voll neuer trefflicher Ansichten.

Wenn nun, wie früher gezeigt worden,\footnote{} diese Himmelssteine als herabgefallene Sterne, und die Gestirne selbst als göttliche Wesen verehrt wurden, deren Einfluss alles Irrdische unterworfen ist, so ersieht man hieraus: dass, wie Görres trefflich sagt, die Urzeit keine andere Geschichte hinter sich habe, als Naturgeschichte, und auch die Mythe in ihr ruhe: denn unter dem Bild der mannigfachen Genien, Intelligenzen, Mittelwesen u. s. f. verstand die erste älteste Schriftsprache die Symbolik nichts anders, als die alles hervorbringenden allwaltenden Naturkräfte; rings von ihnen umgeben, unter ihrer Macht stehend, und auf Erden wie am Himmel, ihre Wirkungen an sich sowohl, als an allen Gegenständen außer sich bemerkend; sagte dem Menschen in den frühen Tagen seines Erdenlebens schon Ahnung und inneres Gefühl, was später heilige Priesterlehre und wissenschaftliche Forschung ihm offenbarten; dass der Standpunkt, worauf er lebt, die Erde im innigsten Zusammenhang, in steter Wechselwirkung mit den außer ihr bemerkten Weltkörpern stehe, und die mannigfachen Naturerscheinungen, die Meteore Wirkungen höherer die niedere Welt beherrschender Kräfte seien, die er aus Furcht und dem Gefühl eigener Schwäche als Götter verehrt. Aus dieser doppelten Ansicht höherer die niedere Welt beherrschender Kräfte, und der Erscheinungen, die auf dieser Erde sowohl, als in der sie umgebenden Atmosphäre durch sie bewirkt werden, geht auch ein doppelter Ursprung des Polytheisms hervor. Jene Verehrung nämlich, die ihren Blick himmelwärts schwingend, die Gestirne und das Feuer als allgemeines Symbol ihrer allbelebenden Lichtnatur verehrt (der Sabäism) dann die niedere, die mit der Erde sich nur befassend, alle mannigfachen, durch unsichtbare Kräfte entstehenden Erscheinungen, in Bildern versinnlicht, (der grobe Fetischismus) der selbst im Holz, im rohen Steine, das übersinnliche Wesen, den Geist, den wundervollen Dämon, dem er diese Wirkungen zuschreibt, verehrt und anfleht.

*) S. 31. u. f. dieser Schrift.

Kehren wir auf die Erzählung und ausführliche Beschreibung, welche Damascius\footnote{} vom Bätylus des Eusebs, und den Zauberkünsten gibt, die derselbe mit diesem Orakel-Steine trieb, so erklären sich, wie mich dünkt, diese Kunststücke durch die innere Beschaffenheit des Steins und dessen Manipulation aus ganz natürlichen Ursachen, nämlich: durch seinen Eisengehalt, und den Magnet; es ist nämlich nicht bloß wahrscheinlich, sondern durch Plinius und anderer ausdrückliche Zeugnisse, wie wir früher gesehen haben, erwiesen, dass die Alten durch Anwendung des Magnets, viele an Wunder grenzende Erscheinungen hervorbrachten.

*) Siehe oben.

Erklärlich wäre nun leicht hierdurch, dass jener Bätylos, der nicht gern in des Arzt Eusebs Hand blieb, und dessen er weniger Herr war, als andere, die gleichfalls Bätylien besaßen, wenn man annimmt, dass diese letzteren von der magnetischen Art waren, und jenen durch ihre Kraft an sich zogen. Denn derselbe Naturkundiger sagt:

Eisen wird vom Magnet angezogen, und hängt gleichsam in einer Umarmung mit ihm selhst.*

Wenn es ferner in der Beschreibung heißt: dass ehe der Stein zum Sprechen kam, er lange in den Händen umher geworfen und bearbeitet werden musste, (ohne dass man ihn fallen ließ) so darf man nur der Kunststücke sich entsinnen, die man fertige Taschenspieler mittelst des Magnets hervorbringen sieht, um in diesen Gaukeleien, die das wundersüchtige Volk jenes Zeitalters, als Wirkung einer dämonischen Kraft anstaunte, für nichts anders, als eine ganz physische Wirkung (des Magnets nämlich) anzusehen; oder, welches ebenso möglich wäre, durch Galvanism, der, (wenn auch dem Nahmen, doch wahrscheinlich der Sache und Wirkungen nach) den Alten nicht ganz unbekannt blieb. Von Damascius wird fernes erwähnt, dass, wenn Eusebius sein Orakel befragen wollte, er es in eine Wand befestigte, und dann eine Antwort von demselben erhielt, die dem Zischen, oder weinerlichen Schrey eines Kindes ähnlich war.\footnote{} Diese Befestigung in der Mauer scheint weniger (wie Dr. Münter meint\footnote{} daher zu rühren, dass Eusebius die Kunst, den Stein in der Hand zu bearbeiten, minder gut verstanden habe, wie andere Gaukler, sondern daraus sich erklären zu lassen, dass diese Steine als Talismane angesehen würden, die man deswegen gern an die Wände der Tempel öffentlicher und Privatgebäude befestigte, um dieselben vor schädlichen Geistern zu bewahren, oder jene, die sich ihnen näherten, daraus zu vertreiben, welcher Gebrauch (führte es uns hier nicht zu weit) aus häufigen Beispielen der chaldäisch-persischen, ägyptisch-phönizischen Astrologie und Dämonologie erwiesen werden könnte.\footnote{} Der Laut, den der Bätylos bei Mitteilung des Orakelspruches (der Sage nach) vernehmen ließe, erklärt sich am nächsten wohl durch die Täuschung der Zuschauer, verbunden mit jenen Betrugsmitteln, deren sich die Beschwörer überhaupt bei den Orakeln bedienten, worunter jenes des Bauchredens, welches Eusebius vielleicht in seiner Gewalt hatte, das nächste und natürlichste scheint.

*) Buch 36. Cap. 25.

*) Falconnet, in der Abhandlung von den Bätylien. Mém. de L'ac. des Inscr. Tome VI. p. 526. führt hierüber eine merkwürdige Stelle an, aus dem seltenen bisher noch Manuskript gebliebenen Buche: Hypopnesticum des Josephus (nicht des berühmten Geschichtsschreibers, sondern eines Christen des 5ten Jahrhunderts). Nachdem die Rede von verschiedenen Bezauberungen war, fügt Josephus hinzu: "`Tempel-Bätylien, eine Art Divination, die mittelst gewisser in den Mauern befestigten Steinen gescheibt, welche Orakel aussprechen. Der Text heißt: τα εν τοῖσ βαιτυλια δια λυθων εν τοῖσ στοιχεοῖσ προσραοστον θων, das Falconnet durch veränderte Leseart also verbessert: δῖα λιθων εν τοισ τοιχοισ προσ χρησαντων wollte man εν τοισ στοιχεοισ beibehalten, so wäre es auszulegen, dass diese Orakel durch die Kraft der auf der Oberfläche der Steine eingeprägten Schrift und Figuren wirksam würden.

*) S. dessen Abhandlung: Vergleichung der Bätylien der Alten mit den Steinen, welche in neuern Zeiten vom Himmel gefallen sind. In Gilberts Annalen der Physik, Th. 21.

*) Nur eines sei hier erwähnt aus El-Makryzy Beschreibung von Ägypten, die Langles in seinen Bemerkungen zu Nordens Reise nach Nubien und Ägypten, Th. III. S. 304. anführt: "`Als der Sultan Al Mahmouhn die Pyramide von Diyze öffnen und untersuchen ließ, kam man nach langem Forschen auf einen Saal mit drei Thüren, am Eingange von einer derselben waren drei Säulen befindlich, von innen ausgehöhlt und in dieser Höhlung befand sich das Bild eines Vogels. Die erste dieser Säulen enthielt eine Taube von einer grünen Steinart; die zweite einen Falken von gelbem Steine; die dritte endlich einen Hahn vom Steine Kedan, einer Art Hämatit. Es waren, sagt El-Makryzy, Talismane, die bestimmt waren, den Eingang der Thüren zu wahren, und böse Geister davon zu verscheuchen; zu demselben Gebrauch dienten wahrscheinlich auch jene mit Keilschrift bezeichneten babylonischen Backsteine, davon vor einigen Jahren mehrere nach London kamen (s. Dr. Hagers Abhandlung über die vor kurzem entdeckten babylonischen Innschriften im Asiatischen Magazin No. III. IV. VI.) -- Entziffert ist bisher zwar keiner dieser Steine geworden; und wenn wir Plinius Äußerung folgten (Nat. Gesch. Buch VII. Kap. 77.) so hätten, nach Epigenes Zeugnis, die Babylonier astronomische Beobachtungen von 520 Jahren auf Backsteine verzeichnet, welches zu leugnen wir keineswegs berechtigt sind; allein jenen auf den ausgegrabenen Steinen befindlichen Zeichen nach zu urteilen (man sehe dieselben in obengemeldter Schrift nach) sind sie eher für magische zu Talismanen dienende Zeichen zu halten; denn wären es astronomische Tafeln, so würde man sie wahrscheinlich nicht tief in Mauern vergraben haben.

Durch die in dieser Abhandlung angeführte Beispiele ergibt sich der Ursprung sowohl der religiösen Verehrung dieser Steine, als die lange Dauer ihres Gebrauches, und indem uns das durch Eusebius\footnote{} überlieferte Fragment des Sanchuniathon, worin es heißt: der Gott Coelus habe diese Steine erfunden, den wahren Sinn aufklärt, den die Alten dieser Mythe beilegten, nämlich Bätylien als dem Himmel entfallene belebte Steine anzusehen, zeigt uns die merkwürdige Stelle des Damascius und andere sie bestätigende in Priscian, Hesichius und dem Etymologicon, dass diese Steine als feurige Kugeln, stets von einem Meteor begleitet, herabfielen. So mächtig ist der Hang zum Wunderbaren, dass der Glaube, den man an diese Steine hatte, vermöge welchem man ihnen die größten Wirkungen als Schutzgötter, Amulette, Talismane, Zaubersteine zuschrieb, von den Zeiten des trojanischen Krieges (wie die Stelle aus dem orphischen Gedicht zeigt) und wahrscheinlich früher im südlichen Asien, Indien, Persien, besonders Chaldäa, dem eigentlichen Vaterlande der dämonologischen Theurgie, bis ins sechste christliche Jahrhundert, wo die größeren Orakel des Heidentums bereits schwiegen, allerwärts erhielt, ja man kann sagen, nie ganz erlosch, selbst in neuerer Zeit nicht, wo der Volksglaube an die Wunder der Donnersteine noch stets lebend ist, so bei wilden Völkern, und jenen, wo das Christentum die Blendwerke des Aberglaubens nicht gänzlich geläutert hat, oder wo Lokalität den Glauben an Visionen, Geister, höhere Zaubermächte, geheime --- den Steinen und Pflanzen -- in wohnende Wunderkräfte befördert, wie bei allen Bergvölkern oder Bewohnern neblichter Thäler.

*) In Photii Bibl.

Der Glaube an das Wunderbare, und der Hang, für unsere Bedürfnisse, unsere Leiden Zuflucht und Hülfe bei überirdischen Kräften zu suchen, ist im menschlichen Gemühte ebenso unauslöschlich, als die Neigung, jedes Ereignis höheren Mächten zuzueignen, und dem Himmel entsteigen zu lassen.
\clearpage
\section*{Verzeichnis der Abbildungen.}
\paragraph{}
\begin{enumerate}
    \item[Fig. 1] Titelkupfer. Rhea, die ihrem Gemahle dem Saturn den in Windeln gelegten Stein, statt des verfolgten Jupiters, zu verschlingen gibt. --- Nach einer antiken Ara aus dem Mus. capitol.
    \item[Fig. 2] Eine Cyprische Münze mit einem Konischen Idol oder dem Steingotte.
    \item[Fig. 3] Emesische Münze auf den Dienst des Helagabolus oder der Sonne deutend.
    \item[Fig. 4] Emesische Münze auf den Dienst des Helagabolus oder der Sonne deutend.
    \item[Fig. 5] Münzen des ΖΕΥΣ ΚΕΡΑΥNIOΣ aus Spanheim Dissert. de Numm. ant. Usu.
    \item[Fig. 6] Münzen des ΖΕΥΣ ΚΕΡΑΥNIOΣ aus Spanheim Dissert. de Numm. ant. Usu.
    \item[Fig. 7] Münzen des ΖΕΥΣ ΚΕΡΑΥNIOΣ aus Spanheim Dissert. de Numm. ant. Usu.
\end{enumerate}
\clearpage
\pagestyle{fancy}
\fancyhf{}
\rhead{}
\cfoot{\thepage}

\begin{figure}[ht]

\begin{subfigure}{0.5\textwidth}
\includegraphics[width=0.8\textwidth,height=0.8\textheight,keepaspectratio]{Fig1.png} 1
\end{subfigure}
\begin{subfigure}{0.5\textwidth}
\includegraphics[width=0.8\textwidth,height=0.8\textheight,keepaspectratio]{Fig2.png} 2
\end{subfigure}
\begin{subfigure}{0.5\textwidth}
\includegraphics[width=0.8\textwidth,height=0.8\textheight,keepaspectratio]{Fig3.png} 3
\end{subfigure}
\begin{subfigure}{0.5\textwidth}
\includegraphics[width=0.8\textwidth,height=0.8\textheight,keepaspectratio]{Fig4.png} 4
\end{subfigure}
\begin{subfigure}{0.5\textwidth}
\includegraphics[width=0.8\textwidth,height=0.8\textheight,keepaspectratio]{Fig5.png} 5
\end{subfigure}
\begin{subfigure}{0.5\textwidth}
\includegraphics[width=0.8\textwidth,height=0.8\textheight,keepaspectratio]{Fig6.png} 6
\end{subfigure}
\begin{subfigure}{0.5\textwidth}
\includegraphics[width=0.8\textwidth,height=0.8\textheight,keepaspectratio]{Fig7.png} 7
\end{subfigure}

\end{figure}
\clearpage
\end{document}
