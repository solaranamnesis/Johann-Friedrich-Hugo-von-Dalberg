\documentclass[a4paper, 11pt, oneside, polutonikogreek, german]{article}
\usepackage[sfdefault]{biolinum}
% Load encoding definitions (after font package)
\usepackage[LGR,T1]{fontenc}
\usepackage{textalpha}
\usepackage{graphicx}
\usepackage{float}
\graphicspath{ {./} }
\usepackage[figurename=]{caption}
\usepackage{listings}
\usepackage{subcaption}
\lstset{basicstyle=\ttfamily}

% Babel package:
\usepackage{babel}
\usepackage{cjhebrew}
% With XeTeX/LuaTeX, load fontspec after babel to use Unicode
% fonts for Latin script and LGR for Greek:
\ifdefined\luatexversion \usepackage{fontspec}\fi
\ifdefined\XeTeXrevision \usepackage{fontspec}\fi

% "`Lipsiakos" italic font `cbleipzig`:
\newcommand*{\lishape}{\fontencoding{LGR}\fontfamily{cmr}%
		       \fontshape{li}\selectfont}
\DeclareTextFontCommand{\textli}{\lishape}

\usepackage{booktabs}
\setlength{\emergencystretch}{15pt}
\usepackage{fancyhdr}
\usepackage{microtype}
\begin{document}
\begin{titlepage} % Suppresses headers and footers on the title page
	\centering % Centre everything on the title page
	%\scshape % Use small caps for all text on the title page

	%------------------------------------------------
	%	Title
	%------------------------------------------------
	
	\rule{\textwidth}{1.6pt}\vspace*{-\baselineskip}\vspace*{2pt} % Thick horizontal rule
	\rule{\textwidth}{0.4pt} % Thin horizontal rule
	
	{\scshape\LARGE "Uber Meteor-Cultus der Alten,\\[1.25pt] vorz"uglich in Bezug auf Steine,\\[1.25pt] die vom Himmel gefallen.\\[1.25pt]}
	
	\rule{\textwidth}{0.4pt}\vspace*{-\baselineskip}\vspace{3.2pt} % Thin horizontal rule
	\rule{\textwidth}{1.6pt} % Thick horizontal rule

	%------------------------------------------------
	%	Subtitle
	%------------------------------------------------
	
	{\scshape Ein Beitrag zur Altertumskunde von Fr. v. Dalberg.} % Subtitle or further description
	
    {\scshape\scriptsize Mit einer Kupfertafel.} % Subtitle or further description
    
    %------------------------------------------------
	%	Cover photo
	%------------------------------------------------
	
	\includegraphics[width=\textwidth,height=\textheight,keepaspectratio]{cover.png}
	
	%------------------------------------------------
	%	Editor(s)
	%------------------------------------------------
    \vspace*{\fill}

	\vspace{1\baselineskip}

	{\small\scshape Heidelberg, bei Mohr und Zimmer.}
	
	{\small\scshape{1811.}}
	
	\vspace{0.5\baselineskip} % Whitespace after the title block

    \scshape Internet Archive Online Edition  % Publication year
	
	{\scshape\small Namensnennung Nicht-kommerziell Weitergabe unter gleichen Bedingungen 4.0 International} % Publisher
\end{titlepage}
\setlength{\parskip}{1mm plus1mm minus1mm}
\clearpage
\vspace*{\fill}
\begin{center}
"`Die Urzeit hat keine andere Geschichte hinter sich, als Naturgeschichte, in ihr ruht sicher auch die Mythe."'
\end{center}
\begin{center}
G"orres Mythengeschichte der asiatischen Welt.
\end{center}
\vspace*{\fill}
\clearpage
\section*{Inhaltsverzeichnis.}
\begin{enumerate}
    \item[] Inhalt.
    \item[] Einleitung.
    \item Ursprung des Elementen- und Meteor-Dienstes.
    \begin{enumerate}
        \item "Uberhaupt.
        \item Insbesondere.
    \end{enumerate}
    \begin{enumerate}
        \item Bei den Indiern,
        \item Tibetanern,
        \item Chinesen, Japanern und "ubrigen s"ud"ostlichen V"olkern,
        \item Im n"ordlichen Asien,
        \item Bei den Persern und Chald"aern,
        \item Arabern,
        \item "Agyptern,
        \item Ph"oniziern,
        \item Griechen.
    \end{enumerate}
    \item Himmel-Steine und deren Verehrung.
    \begin{enumerate}
        \item Ursprung des Steindienstes.
        \item Gebrauch heiliger Steine als Merkzeichen.
        \item Als Alt"are.
        \item Viereckige Steine, Hermen.
        \item Symbole der Volks-Einheit.
        \item Bild der Zeugung Phallus, Lingam.
        \item Rechensteine zu Bestimmung der Zeit.
        \item Vertr"age, B"undnisse, Eide an Steinen geheiligt.
    \end{enumerate}
    \item Stoff und Bestandteile der B"atylien oder heiligen Steine.
    \begin{enumerate}
        \item Viele derselben "achte Aerolithen.
        \item Analyse der Meteorsteine nach den neuesten Erfahrungen.
        \item Einzelne im Altertum bekannt gewordene Aerolithen und Steinregen.
        \item Verschiedene Nahmen, welche die Alten den heiligen Steinen gaben: Jakob- oder Gilead-Stein, Abadir, Alassovid, Pater magnus, magna Mater, der schwarze Stein der Kaaba, Alagabal oder Helagabalos. Dessen Tempel und Dienst als Sonnengott zu Emesa.
        \item Meteorische Luft-, Feuer- und Wasser-Erscheinungen; darauf deutende Mythen: Dioskuren, Pat"acken, Cabiren, die zw"olf gro"sen ph"onizischen G"otter.
        \item Feuer im Steine verschlossen.
        \item Fr"uhe Entdeckung, Feuer aus Steinen zu ziehen. Gebrauch des Feuers; dahin deutende Mythen.
        \item Ursprung der Steine und Metalle nach Hesiods Theogonie und dem Buche Hiob.
        \item Verglichen mit Theophrast, Plinius und andern alten Mineralogen.
        \item Mythische Sagen, die auf Natur und Ausbildung der Metalle und Gemmen Bezug haben.
        \item Von Talismanen, Abraxas, Amuletten, persischen Zaubersteinen; Erkl"arung ihrer Zeichen.
        \item Ihr Gebrauch, ihre vorz"ugliche Heimat, Handel mit denselben, allgemeine Verbreitung dieser Steine.
        \item R"uckblick auf die analoge Natur der alten B"atylien und der Meteor-Steine.
        \item Resultate.
    \end{enumerate}
\end{enumerate}
\clearpage
\paragraph{}
Wenn, wie ein trefflicher Schriftsteller\footnote{Herders Ideen zur Gesch. der Menschheit 1. Th.} sagt: die Philosophie der Geschichte unseres Geschlechts, um diesen Namen zu verdienen, vom Himmel anfangen muss, soll auch gegenseitig, ihren h"ochsten Zweck zu erreichen, sie von der Erde sich zum Himmel erheben. In die Atmosph"are, den Ursitz der Stoffe, der Geb"arerin aller Organisationen, sollen wir aufblicken; denn Luft, dieser allbelebende durchdringende Hauch, aus dem alles hervor, zu dem alles zur"uck geht, ist das umfassende Band, die letzte sch"one Verwandlung, zu welcher die Stoffe, von ihrer schweren Basis befreit, sich erheben, wie das Insekt, das seiner Raupenh"ulle entwunden, sich als entfesselte Psyche emporschwingt, zum "Ather hinauf, wo der anscheinende Tod selbst Stoffe und Fettige zu neuen Umwandlungen findet, wende der Mensch sein Auge als zu seiner Heimat, und die Atmosph"are, die alle Weltk"orper tr"agt und bewegt, wird ihm Aufschl"usse geben "uber so manche Ph"anomene, die er von der Erde allein, welche gleichfalls Leben und Erhaltung von ihr empf"angt, und wahrscheinlich selbst ein Erzeugnis der Atmosph"are ist, nie zu erhalten vermag. Denn die Meteore, woher nehmen sie ihren Ursprung anders, als aus dem uns umgebenden Luftkreis, diesem Ursitz der Elemente, worin elektrische und magnetische Str"ome, brennbare Lufts"aulen, erkaltete Salze, Lichtteile, und andere Bildungsstoffe enthalten sind.

Wenn unter den organischen Wesen nun der Mensch allein zu dieser h"oheren Ansicht geeignet ist, was besonders aus dem vollkommeneren Bau seines Hauptes, und seiner aufrechten Gestalt, nach dem Zeugnis der vorz"uglichsten Anatomen,\footnote{Monro, Kamper, S"ommering, auch Herder in der Geschichte der Menschheit, 1. S. 150 u. 170.} hervor geht, so wird er hierdurch vor allen Tieren auch geeignet, von der niederen Erde hinauf zu blicken in den Sitz alles Lebens, wo Sonne, Mond, das Heer der Gestirne, und die wechselnden Ph"anomene, die durch ihren Einfluss bewirkt werden, und auf die Erde zur"uck wirken, ihm bald die Ahnung einer geheimen h"oheren Ursache dieser Erscheinungen geben; geboren wird mit ihm diese Ahnung, daher unter allen Erdgesch"opfen der Mensch allein ein religi"oses Wesen ist; denn auch die anorganische Sch"opfung, der kalte tote Stein selbst ist dem Einfluss der Atmosph"are unterworfen, ihre belebenden sowohl, als zerst"orenden Wirkungen f"uhlt jedes Tier, aber nur der Mensch --- „nach Gottes Ebenbild geschaffen“ --- hat eine erh"ohte Vernunft, die ihn verm"ogend macht, die Ursache der Dinge zu erforschen, und ein Gem"uht, das, in sich selbst Gottes Ebenbild findend, ihn zur Anbetung jenes Wesens leitet, dessen Macht, Weisheit und G"ute alles schuf und erh"alt.

Die Natur des Unerschaffenen zu erkennen, muss dieser selbst sich dem geschaffenen Wesen durch Offenbarung mitteilen; denn, wenn gleich Himmel und Erde seine Gr"o"se verk"unden, so gew"ahrt gleichwohl die Erkenntnis, die der Mensch aus der Natur (dem Inbegriff alles Gebildeten) zieht, so sehr dessen Betrachtung seine Bewunderung erregen, und sein Herz zur Andacht erh"ohen mag, nur unvollkommene Erkenntnis des einen und ewigen Gottes.

Wenn der rohe Mensch, das Kind der Natur, in seinem hilflosen Zustande k"ampfend mit den Elementen, den m"achtigen Einfluss der waltenden Naturkr"afte f"uhlt, wenn alles um ihn in regem Leben und stetem Wechsel ist, so f"uhrt ihn die kindliche Fantasie von selbst dahin, diesen Erscheinungen ein inneres Leben zu leihen, die Ph"anomene der Natur werden ihm ebenso viele Lebenszeichen, und die wirkenden Ursachen derselben die Elemente, h"ohere "uber ihn gebietende Wesen.

Von dieser Erkenntnis zu ihrer Verehrung, ihrem Dienste, ist nur ein Schritt, aber der Cultus, den er ihnen weiht, ist so einfach und roh, als seine Begriffe. Wenn die Wasser sich empor heben, dass die Meerestiefe ersch"uttert wird,\footnote{Psalm 76.} Regeng"usse aus Wolken str"omen, Hagel die Felder verheeren, die oberen L"ufte donnern, und Blitze wie Pfeile umher fahren, wenn des Donners Stimme br"ullt im Wirbelwinde, und die Blitze das Land erleuchten, dass die V"ogel in den L"uften, und die Tiere des Waldes, und des Meeres Bewohner sich verbergen --- dann erbebt der schwache hilflose Mensch; im Kampf der Elemente sieht er Tod und Vernichtung, drohende Geister, die um Schonung anzuflehen, und (indem er ihnen seiner rohen Denkweise gem"a"s seine eigene Natur leiht) durch Gaben und Opfer zu vers"ohnen, sein erstes dringendstes Gesch"aft ist; erschreckt durch die furchtbaren Meteore, die den friedlichen Genuss des Erdenlebens st"oren, und die sch"onsten (seine Bewunderung anziehenden) Werke der Sch"opfung zernichten, f"uhlt er den Drang, die Wiederkehr dieser verheerenden Erscheinungen, wo m"oglich, zu vermeiden, oder wenigstens sich vor ihren Wirkungen zu sichern. V"ogel, die in fr"uhester Zeit schon als Deuter und Propheten der Zukunft galten, werden befragt, und ist seine Furcht auf die verheerenden Geister, welche die Elemente beherrschen, gerichtet, so bildet seine Fantasie ihm auch vers"ohnende --- das B"ose bek"ampfende Wesen, die, wenn er sie anfleht, ihn hilf leistend retten. In Tr"aumen besonders glaubt er sie zu vernehmen, in dem rauchenden Opfer der Tiere, oder aus D"ampfen und D"unsten der Erde; der tote Stein selbst, B"aume, Berge, Fl"usse, die er von geistigen Naturen belebt glaubt, werden als Orakel von ihm befragt, und diejenige, die durch Alter oder Weisheit ein n"aheres Recht zu deren Deutung sich erwarben, mit besonderer Achtung von ihm geehrt, als Handhaber des Opfergesch"afts, als Priester und Traumdeuter eingesetzt. Blos als schadende oder wohlt"atige Wesen (D"amonen) verehrten, wie die "altesten Geschichtsurkunden lehren, die ersten Menschen die Elemente, ohne ihnen noch bestimmte Namen zu geben, oder bildlich sie durch eigent"umliche Attribute zu unterscheiden. Vor anderem G"otterdienste verehrten die Indier im Budajagna-Opfer die guten und b"osen Genien\footnote{F. Paulo a Bartolomaeo Darstellung der Brahmanischen G"otterlehre. S. 34.} wie durch das Jagam und Homam oder Feueropfer, Sonne, Mond und die Planeten. Dies bezeigt auch der sch"one Hymnus in Sakontala an die Elemente:
\vspace{9pt}
\\
Wasser war des Sch"opfers erstes Werk,\\
Feuer empf"angt die Gaben\\
Anbefohlen im Gesetz:\\
Heilig ist die Opferweihe!\\
\\
\hspace*{1cm} Zeiten misst das Himmelslichterpaar\footnote{Sonne und Mond.}\\
\hspace*{1cm} Und des Schalles F"uhrer\\
\hspace*{1cm} Zarter "Ather, f"ullt das All!\\
\hspace*{1cm} Erd' ist des Geb"uhrens Mutter;\\
\\
\hspace*{2cm} Leben alles atmenden ist Luft:\\
\hspace*{2cm} So in acht Gestalten,\\
\hspace*{2cm} Sichtbar, n"ahr' und segn' euch Gott,\\
\hspace*{2cm} Issa der Natur Verwandler! ---\\

In Menus Gesetzbuche brachte Brama zuerst zehn Herrn der erschaffenen Wesen, diese aber erzeugten sieben andere Menus --- dann wohlwollende Genien und w"utende Riesen (wohlt"atige und zerst"orende Naturkr"afte), himmlische G"anger, Nymphen und D"amonen, endlich Blitze und Donnerkeile, Wolken, und farbige Bogen des Indra, fallende Meteore, die Erde zerrei"sende D"unste, Kometen und Lichtk"orper verschiedener Grade.\footnote{Menus Gesetzbuch nach H"uttners "Ubers. S. 9.}

Ein auf Elemente deutendes Symbol war gleichfalls jene am Berge Meru stehende Feuers"aule ohne Anfang und Ende, deren H"ohe Brahma in hunderttausend Jahren nicht ersteigen konnte, indessen ihr Fu"s im Abgrund stand; so der tibetanische Berg Righiel Lumbo aus vier Elementen zusammengesetzt. Ostw"arts bei den Chinesen finden wir in fr"uhester Zeit die beiden Urm"achte Jang und In. Jang des Vollkommenere vorstellend, daher Himmel, Sonne, W"arme, m"annliche Kraft, Urfeuer; In im Gegensatze das Unvollkommenere, daher Erde, Mond, weibliche Kraft, K"alte, Nacht, Urfeuchte; beide M"achte erzeugten vier Bilder Su-siang, und diese: acht nach verschiedenen Kombinationen verbundenen Gestalten Kua genannt. S"ud"ostlicher hin nach Correa, Japan, Siam, und auf Ceylan zeigen die Mythen "uberall Spuren des fr"uhen Elementen-Dienstes, nur nach Klima und Lokalit"at verschieden. Nordw"arts in Hinterindien bei den skythischen und skandinavischen V"olkern treffen wir gleichfalls nebst der Verehrung von Sonne, Mond, und anderer Himmelslichter, als Untergottheiten: Wolken, Regenbogen, Blitze, Gewitter, Sturm, Hagel, Feuer, Wasser, Erde, Berge, Fl"usse.\footnote{S. Pallas Reisen und Georgis Russland, verglichen mit Herodot und Strabo.} So glaubten die V"olker finnischen, slavischen, gotischen und germanischen Stammes, ihre Helden und V"ater w"urden nach dem Tode in die Wolken versetzt, und erschienen "ofters als Meteore.\footnote{H"aufige Beispiele hievon finden sich in der Edda und in Ossians Ges"angen.} Von den skythischen Scoloten sagt Herodot Buch 4. K. 5., dass sie dem Himmel und seiner Gattin Erde, der Sonne, dem Monde, dem Eisen (oder Mars) und dem Herkules opferten; die sogenannten K"onigs-Skythen aber dem Wasser; sie hatten weder Bilder, Tempel, noch Alt"are, aber den Mars stellten sie durch ein blo"ses Schwert und einen gro"sen Haufen Rei"sig vor. Eine merkw"urdige Sage f"ugt er hinzu, dass n"amlich ein gewisser Targitaus Sohn des Himmels und des Flusses Boristhenes der erste Bewohner dieser Gegend gewesen, und daselbst drei S"ohne erzeugt habe, w"ahrend welcher Regierung vier Heiligt"umer vom Himmel fielen, eine Pflugschaar, ein Joch, eine Axt und ein Goldst"uck; letzteres habe der "alteste Bruder zuerst fallen sehen, als er sich ihm aber gen"ahert, sei es gl"uhend geworden, bei Ankunft des zweiten noch hei"ser, nur als der j"ungste kam, sei die Gluth erloschen, und habe ihn in Stand gesetzt, das Goldst"uck nach Hause zu nehmen, wonach die beiden Eltern ihm Land und Regierung abtraten; weshalb (f"ahrt Herodot fort) die K"onige der Skythen gewisse Goldst"ucke als Heiligt"umer verwahren, und ihnen Opfer bringen.\footnote{Sollte dieser Mythe nicht die Tradition eines vor alter Zeit in jener Gegend gefallenen Meteor-Steins zum Grunde liegen? Gl"uhend fiel das Metallst"uck herab, und das Gl"anzende seiner Feuergestalt verwandelte die Sage in Gold, womit sich noch die Idee vom fr"uhen Gebrauch dieses Metalls und gleich fr"uher Benutzung des Pflugs verband, die Sage scheint demnach einer Zeit anzugeh"oren, wo die Skythen schon ein ackerbauendes Volk waren.}

Die Massageten verehrten vorz"uglich die Sonne; mit gezogenem Schwerte schwuren die Avaren und Hungaren dem Himmel und dem Weltfeuer oder dem Gott Isten (vielleicht Vesta?). So verehrten die alten T"urken und die Mongolen gleichfalls die Elemente. Dass die Anh"anger der uralten Lichtlehre, die Zerduscht, sp"ater nur verbesserte, nebst dem Hauptdienste des Feuers\footnote{Und zwar in Herm-aphroditer-Gestalt als Symbol des m"annlich t"atigen, und weiblich leidenden Princips. "`Persae et Magi omnes (sagt Firmicus de Errore prof. Rel. p. 10.) qui Persiae Regionis incolunt fines, ingnem praeferunt, et omnibus Elementis putant debere praeponi. Hi itaque Jovem in duos dividunt potestatem, naturam ejus ad utriusque Sexus transferentes, et viri et foeminae simulacra ignis substantiam deputantes, et mulierem quidem triformi vultu constituunt, monstrosis eam serpentibus illigantes. --- Virum vero ab actorem boum colentes, sacra ejus ad ignis transferunt potestatem."'} auch den anderen Elementen huldigten, zeigen h"aufige Stellen im Zendavesta, besonders in den B"uchern Izeschnè und Siruzè; denn die Amschaspands, Fervers, und Izeds, welche die Parsis durch Gebete und Opfer sich g"unstig zu machen suchen, was sind sie anders, als Genien der Naturkr"afte und Elemente? wie dies aus dem persischen Weltsysteme deutlich hervor geht. Wir sehen hier Albordi, das gro"se Urgebirgs in des Himmels Mitte, auf ihm ruhend die gro"se S"aule, die den Weltbau st"utzt, bis in die Region des reinen Lichts reichend. Da thront Hougner, Herrscher der H"ohen, denen die Quellen entstr"omen; auf Albordis Gipfel ruht vor allen der erste der Amschaspand, die Sonne, die wie Wasser in den H"ohen die Erde umkreist; ihr zun"achst der Mond, der seinen Lichtglanz "uber die geschaffene Erde ausgie"st. Tiefer stehen die Fixsterne und die Wandelgestirne, in ihren Bahnen unter die Weltgegenden verteilt, jeder Planet, worunter Taschter, der helle Oststern, der erste ist, an diesen h"oheren Himmel schlie"st sich unmittelbar jener der Meteore; aber nicht Feuer allein, auch heiliges himmlisches Wasser, auch der Regen entquillt den Sternen, und wenn die Divs (b"ose Genien) die Welt zerr"utten, dann f"ahrt von Albordis H"ohen ein Stern herab, und befruchtende Wasserstr"ome ergie"sen sich "uber die Erde. Wie es nun, nach Zerduschts Lehre\footnote{S. Bundehesch.} sieben Arten Feuers gibt: Berezeseng, das vor Ormurzd und den K"onigen brennt, Voh-Freuin in Menschen und Tierk"orpern, Oruazescht, in Gew"achsen, Vazescht, vor und aus dem Berg Sapojequier (wahrscheinlich ein alter Vulkan), Sprenescht, K"uchenfeuer; so z"ahlte man auch sieben oder vielmehr vierzehn Arten Wassers: n"amlich Thau, oder Wasser auf Pflanzen, Quellwasser, Regenwasser, Brunnenwasser, Fl"ussigkeiten von Tieren und Menschen, Schwei"s, Mark, Exkremente, Speichel, Oelteile, Dauungssaft, die Fl"ussigkeiten im Inneren der Pflanzen, endlich Milch.\footnote{Kleukers Zend à Vesta im Kleinen 2. 174.}

Wenden wir uns westw"arts, so finden wir gleich fr"uhzeitig diesen Elementen-Dienst in Verbindung mit Sab"aism bei den Arabern und "Athiopiern. Die Ph"onizier, sagt Philo von Biblus aus Sanchuniaton, legten den Namen ihrer K"onige den Welt-Elementen, und verschiedenen ihrer vermeinten G"otter bei, Sonne, Mond, Sterne, und die Elemente waren ihre einzigen G"otter. Von den "Agyptern erz"ahlt Diodor,\footnote{Diodor Bibl. der Gesch. 1. 12.} dass sie nebst Isis und Osiris --- Sonne und Mond --- zuerst das Feuer (Phanes oder Dionysus) und die "ubrigen Elemente verehrten. --- Dasselbe zeigt uns der griechische Cultus; denn auch hier wurden vor anderen G"ottern die Naturkr"afte in ihren Urprincipien und Ph"anomenen oder Meteoren verehrt,\footnote{Von den Telchinen, den ersten Bewohnern der Insel Rhodus, erz"ahlt Diodor Bibl. der Gesch. 5. Buch Kap. 55: "`dass sie die ersten gewesen, welche Bilds"aulen der G"otter gemacht haben, und verschiedene alte geweihte Bilder [wahrscheinlich Hermen oder Hausg"otter] f"uhrten von ihnen den Namen. Zugleich waren sie Zauberer, die ebenso wie die Magier, wenn sie gewollt, Wolken, Regen, Hagel und Schnee heraufbrachten, auch ihre eigene Gestalt verwandelten, in ihrem Unterricht in den K"unsten aber sehr zur"uckhaltend waren."'} so hie"s es in einem alten Gesang, den Pausanias\footnote{Buch 10., Kap. 12. Sie lebte zur Zeit der Peleaden, die, wie Pausanias hinzusetzt, "alter waren, als Phemonon.} der Ph"anis, einer der "altesten Sibyllen, zuschreibt, von Zeus: "`- Jupiter, der war, ist, und sein wird, durch deine Hilfe gibt die Erde ihre Fr"uchte, wir nennen sie daher unsre Mutter. --- Einer merkw"urdigen Vorstellung des Zeus erw"ahnt\footnote{Buch 2. K. 24.} Pausanias; mit drei Augen n"amlich, davon das eine mitten auf der Stirne ruhte, deutend auf die obere, Mittel- und Unterregion des Weltalls, die er beherrscht; ward Zeus nun als das Symbol des Himmels vor anderen G"ottern verehrt, so hatte das Element des Wassers oder Neptun in anderen Gegenden gleichfalls fr"uhen Dienst und Tempel, wie jener uralte, den Pausanias zu Teraphne sah.\footnote{Buch 3. Kap. 20.} Dahin geh"oren die Hydrophorien, besonders jenes uralte Wasserfest, das nach Pausanias Buch 1. K. 18. nahe am Tempel des Olympus, an einer "Offnung, durch welche, der Sage nach, die deukalionische Flut sich verlaufen hatte, zur Erinnerung ihrer durch Wasser vertilgten Voreltern gefeiert wurde; wobei man die interirdischen G"otter anrief, und durch Opfer zu vers"ohnen suchte. Ein "ahnliches Fest seierten die Egineten zu Ehren Apolls, vielleicht weil er der Gott ist, der die vom Schlamm der Gew"asser erzeugte Schlange "uberwand und t"otete. In den Festen, welche zu Hierapolis der syrischen G"ottin gefeiert wurden, waren nach Lucians Erz"ahlung (de Dea Syria) mehrere Gebr"auche, die mit dem athenischen Wasserfeste "Ahnlichkeit hatten, und sich, wie jenes, auf den Elementen Dienst des Wassers bezogen. Die meisten alten V"olker feierten solche Hydrophorien, nicht blo"s wie Boulanger in Antiquité dévolée meint, zum Andenken der allgemeinen Flut, oder einzelner "Uberschwemmungen; vielmehr um durch Opfer und Gebete die Elementar-Geister der Meere, Seen, Fl"usse zu vers"ohnen, und sich geneigt zu machen. Auch der Luft oder den Winden war zu Titane im korinthischen Gebiete ein Altar geweiht,\footnote{Buch 2. Kap. 12.} und die Erde hatte zu Sparta einen Tempel, Gasepton genannt.\footnote{Buch 3. Kap. 12.} Von den alten Pelasgern sagt Herodot,\footnote{Herodot 2. 52.} dass sie anf"anglich den G"ottern, weil diese alles in Wohlordnung gesetzt und in Einteilung gebracht, unter Gebeten zwar mancherlei opferten, aber ihnen noch keine bestimmte Namen gaben, vielmehr erst nach Befragung des Bodonischen Orakels durch dessen Aussage bestimmt wurden, diese h"ohere Wesen Himmel und Erde zu nennen, wozu sie sp"ater (gleichfalls auf Gehei"s des Orakels) den Dionysos (obschon von "agyptischer Herkunft) gesellten. Aber da der menschliche Geist im Fortgang seiner Entwickelung sich nicht gen"ugt an diesem einfachen Dienste, und da seine rege Einbildung "uberhaupt gern in Bildern lebt, suchte er die durch Meteore zwar f"uhlbaren, jedoch in ihren einfachen Bestandteilen nicht anschaulichen Urkr"afte bildlich und symbolisch darzustellen; daher der Ursprung der Hermen, und der durch Einwirkung von Zeit und Lokalit"at in unendliche Formen verwandelte Polytheismus, dessen Bilder und Schemen gleichwohl nichts als Attribute des einigen Gottes sind, dessen reine gel"auterte Verehrung durch ausgeartete, beschr"ankte und kindische Begriffe entstellt wurde, wie gegenseitig die Idee seiner blo"s geistigen Natur nur das Resultat eines reinen, "uber Sinnlichkeit sich erhebenden Gem"uts zu sein vermag. Zwar hat diese reinere, durch Offenbarung dem Menschen bei seiner Bildung mitgeteilte Vorstellung, sich im Geschlechte selbst mitten unter aller geistigen und religi"osen Entartung in einem kleinen Haufen erhalten. Aus Ur (oder Chald"aa), dem Lichtlande, ward durch den Stamm der Abrahamiden die Verehrung eines einigen Gottes bewahrt und bei den Hebr"aern fortgepflanzt, bis durch des Gottmenschen Sendung das Licht reiner Wahrheit in hehrem Glanz schimmernd sich allgemein verbreitete, und dem Polytheismus ein Ende machte. --- So lange hatte derselbe in immer rascherem Fortgange den gr"o"sten Teil der V"olker ergriffen; im Beginnen ihres gesellschaftlichen Zustandes hatten dieselben ihren rohen Begriffen und regen Phantasie gem"a"s alles belebt, was in der Natur sie umgab, daher ihre "au"sere G"otterlehre im wahren Sinne pantheistisch ist, und je tiefer wir ins Altertum zur"uck blicken, je mehr sehen wir die Idee eines einigen Gottes, nach den verschiedenen St"ammen der V"olker geteilt in ebenso viele Lokalg"otter, wie die Kunst oder bildliche Vorstellung dieser Wesen je "alter, je mehr mit vervielfachten Teilen und Attributen "uberh"auft.\footnote{Man sehe z. B. in Fr. Paulo Barthol. Brahminenlehre die Abbildungen der drei ersten Verwandlungen Vischnus, und jene des Shiva, welche die Bildung der Erde aus dem Wasser und den Kampf des Feuers mit den andern Elementen darstellt.}

So roh diese Vorstellungen sein m"ogen, haben sie gleichwohl eine merkw"urdige Deutung, indem sie die ersten Bl"atter in der Geschichtsurkunde der Erd- und Menschenbildung sind; denn die in allen Mythologien erscheinenden Bilder des Chaos, und der aus dessen G"arung entstehenden Feuer- und Wasser-Verheerungen, die Riesenk"ampfe, die verschiedenen aufeinander folgenden, sich immer zerst"orenden G"ottergeschlechter, was sind sie anders, als bildliche, vieldeutende Orakel vom ersten Ursprung der Dinge? --- Aber noch fr"uhere anschaulichere Beweise und Zeugnisse der ersten Urzeit hat unsere Erde aufzuweisen: f"urs erste jene Urgebirge und h"ochsten Felsspitzen, die am fr"uhesten aus dem immer mehr niedersinkenden Gew"asser hervortraten, und l"angst vor der belebten Sch"opfung als einzelne Inseln hervorragten, daher die den "altesten V"olkern eigne Verehrung der Berge und Fl"usse. Ferner alle Metalle und Gemmen (edle Steine), deren Ursprung und Bildung durch Einwirkung der m"achtigeren Elemente, Feuer und Wasser, gleichfalls Zeugnis geben von der Urzeit und der Bildung unseres Erdk"orpers. Endlich --- im Kreis der Meteore, deren Erscheinung der rohe Naturmensch stets einer h"oheren Ursache, einem geistigen Wesen beimisst, jene Feuermassen, die teils als elektrische Flammen in der Atmosph"are schimmern,\footnote{Z. B. die als Sternschnuppen, St. Elms-Feuer, Dioskuren, und unter anderen Namen bekannten Meteore, wozu auch die Nordscheine geh"oren und der als Friedensbild des vers"ohnten Himmels mit der Erde (nach der Mosaischen Sage) so bedeutende Regenbogen; wie gegenseitig Kometen als Boten des Brandes und der Zerst"orung immer die furchtbarsten Meteore waren.} teils als gr"o"sere oder kleinere Steinmassen zur Erde fallen, und, ihrem "au"seren sowohl als inneren Gehalt nach, Spuren eines fremdartigen Ursprungs, einer fernen Heimat tragen. Vom Himmel oder aus h"oherer Atmosph"are fallen sie herab, ein Feuer-Meteor ist ihr Begleiter, sie selbst im Augenblick ihres Sinkens gl"uhend und lichtstrahlend; kein Wunder daher, dass man sie himmlischen Ursprungs und der Verg"otterung wert hielt; denn eben der Glaube, der die Gestirne f"ur belebte geistige, die niedere Welt beherrschende Wesen halten machte, erzeugte auch die Idee, dass diese Feuermassen untergehen, und im Augenblick ihres Erl"oschens oft zur Erde s"anken, welches in der alten Mythik umso gegr"undeter ist, als ihr gem"a"s, die Himmlischen oft unter den Sterblichen wandelten, sich zur Erde herab lie"sen, und dass nicht ungeformte Steine allein, sondern selbst Bilder der G"otter und andere Heiligt"umer, die mit gr"o"ster Ehrfurcht in Tempeln verehrt wurden, vom Himmel fielen.\footnote{Ein Beispiel ist das in Ephesos, der Sage nach, vom Himmel gefallene aus Holz geformte Bild der Diana, von dem wir sp"ater reden werden.}

Die Zeugnisse der Alten von "ofters sich ereigneten Steinregen, und dem Herabfalle einzelner Aeroliten hat man lange teils "ubersehen, teils f"ur m"archenhafte Sagen gehalten, bis neuere Physiker aufmerksam geworden auf die in mehreren Gegenden sich ereignete Erscheinung von Feuerkugeln und sogenannten Meteor-Steinen, den inneren Gehalt, die Bestandteile dieser Massen chemisch untersuchten,\footnote{Chladni, Proust, Reu"s, Klapproth u. a.} und, indem man zugleich die Beschreibung einiger im Altertums bemerkten Steine dieser Art damit verglich, kamen vorl"angst schon einige Altertumsforscher\footnote{Vorz"uglich Falconet in mehreren Abhandlungen der Mém. de l'acad. des Inscript. et belles lettres.} auf die Mutma"sung, dass die von den Alten so religi"os verehrten B"atylien und heiligen Steine, gr"o"stenteils Aerolithen waren, oder mindestens f"ur Steine himmlischen, d. i. au"sertellurischen Ursprungs gehalten wurden.

Ein neuerer Forscher war vorz"uglich bem"uhet, einen aus Quellen gesch"opften Vergleich dieser B"atylien mit den Meteor-Steinen zu machen. Seine Abhandlung kam mir zu Handen, als ich l"angst, angelockt durch Lesung der in Gilberts Journal der Physik und anderen Schriften befindlichen Analysen dieser Steine bem"uht war, Materialien zu einer Untersuchung "uber den Steindienst, und den ebenso merkw"urdigen Meteor-Cultus (wahrscheinlich die fr"uheste Verehrung) zu sammeln. Seine flei"sigen Forschungen\footnote{D. M"unters Schrift "uber die B"atylien der Alten, in den Verhandlungen der gelehrten Gesellschaft von Kopenhagen.} gaben mir wichtige Fingerzeige in manchem, auch bin ich seinem Pfade gefolgt, doch schien mir diese Schrift gleichsam nur Vorarbeit, indem sie den physisch-chemischen Teil der Meteor-Steine kaum ber"uhrt, auch "uber die h"ohere mythische Ansicht, die der Verehrung derselben zum Grunde liegt, und die, wie mir d"aucht, in so genauem Zusammenhange mit der "altesten Theurgie steht, dass ohne Beihilfe derselben sie nicht erkl"art werden kann, nur wenige Winke gibt.

Dies nun zu verfolgen, und den Zusammenhang der Steinverehrung mit dem Dienst der Elemente, oder ihren Erscheinungen in den Meteoren in ein helleres Licht zu setzen, ist der Zweck gegenw"artiger Abhandlung.
\clearpage
\paragraph{}
Welchen Ursprung jene merkw"urdigen Massen auch haben m"ogen, die aus h"oheren Luftregionen teils einzeln, teils in zahlreicher Menge als Steinregen auf die Erde fallen; ob sie im Wasser oder Feuer entstehen? ob, nach Chladni, sie als Teile zertr"ummerter Weltk"orper anzusehen sind, oder ob sie nach La Place dem Monde entfallen; nach Proust\footnote{Gilbert Annalen der Physik, Bd. 24. S. 261.} und anderen hingegen in der Atmosph"are sich bilden, und wenn gleich sie sich in den uns bekannten Gegenden der Erde nicht finden, noch in ihnen finden k"onnen, doch Regionen unseres Erbk"orpers, und zwar den unermesslichen noch unbekannten Polargegenden ans geh"oren, von diesen losgerissen, und aufw"arts geschleudert, in unsern s"udlichen Gegenden niederfallen; diese Erforschungen seien dem Physiker "uberlassen; Tatsache ist indessen, dass von den "altesten Zeiten her Meteor-Steine zur Erde fielen, denen man, wie die Geschichte lehrt, g"ottliche Verehrung bezeigte.

Woher nun, fragt sich, entstand dieser Cultus?

Folgende Hauptursachen lassen sich, glaube ich, hievon angeben:

1. Wenn wir Humes und Boulangers Idee\footnote{Origine of Religion, in Humes Works. Boulangers Antiquité dévoilée.}: dass Furcht der Ursprung aller Religion sei, auch nicht unbedingt, beistimmen k"onnen, so zeigt die Geschichte unseres Geschlechts doch, wie Schrecken vor ungew"ohnlichen Erscheinungen bei rohen sinnlichen Menschen den so nat"urlichen Glauben erzeugen konnte, dass Meteore und au"serordentliche Ph"anomene von unsichtbaren h"oheren Wesen herr"uhren, die, indem sie ihren Wirkungen nach mehr zerst"orender als milder Natur scheinen, man durch Gebete, S"uhnopfer u. d. gl. sich geneigt machen m"usse, ein Glaube, woraus der erste rohe Fetischismus hervorging. Da nun "ofters Meteor-Steine teile einzeln, teils in gr"o"serer Zahl aus h"oheren Regionen herabfielen, war es nat"urlich, in ihnen die Kraft eines sie belebenden in T"atigkeit setzenden h"oheren Wesens zu ahnen.

Eine 2te Ursache ist in der Physik und Kosmologie der alten V"olker zu suchen.

Sterne wurden in der fr"uhesten Zeit, ehe der Polytheismus noch tiefer herabsank, als g"ottliche Wesen verehrt, und ihrem Einfl"usse war alles, was irdisch ist, unterworfen. Diese Verehrung aber gr"undete sich auf die Meinung: vom Dasein gewisser Mittelwesen, die (verm"oge des Systems der Emanation) die ganze Kette der Intelligenzen, Genien, D"amonen bildeten, die ein Ausstrahle des unendlichen Lichtquells oder eine fortgehende Progression von Potenzen aus der Einheit sind; eine Lehre, die allen V"olkern gemein war, und welche man in allen Mythologien wieder findet.

In den fr"uhesten Zeiten nahmen die Indier (wie die Purana lehren) gute und b"ose, himmlische, irdische und unterirdische Mittelwesen an\footnote{Trefflich entwickelt und zusammengestellt findet man die indische Emanations-Lehre in v. Polliers Mythologie des Indous. T. 2. Ch. 12. 13. 14.}; diese Genien (Deiotas) ver"andern nach Willk"ur ihre Form, find als Vorsteher und Leiter "uber die Elemente und alle Wesen von den gr"o"sten zu den kleinsten gesetzt, wie ihr Einfluss sich auch "uber alle Wesen verbreitet, weshalb die Sch"opfung in 15 Regionen (Sourg) geteilt ward, die alle unter ihrer Gewalt stehen, und ihrem guten oder b"osen Einfluss (da dieser Wesen es gute und schlimme gibt\footnote{Daher weise und schwarze Magie.}) unterworfen sind. Sie bilden unter sich eine Hierarchie, an deren Spitze sieben Haupt Deiotas stehen, die die h"oheren Regionen leiten --- andere beherrschen die Erde, Meere, Fl"usse, Quellen, Berge und W"alder; den sieben unteren Regionen (Lock genannt) sind wieder ebenso viele Genien (b"ose, verderbende Wesen, die die reinen Geister betr"ugen, ihnen jedoch gewisserma"sen untergeordnet sind) vorgesetzt. --- Diesen Deiotas vollkommen "ahnlich sind die sieben Amschaspands nebst den oberen und untergeordneten Izeds und Fervers des persischen Magismus; der Chald"aersieben F"unten der oberen Welt; die sieben g"ottlichen Throngeister der Juden und Zephiren der Kabbala, die sieben heiligen Laute der "Agypter,\footnote{Hievon Jablonsky Panth. Aegypt. Proleg. p. 53. -- Zend à Vesta. -- Die heiligen B"ucher und rabbinischen Schriften.} Orphiker und Pythagor"aer; die "Aonen der Gnostiker, wie endlich "uberhaupt die aus "alteren orientalischen Quesen flie"sende D"amonen-Lehre der Griechen,\footnote{Mit welchen auch die nordischen Mythen "ubereinstimmen.} "`Die G"otter, sagt Plutarch, mischen sich, und betreiben nicht selbst die Wahrsagungen, Beschw"orungen u. d. gl., sondern die D"amonen als ihre Diener und Gesch"aftstr"ager; so sind einige Aufseher der Opfer, Vorsteher der Feste und Mysterien; andere gehen als R"acher des "Ubermuts und der Ungerechtigkeit auf Erden umher, andere sind gute wohlt"atige Geister."'

Wie man nun glaubte,\footnote{Plutarch "uber den Verfall der Orakel.} dass diese Mittelwesen der Planeten Gestirnen und Elementen vorst"unden, hielt man auch daf"ur, dass nicht allein Unterg"otter, sondern selbst ausgezeichnet Menschen (Heroen) in Sterne verwandelt w"urden, und als solche am Himmel gl"anzten, aber da sie nicht unfehlbar w"aren, wegen vergangenes Verbrechen ihren Glanz auch wieder verlieren und herabsinken k"onnten, daher der Glaube, dass Sterne belebt seien, und zuweilen auf die Erde herabfielen. Der alten Physik gem"a"s hielt man Sterne f"ur Feuermassen,\footnote{Nach Anaxagoras, Demokrit und Metradorus war die Sonne ein feuriger Klumren, oder ein gl"uhender Stein: so der Mond eine feste gl"uhende Masse, und die Sterne, nach Diogenes, gl"uhende Steine, die oft zur Erde herabfielen und da verloschen. Plutarch. de Placit. Philos.} und da ihre eigentliche Gr"o"se in jener fr"uheren Zeit noch nicht hinl"anglich berechnet war, schien nichts weniger als ungereimt, dass, sobald ihr Glanz (d. h. der sie regierende Geist) erlosch, sie zur Erde sinken konnten. Der gefallene Stern war ein verwandelter entflohener Daimon, und der gl"uhend herab gefallene Stein ein erloschener Stern.\footnote{Les orientaux croyoient, que les Anges sont des Esprits ignées, opinion, qui passa depuis chez les Chrètiens, et qui, si je neme trompe, s'étoit communiquée aux juifs longtems auparavant. Beausobre Hist. du Manichéisme. T. 1. p. 323.}

Eine 3te Ursache der Verehrung, die man sogenannten heiligen Steinen bezeigte, ist in dem Gebrauch und der Anwendung, den man von denselben zu "offentlichen Denkmalen und Merkzeichen machte, gegr"undet. Bevor wir nun zur n"aheren Bettachtung der Aerolithen-Verehrung "ubergehen, wird es nicht "uberfl"ussig sein, einen n"aheren B"uck auf. Entstehung des Stein-Cultus "uberhaupt zu werfen.

In jenem Weltalter, wo Sinnbilder, Schriftz"uge, Buchstaben noch nicht erfunden waren, und rohe Menschen keine anderen als gleichfalls rohe Werkzeuge hatten, ihre Ideen aufzuzeichnen, mussten Steine als Merkzeichen vorz"uglicher Ereignisse, die zugleich als Vereinigungspunkt bei feierlichen Handlungen und Vertr"agen, oder als Grenzbezeichnungen dienen konnten, eine Achtung gewinnen, die allm"ahlich bis zur Verg"otterung erh"oht ward, und durch den geheimen Sinn, den sp"atere Mysterien, Priester, Hierophanten diesen Gegenst"anden beilegten, eine noch h"ohere Verehrung erhalten.

Aus dem h"auslichen Feuerherde, der bei jedem Volke im nomadischen Zeitalter in der Mitte des Zeltes stehend, die Familien zu gemeinsamen Verrichtungen, Gebeten, traulichen Gespr"achen sammelte, und ihnen darum so heilig war, als ihre Laren, entstand der Vesta-Dienst, und mit ihm die Bewahrung des reinen Feuers, auf dem allgemeinen Heerde oder Steinaltar, als dem Mittelpunkte des ganzen Staates. So ging aus dem fr"uhen Gebrauch roher, zuerst unf"ormlicher, dann gehauener viereckiger, endlich mit Kopf und menschlichen Gliedma"sen versehenen Steinen (woraus allm"ahlich die Bilds"aulen entstanden) der Dienst des Jupiter lapis oder horcus,\footnote{Auch des ΖΕΥΣ ΚΕΡΑΥΝΙΟΣ oder Fulminatoris, dessen Dienst in den "altesten Zeiten schon herrschend war, und dem Seleucus Nicator, als er die Stadt Seleucia am Meer erbaute, unter dem Symbol des Blitzes Tempel und Alt"are errichtete, wie besonders eine M"unze dieses K"onigs zeigt, auf deren R"uckseite sich ein befl"ugelter Donnerkeil befindet. S. Spanheim de Praestantia et Usu Numm. antiq. p. 393. Zwei andere M"unzen in demselben Werke zeigen den befl"ugelten Blitz auf einem Tische oder Altare liegend; s. Kupfertafel No. 6, 7, 8.} des Terminus, Pales, der "Acker, G"arten und Weg- Genien, Hauslaren und Hermen hervor."' Die Griechen, sagt Pausanias Buch 7. verehrten anf"anglich rohe Steine statt G"otter."' So stellte, gleichfalls nach Pausanias Buch 9. K. 24. 27. ein blo"ser Stein in B"ootien den Herkules, zu Thespis den Cupido, zu Orchomenos die Grazien, zu Theben den Bacchus, und nach Herodian zu Paphos die Venus in der Gestalt eines Ecksteins oder einer Pyramide vor.\footnote{Von diesem letzten sind die Nachrichten zu unsicher, um bestimmt zu sagen, dass er ein Meteor-Stein gewesen, so von mehreren anderen, z. B. dem Stein des Jupiter Caseos, der Diana im Tempel zu Laodic"aa --- (von dem Eckhel in Mus. Caesar. Abbildung auf M"unzen anf"uhrt) vom Stein im Tempel zu Perga, zu Calchis in Syrien, zu Flavia Neapolis (dem alten Sichem) s. gleichfalls Eckhel und Pellerin Recueil de Medailles, und andere mehr; aber von jenem zu Orchomenas sagt Pausanias 11. 100. 38. und 9. 25. bestimmt, dass er vor dem trojanischen Kriege zu Zeiten des K"onigs Eteokles vom Himmel gefallen.} Ein Beispiel zeigt die M"unze Iaus Vaillaut Num. Graec. Imp. Tab. 4. No. 14., die ein Konisches Idol mit 2 Tauben und Leuchtern aus der Insel Cypern (wo der Venus-Dienst herrschte) vorstellt. Auch bei den Mexikanern findet sich nebst dem Elementen-Dienste die Verehrung der B"atylien oder heiligen Steine; merkw"urdig ist die Nachricht, welche Alexander von Humbold im 2ten Hefte der Vues des Cordilieres; Seite 94, von der Gottheit der Tolteques gibt: "`ihr vornehmster Gott hie"s Tlalocteuctli, er war zugleich Gott des Wassers, der Berge, und des Gewitters. Diesem Bergvolke waren die hohen in stete Nebel geh"ullten Gipfel der Gebirge der geheimnisvolle Ort, auf dem der Donner erzeugt wird. Dahin versetzten sie den Thron des gro"sen Geistes Teotl, jenes unsichtbare Wesen Ipalnemoani, und Tloque Nahuaque genannt, weil er nur durch sich selbsten ist, und sich allein umschlie"st. Von dieser kaum ersteiglichen H"ohe herab kommt der Orkan sowohl, der die friedlichen H"utten zerst"ort, als der wohlt"atige Regen, der die Felder erquickt. Auf einem der h"ochsten Berge hatten die Tolteques dem Gotte Tlalocteuctli eine Bildf"aule errichtet, aus einem wissen Steine, den sie f"ur g"ottlich hielten (teotetl), nur roh war sie aasgehauen, denn dieses Volk, "ahnlich darin den Orientalen, hatte abergl"aubige Verehrung f"ur gewisse Farben der Steine. Vorgestellt ward diese Gottheit mit dem Blitz und Donnerkeil in der Hand, auf einem cubusf"ormigen Stein sitzend, eine Vase vor sich gestellt, wo rinne man ihm Caoutchoue und verschiedene Erdsamen opferte. Derselbe Cultus findet sich bei den Aztequen, die ihn bis zum Jahr 1317 der christlichen Zeitrechnung beibehielten, wo der Krieg, den sie mit den Einwohnern der Stadt Xochimileo f"uhrten, die erste Veranlassung zur Einf"uhrung der Menschenopfer ward."'

Bemerkenswert ist die Analogie dieses mexikanischen Stein-Gottes mit dem Alagabal der- Syrer, dem schwarzen Stein der Kaaba, dem Alasovid, Abadir, Dusares der arabischen St"amme, von denen wir sp"ater reden werden. Wenn es "uberhaupt scheint (wie Hr. v. Humbold aus wahrscheinlichen Gr"unden zeigt), das die Nationen, welche Amerika bev"olkern, sich fr"uhzeitig vom ersten Wohnsitze des Menschengeschlechts entfernt, und ihrer eigenen Leitung "uberlassen, lange herumirrend, allen "Ubeln und Unbequemlichkeiten eines herumirrenden Lebens unterworfen, sich sp"ater entwickeln, und zu einem h"oheren Grade von Kultur erheben konnten so finden sich bei ihnen dagegen d Spuren des urspr"unglichen Natur-Cultus, wie fr"uher bei den V"olkern des alten Kontinents in seiner primitiven Form wieder. Wir d"urfen z. B. nur die Zwergen"ahnliche Basalt-B"uste der mexikanischen Priesterin im 1. Hefte des Humboldischen Atlas pittoresque (die doch eher eine mexikanische Haus- oder Schutz-Gottheit zu sein scheint), wie "uberhaupt die meisten monstr"osen Abbildungen der Aztequischen Gottheiten anblicken, so zeigen sie eine auffallende "Ahnlichkeit mit jenen ph"onizischen und alt-"agyptischen zwergartigen G"otterchen, welche Herodot unter dem Namen Pat"acken erw"ahnt, und die als Schutzg"otter in Tempel und Vorhallen gestellt wurden.

"Uberhaupt war im Altertume der Glaube herrschend, dass die Gottheit ihr Bildnis selbst oft vom Himmel zur Erde gesandt habe, es mag nun dieses aus Stein oder einem andern Stoffe bestanden haben, so jenes der Diana von Ephesos, welches nach Plinius Zeugnis\footnote{Nat. Gesch. Buch 16. Kap. 77.} aus Weinstock geschnitzt, und t"aglich mit R"urden getr"ankt wurde, um die Gliederf"ugungen beweglich zu erhalten, dass es f"ur ein vom Himmel gefallenes Bild galt, "alter sogar als Bacchus und Minerva, wie Plinius versichert, zeigen die Worte in der Apostelgeschichte\footnote{Kap. 29.}: "`Ihr B"urger von Ephesus! ist denn irgend ein Mensch auf der Welt, der nichtwei"s, dass die Stadt Ephesos die Dienerin des Tempels der Diana, und des vom Himmel herab gefallenen Bildes ist?"'

Jedes Volk hatte in der Mitte seines Versammlungsortes einen solchen Stein als Symbol seiner Einheit (την πολιν εστηαν), einen Lokal-Schutzgott (Δαιμον οιχοδεσποτην), und der Gott einen ihm dienenden Priester, nebst einem Hause (das sp"ater zum Tempel ward). Mit diesem rohen uralten Fetischismus verband sich fr"uhzeitig ein geheimer Sinn, den die Priester und Mysterien dem Stein als Symbol beilegten,\footnote{Auch den Juden war der heilige Stein ein Symbol der Gottheit; so sagt Jesaias 8. 14.\\
"`Jehovah Zebaoth -- den seht f"ur heilig an,\\
Nur der sei eure Furcht, nur der euer\\
\hspace*{1cm} Schrecken,\\
Dann dient er euch zur Sicherheit, gleich\\
\hspace*{1cm} dem heil'gen Stein."'\\
Nach Justis "Ubersetzung in den Blumen althebr"aischer Dichtkunst.} und indessen das rohe Volk im exoterischen Dienste seinen Fetisch, den rohen Holz- oder Stein-Block verg"otterte, deutete die geheimere Einweihung denselben als Darstellung und Symbol einer m"achtigen Naturkraft, n"amlich als Bild der Zeugung und der sich stets erneuernden Zeit, woher im fr"uhesten Zeitalter der Lingam-Stein entstand, der mit der Tradition durch alle Cultus gehet.\footnote{Einen dahin sich beziehenden Gebrauch zweier indischen V"olker (der Zechien und Albarachen), deren Sitz und Ursprung jetzt schwer anzugeben ist, f"uhrt ein wenig bekannter Schriftsteller, Vincentius Bollovancensis in Speculo histor. Cap. 4. an; die Stelle, deren Ouseli in seinen Noten zum Min. Felix. Lugd. Batav. 1672. p. 17. erw"ahnt, ist zu merkw"urdig, um hier nicht ausf"uhrlich zu stehen:\\
\hspace*{0.5cm} Duarum Indiae gentium, quae vocantur Zechiam et Albarachuma, antiqua consuetudo fuit, nudos et decalvatos, magnisque ululatibus personantes Simulachra Daemonum circumire, angulos quoque osculari, et projicere lapides in acervum, qui quasi pro honore Diis exstruebatur. Inde est, quod in libro Salomonis dicitur; qui projicit lapidem in honorem Mercurii. Faciebant autem hoc bis in anno, sole scilicet existente in primo gradu Arietis, et rursus, cum esset in primo gradu Librae: hoc est, initio Veris et Autumni. Haec ergo consuetudo cum ab Indis ad Arabes descendisset; eamque suo tempore apud Mecham in honorem Veneris Mahumed celebrari reperisset; sic illam manere praecepit, cum tamen cetera idololatriae vestigia removisset. Illud vero soli veneri in illa celebratione dicitur exhiberi solitum, ut lapilli retro, id est, sub genitalibus membris projicerentur, eo quod Venus maxime partibus illis dominetur. Unde id adhuc hodie fit in domo Dei illicita quam vocant.}

Dieser Lingam oder Phallus-Dienst war jedoch in seinem Ursprunge nicht so unrein und obsc"on als Zoega\footnote{De orig. et usu obelisc.} und Friedr. Schlegel\footnote{Von der Weisheit der Indier.} ihn schildern; durch Entartung, besonders Einf"uhrung der Bacchischen Orgien mag er es sp"ater geworden sein; seinem Ursprunge aber, und dem inneren Sinne der alten Religion gem"a"s, war diese Idee gewiss heilig und rein. Denn die Indier, (und wie sie alle alten V"olker) glaubten: sowohl Holz als Stein enthielte das Elementar-Feuer, und mit ihm das Zeugungs-Prinzip. Die Sonne nebst den. "ubrigen Sternen seien Steine. Im Steine lie"sen sie die Gottheit wohnen, und sich in Stein verwandeln. --- So Krischna am Ende seines Lebens, nach der griechischen Tradition der Stein des Chronos, den in Windeln gelegt, Rhea ihrem Gemahl statt des verfolgten Zeus zu verschlingen gibt.\footnote{S. das Titelkupfer nach einer antiken Ara im Museo Capitolino.} Pausanias traf ihn noch im Tempel, und er ist offenbar ein Bild der Zeit, so Niobe mit ihren 12 Monat-Kindern; des Sisyphus Rad; der Mythos des aus Steinen Menschen schaffenden Deukalions u. s. f. Der-Stein-Cultus schien demnach nicht sowohl aus einem rohen Fetischismus, wie Zoega meint, ausgegangen zu sein, als: eine durch Verwilderung der in der fr"uhesten Zeit aus dem Ursitze des Menschengeschlechts in verschiedene Weltgegenden auswandernden St"amme entstandene Ausartung der urspr"unglich reinen. Idee, welche die erste Natur-Religion unter diesen Symbole verstand.

Von der urspr"unglichen Deutung des Steins auf Zeit kommen die Mahl- oder sogenannten Rechensteine; im Hermes (von dem alle Zeitrechnung kam) wurden sie in ganzen Haufen zusammengelesen, die ερμαχες ober ερμειοι φιλοι hie"sen,\footnote{Kanne allgemeine Mythologie.} der Cultus verga"s den Ursprung der heiligen Sitte, und jeder Wanderer legte zu dem Haufen noch einen neuen Stein. In Arabien erhielt sich die Sitte lange nach Muhameds Zeiten und die Geschichte der Erzv"ater erw"ahnt diesen Gebrauch. Dahin k"onnen wir auch die 360 Steinegef"a"se der "agyptischen Priester auf der Nil Insel n"achst Philoe rechnen, mittelst welchen sie, indem sie selbe bei jedesmaligen Wiederbeginnen des b"urgerlichen Jahres anf"ullten, die Zeit- oder Zahl ihrer Jahrestage anzeigten.\footnote{S. Phamenophis. S. 96.}

B"undnis, Vertr"age und Eide wurden daher stets an solchen Steinen geheiligt. So hatte der Libanon einen H"ugel, der wie Jacobs Steinhause \<gl`d> Gilead hie"s. Dem Ursprung den Sitten, bei Steinen zu schw"oren, deutet selbst die Wurzel der Sprache an. So hei"st: \<`bdy> h"aufen, und \<`bd> ein harter Stein. \<`yd> und Zeugnis geben, \<`d> Zeuge, welches in der Grundbedeutung Zeit, Zeugung und Feuer hei"ste und in der Zusammensetzung Gilead \<gl`d> sowohl. Steinhaufe des Zeugnisses, als der Zeit bedeutet.

Diese Einleitung f"uhrt uns auf n"ahere Erforschung des Stoffes und inneren Gehalts jener Steine, die man im Altertume zu religi"osen oder dem gemeinen Wesen dienlichen Zwecken gebrauchte, und darum f"ur besonders heilighielt. Hier sprechen. nun alle Zeugnisse, dass, wo nicht alle, doch gewiss eine gro"se Zahl derselben wahre Aerolithen gewesen, deren Beschreibung ganz mit jener "ubereinstimmt, die uns die Physiker von den in sp"ateren Zeiten gefallenen Meteor-Steinen geben. Die Rinden dieser Steine (sagt Reu"s\footnote{"Uber den Steinregen bei Lissa, im Journal der Chemie, Physik und Mineralogie, 8. Band 2. Heft, S. 457.}) sind dunkelschwarz, stellenweis ins Braune ziehender Farbe, teils matt, teils schwach schimmernd, und an den sammetschwarzen Stellen non Pech-, an andern von schwachem Metallglanze; sie zeigen zahlreiche gr"o"sere und kleinere Eindr"ucke und Erhabenheit, wie sie ein weicher dehnbarer K"orper annimmt, wenn man ihn mit dem Finger dehnt oder kratzt. Sie f"uhlen sich im Ganzen ziemlich glatt, nur hie und da etwas rau an. In Hinsicht ihrer schwarzen Kruste bemerkt man zwar Verschiedenheit, einige sind dunkelschwarz, andere haben ein pechartig metallisches schwach schimmerndes Ansehen, und sind etwas m"urbe; dieser Verschiedenheit ungeachtet ist doch ihre Gleichartigkeit nicht zu verkennen, und man bemerkt beim ersten Anblick derselben ihre Abstammung aus h"oheren Regionen. Ihre vorz"uglichen Bestandteile sind, nach Klaproths neuester Untersuchung\footnote{S. Analyse des Meteor-Steins von Lissa in demselben Journal.}: Eisen, Nickel, zuweilen Chromium, Mangan, Kiesel-Erde, Bittersalz-Erde, Alaun-Erde, Kalk, Schwefel, wovon (wie er hinzusetzt) das Eisen als gediegen anzunehmen. Dieses metallischen Eisens wegen wirken auch die Aerolithen st"arker oder schw"acher auf die Magnetnadel, und werden von ihr angezogen.

Wie verschieden dieselben in ihrem Gewichte seien, ist daraus abzunehmen, dass man Steine von der keinen Dimension eines Zolls bis auf Massen von 3 bis 400 Pfund, ja selbst von 14 Zentnern (wie jener ber"uhmte Eisenstein, der in Sibirien niederfiel) kennt.\footnote{S. die erw"ahnte Abhandlung, S. 457.} Unbestimmt ist ihre Form und Zahl; bald zugerundet-eif"ormig, mit Ecken und Kanten; in der Mitte eingedr"uckt, mehrseitig, pyramidalf"ormig, an den Seiten abgestumpft, oft ganz kugelrund u. s. f. Gleichverschieden sind sie an Zahl, und in den ihren Fall begleitenden Nebenumst"anden, indem sie teils einzeln, teils in geringer Zahl, teils als f"ormlicher Regen (der Steine auf Meilenwegs umher streut, welche H"ugel und kleine Berge bilden, wie k"urzlich zu L’Aigle in der Normandie\footnote{S. Journal der Chemie am angef"uhrten Orte.}; bald fallen sie bei ganz heiteren Tagen, bald bei st"urmischem wolkigtem Himmel, aber immer (sagt Reu"s) ist beim Herabfallen ein au"serordentliches Get"ose zu h"oren, das mit einem Knalle aus Kanonen, Pelotonfeuer, Wirbeln auf Trommeln, t"urkischer Musik, einem orgel"ahnlichen Pfeifen und Sausen in den h"oheren Luftregionen verglichen wurde; und in vielen, ja den meisten F"allen bemerkte man dabei eine Richtung von S"udwest nach Nordost, so dass die n"ordlichsten Steine zuerst, die s"udlichen zuletzt niederfielen.\footnote{Journal f"ur Chemie und Physik. S. 455.} Neuere Berichte erz"ahlen die au"serordentliche Wirkung solcher Ph"anomene auf die Gem"uter der Augenzeugen. Ein Meteor dieser Art kam im 1. 1789 "uber eine Gegend unweit Worms, woselbst ich mich auf einem, meiner Familie geh"origen Schlosse befand. Im Sommer bei heiterem ganz wolkenlosem Himmel entstand gegen Abendzeit ein immer zunehmendes Rauschen in der Luft, dass nicht sowohl dem Rollen des Donners, als dem l"armenden Zug eines Kriegsheeres- zu vergleichen war, und immer zunahm, je n"aher es unserem Wohnsitze kam. Die erschrockenen Landleute, die eben noch auf dem Felde waren, liefen, ein nahes Ungl"uck ahnend, ihren H"ausern zu, indessen das unsichtbare Meteor (ohne Schaden zuzuf"ugen) "uber unsere H"aupter hinrollte; alle Bewohner des Schlosses vernahmen dessen wunderbare Laute; aber unter den Anwesenden befand sich keiner im Falle, dies Ph"anomen gleich nach seiner Erscheinung zu verfolgen; erst nach einiger Zeit erfuhr man, dass eine Feuerkugel nicht fern des Ortes niedergefallen sei, nach der man sich gleichfalls nicht weiter umsah, die aber, den seither "uber Aerolithen bekannten Tatsachen nach, unbezweifelt ein Meteor-Stein war.

Beispiele der Art lassen mutma"sen, welche tiefe Eindr"ucke von Furcht und Erstaunen solche Ph"anomene in fr"uherer Zeit auf rohe Menschen, die jede au"serordentliche Erscheinung als z"uchtigende oder Schaden bringende Wirkung einer Gottheit ansahen, erregen mu"sten.\footnote{Noch gegen Ende des 17ten Jahrhunderts wurde ein in der Ortenau gefallener Meteor-Stein f"ur ein sichtbares Zornzeichen des Himmels gehalten In einem seltenen B"uchelchen (gedruckt 1671) wird diese Erscheinung erz"ahlt. (Herr Gilbert hat sie im 1ten St"uck der physischen Annalen f"ur 1809 mitgeteilt.) Der Berichterstatter sagt unter andern: "`dass dieser Stein wie die Donnerfeile in der Luft generiert worden, werde ich mich schwerlich "uberreden k"onnen, weil er ein mineralisch Erz zu haben scheint, und nicht, wie andere dergleichen Steine, die frisch bekommen werden, nachdem sie herunter gefallen, nach Schwefel gerochen, oder hei"s gewesen; sondern will viel ebender zugeben, dass diese Steine, weil man sie an unterschiedenen Orten so weit von einander geh"ort, aus Verh"angnuss Gottes vom b"osen Geist und seinem Anhang auf Erden gesammelt, in die Luft gef"uhrt, und von da wieder zerstreut worden."' --- Dieser Meinung gem"a"s, h"alt der Verfasser sie demnach f"ur ein Prognostikon der steinern T"urken Herzen und grimmigen Hundes Art, die sie gegen das teure Christenblut zu ver"uben pflegen! In der Beschreibung dieses Luft-Steines verdient besonders bemerkt zu werden, dass einer der Anwesenden erz"ahlt: Er habe etwas "uber sich hinausfahren sehen, wie eine gl"uhende Kugel, davon er niedergesunken; ein anderer sah etwas vom Grunde "uber sich spritzen, fand daselbst ein Loch, und den darin liegenden Stein anderthalb Fu"s tief. Worauf der Verfasser jener Beschreibung die Hypothese begr"undet, dass dieser Stein aus der Erde in die Luft geschoben, und dann herabgefallen sei. Eine Meinung, die viel "Ahnlichkeit mit jener neuen des Herrn Patrin (Annalen der Physik, 10. St"uck f"ur 180. S. 1891 hat, der die Bildung der vulkanischen Materien "uberhaupt aus einer chemischen Verbindung der gasf"ormigen im Innern der Erde zirkulierenden Fl"ussigkeiten erkl"art, welche durch die mineralische Assimilation zu Steinen und Metallen werden; denen "ahnlich, von welchen man annimmt, dass sie auf nassem Wege gebildet worden find. --- Der angef"uhrte Meteor-Stein wog 10 Pfund, war, wie alle "ubrigen, an Farbe "au"serlich schwarz, von Innen grau, etwas l"ochericht, wie mit dem Finger eingedr"uckt, seine Form beinah die eines Hundskopfs ohne Ohren u. s. f.} Selbst Blitze, sagt Plinius,\footnote{Nat. Gesch. Buch 2. Kap. 53.} glaubten die R"omer, w"urden von neun G"ottern geworfen, und es g"abe derselben elf Arten, davon drei dem Jupiter zugeh"orten; aber nur zwei blitzende Gottheiten g"abe es, der Zeus, welcher am Tag, und der Summanus, der bei Nacht die Blitze wirft. --- Auch Steinregen (sagt derselbe Naturkundiger) sendet die Gottheit "ofters herab; ein Beispiel liefert er in der Erz"ahlung des bekannten Steinregens, der sich w"ahrend dem Konsulat des Marius ereignete\footnote{S. Nat. Gesch. Buch 2. Kap. 58. "`Man bat uns erz"ahlt, dass zu Zeit des zimbrischen Kriegs, und noch "ofter, sowohl vor- als nachher, ein Ger"ausch der Waffen und der Schall einer Trompete geh"ort worden sei. Im 3ten Konsulat des Marius sahen die Ameriner und Tudertiner Waffen am Himmel, die vom Morgen und Abend her so lange gegeneinander fuhren, dass die auf der Abendseite zur"uckgetrieben wurden."' --- Im folgenden Kapitel erw"ahnt Plinius dreier, ihm bekannt gewordener Meteor-Steine: einen, den Anaxagoras im 2. Jahr der 78. Olymp. vorhergesagt, dass er aus der Sonne fallen w"urde, und wirklich fiel er zu Aegospotamos (desselben erw"ahnt Aristoteles in Meteorologia 100. 7. Δ) in Thrazien nieder; er war von der Gr"o"se einer fahrbaren Last, schwarz an Farbe, und an seiner Kruste angebrannt, weil, sagt Plinius, in der Nacht, da er fiel, eben ein Komet brannte; --- ein anderer im Gymnase zu Abydos; ein dritter zu Cassandria im Makedonischen; endlich einen, den er selbst kurz nach seinem Falle gesehen, im Vocontischen Gebiete. "Ubrigens hat von Steinregen, die in fr"uherer und sp"aterer Zeit fielen, Chladni ein, soviel m"oglich, vollst"andiges Verzeichnis gegeben in Gilberts Annalen der Physik 15. 310., worauf ich, um bekannte Dinge nicht zu wiederholen, verweise.}; und am Schlusse dieser Schilderung, Kap. 59, wo er von jenem Meteor-Steine spricht, den Anaxagoras, wie die Sage erz"ahlt, l"angst vorherberechnet hatte, sagt er bestimmt, dieser Aerolithe werde im Gymnas zu Abydos in besonderer Ehre gehalten. --- Colitur (ist sein Ausdruck), welches Gro"se, sein deutscher "Ubersetzer, wie mich d"unkt, unrichtig mit aufbewahrt ausdr"uckt, da in der n"achstfolgenden Periode es hei"st: und die Kolonie, welche man Potid"aa nennt, des Steines (n"amlich der Verehrung wegen), so diese Kolonie dem Stein bezeigte, der wahrscheinlich ihr Lokal- und Schutzgott war, hierher gef"uhrt ward.\footnote{Von Steinregen, die zu verschiedenen Zeiten sich ereigneten, verdient besonders jener im Jahr Christi 769 gefallene erw"ahnt zu werden, dessen Aboulfaradi oder Gregorius-barhebraeus in seiner syrischen Chronik gedenkt, und der aus schwarzen Steinen bestand. Von einem andern im Jahr Christi 893 gefallenen spricht derselbe Schriftsteller. Beide best"atigen die Mutma"sung, dass die schwarzen Steine auf dem sogenannten Gileads-H"ugel in Syrien (einer Gegend, wo sich gew"ohnlich nur wei"se Kalksteine vorfinden) gleichfalls Aerolithen oder Produkte eines Steinregens sind. S. Silvestre de Sacy Noten zu seiner franz"osischen "Ubersetzung von Abd-Alatifs (eines arabischen Arztes aus Bagdad) Beschreibung "Agyptens. Paris 1810. 4. p. 505.}

Aber fr"uher noch, als diese Epoche, wurden heilige Steine, Διιπετρα, verehrt, die der Beschreibung nach alle Eigenschaften wahrer Aerolithen hatten. Aus Sanchuniaton, den Eusebius\footnote{Praepar. Evang. 50. 1. 100. 10.} anf"uhrt, lernen wir, dass die G"ottin Astarte einen Stein fand, der als Stern vom Himmel gefallen war, und nachdem sie ihn aufgehoben hatte, denselben der Stadt Tyrus weihte.\footnote{Man erinnere sich hierbei, dass Astarte oder Alilat die G"ottin war, die nach Herodot B. 3. K. S. die Araber unter dem Bild eines Steines verehrten.} Man liest ferner in einem alten, dem Orpheus zugeschriebenen Gedichte: Von den Steinen, die Beschreibung eines Steines, Ophites genannt,\footnote{Ille enim Phoebus Apollo lapidem vocalem habendum Sideriten verum dedit, quem hominibus aliis placuit vocare anima carentem Ophiten funestum, subasperum, durum, nigrum, spissum circa ipsum vero circulo ab omni parte undique fibrae rugis similes, insculptae extenduntur.\\
Orph. λιθιχα v. 16-21.\\
Edit. Gesneri. Lipsiae 1764.} der (wie es darin hei"st) der wahre Siderit sei. Apollon (so erz"ahlt das Gedicht) gab diesen Siderit (Stern-Stein) --- anders nennen ihn Ophites --- dem Trojaner Helenos; der Stein, der die Gabe zu reden hatte, ist hart, schwer an Gewicht, und schwarz an Farbe. Viele im Kreis laufende Streifen sind auf seiner "au"seren Rinde zu sehen. Helenos gebrauchte denselben mittelst verschiedener bei Beschw"orungen "ublicher Gebr"auche zu Bezauberungen und Wahrsagerk"unsten, wodurch er, wie es hei"st, den Untergang Trojas vorgesagt habe.\footnote{S. Falconet sur les Baetylos in den Mém. de l'acad. des Inscr. T. 6. p. 514.} Ein anderer Stein, den Plinius Astroides\footnote{Plin. Hist. nat. 1. 37. 100. 9.} nennt, und dessen sich, wie er hinzusetzt, Zoroaster zu seinen magischen K"unsten bediente, ist unbezweifelt ein Aerolith, dem man magische Kr"afte zuschrieb. In denen von ihm "ubrig gebliebenen Orakeln wird vorgeschrieben, einen solchen dem Himmel entfallenen und Gott geheiligten Stein zu opfern, so oft sich ein b"oser Daimon der Erde n"ahere; und im Leben dieses Weltweisen sagt Porphyr: Als der weise Perser sich in der Insel Kreta befunden, sei er vom Priester Morgos, einem der id"aischen Daktylen mit einem Donnersteine zur Einweihung bereitet worden, eine Sage, die umso m"oglicher ist, da ihr ein naturhistorisches Factum zum Grunde liegt, und es scheint, dass in fr"uhesten Zeiten auf dem Ida ein Steinregen gefallen sein m"usse, da man auf dessen Gipfel nach Plin. Nat. Gesch. 37. B. K. 61. id"aische Daktylen oder fingerf"ormige Steine von wei"sblaulichter Farbe in gro"ser Menge findet, woraus sich leicht der sp"ater entstandene Mythos von Zeus Geburt am Ida erkl"aren l"asst. Denn in Kreta (nach Dik"aarch die "alteste Insel) wohnten die Eteo-Creter oder Id"ai Daktili, eingewanderte Kolonisten aus Phrygien, die sich besonders um den h"ochsten Berg der Insel ansiedelten, und ihm den Namen des phrygischen Ida gaben. Von Cres, ihrem K"onige, nahm die Insel den Namen. Die Id"ai Daktili aber brachten die erste Kultur sowohl als den Dienst der gro"sen G"ottin dahin, und der Cabiren; die ersten Waffen wurden von ihnen erfunden; dieser Eisen-, vielleicht auch Magnetreiche Berg, der eine Wetterscheide der Gegend war, woran sich "ofters meteorologische Ph"anomene zeigten, musste daher der Schauplatz gro"ser Naturerscheinungen sein, und der Himmel (dem Mythos gem"a"s), als erzeugender Vater, kam -- sich verbindend --- zur m"utterlichen Erde; der junge Zeus fand hier seine Wiege, und die Korybanten, die ersten Priester der Insel, pflegten seiner Kindheit; was Uranos aber aus seiner H"ohe durch Meteore herabsandte, wurde als Erzeugnis von ihm angesehen; so waren auch die daktylischen Steine von ihm gesandt worden. Ob sie nun den ersten Bewohnern, oder diese jenen den Namen gegeben, bestimmen die Nachrichten nicht, aber das sagen die "altesten Geschichtsschreiber: Scepsius, Pherecides,\footnote{S. Hermanns Handbuch der Mythologie 3. Th. S. 160 u. f.} Herodot,\footnote{Buch 3. 37.} und sp"aterhin Diodorus,\footnote{Buch 5.} dass diese id"aischen Finger (Daktylen), die f"ur Zauberer galten, und sich auf Magie, Ordens-Einweihungen, geheime Wissenschaften legten, wodurch sie die wilden Bewohner aller L"ander, zu denen sie sich begaben, in Erstaunen setzten, zuerst den Gebrauch des Feuers, des Eisens und die Bearbeitung desselben erfanden, weshalb sie f"ur S"ohne des Vulkans galten, und verdienten als G"otter verehrt zu werden. Am Ida, den sie bewohnten, waren auch die ersten Erzgruben, und des Berges Schoos barg einen reichen Vorrat des ergiebigsten Eisens. --- Von des Zeus Geburt, Jugend und zarter Pflege durch die Nymphen in einer H"ohle des Berges liefert die Sage die lieblichsten Mythen, denn das Kind, gef"uttert mit Milch und Honig, an der Brust der Ziege Amalthea s"augend, ward von freundlichen Bienen gen"ahrt, die, wie Diodor (Gesch. Buch 5. Kap. 70.) erz"ahlt, der Gott, um das Andenken seiner Geneigtheit gegen dieselben zu bewahren, in ihrer Farbe einem goldgelben Erze "ahnlich machte, und sie so verwandelte. -- Also nicht in Gold selbst, wie die Worte deutlich sagen, sondern in ein dem Golde an Farbe "ahnliches Erz, worunter, wie mir scheint, kein anderes zu verstehen ist, als der bei allen Erzg"angen so gew"ohnliche Schwefelkies, Markasit, den Agricola, seiner Analogie mit dem Magnet wegen, mica Magnetis nennt, und Langius in Hist. Lap. Helv. p. 21: μεταλλολιθος. Da die Biene aber das Bild der Arbeit ist, so scheint die Mythe in dieser Allegorie die flei"sige Ausf"orderung des Eisens, eines Geschenkes, das der Gott dem Berge gab, symbolisiert zu haben. Vielleicht hat es auch Bezug auf das Feuer selbst, das dem Himmel oder Zeus heilig war, denn die Alten bedienten sich des Markasits zum Feuerzeug, gleich anderen Kieselarten, weshalb er den Namen Pyriten (Feuerstein) erhielt.\footnote{Versuch einer Lithurgik oder "okonomischen Mineralogie von Schmieder. Leipzig 1804. 2. Teil, S. 524.}

Mehrere Zeugnisse "uber dem Himmel entfallene und deshalb f"ur heilig gehaltene Steine lie"sen sich aus alten Schriftstellern anf"uhren.\footnote{Man lese vorz"uglich Bochard Canaan 50. 2. Selden de Dissyris, Zoega de Usu et Orig. Obelisc. Falconet in der angef"uhrten Schrift.} Wir eilen jedoch unserem Zwecke n"aher, indem wir den eigentlichen Namen ber"uhren, den sie im Altertume trugen.

B"atylien hie"sen sie, ein Name, der nicht urspr"unglich griechisch, sondern, wie Eusebius\footnote{Praepar. Evangel. 50. 1. 100. 10. S. auch Hesychius, Priscian, das Etymologicon.} nach Philo von Biblos, dem "Ubersetzer des Sanchuniaton, zeigt, ph"onizisch ist; belebte Steine nennt sie dieser Geschichtsschreiber, und sagt, der Gott C"olus habe sie mit k"uhner Kunst gebildet; aber Bet"ul war einer der vier Kinder des Himmels (Uranos) und der Gea (Erde), deren drei andere Saturn, Dago und Atlas hie"sen.

Diesen B"atylos (Sohn des Himmels), der Ph"onizier Gott, muss man daher sorgf"altig von den kleineren B"atylien unterscheiden, die, wie der Aberglaube w"ahnte, von kleinen Gottheiten bewohnt werden, welche ihnen magische Kr"afte mitteilten.

Den Namen B"atylos leitet Bochard\footnote{Canaan 50. 2. Cap. 2. p. 75.} vom Steine Jacobs, den er auf seiner Flucht nach Mesopotamien an einem Orte des Libanons fand, und nachdem er ihm in der Nacht, wo er den bekannten Traum hatte, zum Hauptk"ussen gedient, ihn\footnote{Genes. 28. 18. von diesem Gebrauch heilige dem G"otterdienste geweihte Steine zu salben. S. Mnuc. Felix Edit. Lugduni 1672. p. 15.} (einem alten urspr"unglichen Gebrauche nach) mit Oel salbte, zugleich ausrufend: Der Herr ist wahrhaft an diesem Orte, und ich wusste es nicht, und den Ort, der vorher Luz hie"s, Bethel nannte, das ist: Haus des Herrn, wovon der Stein den Namen erhielt, und bei den Nachkommen in besondere Verehrung kam, die aber bald bei den nomadischen Kananiten in Abg"otterei ausartete, weshalb er auch, wie die j"udische Tradition erz"ahlt,\footnote{Quanquam ille Cippus amatus fuit a Deo temporibus Patriarcharum, postea tamen edit eum, propterea quod Chananaei deduxerunt illum in ritum Idolatriae. Bochard Canaan p. 785.} als Gott zuwider durchs Gesetz verworfen ward.

Nicht unfern von diesem Orte zu Gilead,\footnote{Nach den mosaischen Nachrichten die eigentlichen Vor-Alpen des Libanons.} dem Steinh"ugel oder Steinberge, hatte fr"uher schon Jacob, eh' er von seinem Schwiegervater mit S"ohnen und Frauen schied, Steine zu einem Haufen gesammelt, die der eine Sahadutta, der andere Gilead nannte, aber Jacob hatte zuvor einen einzigen Mahlstein errichtet. Genes. 21 --- 45. Dieser Stein muss von betr"achtlicher Gr"o"se gewesen sein, indem er ihn als S"aule aufrichtete, die unbeweglich stand, und zu nichts anderem, als zum Altar dienen konnte; so sagt die Tradition, und setzt hinzu: seine Farbe sei schwarz gewesen.

Wenn hieraus auch nicht zu erweisen ist, dass es derselbe Stein sei, der, wie die Moslemin vorgeben, noch heutzutage sich in der Kaaba zu Mekka befindet, so ist doch so viel sicher, dass die Verehrung, die man ihm seit den Zeiten dieses Patriarchen bezeigte, dem Aberglauben Veranlassung gab, die gro"sen schwarzen Steine, die man unter verschiedenen Benennungen im Orient verehrte, f"ur diesen Stein auszugeben, dessen Heimat Syrien und der Libanon ist, dem er gleichwohl als eigent"umliche Steinart nicht zugeh"oren kann; denn wie ein unterrichteter Reisender,\footnote{Volney Reise nach Surien und "Agypten, deutsche "Ubersetzung Th. 1. S. 224 u. s. f.} dem wir in aller Hinsicht glauben d"urfen, berichtet, besteht die Grundlage dieses Gebirges, wie "uberhaupt von ganz Syrien, aus einem harten, wei"slichten kieselartigen Kalksteine, so dass die schwarzen gr"o"seren und kleineren Steine, die zu verschiedenen Epochen sich dort vorfanden, dieser Gegend fremd waren, und au"ser-Tellurisch durch irgend ein Meteor dahin gebracht sein m"ussen. Steinregen und dergleichen Ph"anomene scheinen "uberhaupt zu jener Zeit in Ph"onizien und Syrien sich "ofters (ob aus atmosph"arischen\footnote{Was dadurch begreislich wird, dass diese Bergkette als Hauptwetterscheide jener Gegend, wie Volney sagt, in meteorologischer Hinsicht "au"serst merkw"urdig ist.} oder anderen Ursachen) ereignet zu haben. So jener zu Josua Zeiten: "`Gott lie"s bis gen Aseka gro"se Steine "uber sie regnen, so dass durch diesen Steinhagel eine gr"o"sere Zahl Kanaaniten umkam, als durch das Schwert der Israeliton,\footnote{Josua 10. 11.} und der Steinregen bei Mose 28. a4., womit der Herr sein Volk bedrohet: "`der Herr wird deinem Volke Staub und Asche f"ur Regen geben vom Himmel, bis du vertilget werdest."`

Jener Jacobs-Stein, der wahrscheinlich doch nicht mehr wog, als die 14 Zentner schwere Eisenmasse, welche in Sibirien niederfiel, ist den vorerw"ahnten Umst"anden gem"a"s unbezweifelt ein Aerolithe gewesen, und die Verehrung, die man ihm (seines Ursprungs wegen) bezeigte, artete in sp"ateren Zeiten durch Aberglauben in Abg"otterei aus.

Wie nun diese Steinmasse als Symbol einer Gottheit auf den Spitzen des Libanon stand, so verehrte man "ostlich an Indiens Grenzen zu. Nepal\footnote{In Dappers Asia steht p. 111 nach Dèlla Vallès Beschreibung eine treue Abbildung des Mahadeva (oder Gott der Zeugung) als Lingam. S. ferner Account of the Kingdom of Nepal im 2. Th. der Asiatik Researches, 8te Ausgabe S. 307, von einem im n"ordlichen Europa niedergefallenen Aerolithen liefert Bartholin eine Beschreibung in ist. Anatom. Centur. IIl. et 4. p. 337.} (unfern Benares) den schwarzen Stein als Bild des Mahadeo (den Gott der Liebe und Zeugung), auch in Cachemir verehrte man einen vom Himmel gefallenen Stein, einen anderen als Lingam in der Pagode von Perwuttum; so wie westw"arts in Griechenland den viereckigten schwarzen Stein Saturns, der zu Pausanias Zeiten noch im Apollo-Tempel zu Delphos bewahrt,\footnote{Pausan. Griechenland B. 10., Kap. 24.} t"aglich mit Oel bestrichen und roher Wolle umwickelt wurde. Dahin kann man gleichfalls jenen Stein von Pessinunt z"ahlen, welcher der Cybele heilig war, und im zweiten punischen Krieg nach Rom gebracht wurbe.

Es zeigten sich (sagt Appian vom Hannibalischen Kriege, Kap. 56.) zu jener Zeit zu Rom schreckliche Wunderzeichen am Himmel, weswegen die zehn M"anner die Sibyllinischen B"ucher nachschlagen mussten, und aus denselben antworteten:

"`Es werde in jenen Tagen zu Pessinus in Phrygien, wo die Mutter, der G"otter verehrt wird, etwas vom Himmel fallen, dass man nach Rom bringen m"usse;"' nicht lange darnach sei die Nachricht gekommen, es sei wirklich herabgefallen, worauf denn das Bildnis der G"ottin nach Rom geholt wurde, an welchem Tage die R"omer (setzt Apian hinzu) wirklich noch das Fest der Mutter der G"otter feiern. --- Die Gr"o"se, Form und Farbe des Steins beschreibt Arnobius advers. Gentes 50. 6. et 7. allatum ex Phrygia --- - --- nihil quidem aliud nisi lapis quidam non magnus, ferri manu hominis sine ulla impressione, qui posset, Coloris fulvi atque atri, angulis prominentibus inaequalis, at quem omnes hodie ipso illo videmus in signo Oris loco positum indolatum, et asperum et simulacro faciem minus expressam simulatione praebentem. --- Die Worte: in Signo Oris, zeigen, dass der Stein einem Munde glich, und man ihn deswegen an des Mundes Sselle ins Antlitz der G"ottin einfasste, wodurch, dem geheimen Sinn nach, die Bilds"aule diejenige Gottheit wurde, die man im Stein verborgen glaubte die Orakel gingen aus diesem Munde (dem geheiligten Steine) hervor, welches Prudentius in folgenden Versen beschreibt: Lapis nigellus evehendus essedo Muliebris Oris clausus argento sedet. Diesen Woxten Muliebris Oris gem"a"s, auch des Umstandes wegen, dass der- Stein von sehr unbetr"achtlicher Gr"o"se\footnote{Welches jedoch Banier in einer Abhandlung "uber die Mutter der G"otter im 5. Band der Mém. de l'acad. des Inscr. p. 244. leugnet, haupts"achlich aus dem Grunde, weil, w"are der zu Pessinunt gefallene Stein von unbetr"achtlicher Gr"o"se gewesen, er nicht so leicht bemerkt, und in fernen Gegenden bekannt worden w"are. Vielleicht aber fielen mit demselben noch mehrere kleine, und einer dieser minder gro"sen Steine konnte ja nach Rom gebracht worden sein. -- Zugleich bemerkt Banier, dass die pessinuntischen Priester der gro"sen G"ottin auf dem G"urtel, der sich um den Leib schloss, kleine geweihte B"atylien trugen; ob diese gleichfalls vom Himmel gefallen waren, wird nicht angegeben. --- Aber von jenem Hauptsteine der G"ottin sagt auch Arnobius, er sei von betr"achtlicher Gr"o"se gewesen, und habe (der alten Sage nach) f"ur jene Felsmasse gegolten, von der Deukalion die Steine abschlug, aus denen er Menschen bildete; der nach Rom gebrachte muss also doch wohl von kleinerer Art gewesen sein.} gewesen sein m"usse, indem bei seiner Ankunft in Rom im Jahre 548 die r"omischen Damen ihn, wie Livius versichert, wechselseitig von Hand zu Hand bis zum Tempel der Victoria trugen,\footnote{In terram elatam tradidit (scipio Nasica) ferendam Matronis --- - --- eae per manus, succedentes aliae aliis, in aedem viotoriae pertulere. Liv. 17. v. 16.} h"alt Falconet (in den Mém. de l'acad. des Inscr. T. 23. Dissert. de la Mere des Dieu) diesen Stein f"ur einen Hysteriolithen der "Ahnlichkeit wegen mit einem Munde, welche Form dem fr"uhen Aberglauben Anlass gegeben habe, ihn als den Mund einer G"ottin zu verehren, der esoterische Cultus aber habe darunter die Natur, als Urquelle aller Wesen symbolisiert, in welchem Sinne auch Iren"aus von den Kainiten, einer christlichen Sekte der fr"uheren Jahrhunderte, sagt: Cainiti Hysteram fabricatorem Coeli et Terrae vocant. 50. 1. contr. Haereses 100. 35. wie dasselbe Wort zuweilen auch f"ur die Mutter aller Wesen galt. "Ahnliche B"atylien (wie Salmasius zu Lambridii Helo gabal. Edit. Lugd. Battav. 1671. T. 1. p. 801 zeigt) waren auch anderen Gottheiten, dem Jupiter, Saturn, der Sonne geweiht, und von derselben Gattung war jener Stein, der im Eingang des Tempels der Diana zu Laodic"aa stand.

Abadir,\footnote{Bochard Chanaan 50. 2. 100. 2. p. 786. leitet den Namen vom Ph"onizischen: Eben-Dir oder: Aban-Dir Lapis sphaericus, Talis enim (setzt er hinzu) Boetyli forma.} der gro"se m"achtige Herrscher, -- Pater magnus, der Gott des Berges, hie"s auch dieser Steingott, und die Nabat"aer (ein arabischer Stamm) verehrten ihn unter dem Namen Dusares, Teusares,\footnote{Suidas sagt hievon: "`Theusares --- Dusares, id est Deus Mars qui Petrae in Arabia maxime colitur. Simulacrum ejus est Lapis niger, quadratus, informis, altus pedes sex, latus duo, et aurea basi impositus."' Weitl"aufiger hier"uber Zoega de usu et orig. Obelisc. p. 205.} dessen Cultus sp"ater sich bis nach Gro"sgriechenland verbreitete, unter dem Namen Abadad in Persien, Alassovid bei den Arabern, kommt er noch nach, Mahameds Zeiten vor, und in vielen St"ammen des s"udlichen Arabiens bei den Patrinsern, Adr"anern, Bostrenen, Dacherauern wurden, wie bei den Nabat"aern, B"atylien oder Steing"otter verehrt. In der alten Coelo- Syrien-Stadt Emesa,\footnote{Heut zu Tage Hems oder Hims. S. Mannert Geogr. der Griechen und R"omer. 6. Th. 1. Heft, S. 458.} wo nach Pococks Bericht in der N"ahe sich sine Menge schwarzer Steine finden, woraus sich allm"ahlich ein H"ugel formte,\footnote{Gleich jenem gro"sen Steine zu Bethel, umgehen von einem H"ugel kleinerer B"atylien.} bildete sich, nach Strabo, sehr fr"uhe schon ein Sammelplatz der Verehrung mehrerer arabischen St"amme, welche die Sonne unter dem Bilde eines an jenem Orte aufgefundenen schwarzen runden, spitzig zulaufenden Steines anbeteten, dem in der Folge ein pr"achtiger mit Gold und Silber ausgeschm"uckter Tempel erbaut wurde.\footnote{Herodian 5. 100. 5.} Heliogabal, Gabal, Alagabal, auch Malach Bilos, und zwar wie Selden vermutet, von Moloch-Bel oder Baal, hie"s dieser Gott; Selden, der gleich Bochard\footnote{Chanaan 50. 2. 100. 2. p. 797.} denselben mit Recht als ph"onizisch annimmt, leitet den Namen von Ahgol-Baal\footnote{Selden de Dis Syriis Syntagma 2. p. 220 et seq.} oder Agalibal, dem runden zirkelf"ormigen schnellen Gott\footnote{Deus rotundus, circularis aut volubilis, ut dicebant ii, qui sphaeram mundi Deum sentiebant apud Ciceronem de Nat. Deor. 2.} her, und ist der Meinung, nicht die Sonne, sondern Zeus oder Jupiter sei unter diesem Symbol verehrt worden. Andere setzen den Uranos, andere Bel an die Stelle, es sei nun dieses, oder wie es wahrscheinlicher und allgemein angenommen ist, die Sonne selbst der Emeser Gott gewesen, immer werden wir auf das Urelement Feuer zur"uck gef"uhrt, und der Meteor-Stein, dem h"oheren "Ather entfallen, in seiner sph"arischen Pyramidalform ist ein schickliches Symbol der Gottheit; indem, wie wir aus Platon wissen, die Pyramide die erste urspr"ungliche Form, als der Urstoff des Feuers angesehen ward (s. Plutarch "uber den Verfall der Orakel) und die Kugel oder Sph"are eines der "altesten Symbole der Gottheit war,\footnote{Von der sph"arischen Form des Alls, oder der Welt, und dem Kreise als Bild der Gottheit s. Aristoteles de mundo 50. 2. 100. 14. D. Meteorologia 100. 7. item 50. de Zeone et Gorgia. --- Plutarch de Plac. Phil. 100. 6 --- Auch den "Agyptern war die Kugel das Symbol des Universums. S. Kircheri Sphynx p. 25. --- Dass noch jetzt in Indien die Gottheit unter der Gestalt einer Kugel verehrt wird, lesen wir in Haafners Reise nach der K"uste Koromandel:
\hspace*{0.5cm} "Am Meeres Ufer sahen wir einen alten verlassenen Tempel, der wahrscheinlich einst dem Ischuren oder allerh"ochsten Wesen geweiht war, denn man sah an dem Geb"aude keine Gottheiten abgemalt oder ausgehauen. Die indischen Pundits sagen, das h"ochste Wesen, das sich in zahllosen Werken offenbart, sei so erhaben, dass es nicht durch Figuren dargestellt werden k"onne. Da diese Figur keine und doch alle Gestalten hat, so wird sie unter dem Bilde einer steinernen Kugel auf einem Fu"sgestell in der Mitte des Tempels vorgestellt, nie wird dieses Symbol in Prozessionen umhergetragen, die Tempel haben keine T"anzerinnen, und "offentliche Feierlichkeiten werden in ihnen nicht angestellt; man opfert diesem Wesen nichts als Feldfr"uchte, und der Dienst, den die Brahminen in seinem Tempel versehen, besteht blo"s in Lobges"angen und Gebeten. --- Dies ist das einzige allerh"ochste Wesen der Indier, das Breem Brrm, Gott, oder das h"ochste Wesen, der Tausendnamige hei"st.} wozu in Hinsicht der Steinmassen, die auf unsere Erde herab fallen, noch in Erw"agung kommt, dass Diogenes die Sterne f"ur bimsteinartige und gl"uhende Steine h"alt,\footnote{Plutarch de Placit. Philos. 100. 13.} die oft auf die Erde herab fallen und da verl"oschen; und Anaxagoras (der uralten orientalischen Tradition gem"a"s) behauptete, der die Erde umgebende "Ather sei seiner Natur nach feurig, und rei"se durch die Heftigkeit seines Umschwungs Felsenst"ucke von der Erde mit sich fort, die er mittelst der Entz"undung in Sterne verwandle. Zu dieser Meinung ward Anaxagoras wahrscheinlich durch jene im zweiten Jahr der 75. Olympiade zu Aegos Potamos in Thrazien niedergefallene Steinmasse, deren Fall er, einer alten Sage nach, berechnet und vorhergesagt, bewogen. Mehrere Schriftsteller versichern einstimmig, nach Plinius Worten: \emph{Praedixisse Coelestium Litterarum scientia quibus diebus saxum casurum esse e sole, idque factum inter diu} --- sollte man denken, es sei dabei eine Sonnenfinsternis, oder sonst ein Himmels-Ph"anomen vorgefallen; da aber die Geschichte hier"uber schweigt, ist diese Vorhersagung wohl zu jenen Dichtungen und Traditionen zu z"ahlen, die im Altertume so h"aufig sind. Von diesem Steine gibt auch Diogenes Laert. 2. 10. Zeugnis: φασι δ'αυτον προειπειν την περἰ Αιγος ποταμον γενομενιν τοῦ λιθοῦ πτωσιν, ὀν ειπεν εκ του ηλιου πεσεισθαι. So merkw"urdig war derselbe dem Altertume, dass selbst der Parische Marmor seiner erw"ahnt:
\vspace{1pt}
\\
αφ' ὁυ εν Αιγος ποταμοις\\
ὁ Λιθος επεσε --- ---\\
\vspace{1pt}

Aus allen Zeugnissen, die wir im Aristoteles,\footnote{In den B"uchern der Physik, Meteorologie, de mundo.} Plutarch\footnote{de Placit. Philos.} und Stob"aus\footnote{Stobaei Ecclog. physicae 100. 15. 1.} gesammelt finden, erhellt der allgemeine Glaube der alten Welt an die Feuer-Natur der Gestirne, und die Meinung, dass teils erloschen, teils als Lichtbothen sie zuweilen zur Erde sinken, "uberhaupt ihrer Natur nach freundlich leuchtende, leitende Wesen sind, g"unstig dem Wanderer und Schiffenden auf unbekanntem Meere. Das waren in den "altesten Zeiten schon die Plejaden, die hellen Morgen- und Abendgestirne, besonders die himmlischen Dioskuren, deren Feuer-Gewalt in der Atmosph"are so m"achtig wirkte, dass ein alter Mythos sie mit den Cabiren vermischend (wahrscheinlich ihrer Feuer-Natur wegen), f"ur S"ohne des Vulkans und der Cabeira, die selbst eine Tochter des Proteus war,\footnote{Handb. der Mythologie von Herrmann 3. Th. S. 172.} ausgab. --- Wie nun herabgefallene Steine f"ur erloschene leblose Sterne galten, wurden auch die Feuer- und Lufterscheinungen, die in der Atmosph"are teils auf der Erde, teils im Meer sich zeigten, als meteorische, den Gestirnen zugeh"orige oder entsunkene Teile angesehen, die der Mythos bald in Sterne, und, da Sterne g"ottlicher Natur waren, in G"otter verwandelt, so bewirkten, nach Diodors Erz"ahlung, die wohlt"atigen Zwillingsbr"uder bei drohendem Sturme die Rettung ihrer Gef"ahrten der Argonauten durch zwei Sterne, die auf ihre K"opfe fielen, und dies, f"ugt Diodor\footnote{Im 4ten Buch 100. 43. der Geschichte. Diese Sterne aber waren nichts als elektrische Funken, die bei Meeres St"urmen "ofters vorkommen, und heutzutage unter dem Namen St. Elms-Feuer bekannt sind.} hinzu, gab die Veranlassung, dass Seefahrende, die Sturm litten, den samothrazischen Gottheiten (in deren Geheimnisse Castor und Pollux eingeweiht waren) Gel"ubde taten, und die Erscheinung der Sterne als ein Sichtbarwerden der Dioskuren. betrachteten. Am samothrazischen Seehafen standen ihre Bilds"aulen als Schutzg"otter, mit deren Errichtung es folgende Bewandtnis hat: vor ihnen, in den fr"uhesten Zeiten standen an derselben Stelle zwei Statuen des Himmels und der Erde, die gro"sen cabirischen oder ph"onizischen Gottheiten, deren Varro de Ling. lat. Lib. 4.\footnote{Principes Dei Coelum et terra, und hins zuf"ugt: sunt Tautes et Astarte apud Phoenicos, ut idem principes in latio Saturnus et Obs. Terra enim et Coelum ut Samothracum initia docent, sunt Dei magni et hi, quos dixi multeis nominibus --- nam neque ut vulgus putat, hi Samothraces Dii, qui Castor et Pollux. Sed hi Mas et foemina --- --- Divi potes, et sunt pro illeis, qui iis Samothrace haec sunt Coelum et Terra.} erw"ahnt. Dieser Gottesdienst, urspr"unglich aus Phrygien und Thrazien, blieb unver"andert bis zu Ankunft der Pelasger, wozu (vermutlich noch vor deren Ankunft) die Hecate, eine thrazische Landesgottheit, kam, die in Samothrazien eine ihr eigne furchtbare H"ohle ober. Grotte, Mysterien und Opfer erhielt. Zu diesen Landesgottheiten brachten die Pelasger bei ihrer Einwanderung ihre eigene Gottheiten, die Ceres, Proserpina, und die drei Dioskuren mit (denn vor den zwei sp"ateren kannte die "altere Mythe drei derselben). Diese 5 auf Samothraze eingef"uhrten Gottheiten behielten jedoch aus Achtung an dem Altertum den Nahmen der "alteren bei, und wurden wie jene Cabiren (Καβαιροι) benahmt. Den Ursprung dieser Cabiren, wie schon der N"ahme gibt, m"ussen wir in Ph"onizien suchen, denn Cabir \<kbyr> gilt bei den Hebr"aern und Arabern f"ur Gro"s, daher bei den Sarazenen, wie wir aus Cedrenii Chronicon\footnote{S. Is. Vossius de Orig. Idolatriae Lib. 2. p. 467. --- Dasselbe best"atigt der Orubische Hymus 37, und Tertullian de Spectac. 100. 8.} lernen, Cabar, Alla, die gro"se Cubar oder G"ottin, so viel als Venus, Astarte, der Abendstern hie"s. --- Gatterer, in einer Abhandlung in den G"ottingischen gelehrten Akten, der alles aus "Agypten herleitet, und auf den Kalender beziehet, hat die meisten G"otter, auch die Cabiren, vom Nile hergeleitet; aus der Ursache haupts"achlich, weil im Tempel zu Memphis ihre Bilds"aulen, die Cambyses zerst"orte, verehrt wurden; allein Herodot, der Buch 3. Kap. 37 die Sache erz"ahlt, sagt blo"s: "`Cambyses trat in den Tempel der Cabiren, dessen Eingang das Gesetz nur dem Priester gestattete, und lie"s alle darin befindliche Bilds"aulen verbrennen."' Sie glichen jener des Vulcan, dessen S"ohne, wie man sagt, die Cabiren waren. Was aber den Griechen Vulcan war, nannten die "Agypter Phthas, den belebenden Weltgeist; und die vier Unterg"otter, seine S"ohne, m"ogen sonach entweder die Elemente, oder die f"unf Sinnlichkeiten, die in der indischen und persischen Kosmologie gleichfalls erscheinen, vorgestellt haben. --- Vor dieser Stelle noch sagt er: "`auch in den Tempel des Vulcan trat Cambyses, und auf tausenderlei Weise verspottete er die Bilds"aule dieses Gottes. Sie glichen jenen Pat"aken, die die Ph"onizier an das Hinterteil ihrer dreirudrigen Schiffe befestigen, und um denen, so noch keine gesehen, einen. Begriff davon zu geben, gen"ugt zu sagen, dass sie Pygm"aen glichen; es waren aber, so viel sich von einem Gegenstande sagen l"asst, "uber den, au"ser Homer, alle alten\footnote{S. hier"uber Larchers Roten in seiner "Uberhebung des Herodot. Tome 3. p. 303.} Schriftsteller schweigen, diese Pat"aken kleine ungestaltete Figuren mit runden dicken K"opfen und B"auchen, wahrscheinlich Laren und Hausg"otter der Ph"onizier, die sie als Schutzgeister auf ihren Seereisen mit sich nahmen. Gatterer (a. a. Orte) indem er sie aus "Agypten herleitet, h"alt sie f"ur Symbole der 5 "agyptischen Schalttage\footnote{Das f"ur und wider diese Meinung s. in Hermanns Handb. 3. Th. 15.} z es mag sein, dass die "Agypter diese Tage ebenfalls unter solchen kleinen Bildern verehrten, oder ihre Steing"otterchen an die Stelle jener ph"onizischen Pygm"aen setzten, ihr Ursprung, wie der N"ahme bleibt, was auch Herodot bezeigt, nichtsdestoweniger ph"onizisch. Es waren mit der Bilds"aule Vulcans, ihres Vaters, der, wie Herodot sagt, ebenso vorgestellt wurde, dieser G"otterchen 5; aber nach einem Scholiasten des Appollonius, den Bochard Chanaan Buch 1. Kap. 12. S. 427. anf"uhrt, waren der cabirischen Gottheiten nur vier, namentlich: Arieros, Ariokersa, Ariokursos und Casmilus, die der Scholiast also deutet: den ersten auf die Demeter, den zweiten auf die Kore oder Proserpina, den dritten auf den Hades, den vierten auf Hermes; die drei ersten stimmen ganz mit der Idee der Kybele, oder gro"sen G"ottin "uberein, der vierte, Casmilus, oder Camilus (Mercur, Thot), war eigentlich ein die andern bedienender Gott. Diese Cabiren nun sind dieselbe, welche durch Ph"onizier oder Phrygier nach Samothrazien kamen. Wie gesellten sich aber zu diesen das Zwillingspaar Castor und Pollux? -- Aus Pausanias (3.-12.) wissen wir, dass sie 40 Jahre nach ihrem Gefechte mit dem Idas und Lyn"aus verg"ottert, und (wie Clem. Alex. Strom. 1. S. 382. sagt) drei und f"unfzig Jahre nach Herkules unter die G"otter versetzt worden. Ferner berichtet Pausanias 3. 26., dass die Dioskuren auf der kleinen an der K"uste Lakoniens gelegenen Insel Pephnos, (worauf, einer alten Sage nach, sie geboren waren), durch zwei kleine erzne S"aulen, die l"angst ohne Bedeutung dastanden, und wahrscheinlich von Ph"oniziern oder "Agyptern aus Dankbarkeit f"ur eine gl"ucklich vollbrachte Reise ans Ufer gestellt waren, abgebildet wurden. Diese S"aulchen waren zugleich Schutzg"otter der Schifffahrt; und da, durch diese Idee verleitet, man sie zu den Cabiren gesellte, ward man umso leichter bewogen, sie mit jenen am samothrazischen Seehafen stehenden zu verwechseln, als alle Cabiren wie Kinder, oder Pygm"aen, das hei"st: Bilderchen mit dicken B"auchen, gro"sen Munden, Augen und Ohren (Zwergen) vorgestellt wurden, wozu noch kommt, dass in Attika beide, die Cabiren und Dioskuren, den Ehren-Nahmen Anaktes\footnote{Cicero de Nat. Deorum 3. 21.} gemeinschaftlich trugen. Bald wurden beide zu Sparta, in Attika, Samothrazien, Lemnos, mit einander verwechselt, ja das Ansehen der "alteren bald durch den Dienst der j"ungeren Dioskuren, Castor und Pollux verdr"angt, so, dass wenn von Cabiren und Dioskuren die Rede war, man immer an diese als die bekanntesten dachte, und da in Ermangelung des Kompasses die Alten bei ihren Seefahrten sich blo"s nach dem Laufe der Sterne richteten, war es nat"urlich, dass die verstirnten Heroen Besch"utzer der Schifffahrt wurden, und die leuchtenden Schiffe umgebenden Meteore wurden angesehen als unmittelbar von den g"ottlichen Dioskuren gesandt.

Diesen Cabiren-Cyclus, den wir, alten Sagen gem"a"s, hier in gedr"angter K"urze sammelten, haben Mythologen und Dichter so vermischt, dass das widersprechendste Ganze daraus entstand. Kl"arer finden wir die Sache in den Historikern, besonders den "alteren vorgestellt, und nach kosmologischer Ansicht ist, je weiter wir ins Altertum dringen, die fr"uheste Vorstellung gewiss die der Natur-Weisheit gem"a"seste, n"amlich: dass die zwei fr"uheren Bilds"aulen auf Samothraze das Symbol des Himmels und der Erde, des m"annlich und weiblichen, oder (wie Varro sagt) des trockenen und feuchten Prinzips sind. Dann die drei "altesten Dioskuren, Tritopatreos, Kabuleos und Dionysios, Symbole des Einflusses der Gestirne oder des h"oheren Himmels auf die Erde; endlich die vier Cabiren der Ph"onizier [zu denen sich jene f"unf "agyptischen G"otter (gleich falls S"ohne des Hephaistos) anreihten, Symbole der Elemente: Luft, Feuer, Wasser, Erde] und wollte man das f"unfte hinzusetzen, des "Athers. Wir finden demnach, um auf den Stein-Cultus zur"uckzukehren, in diesen alten missgestalteten G"otterchen die erste Veranlassung zur Anwendung der B"atylien, in denen man die Elemente und Naturkr"afte, so wie in den Pat"aken die kugelf"ormigen leuchtenden Himmelsk"orper verehrte. Wenn man nun zu den vier urspr"unglichen Cabiren die f"unf "agyptischen Tagesg"otter reiht, und mit diesen die drei "alteren Dioskuren verbindet, so geben diese zusammen die zw"olf ph"onizischen Hauptg"otter, gleichfalls auf die Urelemente deutend, deren "alteste Abbildung in rohen B"atylien teils die Abraxas, andern Teils die Hermen veranlassten, aus denen sp"aterhin die geformteren Bilds"aulen hervorgingen.

Syrien und Ph"onizien ist, wie wir gesehen haben, die vorz"ugliche Heimat der B"atylien, und wie Emesa ein Hauptort religi"oser Vereinung, zugleich aber auch Stapel- und Sammelplatz des Handels vieler umher liegenden St"amme und V"olkerschaften geworden, so verbreitete sich auf demselben Caravanen-Wege und in den T"alern des Libanons der Sonnendienst vom Haupttempel in mehrere benachbarte St"adte, worunter die betr"achtlichsten Heliopolis (oder Balbeck) und die Palmenstadt Palmira waren, deren Ruinen noch heut zu Tag von ihrer ehemaligen Pracht zeugen, wo aber "uberall der schwarze viereckige Stein den Gott abbildete. Als die R"omer Syrien erobert hatten, ging dieser Sonnendienst nach Rom, ein Emmesenet Priester des Helagabal-Tempels (der falsche Antonin) der, wie es "ublich war, als Oberpriester den Namen des Gottes annahm,\footnote{Sacerdos Dei Solis elagabali.} jener reitzende J"ungling (wie Julian in den C"asaren ihn nennt), gleich ber"uhmt durch seine Sch"onheit und seine Ausschweifungen, bestieg den Thron des Reichs, und auf M"unzen sowohl\footnote{Les Caesars de l'empereur Julien traduits du grec par Spanheim avec des remarques et preuves, p. 46. de preuves.} als Denkmalen\footnote{Mehrere derselben in Rom und Neapel befindlich f"uhrt Selden an de Diis Syris p. 220 et seq.} wird das Bild des Gottes unter der Gestalt eines gro"sen Steins oder H"ugels auf einem Wagen ruhend, vorgestellt. Dass die Steinmasse, die den emmesener Gott darstellte, ein wahrer Aerolithe gewesen, zeugen Herodians Worte: er sei schwarz von Farbe gewesen, und, wie man versichere, vom Himmel gefallen. Den Stein lie"s Helagabal nach Rom f"uhren, von wo er jedoch nach dessen Tode wieder nach Emesa zur"uckgebracht wurde. Einen prachtvollen Tempel erbaute er ihm in der Vorstadt, und f"uhrte den Sonnendienst ein, der mitten im Sommer gefeiert wurde, er selbst blieb, was er vorher gewesen, der oberste Priester. Die gl"anzendsten Fest wurden dem Gotte gefeiert, und Herodian gibt in der Beschreibung eines ihm zu Ehren angestellten "offentlichen Umganges, eine merkw"urdige Schilderung der dabei "ublichen Gebr"auchen. Den Gott selbst, hei"st es,\footnote{Herodian Buch 5. Kap. 6. verglichen mit Lampridii vita Antonii. Heliogabali in script. rei aug. mit Casaubons, Salmasius und Gruters Noten.} lie"s er auf einen goldenen mit kostbaren Steinen besetzten Wagen setzen, und ihn darauf aus der Stadt in die Vorstadt fahren; der Wagen war mit einem Zug der sch"onsten wei"sen Pferde bespannt, mit dem reichsten Geschirr geschm"uckt. Der Gott selbst hielt die Z"ugel, denn kein Mensch durfte den Wagen besteigen; alle nur standen herum, als wenn der Gott selbst f"uhre. Helogabal ging vor dem Wagen, lief zuweilen aber zur"uck, sah die Gottheit an, zog die Z"ugel r"uckw"arts, und sah w"ahrend dem ganzen Weg die Gottheit best"andig an. Damit er nicht ansto"sen oder fallen m"oge, lie"s er die Stra"sen mit Goldstaube bestreuen, die Soldaten hielten ihn zu beiden Seiten, und sorgten, dass er im Fahren sicher sein m"ochte; das Volk lief zu beiden Seiten des Wagens mit brennenden Fackeln, streute Blumen und Kr"anze. Die Statuen der "ubrigen G"otter, nebst den kostbarsten, die in Tempeln verwahrt wurden, die kaiserlichen Insignien und pr"achtigsten Hausger"ate, wurden vorgetragen; seinen Gott noch mehr zu ehren, lie"s er die Statue der Urania oder der ph"onizischen Astarte, die eigentlich den Mond vorstellte (Dido soll sie in Karthago haben errichten lassen, und vielleicht ist sie eben der pessinuntische Stein, der im zweiten punischen Kriege nach Rom kam?), in den Sonnen-Tempel zur Statue des Gottes stellen, um sich mit ihm als der Sonne zu verm"ahlen, wozu er betr"achtliche Sch"atze und Kostbarkeiten als Heiratsgut gab. Auch m"ussten Rom und ganz Italien das Hochzeitsfest feiern. Mehrere M"unzen von Elagabal, Caracalla, Alexander Severus, welche die emeser Gottheit unter dem Symbol eines Steines gew"ohnlich mit einem oder mehreren dar"uber schwebenden. Sternen) wahrscheinlich den Ursprung des Steins anzudeuten, zuweilen auch mit einem halben Monde (auf die Astarte deutend) vorstellen, finden sich in Vaillant, Eckels M"unzsammlungen, und in Spanheims Noten zu Julians C"asarn S. 87. -- und 47. der Proben. Aus letzterem sind die dieser Abhandlung beigef"ugten M"unzen Nr. 2. 3. 4. genommen, davon die vordere von Caracalla, die andere von Alexander Severus ist. Nr. 4. ist eine goldene M"unze, worauf drei Sterne als Symbole des Steingottes erscheinen. --- Andere nicht seltene M"unzen Helagabals selbsten haben die Umschrift: Sancto Deo Soli Helagabalo.

Gleiche Bewandtnis hat es mit Dusares von dem Maxim. Tyr. Diss. 8. Cap. 8. sagt: rabes, quem colunt non novi, at Simulaerum vidi, lapis erat quadrangulis.\footnote{S. auch Elem. Alexandr. 100. 4. item Arnob. contra Gentes. 50. 6.} Desgleichen die Stelle im Porphyr: die Dumatenier (ein arabischer Stamm) pflegen j"ahrlich einen Knaben, den sie vorher geopfert haben, zu begraben, und zwar an einem Steinaltar, der ihnen zur Abbildung der Gottheit dient.\footnote{Porph. de Abstinentia 100. 2.} Ob der in der Kaaba noch jetzt befindliche Stein (wie Sage und frommer Aberglaube w"ahnt), jener Betel-Stein, Dusares, oder Alagabal sei, l"asst sich bezweifeln; wahrscheinlicher ists, dass da Mekka und Medina, jetzt die Hauptsitze des Islamisms sind, so wie sie es in fr"uheren Zeiten vom Sterndienste waren (denn in Mekka stand ein Tempel\footnote{S. Mohsen Fan Dabistan deutsche "Ubersetzung. Aschaffenburg 1809.} des Mondos), die arabischen Coraischiten, die im Besitze dieser beiden Oerter w"aren, und dies Gestirn unter dem Bild des Steins verehrten, eine in der Gegend aufgefundene Steinmasse dort aufrichteten, und als eine vom. Himmel herab gefallene Masse f"ur ein schickliches Bild des Gestirns ansahen.

So fabelhaft die arabischen M"archen "uber diesen Stein immer sein m"ogen, sind sie doch darum nicht zu "ubergehen, weil sie einiges Licht auf die meteorische Natur desselben werfen.

Das Merkw"urdigste an diesem Haus (sogt Niebuhr, Beschreibung Arabiens, S. 312. u. f. Koppenhagener Ausgabe) ist der schwarze Stein Hhadjar-el-assouad genannt, der in der Wand auf der s"ud"ostl. Seite, nur wenig von der Erde erhoben, sich eingemauert findet. Die Araber behaupten, der Engel Gabriel habe ihn vom Himmel zur Erbauung der Ka'abah gebracht; eine Mythe, die den himmlischen Ursprung des Steins ganz nach der orientalischen Theurgie, verm"oge welcher die Naturkr"afte unter dem Bilde geistiger Mittelwesen dargestellt wurden, bezeichnet. Der Sage nach soll er anf"anglich wei"s und schimmernd gewesen sein (vielleicht weil er als ein gl"uhender Stein herabfiel), nachher aber w"are er der Tr"anen willen, die er f"ur die S"unden der Menschen vergoss, ganz schwarz geworden, und habe seinen ersten Glanz verloren. Nichts in der Welt ward mehr verehrt, als dieser Stein, der in Silber eingefasst ist, und von jedem Muselmann ber"uhrt werden muss, so oft er die Kaaba umgeht. -- N"achst den vier kleineren H"ausern, den vier verschiedenen Sekten der Sunniten geh"orig, und dem Magam-Hhasaret-Ibrahim, dem Orte, wo Abraham gebetet haben soll, w"ahrend man die Kaaba erbauet, ist noch ein anderer Stein daselbst, der vielleicht eher der "alteste arabische Stein-Gott, Bethilos, Jacobs-Stein, oder Dusares sein m"ogt, indem sie ihn so wenig als einen andern (Ismael-Stein genannt) in Ehren achten, aus der Ursache vielleicht, weil der Prophet gegen die abg"ottischen Bilder der alten Sternanbether (deren der Koran unter dem Namen Alassavid erw"ahnt) so streng eifert.\footnote{Bernh. von Breitenbach, der im 15ten Jahrhundert Jerusalem und das Morgenland bereiset, gibt im Itiner. Hierosolymit. Mogunt. 1486 folgende Nachricht von der Verehrung des wei"sen und schwarzen Steines in der Kaaba: "`Duo filii Loth, Ammon scilicet et Moab, hanc domum honorabant, ibique duo colebant idola, unum ex albo factum lapide, quod Mercurium; alterum ex nigro, quod Camos appellabant. Et istud quidem ex nigro lapide, in honorem Saturni, alterum ex albo in Martis honorem, venerabantur. Et bis in anno ad haec idola adoranda eorum ascendebant cultores. Ad Martem quidem quando sol primum intrat arietis gradum, quoniam aries honor est Martis, im cujus discessione, ut mos erat, lapides jaciebantur. Ad Saturmum vero, quando sol primum gradum librae ingrediebatur, quia libra honor erat Saturni, sicque nudi ac tonsis capitibus thurificabant. Arabes quoque cum Ammonitis et Moabitis haec idola adorabant: longissima post tempore veniens Mahumet, pristinam gentis consuetudinem nolens tollere, quasi mutato quodammodo more, inconsutis opertos tegumentis domum circumire permisit. Sed ne videretur idolis sacrificare praecipere Saturni simulacrum in pariete in angulo domus constituit, cujus ne appareret facies, dorsum tantum extra posuit. Idolum vero Martis, quod undique erat sculptum, subtus terram misit, lapidesque supposuit. Hominibus autem, qui ibi ad adorandum conveniunt, lapides istos osculari praecepit, et humiliatis tonsisque capitibus intra crura lapides retro jactare, qui et dorsa denudabant, quod est signum pristinae legis, et ad effugandum daemones, se hoc modo lapides jacere dicunt, quos clam in eo ritu potius venerantur. Et haec est illa praeclara Muhammeti industria, imo malitia, ut cum a ceterorum cultu idolormm suos inhihuerit, istud tamen in honorem Veueris apud Meccham suam fieri permisit."'}

Fr"uher haben wir aus Sanchuniathon gelernt, dass man diese einzeln gr"o"seren Steinmassen von den kleineren B"atylien unterscheiden m"usse, denen die Alten verm"oge eines in ihnen wohnenden Gettes oder Daimons magische Kr"afte zuschrieben, und desfalls sich ihrer als Talismane, Amulette zu ihren Beschw"orungen bedienten, ein Gebrauch, der von den "altesten Zeiten her sich bis sp"at in die christliche Epoche\footnote{S. hier"uber Fouchers Abhandlung von der Religion der Perser mit Kleukers Anmerkungen in dem deutschen 8end-vi-Vesta 3. Th. S. 104. 170. 190.} erhielt, ja durch die Gnostiker, Valentinianer und Sch"uler des Basilides noch lebhafter erneuert warb.

Ein verl"assiger Schriftsteller Photius\footnote{Photii Biblioth. p. 1048. ad calcem; Zoega de Orig. Obelisc. p. 202.} liefert im Auszug von Damascius Leben des Isidors die Beschreibung einer dem Arzte Asklepiades widerfahrenen Erscheinung, welche auffallend mit den Umst"anden "ubereintrist, die nach den neuesten Erfahrungen das Herabfallen der Aerolithen begleiten.

"N"achst Heliopolis in Syrien (erz"ahlt er) habe er, als einsmals er den Libanon erstiegen, daselbst eine Menge herabgefallener B"atylien gesehen, von denen der Aberglaube viele Wunder erz"ahlt, und davon ihm selbst und dem Isidorus folgendes wiederfahren: ich s"ahe n"amlich (f"ahrt er fort) einen in der Lust schwebenden B"atylos, bald mit einem Gewand bedeckt; bald in den H"anden dessen, der ihm diente, er hie"s Eusebius und erz"ahlte zugleich folgendes: zur Nachtszeit sei er einst von Emesa nicht allzufern gegen jene Berghohe gewandelt, worauf der a pr"achtige Pallas-Tempel erbauet ist; als er Zeit lang dort am Fu"s des Berges vom Wage seitw"arts gesessen, habe er eine feurige Kugel vom Himmel herabfallen, und einen m"achtigen L"owen dabei stehen sehen; der L"owe sei sogleich verschwunden, und er, als das Feuer schnell verloschen war, zur Kugel gelaufen, die er f"ur einen B"atylos erkannt; er habe den Stein sogleich aufgehoben, und befragt, welchem Gott er zugeh"ore? worauf er ihm geantwortet: dem Genn"ao, einer Gottheit, so die Einwohner von Heliopolis im Tempel des Jupiters unter der Gestalt eines L"owen verehren. Eusebius(f"ahrt die Erz"ahlung fort) war aber nicht Herr dar Bewegung des Steines, er rief ihn nur durch Gebete an, und erhielt dann vom Steine durch einen dem Zischen "ahnlichen Laut Antwort auf seine Fragen.\footnote{"Ahnliche Vorschrift gibt das Orphische Gedicht Aeza, Edit. Gesneri, p. 324. "`et tu quandoquidem vocem Deorum vis audire, sic facias, ut miraculum animo tuo, intelligas: quado enim valde laboraveris illum manibus jactare et commovere, subito edit vocem recens nati infantis, nutricis in sinu plorando lac efflagitantis. Oportet vero te constanti animo curare eum semper, ne forte infirmo timore solutus, e manibus in terram abjiciens iram difficilem excites immortalium. Tum aude de vaticiniis interrogare umnia enim tibi vera. Eumque postea proprius ad oculos admovens, quando laveris, intuere divinitus enim exspirantem intelliges.} Der Stein war, wie die "ubrigen dieser Gattung, rund, von m"a"siger Gr"o"se, an Farbe schwarzbl"aulich, hie und da mit Linien und Figuren, die der Aberglaube f"ur Zauberzeichen und Z"ahlen ansah. Damascius nennt sie Buchstaben, γραμματα εκ τω λιθω γεγραμμενα bezeichnet. Nach Isidors Meinung werden die Steine durch einen in ihnen wohnendest D"amon beherrscht, nicht zwar von einem b"osen, der Materie anh"angenden, noch auch einem ganz reinen, sondern von einem jener Mittelwesen, von denen der ganze "Ather erf"ullt ist. Mehreren Gottheiten, dem Saturn, Jupiter, Mars, der Sonne, und anderen sind diese Steine geweiht.

Abgerechnet alles Wunderbare, was in der Erz"ahlung liegt, und gnostischen Schw"armern, wie Eusebius und dessen Biograph waren, ganz eigen ist, liefert die Beschreibung dieser Erscheinung uns ein anschauliches Beispiel eines Meteors, und die herab gefalleng gl"uhende Kugel ist nichts anders, als ein Aerolithe.

B"atylien, sagt zudem Plinius nach Sotakos, sind Steine, die auf dem Libanon bei Heliopolis gefunden werden. An einer anderen Stelle\footnote{Hist. nat. Lib. 37. 100. 51.} erw"ahnt dieser Naturkundige, nachdem er die verschiedenen Asteriden oder Sternsteine durchgegangen ist, einer Gattung derselben, die er Ceraunia nennt; Sothacus (sagt er) nimmt zwei Gattungen Ceraunien an; n"amlich einen schwarzen und einen r"otlichen, und sagt, dass beide einer Art "ahnlich seien. Mit denen, welche schwarz und zugleich rund sind, k"onne man St"adte einnehmen, und Flotten erobern, und sie hie"sen Betuli; --- Die l"anglichten, kristallischen und himmelblauen an Farbe, die besonders in Karamanien w"uchsen, hei"sen Cerauniae. Sie nehmen auch noch einen sehr seltenen Stein dieses Namens an, der von den Parthischen Magiern sehr gesucht wird, weil er an Stellen gefunden wird, die vom Blitz getroffen sind. Wirklich finden sich\footnote{Plin. Nat. Gesch. 37. 52.} solche Steine auf Inseln des roten Meeres, und, wie Reisende berichten, in den s"udlichen Gegenden Persiens, aber nicht dort allein, auf der ganzen Erde find sie zerstreut, und der Glaube, dass sie Erzeugnisse des Blitzes feien, findet sich bei allen V"olkern. Sie bedienten sich ihrer zu gottesdienstlichen Gebr"auchen, oder als Streit"axte. H"aufig findet man sie (im Norden am meisten) teils frei in der Erde liegend, teils in alten Grabm"ahlern verschlossen. Nat"urlich war es, dass Steinen, die man durch Himmelsfeuer gebildet glaubte, h"ohere Kraft zugeschrieben wurde. Dem Naturmenschen aller Gegenden wird der Gebrauch des Feuers erst durch Reibung zweier Steine oder H"olzer aneinander bekannt; die Erkenntnis aber, dass dem Stein diese Feuerkraft innewohne, setzt, wenn nicht zuf"allig der Wilde zu dieser Entdeckung gelangte, fr"uheres Anschauen, "au"sere Veranlassung zum voraus. Zu der Ahnung, dass Feuer aus dem Steine zu ziehen\footnote{Ignis ubique latet, naturam complectitur omnem.} sei, k"onnen ihn haupts"achlich drei Ursachen bringen:
\begin{enumerate}
    \item Der Blitz, der im Augenblick des Herabfallens B"aume spaltet und entz"undet,\footnote{Der Blitz war nach Proclus (in Timaeum p. 34.) ein Symbol der Demiurgischen Kraft, welche die Welt schafft und belebt.}
    \item die durch unterirdische, oder andere Ursachen auf der Oberfl"ache der Erde entstehenden Br"ande, und die furchtbaren Vulkane, deren Feuerschl"unde nicht blo"s gl"uhende Steine in gro"sen Massen, sondern selbst Feuerstr"ome auswerfen, die erk"altet sich wieder zu felsharten Massen bilden; endlich
    \item die "uberall und in den fr"uhesten Zeiten entfallenen Himmels-Steine, deren Ankunft meist ein feuriges Meteor begleitet. Feuer ist das furchtbarste und wirksamste Element, das in den "altesten Dichtungen, die uns Nachricht von der Welt-Sch"opfung geben, zuerst aus dem Schoos der Nacht erschien. In Hesiods Theogonie, dieser ehrw"urdigen Urkunde, die gewiss aus Traditionen des Orients entstand, wird, nachdem "Ather und Hemera aus dem Schoos des Erebos entstiegen waren,\footnote{Theogonie nach Vo"s "Ubersetzung Vers 123 u. f.} von Uranos und Gea (Himmel und Erde) Hyperion geboren, der mit Thia, seiner Schwester, den Helios, (Sonne) Selene (Mond) und Eos die Tags- und Morgeng"ottin erzeugt, von welcher (Astr"aos Gattin) nebst Boreas, Zephyr und Nothos, Phosphorus oder Hesper und die Gestirne geboren wurden. Hyperion, eine feurigleuchtende Masse, brachte der kalten Natur die erste W"arme, er umfasste Sonne und Mond, die sp"ater erst aus einer Masse geschieden wurden.\footnote{Die sp"atere Erscheinung des Mondes bezeugen nicht allein Plutarch, Lucian und andere Schriftsteller, auch die "Uberlieferung mehrerer V"olker, der Arkadier besonders, die sich "alter als der Mond, oder vorhanden, ehe der Mond die Erde beschien, angeben, stimmen damit "uberein, und im Alexis macht Hemsterhuys (s. dessen Werke 3. Teil) es mehr als wahrscheinlich, dass der von allen Sternkundigen angezeigte Komet (der erste seit der Weltsch"opfung, der sich der Erde n"aherte, im Zeichen der Fische ums Jahr 2312 vor der gew"ohnlichen Zeitrechnung) eben derselbe gewesen, der durch seine Ann"aherung an die Sonne in einen bimsteinartigen, anagebrannten, verglasten K"orper verwandelt, und (in seiner ferneren Laufbahn gehemmt) nun bestimmt worden sei, bei der Erde zu verbleiben, wodurch unbezweifelt alle Atmosph"ar- und Gestalter"anderungen unserer Erde geschahen, durch die ihr erster besserer Zustand im Physischen sowohl als Geistigen (das sogenannte goldene Weltalter) verschwand; darum auch hat das Andenken dieses ersten furchtbarsten aller Meteore sich so lange im Andenken aller V"olker erhalten.} Der Tiefe entstieg er gleich der Sonne, von der Erde zum Himmel, und erw"armte von dort die Natur, wie er die gr"o"seren Feuermassen, Sonne und Mond, erzeugte, so Kojos sein Bruder die gr"o"seren Sterne; aber Asteria und Letho (von ihnen erzeugt) waren G"ottinnen der kleineren Sterne,\footnote{S. Kanne Mythologie der Griechen, S. 32.} die aus jener Feuermasse sich zum ersten male schieden, und in Gesellschaft der Letho die erste sternhelle Nacht brachten, denn ohne Dunkel der Nacht erscheinen (nach dem alten Glauben) keine Sterne, Letho das Dunkel begleitete daher gleich von Geburt an ihre Schwester Asteria, weshalb auch die Orphiker ihnen die Hekate beigesellten, und (der Magie wegen, weil Sterne und Zauberei im engsten Verh"altnisse stehen) zur Tochter derselben machten. Wie nun Theogonie uns die Entstehung aus dem Chaos und die ersten Uranf"ange zeigt, so finden wir in ihr auch das Bild des Elementen-Kampfs zur Vollendung der Wesen, und wie die drei aufeinander folgenden Himmels-Reiche des Uranos, Chronos und Zeus, in physischer Hinsicht nichts anders, als die fr"uheren Revolutionen im Weltall darstellen, so enthalten die Riesen und Titanen K"ampfe, die bei Ver"anderung eines jeden dieser Himmels-Regierungen vorfielen (gleich jenen indischen Sagen bei Erscheinung eines neuen Jug oder Zeitalters), ein sinnlich anschauliches Bild der mittelst g"arender Elemente durch die Zeugungskraft des j"ungeren Gottes gebildeten Wesen, sowohl Lebenden, als Pflanzen, Erden und Metallen. Die Stelle, worin Hesiod die Tiefe des Tartaros schildert, und das Bild vom Falle des ehrnen Ambosses, dessen er sich dabei bedient, bezeigt nicht allein das hohe Alter der Metallurgie, auch die Tiefe, worin die Metalle im Schoos der Erde verborgen liegen, wird darin bezeichnet; denn alles, Fu"sboden, Pforten, Mauern, ist hier von Erz.
\end{enumerate}
\paragraph{}
"`--- --- Gleich fern von der Erde ist des Tartaros finsterer Abgrund. Wenn neun Tag' und N"achte dereinst ein eherner Amboss Fiele vom Himmel herab, am zehenten k"am' er zur Erde; Wenn neun Tag' und N"achte sodann ein eherner Amboss Fiele hinab vom der Erd', am zehenten k"am' er zum Abgrund. Ehrnes Geheg' uml"auft den Tartaros; aber umher ruht Dreifach gelagerte Nacht an dem Eingang; oben herab dann Wachsen die Wurzeln der Erd' und des ungeb"andigten Meeres. Allda sind die Titanen im nachtenden Schlunde des D"unkels Eingehemmt, nach dem Nathe des schwarz umw"olkten Kronion, Tief in der dumpfigen Kluft, am Rand der unendlichen Erde. Keiner vermag zu entfliehn; denn es schloss Poseidon den Ausgang Fest mit eherner Pfort', und rings umschr"ankt sie die Mauer."'

Nachdem Hesiod uns in den Abgrund gef"uhrt hat, schildert er mit der alten Dichtungen eigenen Kraft die Beschaffenheit der tiefen Erdschk"unde; und im Bild des furchtbaren Typh"aeos, des vielk"opfigen Drachen, Sohn der Gata, steht eine Szene vor uns schauderhast erhaben, vom inneren Lehen, Tosen, Sausen, und G"aren entgegen gesetzter Elemente --- Feuer, Luft und Wasser in den verschlossenen Bergschluchten:

"`So aus den H"auptern gesamt, wenn er schauete, brannt' es wie Feuer. Auch war hallende Stimm' in allen entsetzlichen H"auptern, Von vielartigem Wunderget"on: denn in h"aufigem Wechsel Lautete jetzt f"ur die G"otter verst"andliches; jetzo hinwieder Scholl es, wie dumpfes Gebr"ull des in Wut anrasenden Stieres; Jetzo gleich, wie des. L"owen von unaufhaltsamer K"uhnheit, Jetzo gleich dem Gebelfer der Hundeleine t"onet es seltsam, Jetzo wie gellendes Pfeifen, dass rings nachhallten die Berghohen."'

Aber des Berges H"ohlen entsteigen D"unste in die h"ohere Atmosph"are, Wolken bilden sich, und aus ihnen die zerst"orenden Wetter.

"Ernst nun schwang er die Donner, und donnerte; rings in dem Aufruhe Toste das Land grauenvoll, und der w"olbende Himmel von oben, Auch des Okeanos Strom, Meerflut und tartarischer Abgrund. Ja dem unsterblichen Fu"s erbebten die H"ohn des Olympos, Als sich der Herrscher erhub; und tiefauf dr"ohnte das Erdreich. Beiden entloderte Brand, um das finstere Meer sich verbreitend, Hier von dem Donner und Blitz, und dort von der Flamme des Scheusals, Von glutwirbelndem Sturm und zuckendem Strale der Wetter. Auf nun brauste die Erd', und der Himmel umher, und die Meerflut; Und die Gestad' umtobt' unermessliches Wogenget"ummel, Durch der Unsterblichen Schwung; und es schwankte das All in Ersch"utterung."'

Und wie das Feuer aus der wetterschwangeren Wolke sich zur Erde senkt, begleitet von Regan und Hagel, also entstr"omen aus den tiefsten Schl"unden verheerende Feuergluthen; --- das Meer erbrau"st, und durch die Kraft der Wasser, Luft- und Feuer-Meteore treten neue Gestalten mannigfacher Gestirne, Erde, Metalle, aus dieser gro"sen Werkstatt hervor.

"Lodernde Glut entstr"omte dem niedergedonnerten Herrscher, In des Gebirgs Waldthalen, von Felsabh"angen umdunkelt, Wo er erlag; weit brannte die m"achtige Erd' in des Wetters St"urmischer Loh', und zerfloss, dem schmelzenden Zinne vergleichbar, Welches der J"unglinge Kunst im wohlgeh"ohleten Tiegel, Gl"uhet; oder wie Eisen, das stark vor allem Metall ist, In des Gebirgs Waldthalen von flammender Hitze geb"andigt, Schmilzt in dem heiligen Grund, durch k"unstliche Hand des Hef"astos: Also zerschmolz auch die Erd' in stralender Lohe des Feners."'

So reichen Physik und Fabel sich die Hand; und wie letztere im Mythos das Bild der Entstehung der Wesen durch G"arung und Verbindung der Elemente zeigt, so sagt der "alteste uns bekannte Mineraloge schon: "`die Anh"aufung (Bildung) der Steine wird teils durch Hitze, teils durch K"alte bewirkt, auch k"onnen einige durch beide entstehen, "uberhaupt scheinen die Erdarten durch Feuer zu erh"arten. "Ubrigens werden alle Mineralien durch jene entgegengesetzten Kr"afte bald gebildet, bald aufgel"o"st."'\footnote{Theophrast von den Steinen, §. 3. Sein deutscher "Ubersetzer, D. Schmieder, bemerkt dabei der Gedanke ist gut chemisch, nur wird er durch Mangel der Kunstw"orter entstellt; genau genommen gibt es wirklich keine andere Umbildungskr"afte f"urs Mineralwich, als Hitz. und K"alke. Alle nasse Aufl"osungen, Schmelzungen, Verfl"uchtigungen, Verwitterungen haben doch nur Orts"anderungen des warmen Stoffs zum Grunde.}

Dass diesem gro"sen Naturkenner das Gesetz der Kristallisation nicht unbekannt geblieben, zeigt folgende Stelle:\footnote{Schade, dass sein gr"o"seres Werk von den Metallen verloren gegangen.}

"`Man muss sich im Allgemeinen vorstellen, dass die Steinarten aus reinen und gleichf"ormigen Fl"ussigkeiten entweder f"ur sich, oder beim Durchseihen einer fremden Masse entstanden sind. Davon sind denn die Gl"atte, Dichtigkeit, der Glanz, die Durchsichtigkeit, und andere Eigenschaften der Steine herzuleiten. Je gleichf"ormiger und reiner der Urstoff war, je regelm"a"siger die Bildung erfolgte, desto mehr sind auch den Produkten jene Merkmale eigen."'

Dies Entstehen zusammengesetzter Wesen aus einfachen Elementen nun ward nebst anderen Gegenst"anden, nach dem Zeugnis alter Schriftsteller, in den Mysterien gelehrt, und symbolisch dargestellt. Nach der indischen Lehre (um nur die vorz"uglichsten anzuf"uhren) ward Porsch als das Bild der ganzen Welt, und Archetyp der Sch"opfung angesehen. Genau nach der Lehre der Vedas beschreibt der Brahman Sandales, Zeitgenosse des Bardesanes,\footnote{Stobaeus 50. 1. p. 141.} dessen Abbildung. In einer H"ohle des h"ochsten Berges auf der Erdmitte sei eine Bilds"aule aufgerichtet, zehn Cubitus hoch, die Arme gekreuzt, die rechte H"alfte des Gesichts, Arm, Fu"s und die ganze Seite m"annlich, die linke weiblich, beide kunstreich und geschickt verbunden. Auf der rechten Brust gebildet sei die Sonne, auf der linken der Mond, auf beiden Armen aber seien durch die Kunst des Bildners viele Engel und alles "Ubrige, was die Welt befasst, Himmel, Berge, Meer, der Strom des Ozeans, Pflanzen und Tiere, und was sonst in der Natur existiert, vorgestellt. Dies Bild habe Gott seinem Sohne gegeben, als er die Welt gr"undete, um an ihm ein sichtbar Vorbild seines Werks zu haben. Es sei nicht von Silber noch Gold, Erz oder Stein, am "ahnlichsten noch einem harten Holze, und doch nicht Holz. Auf dem Haupte der Gestalt sitze das Bild Gottes wie auf einem Throne. Hinter dem Bilde sei tiefes Dunkel durch die H"ohle, man gehe mit Fackeln hinein, und dem Reinen offene sie sich weit, Unreine aber k"onnten nicht hindurch. Zur Zeit des Sommers und im Herbste versammelten sich die Brachmanen in ihrer N"ahe, um das Bild zu sehen, und sich selbst zu pr"ufen; sie unterredeten sich dann miteinander "uber die Gestalten auf dem Bilde, die nicht leicht verst"andlich seien, teils ihrer Menge wegen, teils weil nicht jedes Land alle Pflanzen und Tiere tr"agt. --- So wurden auch in den Mithras Mysterien den Eingeweihten die Urgeschichte der Sch"opfung gelehrt: die dieser Gottheit (der Sonne) geweihte H"ohle auf dem h"ochsten Berge Persiens durch unterirdische Blumeng"arten, Quellen und Fl"usse versch"onert, stellte das Universum dar, und die in gleicher Entfernung aufgestellten Bilds"aulen waren Symbole der Elemente und Himmelszeichen.\footnote{Porphyr. de Antr. Nymph. p. 234.} Sterne, Konstellationen, Tiere, Pflanzen, Metalle, wurden nebst dem Hauptbilde des Mithras, auf dem Stier reitend, mit der Aufschrift: Gott dem un"uberwindlichen Mithras, darin vorgestellt, die Priester sowohl, als Eingeweihten, erhielten jedes den Rahmen eines Tieres, und die Neophyten wunden nicht anders, als nach den furchtbarsten Pr"ufungen aufgenommen. Gleiche Einweihungen begleiteten in Lemnos den Dienst des Vulkan, des Zeus auf Kreta, der g"ottlichen Mutter in Syrien, der Cabiren auf Samothraze, ja die Orgien, und die Mysterien des Adonis waren nichts, als geheimer Naturdienst. In ihnen wurde vorz"uglich gelehrt und durch Anschauungen versinnlicht die allm"ahlige Entstehung aller Wesen, vorz"uglich der Steine und Metalle, denen man (ihrer verwandten Natur wegen mit den Gestirnen) geheime und h"ohere Kr"afte zuschrieb, wie auch gegenseitig jedes Metall ein ihm analoges Gestirn hat. Bed"urfnis und Luxus machten sie fr"uhzeitig dem Menschen notwendig, und ihre Entdeckung hatte die Natur ihm, ohne dass er selbst desfalls tiefe Gruben zu befahren bedurft h"atte, noch zur Hand gelegt, indem in der ersten Zeit, wie noch jetzt, durch Wasserf"alle von hohen Bergen herab, durch Str"omungen der Fl"usse und Regen h"aufige Metalle in kleinen und gr"o"seren Massen an Tag gef"ordert werden, auch Vulkane, und die in der Urgeschichte so h"aufigen Erdbr"ande lehrten die Menschen bald das Schmelzen der Erze. Gold, Silber und Kupfer\footnote{Tubal, der nach Mose 1. 100. 4. 22. alles zu h"ammern versuchte, und Eisen und Kupferschmied ward, zeigt uns die Kunst, Erz zu verarbeiten, schon vor der allgemeinen Flut.} wurde am fr"uhesten zu h"auslichem Gebrauch sowohl als zum Cultus verarbeitet. Zeugnisse hier"uber sind aus den heiligen B"uchern und profan Geschichtsschreibern h"aufig; seltsam aber ist, dass, Ger"atschaften aus Gold und Silber gearbeitet, die zum blo"sen Luxus dienten, sich eben so fr"uhzeitig vorfinden, als jene, die blo"se Notwendigkeit erfand; selbst goldene Massen, wie die Bewohner Perus und Mexicos sie hatten,\footnote{Dass die Peruaner in fr"uhester Zeit schon das Kupfer schmolzen, und zur Bearbeitung ihrer prachtvollen Denkmale aus ungeheuren Porphyr-Massen nicht blo"s steinerne Arten, sondern auch Werkzeuge gebrauchten, die aus Kupfer und Zinn vermischt waren (was die Alten Aes χαλκος nannten), zeigt Alex. von Humbold im 2ten Cahier das Vues des Cordellieres p. 117. aus mehreren Tatsachen, vorz"uglich durch eine in einem alten Silber-Bergwerke aus der Zeit der Incas vorgefundenem Messer oder Bohrer, welches er nach Europa zur"uckbrachte, und das nach Vauquelins Analyse 0.94 Th. Kupfer, und 0,06 Th. Zinn enthielt, es war 12 Zentimer lang und 2 breit; an Farbe und Gehalt den Axten und Opfermessern der alten Gallier "ahnlich. In einem trefflichen Mémoire "uber die Bronze der Alten, von Mongez (Tome 5. des mémoires de l'institut national des sciences et arts, auch G"ottinger gelehrte Anzeigen von 1805. S. 159) zeigt dieser Gelehrte durch chemische Analyse verschiedener alter Schwerdte, Pfeile u. a. Waffen, dass die gew"ohnliche Proportion der alten Bronze f"ur Gewehr ein Zehntel Zinn sei, ungef"ahr die Mischung f"ur unsere Kanonen.} da die Spanier in diese L"ander eindrangen, bedienten sich schon die "Agypter, Lydier, die Bewohner Beticas (des heutigen Protugalls, als die Kartaginenser zum erstenmal dahin kamen) und andere alte V"olker.\footnote{Beweise hievon sind mit vielem Flei"se gesammelt in Goguet Orig. des Loix arts et sciences. Liv. 2. Ch. 4. Die Fortschritte und Verbesserungen der Schmelzkunst geh"oren "ubrigens in die Geschichte der Metallurgie.} Ist dem Golde, Silber und Kupfer ihr Vorzug und fr"uher Gebrauch nicht abzustreiten (dessen Ursache vorz"uglich in ihrer leichteren Behandlung und Schmelzbarmachung, wie gegenseitig in der H"arte und Spr"odigkeit des Eisens zu suchen ist), so erscheint gleichwohl sein Gebrauch in den fr"uhesten Zeiten.\footnote{Schon in der Mythe des "agyptischen Welt-Eyes zeigt sich Gold und Silber als die edelsten und ergiebigsten Metalle, denn der Fabel gem"a"s teilte sich das Ei, als nach Verlauf eines Jahres es zerspaltete in zwei H"alften, die eine war Gold, die andere Silber, und die silberne war die Erde, die goldene die Sonne.}

Das "alteste Buch, Hiob (28.) erw"ahnt dessen gleichzeitig mit dem Golde. "`--- Silber hat der Mensch gefunden, Und den Ort des Goldes, das der K"unstler gie"st, Eisen nimmt er aus der Erde, Und Steine schmelzet er zu Kupfer, Er macht der Finsternis ein Ende, Alle verwahrten Sch"atze forscht er auf, Den Stein der acht und der Schatten, Am Fu"s des Berges bricht ein Strom aus, Von ihrem Waldbach vergessen, Versiegen die armen Str"ome wieder, fern von dem Menschen herumirrend. Ein Erdreich, aus dem oben Speise w"achst, Wird unten als vom Feuer umgew"uhlt; Seine Steine sind der Ort des Lazurs, Der mit goldnem Staube bezeichnet ist, Diesen Fu"ssteig kannte kein Raubvogel, Das Auge des Falken hat ihn nicht erblickt."'

Des Eisens H"arte r"uhmen die mosaischen B"ucher auf eine Weise, die dessen fr"uhzeitigen Gebrauch zeigen.\footnote{S. ferner Hiob 100. 19. 20. 40. 41.} Sie reden von eisenhaltigen Berggruben, vom eisernen Bette Ogs, K"onigs von Basan, von Schwerdtern, Degen, Messern, Degengriffen aus Eisen verfertigt, welches notwendig die Kunst voraussetzt, Eisen zu h"arten und in Stahl zu verwandeln.\footnote{Deuter. Cap. 4. 5. 20. Cap. 8. 5. 9. Auch Wummer 35. 16. Levit. 100. 1. 17.}

Eisen ist das allgemein verbreiteste, obschon meist tief vergrabene Metall; auf der ganzen Erde ist es befindlich, aber wo es am vorz"uglichsten erscheint, ist Rorden, denn am Kaukasus und in Sibirien sind noch jetzt die ergiebigsten Eisengruben. Dahin auch versetzt der Mythus den Prometheus, den Zeus (weil er den G"ottern das Feuer stahl, und den Menschen dessen Gebrauch samt den K"unsten, die durch Feuer sich ausbilden, lehrte) auf den Kaukasus verbannte, Heph"astos (der g"ottliche Schmied) musste mit eisernen Banden ihn dort fesseln, und die daher entstandene Sage, dass Prometheus den ersten eisernen Ring getragen, ist ein Dichter-M"archen auf die fr"uhe Bearbeitung des Eisens, und den Gebrauch, Ringe zu tragen.

So k"onnen Typhon und Horus nach einer andern Mythe gleichfalls als Symbole des Eisens und dessen Ausbeute gelten. Denn Typh"aos, der Riese, der von Zeus tief in den Abgrund geworfen ward, bewohnt die eisenhaltigen H"ohlen des Tartaros, weshalb auch Eisen f"ur seinen Knochenbau gilt, wie gegenseitig Magnet, weil er das Eisen an sich zieht, die Knochen des Horus genannt wird.\footnote{Plin. Nat. Gesch. 33. 3. und 37. 1.}\footnote{Denn (sagt Plutarch Isis und Osiris) wie das Eisen sich oft von diesem Steine anziehen l"asst, und ihm zu folgen scheint, ebenso pflegt die belebende heilsame vern"unftige Bewegung der Welt jene typhonische Hartn"ackigkeit gleichsam mit sch"onen Worten an sich zu ziehen und zu erweichen diese aber rei"st sich dann wieder los, kehrt in sich selbst zur"uck, und "uberl"asst sich einer schrankenlosen Freiheit.}

Typhon, der sp"ater erst in den griechischen Mythos aufgenommen ward, ist syrischen oder "agyptischen Ursprungs. Dort bedeutete er (nach Plutarch "uber Isis und Osiris --und im Buch von der Mondsscheibe) das d"urre trockene auch im Gegensatz des Osiris (der feuchten befruchteten Mondwelt) die Sonne als versengende ver"odende Kraft; nach einigen Fabeln ward er von Zeus unter den "Atna geschleudert, und dieser Berg auf ihn gew"alzet; aber nach Homer liegt er unter den arimischen Gebirgen vergraben. Diese Arimi aber sind (Heyne Excurs 1. zum Virgil 9. 3ter Th. S. 313.) in Asien, am ersten in Zilizien zu suchen, welches nach Strabo, nebst Phrygien, Mysien und Lydien von unterirdischen Feuern gebrannt hat; aber Typhon, nach der "agyptischen Mythe, bedeutete auch das Meer, in dem der Nil verschwindet, oder im Allgemeinen das feuchte schlammige Prinzip, aus dem das Weltey erzeugt ward; auch aus Wasser, wie aus Feuer, kamen Wesen hervor (wie neuere Forschungen am Basalt erwiesen haben). Denn nicht Feuer allein, auch Wasser ist Ursache und wirkende Kraft vieler der bedeutendsten Meteore. Am ersten scheinen die Ph"onizier und Phrygier mit Bearbeitung des. Eisens sich besch"aftigt zu haben, und als kunstfertige Bergleute zeigt die fr"uheste Geschichte uns die Daktylen am Ida, aber vorz"uglich die Kyklopen ber"uhmt durch Hammer- und Mauer-Arbeiten, jener flei"sige, regsame Volksstamm des alten Thrinakias (Sizilien), die nicht dort allein, auch in Griechenland und Latium viele noch bis jetzt unzerst"orbaren Denkmale ihrer Kunst erbaut haben.\footnote{Nach der fr"uheren Theogonie waren noch vor Zeus die ersten Riesen jene vielarmigen starken Riesen Hesiods, die Urheber des ersten Gewitters, und die drei m"achtigen Kyklopen Briareos, Gyges und Kottos (w"ortlich: der Gewaltige, der Starke, der Schl"ager) bereiteten dem Jupiter Blitz und Donner.}

Zeus (der Urbedeutung des Worts nach Ζῆν Leben,\footnote{Ζην ist, wie Kanne Mythologie der Griechen zeigt, die "alteste Form von Ζευς aus Ζην wurde Ζεις und nachmalen Ζευς.} und Θεος, das Laufende) war als Naturgeist, dem die ihm untergeordneten Elementargeister (starke Riesen), Donnerkeile und Blitze berieten, die wirkende Kraft im Gewitter, das lebende Prinzip, ein Meteor, dem (als herrschenden Gott der h"oheren Regionen) Wolken, Blitz und Donner zu Gebot stehen. Aber durch Prometheus kam das Feuer zur Erde, wie gegenseitig durch Typhon (den Flammen spr"uhenden Drachen). in der Erden Schoos, tief verschlossen in das Innere der Metalle; entlockt jedoch wurdes ihnen wieder durch Vulcan (nach den Orphikern das Symbol der Naturkraft, daher er der Starke, Kraftvolle hei"st,\footnote{Orph. Hymne 65.} das unerm"udete Feuer, in flammenden Blitzen gl"anzend, den Sterblichen Licht bringend, der handfeste, ewig arbeitende K"unstler, Teil des Weltalls, das lautere Element, der Allvorzehrende, Allb"andigende, dessen Glieder, das Licht, die Sonne, der Mond, die Gestirne sind, der Seelige, der alles, selbst die sterblichen K"orper bewohnt, des Feuers rastlose Wut d"ampft, und Lebens-W"arme mitteilt). Vulcan mit den Kyklopen, Symbole des Flei"ses und der Kunstfertigkeit, sind die Urheber der durch Kunst bewirkten Ph"anomene, wie oben am Himmel Zeus die nat"urlichen Meteore erregt; aber dem Eisen, das vorz"uglich vom g"ottlichen Schmiede (Vulkan) bearbeitet wird, d. h. von allen Metallen sich auf Erden am meisten findet, gab die Natur einen ihm verwandten Gef"ahrten zur Seite, den Magnet, der am liebsten auch sich findet, wo Eisen liegt, und mit ihm verbunden so manche Wunder erzeugt. In der fr"uhesten Zeit war er bekannt, denn nach Aristoteles (de Anima 50. 2. 1.) erw"ahnte schon Thales seiner. Im orphischen Gedichte: von den Steinen, wird er vorgestellt als ein sch"oner J"ungling im Gefolge des Medea (vermutlich der magischen Kraft wagen, die dem Stein zugeschrieben wurde); nach einer andern Sage ward er durch Zufall, von einem Sch"afer auf den Berg Sipylos entdeckt, der ihm seinen Namen gab; dasselbe erz"ahlt Plinius, vom lydischen Steine; doch ists eine blo"se Verwechselung; denn dieser hat nichts von der magnetischen Natur. Siderites hei"st dieser von seiner Analogie mit dem Eisen; sein bedeutendster N"ahme, auf seine Kraft und Wirkung deutend, ist nach Platon im Ion und Tim"aeos, der des herkulischen Steins ηραχλεια λεθοσ\footnote{Eigentlich von der Stadt Heraclea benahmt, in deren N"ahe er vorz"uglich gefunden wird, allein wahrscheinlich hat die Stadt sowohl als der Stein den Nahmen von Heracles erhalten.} Aristoteles (de Anima 50. 1. 100. 2.) nennt ihn vorzugsweise den Stein η λιθος und schrieb nach Diog. Laert. eine eigene (aber verloren gegangene) Abhandlung von demselben. --- Ja er behauptet sogar, Thales habe ihn belebt geglaubt; Plinius (Nat. Gesch. 36. 38.), der f"unf Arten derselben angibt, nennt als vorz"uglich den "athiopischen und arabischen Androdamos, der schwarz ist, und H"arte hat, auch im inneren Afrika gefunden wird, und Silber, Erz und Eisen anzieht. Am st"arksten r"uhmt er den H"amatit, der (wie er sagt) den Magnet selbst anzieht. Theamedes hingegen ist der negative Magnet, der Eisen, statt anzuziehen, von sich st"o"st; auch diese Eigenschaft war den Alten nicht unbekannt, denn nebst Plinius nennt Marcellus Empiricus des Theodosius Arzt im Buch de Medicam. den Magnet Lapidem antiphyson, qui ferrum trahit et abjicit, welche Eigenschaft des ein-entgegen blasens Claudian epigram. 14., wo er vom martialischen Eisen redet, zu meinen scheint:

"`Ille lacessitus longo spiraminis actu --- ---"'

und Auson in der Beschreibung von Arsinoes Bildf"aule.\footnote{In Mosella Idyll. 3.}

"`Spirat enim tecti testutidine totus Achates\\
\hspace*{0.5cm} Afflatamque trahit ferrato crine puellam."'

H"aufig sind in den Alten die Stellen, die von der Anziehungs-Kraft des Magnets reden. Wie zart ist folgende von Claudian Epigr. 4.

"`Pronuba fit Natura Deis ferrumgue maritat.\\
\hspace*{0.5cm} Aura tenax --- --- ---\\
\hspace*{0.5cm} Flagrat anhela silex, et amicam saucia sentit,\\
\hspace*{0.5cm} Materiem, placidosque Chalybs cognoscit\\
\hspace*{1.5cm} amores."'

und jene des Lucretius 50. 6.

"`Exultare etiam Samothracia vidi\\
\hspace*{0.5cm} Et ramenta simul ferri furere intus ahenis\\
\hspace*{0.5cm} In scaphiis lapis hic magnes cum sub dictus\\
\hspace*{1.5cm} esset."'

Von dieser Anziehungskraft des Magnet sagt Plinius\footnote{Naturgeschichte 36. 25.}: "`was ist tr"ager, als der starre Stein? und doch gibt Natur dem Magnet Gef"uhl und H"ande. Was widersteht st"arker als das starre Eisen, aber hier gibt es nach, und nimmt Sitte an; es wird von Magnet-Stein angezogen, und die Materie, welche alle Dinge z"ahmt und beherrscht, l"auft, ich wei"s nicht welchem Nichts entgegen, steht still, wenn sie ihm n"aherkommt, wird gehalten, und h"angt in einer Umarmung fest. Auch Platon gibt\footnote{Im Ion.} in der Beschreibung jener Kette von Eisen, deren Ringe fest gehalten werden durch den obersten, der an einem Magnet h"angt, ein sch"ones Bild davon. Philo, Galenus, Nemesius, erw"ahnen seiner, und Augustin spricht\footnote{De civit. Dei 21. 4. Er bemerkt zugleich, dass der Magnetstein zuerst aus Indien gekommen.} mit Verwunderung von seinen Eigenschaften.

Eine gleich wunderbare ist die Wechselwirkung zwischen ihm und dem Eisen, denn wird dieses vom Magnet angezogen, so erh"alt gegenseitig der Magnet durch Ann"aherung und Ber"uhrung von Eisenteilchen neues Leben, weshalb Plinius ihn ferrum vivum nennt. Claudian schildert diese Wirkung in folgenden Versen:

"`Ex ferro meruit vitam, ferrique rigore\\
\hspace*{0.5cm} Vescitur; has dulces epulas, haec pabula\\
\hspace*{1.5cm} novit:\\
\hspace*{0.5cm} Hinc proprias renovat vires."' Epigr. 14.

Wie und durch welches Medium in physischer Hinsicht\footnote{Thales und andere Weisen hatten nach Aristoteles die Wunderkr"afte des Magnets durch eine in ihm wohnende, ihn belebende Seele am besten zu erkl"aren geglaubt.} der Magnet auf das Eisen wirkt, hat Plutarch am ersten gezeigt, und zwar den "Ather, oder die in Bewegung gesetzte Luft als das Prinzip davon ausgenommen, seine Worte sind\footnote{In den Platonischen Fragen.}: der Magnet gibt gewisse schwere, windartige Ausfl"usse von sich, wodurch die n"achste Luft angesto"sen wird, so dass sie die vor ihr befindliche verdr"angen muss; diese geht nun im Kreise herum, und zieht dann auch das Eisen mit Gewalt an sich, denn das Eisen hat viele raue Stellen, G"ange und "Offnungen, die wegen ihrer Ungleichheiten zum Eindringen der Lust sehr schicklich sind; so dass diese statt abzugleiten, sich leicht festsetzen, und lange genug darin verweilen k"onnen, um das Eisen in Bewegung zu setzen, und mit Gewalt nach dem Steine hinzusto"sen. Die Meinung der Alten, dass dem Magnet ein belebender Geist innewohne, hat ihn zum Stein der Liebe und Sympathie\footnote{Daher wohl sein franz"osischer Name: Aimant, wie aber Sympathie und Attraktion ihm eigen ist, so auch Antipathie, denn er soll, wie die Alten w"ahnen, nur bei Tag wirken; manche ihm heterogene Dinge, als Zweifeln, Oel, Mercur schw"achen seine Kraft.} gemacht, und diese Eigenschaft eignen ihm alle Lithographen vom Verfasser der orphischen Steinschrift bis zu Albertus magnus und Paracelsus zu. Zugleich hat Magie sich seiner bem"achtigt, und nicht wenige ihrer Zaubererscheinungen sind durch ihn bewirkt worden.

Unter den k"unstlichen Wundern, die ihren Ursprung dem Magnet zu danken haben, verdient zuerst jener Tempel Erw"ahnung, den nach Plinius (Nat. Gesch. 34. 42.) der Architekt Dinochares auf Gehei"s des Ptolem"aus erbaute, und darin die eiserne Statue der Arsinoe schwebend anbrachte; Plinius sagt zwar, der Tod des Ptolem"aus habe den Bau verhindert, allein Auson beschreibt die Bilds"aule als wirklich vorhanden und schwebend:

"`Conditor hic forsan fuerit Conditor aulae\\
\hspace*{0.5cm} Dinochares, quadroqui in fastigio cono\\
\hspace*{0.5cm} Surgit, et ipsa suis consumit piramis umbras:\\
\hspace*{0.5cm} Jussus ob incesti qui quondam foedus amoris\\
\hspace*{0.5cm} Arsinoen Pharii suspendit in aere templi."'

Augustin de Civit. Dei Lib. 21. Cap. 8. best"atigt es, wofern dieser Kirchenlehrer nicht eine Bilds"aule der Sonne in Serapis Tempel zu Alexandrien darunter verstand, die gleichfalls schwebend erschien. Allein Ruffin (in der Kirchengeschichte Lib. 2. 100. 23.) sagt: die Statue sei blo"s durch einen oben am Gew"olbe angebrachten Magnet (vielleicht war das ganze Gew"olbe von Magnet-Stein) festgehalten worden, statt dass jene der Arsinoe zwischen zwei ihr zur Seite stehenden Magneten gleichsam schwebend erschien; mehrere solche Bilds"aulen, teils schwebend, teils bei gewissen Feierlichkeiten, wie jene des Apoll durch Magnet k"unstlich in Bewegung gesetzt, deren Lucian de Dea Syria gedenkt, werden im Altertume erw"ahnt,\footnote{Falconet Dissert. sur ce que les Anciens ont lu sur laimant in mem. de l'acad des inscript. T. 4.} zum Beispiel das Sonnenbild im Tempel des Serapis, der Magnet, der, nach Kircher de Art. Magnet. Lib. 1. Cap. 1. die ehrnen K"alber des Jeroboams im Tempel, und nach den Rabbinen, jener, der den ehrnen Kranz der Ammoriten festhielt, eine Statue der Sieges-G"ottin, die zwischen vier Bilds"aulen schwebend hing; ein Cupido, dessen der K"onig Theodorich in einem Brief an Boetius erw"ahnt, ein ber"uhmter gleichfalls schwebender Bellerophon zu Pferd, u. a. m. --- Von Mahomeds Sarg hatten die Araber gleiche Sage, die sich aber dahin beschr"ankt, dass nach Zeugnis eines alten Reisenden\footnote{Gabriel Bremond Descript. de l'Egypte 1679.} oberhalb demselben ein gro"ser Magnet-Stein angebracht ist, woran schwebend ein goldener halber Mond h"angt.

Besondere Verehrung bezeigten dem Magnet als heiligem Steine die Hebr"aer; dass er ihnen in der W"uste schon bedeutungsvoll war, lernen wir aus dem Torah, worin unter andern Rabbi Isaac Abarbanel sagt- zur dritten Ordnung geh"oren drei seltene, aus den Gegenden der St"amme Gad, Asser und Isaschar kommende Steine Lessem, Schebo und Aschlamah,\footnote{Die Talmudische Tradition hat, wie in so vielen, auch hier alles untereinander geworfen. Der wahren Auslegung gem"a"s ist Aschlamah der Amethyst, Schebo der Achat, und Lessem der Lynkurer oder Luxstein, der seiner anziehenden Kraft wegen von Theophrast schon f"ur eine Art Bernstein angesehen ward, und deswegen Analogie mit dem Magnet hat. Diese drei geh"orten zu den zw"olf Steinen, welche im Brustbild des hohen Priesters befindlich waren, und worauf die Namen der zw"olf St"amme graviert wurden. Die "ubrigen hie"sen: Odem (Sarder), Pithera (Topaz), Barechet (Smaragd), Rophet (Karfunkel), Saphir (Saphir), Jahalom (Jaspis), Tharsis (Chrysolit), Schohem (Onix), Jaspe (Beryll). S. Calmet Bihl W"orterbuch. Nach Hartmanns neuesten scharfsinnigen Untersuchungen in der Hebr"aerin am Putztische, 1. Th. S. 278 und 3. Th. Anmerk. 137. prangten im priesterlichen Brustschmucke folgende Steine: der Karneol, Smaragd, Karfunkel, Jaspis, Lazurstein, Lyncurier, Amethyst, Chrysolith, Achat, Topaz, Onix, Sardonir. Sprachforscher m"ogen "uber die Richtigkeit seiner Bemerkungen und den "achten Sinn der hebr"aischen Benennungen dieses Steine entscheiden.} denen aus dem Tierkreis die Zeichen der Waage, des Skorpions und Bogensch"utzens, und aus den Monaten Thisli, Marschevan, und Choslen, (der 1ste, 2te und 3te Monath) entsprechen. Unter diesen Steinen ist Lessem der Amethyst, Schebo der Magnet, bekannt durch die Eigenschaft, Eisen anzuziehen, Aschlamah aber eine Steinart, die den Schlaf bef"ordert. Steine, besonders die edleren, waren, nach Proklos de Magia, mehr als Pflanzen, in besonderer Verwandtschaft mit Sternen und Geistern; so sollte der Jaspis die Opfer den G"ottern angenehm machen, der Achat die Gatten unaufl"oslich verbinden, so hatte der Lapis Solaris und Lunaris Bezug auf Sonnen- und Mond-Dienst, und wie jedes Mineral einen ihm eigenen Stern hatte, so geh"orte dem Polar-Stern der Magnet, der, wie das Trinum Magicum sagt: offenbar die Verwandtschaft der untern mit der oberen Welt zeigt. Talismane wurden daher zu Siegelringen "ofters in Magnet-Steine geschnitten. Ein solcher war der Skarab"aus, den der Engl"ander John Gr"av im 17ten Jahrhundert aus dem Orient brachte, und der aus dem lebhaftesten kr"aftigsten Magnete bestand.\footnote{Kircheri Magnes. pag. 13.} Wodurch Magnet-Stein aber als F"uhrer und leitendes Wesen unter allen Steinen seine h"ohere astralische Kraft am meisten bew"ahrt, ist der Compass; eine Entdeckung, die unter die gl"ucklichsten und wichtigsten der Weltgeschichte zu z"ahlen ist. Die allgemeinste auch neuerlich noch behauptete Meinung ist\footnote{Annales des Voyages de l'histoire et de la Geografie par Maltebrun. Cahier 29. p. 252.}: Flavio Gioja von Amalfi habe im 16ten Jahrhundert zuerst den Gebrauch der Magnetnadel gefunden; allein D. Hager in einer k"urzlich herausgegebenen Schrift\footnote{Memoria sulla Bussola orientale. Pavia 1809.} hat aus chinesischen Urkunden erwiesen, dass dies Volk sich des Kompasses schon 1100 Jahre vor der christlichen Zeitrechnung bedient, und die Abweichung der Magnetnadel gekannt habe. Von ihnen empfingen ihn durch den Handel die Araber im 9ten Jahrhundert; denn als die Portugiesen auf ihrer ersten Umschiffung Afrikas an die Ostk"uste dieses Erdteils kamen, waren die Araber l"angst in Besitz des Kompasses; durch die Kreuzz"uge aber waren Amalsitaner bereits mit Syrien, Palestina und Arabien in Verbindung. Auf diese Weise also kam er wahrscheinlich nach Europa.

Hat der Magnet durch seine innere Kraft und magische Wirkung sich ein so hohes Ansehen erworben, dass er den Rang vor allen Steinen erhielt, aus dem Grunde, weil unter allen er am meisten Analogie mit der h"oheren siderischen Region, und desfalls den wichtigsten Einfluss auf die Wesen der niederen Erde haben m"usse, so wurden Metalle und Steine, (der Diamant selbst) umso h"oher gesch"atzt, als sie die Anziehungs-Kraft gleich dem Magnet besa"sen, und dadurch ihm verwandt schienen. Daher der Diamant in "alterer Zeit bei Borel und anderen Schriftstellern "uber Steine oft mit dem Magnet verwechselt, und beiden der N"ahme Adamas gegeben wird. -- Die n"achste Stelle in dieser Hinsicht nimmt das succinum (Bernstein) ein, den Griechen bekannt unter dem Namen ηλεκτρον von ηλεκτορ, Sonne, seines Glanzes wegen. Von seiner anziehenden Kraft redet schon Theophrast, glaubt den Stein aber ein Mineral.\footnote{Theophrasts Lynk"ur war wahrscheinlich auch eine Art Bernstein.} Seiner eigentlichen Natur gem"a"ser\footnote{Er ist n"amlich ein durch unterirdische G"arung ver"andertes und verh"artetes Fichtenharz.} beschreibt ihn Plinius (Naturgeschichte 37. 11.) durch den Mythos der Schwestern des vom Blitz getroffenen Ph"atons, die durch ihr Heulen in B"aume verwandelt wurden, und von deren Tr"anen j"ahrlich am Fluss Eridanos das Elektrum entstehe.

Einer Menge anderer Steine schrieben die Alten h"ohere Kr"afte zu, und reihen sie deshalb unter die dem Cultus vorz"uglich geeigneten. Sie aufzuz"ahlen, w"are eine ebenso m"uhsame Arbeit, als ihre Nahmen nach heutiger Nomenklatur bestimmen zu wollen. Was das orphische Gedicht von den Steinen sagt, hat vorz"uglich Bezug auf Magie und Heilkunde, die in ihrer Kindheit unter der Herrschaft der Divination stand, und am liebsten durch magische Mittel, Beschw"orungen, Zauberformeln, Amulette, heilte, weswegen auch in jener Zeit, wie noch jetzt bei den Wilden, Priesterw"urde und das Amt eines Arztes verbunden waren.

Unter den Alten hat Theophrast am meisten Bestimmtheit; allein Plinius, der oft auf blo"ses H"orensagen Nachrichten sammelte, und ohne Kritik h"aufig Steine mit mehreren Nahmen belegte, hat diese Materie so verworren, dass, vermehrt durch die empirischen Nachrichten und Zus"atze der Lithographen des Mittelalters, es mit allem Scharfsinne kaum m"oglich ist, die eigentliche Natur aller bei den Alten vorkommenden Steine zu bestimmen. Gleiwohl hat verbesserte Physik auch in dieses Fach helleres Licht gebracht, und es ist nicht zu bezweifeln, dass die wichtigen Fortschritte der Chemie in der Analyse der Erbsubstanzen, auch "uber die Natur und inneren Bestandteile der Steinarten uns neue Aufschl"usse geben werde.

Wenn wir auf den fr"uhesten Steingebrauch zur"uckgehen, der darin bestand, durch Reibung dem Menschen den Gebrauch des Feuers zu lehren, sind wohl jene Steine, die am leichtesten entz"undbar sind, und Feuerfunken von sich geben, diejenigen gewesen, die am fr"uhesten hervorgesucht, und weil sie heilige Gluth des Himmels, oder nach der fr"uheren Sprache zu reden, den Feuergeist enthielten, f"ur B"atylien galten. "Uberhaupt k"onnen wir annehmen, dass je mehr die Steine nach inneren und "au"seren Bestandteilen, meteorischen Ursprung, astralische Verwandtschaft, magische und h"ohere Wirkung "au"sern, umso mehr wurden sie als heilige Steine angesehen; denn nicht der Stein f"ur sich selbst, der Elementar-Geist, dass ihn belebende Princip, ward in ihm verehrt.\footnote{Die Alten, unbekannt mit der neueren Chemie, benannten die Steine weniger nach ihrem inneren Gehalte, als nach "au"seren Merkmahlen, nach ihrer Anwendung und N"utzlichkeit.}

In dieser Hinsicht k"onnen wir, um sie einigerma"sen zu klassifizieren, sie in nachfolgende Ordnung einteilen: 1) in feuergebende; 2) als solche, die durch Feuer oder sein entgegengesetztes Element Wasser erzeugt werden; 3) Steine, die durch Glanz, Farbe, Durchsichtigkeit, Form, oder andere Eigenschaften sich auszeichnen.

Unter den ersten (Feuer gebenden) ist vorz"uglich der Schwefelkies, Markasit, zu nennen, das gemeinste Mineral in der Natur, dass darum sowohl, wie seiner H"arte wegen, noch vor dem Feuerstein und anderen Kieselarten zum Feuerzeug gebraucht wurde, weshalb es auch den Nahmen Pyrit erhielt. Die h"artesten und feuerreichsten nannte man Pyrites vivos, und gebrauchte sie in Feldlagern; da er sch"one Politur annimmt, so ward er von den Peruanern, Griechen, R"omern, "uberhaupt dem ganzen Altertume zu Spiegeln gebraucht. Der eigentliche Feuerstein, der oft in Hornstein, zuweilen in Calcedon "ubergehet, ist unter allen Fossilien, teils wegen seiner H"arte, teils weil er am meisten Reibelektrizit"at besitzt, am dienlichsten zum Feuerzeuge.\footnote{Schmieder Versuch einer Lithurgik oder "okonomischen Mineralogie. Leipzig 1803.} "Uber ihn herrschte jedoch unter den Alten viele Dunkelheit, denn Plinius nennt mehrere Steinarten Silex; unser Feuerstein scheint sein Silex globosus zu sein, der sp"aterhin des Gebrauches wegen Pyromachus und Silex creta ceus hie"s. Wenn und wo der Gebrauch des Feuerzeugs aufkam, ist nicht zu bestimmen, aber seine Erfindung verliert sich in die Zeit der Mythe; ihr gem"a"s war, wie Plinius erz"ahlt,\footnote{Ignem e silice Pyrodes cilicis filius, eundem adservare in ferula Prometheus monstravit. Hist. nat.} Pyrodes der erste, der Feuer aus dem Kieselsteine schlug, Prometheus aber erfand eine Art von Lunte (Ferula), weshalb ihm wahrscheinlich Schuld gegeben ward, dass er heimlich das Feuer vom Himmel entwendet habe. Die ersten Worte sind aber blo"s mythisch zu nehmen, denn Pyrodes ist (wie Schmieder in der Lithurgik 2. Th. S. 174. bemerkt) dasselbe, was im franz"osischen Feu portatif, im Deutschen Feuerzeug bedeutet, und Cilix ist nichts anders als Silex. Vom Verfahren der Alten beim Feuerschlagen belehrt uns ebenfalls Plinius.\footnote{Hi exploratoribus castrorum maxime necessar"u, qui clavo vel altero lapide percussi, scintillas edunt, quae exceptae sulphure aut fungis aridis, vel foliis, dicto celerius ignem trahunt.}

Dieses Fossils (das erst nach Erfindung des Schie"spulvers und des Feuergewehrs seine wahre Anwendung erhielt\footnote{Daher neuerer Name: Flintstein.}), bediente man sich in "alteren Zeiten auch als schneidenden Instruments zu Opfer-Messern und Streit-"Axten; man findet dergleichen h"aufig noch in Grabm"ahlern.\footnote{So bei den Juden und den Priestern der Cybele zur Beschneidung daher er (vielleicht von Sicilex, Sicilis einem schneidenden Instrumente, und scindere schneiden) seine Benennung erhielt.} "Uberhaupt gebrauchte man alle kieselartigen Steine zum h"auslichen und religi"osen Dienste. Jener Pessinuntische, von dem es immer noch zweifelhaft bleibt, ob er ein Aerolithe gewesen, war vielleicht von dieser Gattung, denn bestimmt wird er religiosa Silex genannt; und warum gab man den Sand, Kalch, Kreide oder Seifenartigen Steinen nicht diesen Nahmen? mir deucht eben, weil sie in Hinsicht des Feuers zu religi"osem Zwecke nicht so brauchbar als jene waren.

Hier m"ussen wie auch des Fossils erw"ahnen, das unter dem Namen Donner-Steine bekannt ist, und wor"uber in alten sowohl als neueren Zeiten der Volksglaube allgemein war, dass es Donnerkeile seien, die der Blitz in die Erde vers"anke. Man findet sie h"aufig in Norden, besonders zu Streit"axten, Mossern und Waffenger"aten verarbeitet, oft in uralten Grabst"atten; deutlich erkennt man an ihnen die urspr"unglich parallelpipedale Gestalt; an einem Ende schliff man sie spitz zu, am andern wurden sie durchbohrt; andere findet man blo"s angebohrt, und zuweilen mit Stielen versehen. Die durchbohrten hing man nicht selten an Baum"aste, man hielt es f"ur ein Kennzeichen achter Donnerkeile, wenn sie bei Gewittern zitterten, und wenn ein fest um sie gebundener Faden im Feuer nicht verbrannte, welches, bei jedem andern Steine der Fall ist. Viele derselben sind wahrscheinlich achte Aerolithen, wor"uber Morhoff, der den Stein Lapis fulminaris nennt, die scharfsinnige Bemerkung "au"sert: "`manche k"onnten an der Oberfl"ache der Erde selbst vom Blitz aus den vorhandenen Materialien auf der Stelle gebildet worden sein. Ob die bekannten id"aisch-daktylischen Steine auf Kreta von der meteorischen Art, oder vielleicht Meeresprodukte, n"amlich versteinerte Belemniten find? ---. k"ame auf n"ahere geologische Untersuchung Viele dieser Donnersteine aber haben nach genauer Analyse, alle Kennzeichen des Lydit oder sogenannten Probiersteins, der ein blo"s in Schuttgebirgen umkommender jaspisartiger Kiesel-Schiefer, an Farbe meist schwarz sehr hart, und politurf"ahig ist.\footnote{S. Schmieders Lithurgik 2. 112. und Schr"oders Lithografisches Lexikon.} Diejenigen Steinarten, welche die Grundlage der Urgebirge machen, und "alter als die organische Sch"opfung sind, die Granit, Gneus, Syenit, Graustein, Porphyr, Tonschiefer, Urtrapp, Serpentin, Urkalk, sind ihrer H"arte und Unverwitterlichkeit wegen zu Unterlagen, S"aulen, Tempeln, Alt"aren, und allen dauerhaften Monumenten geeignet; die "Agypter allein mussten durch die H"arte ihrer Arbeitswerkzeuge ihnen bestimmte plastische Formen abzugewinnen; der Marmor hingegen ist seiner weichen zart geschmeidigen Natur und Empf"anglichkeit zur sch"onen Politur wegen, am geeignetsten f"ur plastischen Gebrauch, um die Formen der G"otter abzubilden, und dadurch das h"ochste Ideal der Kunst zu erreichen, was die Griechen ihren Gestalten auch zu geben wussten.

Unter den durch Feuer erzeugten Steinen nennen wir zuerst die aus der h"oheren Region zur Erde herabgefallenen Meteor-Steine; ihr Entstehen sei auch, welches es wolle; immer hatten sie (als dem Himmel entfallen oder mindestens in h"oherer Atmosph"are gebildet) den n"achsten Anspruch auf Verehrung. Gleich schreckhafte Meteore entstehen durch Vulkane, und die Steine; die aus dem inneren Erdenschlund ausgeworfen werden, mussten, da sie in fr"uherer Zeit noch h"aufiger waren, als jetzt, teils durch ihre mannigfachen Bildungen, teils durch den Nutzen, den man bald aus ihnen zu ziehen lernte, bald die Aufmerksamkeit an sich ziehen. Dergleichen sind Laven und ihre mannigfachen Untergattungen. Der Bimstein\footnote{"Uber den Theophrast K. 14. 19. 20. schon sehr klare Einsicht hatte.} z. B. insbesondere der Obsidian, der nach neuesten Forschungen von Humbold und andern ein eigentlich vulkanisches Edukt, und die Mutter des Bimsteins ist, dem er durch Feuer "ahnlich wird. --- Andere durch Feuer erzeugte Fossilien sind jene, die ihr Entstehen den in manchen Gegenden so h"aufigen Erdbr"anden verdanken. Bergharze, Asphalte, Naphten, wozu auch der Gagat geh"ort, die verh"artet einer sch"onen Politur f"ahig sind, und wie die Kunstgeschichte zeigt, in Babylon besonders und andern Gegenden Vorderasiens, wo diese Br"ande h"aufiger waren, vielfach verarbeitet wurden. Manche Steinkohlen, vorz"uglich jene, die viel Schwefelkies enthalten, k"onnen gleichfalls dahin gez"ahlt werden; die helmartigen schwarzen Steine, die nach Plutarch im Eurotas gefunden werden, und als geweihte Steine h"aufig im Tempel der Minerva Chol-Kiokos niedergelegt wurden, waren vielleicht von dieser Gattung.\footnote{Plutarch Abhandl. von den Fl"ussen.}

Da h"aufige Erfahrungen uns belehrt haben, dass nicht durch Feuer allein, sondern durch Wasser auch Steine entstehen, und prismatisch in gr"o"seren oder kleineren Massen sich bilden, so k"onnen wir vorz"uglich den Basalt dahin rechnen, ein Fossil, das im Altertum mehr noch als jetzt, seines sch"onen Schliffes wegen, den er annimmt, teils zu gr"o"seren Werken der Kunst, teils zu Talismanen und Skarab"aen benutzt ward. Gleiche Ursache, n"amlich Wasser, erzeugte auch jene "Uberbleibsel der ersten Sch"opfung, die Tr"ummern verloren gegangener Muschel- und Schaaltiere, die als Versteinerungen auf den h"ochsten Gebirgen oft --- Familienweis gelagert --- sich vorfinden. Die Echiniten besonders, und Belemniten mit ihren Untergattungen, von denen aus mehreren Beweisstellen des Altertums, die Falconet in der Abhandlung: Sur la Pierre de la mère des Dieux\footnote{Mém. de l'acad. des Inscr. p. 23.} gesammelt hat, erhellet, dass man ihnen Wunderkr"afte zuschrieb, und sie deshalb g"ottlich verehrte, besonders hatte dieses statt mit den Priapolithen und Hysterolythen, der heutigen Venus-Muschel, die Plinius Diphyes nennt, wobei er sagt; ut concubitu venereo aptum dieris, nisi lapis esset; und zwar ihrer "Ahnlichkeit wegen mit dem indischen Ioni und Lingam.

Wir kommen jetzt auf diejenigen Steine, die durch Glanz, Form, Farbe-Mischung und andere Eigenschaften besonderen Wert erhielten: die Edelsteine n"amlich, jene prachtvolle sch"one Erzeugnisse der anorganischen Sch"opfung, der Schmuck, die sch"onste Bl"uthe des Steinreiches, deren Schimmer und Strahlenglanz in den fr"uhesten Zeiten schon aller Augen entz"uckte, und als h"ochste Seltenheit aufgesucht wurde. Unsere Schrift w"urde ihre Grenzen "uberschreiten, wollten wir sie alle hier aufz"ahlen und nach ihrem inneren Gehalte erforschen, die festere Textur der Teile, ihre Durchsichtigkeit, H"arte, Gl"atte und Zartheit der Politur gibt ihnen zwar hohe Vorz"uge vor den gemeinen Metallen; gleichwohl haben sie (den Demant ausgenommen, der seiner inneren Natur nach reiner kristallisierter Kohlenstoff ist) den Metallen und deren Mischungen ihren h"ochsten Schmuck, die Farbe zu danken.

Asien, die Wiege des Menschen-Geschlechts, besonders jenes an Produkten so gesegnete Land Indien, ist auch die Gegend, aus deren reichhaltigen Gruben im hohen Altertume schon die vorz"uglichsten Steine kamen, die teils f"ur Gegenst"ande des Luxus, teils zu gottesdienstlichen Gebr"auchen bestimmt waren, um priesterliche Gew"ander, Tempel und Alt"are zu schm"ucken.

Aus diesem Lande der "altesten Mythen kamen zuerst auch jene Sagen der Wunderkr"afte und heilbringenden Eigenschaften, die man vielen unter diesen Steinen zuschrieb, und da der weite Handel, der damit getrieben wurde, sie durch viele H"ande oft unkundiger leichtgl"aubiger Steinh"andler gehen lie"s, vermehrte sich noch der Wunderglaube an magische Wirkungen dieser Steine, wozu unleugbar noch ihr innerer Gehalt und manche Wirkungen, z. B. die magnetische und elektrische Kraft derselben beitrugen. So hingen sie auch als Arzneimittel, innig mit der "alteren Heilkunde sowohl, als mit Magie und Astrologie zusammen, davon die Orphische Steinschrift, Theophrast und Plinius im 37ten Buche die "uberzeugendsten Beweise liefern.

An diesem Glauben der Wirkungen gewisser Steine auf dies oder jenes Uebel, hing das ganze Mittelalter, und noch heut zu Tage ist er im Orient herrschend.

Manche Eigenschaften dieser Steine sind auch wirklich so auffallend, dass Menschen, denen die chemischen Bestandteile derselben nicht bekannt find, diese Ph"anomene leicht f"ur Wirkungen h"oherer Kr"afte gelten k"onnen. Ich will nur einige der merkw"urdigsten nennen. Denn alle im Plinius vorkommenden Steine dieser Art lassen sich kaum mehr mit Genauigkeit bestimmen, da er unter die Gemmen so viele rechnet, die nicht dazu geh"oren; wie z. B. nur er vom Ammons-Horn sagt: Cornu ammonis inter sacratissimaa Aethiopiae gemmes refertur. Besonderer Erw"ahnung verdient 1) der Turmalin, ein Fossil, von dem zwar die alten Nachrichten schweigen, dessen Dasein ihnen doch h"ochst wahrscheinlich bekannt sehn musste, da es in Ceylon, dem alten Tabrobane, mit welcher Insel bekanntlich ein sehr alter Handelsverkehr war, zu Hause ist, weshalb dieser Stein wohl auch unter denjenigen Gemmen, deren Plinius erw"ahnt, ohne sie genau zu bestimmen, leicht verborgen sein konnte. Die au"serordentliche Polarit"at\footnote{Und zwar ist der eine Pol vositiv, der andere negativ elektrisch. Wenn aber zwei erw"armte Turmaline auf Papier "uber Wasser schwimmen, so verhalten sie sich wie zwei Magnetnadeln, indem sich die gleichnamigen Pole absto"sen, und die ungleichnamigen anziehen.} dieses krystallf"ormigen Sch"orls, und seine Elektrizit"at, verm"oge welcher er im Sonnenschein, in hei"ser Asche, auf hei"sem Eisenblech oder durch Reiben erw"armt wird, machen ihn vorz"uglich merkw"urdig.

2. Der Lazur-Stein, Cyanus, den die Alten auch Saphir nannten, der heutige Lapis Lazuli, verdiente besonderer Erw"ahnung, weil (wie wir gesehen haben) schon Hiob ihn nennt, und Theophrast Kap. 30. 38. 47. 50. ihn beschreibt, sehr verschieden ist er jedoch von unserem Saphir, der weit durchsichtiger ist, und nach der Beschreibung, die Plinius von der Asteria, einem in Indien und Caromanien einheimischen Steine macht, treffen beide ihrer Natur nach aufs genaueste zusammen, indem Lehmann und andere Lithographen gleichfalls an ihm entdeckt haben, was man des Plinius Aussage nach f"ur Fabel hielt, n"amlich: dass angeschliffen, er den Schimmer von mehreren "ubereinander liegenden Sternen\footnote{Br"uckmann Abhandlung von Edelsteinen; ach Schmieder Lithurgik 2. 260.} zeigt, daher ihn die Neueren Girasol nannten, und ihn bald f"ur Calcedon, Kazzenauge-Opal, bald f"ur Kristall hielten. "Uberhaupt gaben die Alten den Namen Lapides Stellares allen Edelsteinen, die, wenn sie geschliffen sind, einen stern"ahnlichen Glanz geben, und denen man ihrer anscheinenden Analogie wegen mit jenen helleren Lichtern der Himmels-Region, auch h"ohere geheime Kr"afte zuschrieb. Plinius\footnote{Nat. Gesch. Buch 37.} gibt als solche vorz"uglich an: die Asterie, Astria, Astroides, Astrobol, Akopos, Antipatos, das Belus-Auge (ein babylonischer Stein) Sonnen-, Mond- und Komet-Steine, den persischen Mythras, Heliotropos, Calcophonas, Karfunkel (der heutige Granat), H"aphestites, worunter jedoch mehrere eher zu Versteinerungen zu rechnen. Zum Kunstgebrauch w"uhlten die Steinschneider vorz"uglich den Praser,\footnote{Ein gr"uner Quarz, dem Juweliere oft, jedoch uneigentlich, den Namen Smaragd-Mutter geben.} Calzedon, Onix, Jaspis und Achat. Die Steine, aus der die sogenannten Amulette, Abraxas, Talismane, Lapides divi oder vivi best"unden, waren also eben so verschieden, als der Gebrauch, den man von ihnen machte, und man muss sie hiernach ebenso sorgf"altig, unterscheiden, als nach der Gegend, woher sie kommen; so sind zum Beispiel Harz- und Asphaltischer Natur die babylonischen mit Keil- oder Pfeilschrift bezeichneten Backsteine, die am wahrscheinlichsten Zauberformeln, oder, (wie Plinius versichert) astronomische Beobachtungen enthielten.\footnote{S. D. Hagers Abhandlung "uber die vor kurzem entdeckten babylonischen Inschriften, in Klaproths asiatischem Magazin.} Die persischen Zylinder, deren Farbe gew"ohnlich wei"s oder blaulicht, zuweilen auch schwarz ist,\footnote{Mehrere derselben liefert Cailus Recueil d'antiquités, und Montfaucon ant. expliquée.} worauf sich gew"ohnlich Abbildungen mit oder ohne Schrift befinden, sind ebenfalls zu den B"atylien zu rechnen. Was die "agyptischen betrift, so ergibt sich aus der h"aufigen Gemeinschaft, die in fr"uhen Zeiten zwischen diesem Lande, Babylonien, Chaldea und Persien war, dass von dort aus h"aufig geweihte Steine oder Talismane vorkommen, die den persischen auffallend "ahnlich sind,\footnote{Beispiele sehe man in Cailus Recueil d'antiquités. Tom. 5. pl. 12. 13. 14. 17. Ferner Tome 1. planehe 17. Tome 2. pl. . Tome 3. pl. 1. No. 4. T. 4. pl. 21. 22. T. 6. pl. 19. 20. 21. 22. T. 7. pl. 6. Auch Montfaucon antiquite epl. . 2. part. 2.} wie "uberhaupt sich immer mehr zeigt, dass der mythische Stoff aller V"olker aus einer und derselben Quelle floss; daher auch eine die andere aufschlie"st und erkl"art. Ferner scheint erwiesen, dass von allem Aberglauben, wodurch reine Gottesverehrung in Abg"otterei ausartete, der Fetischen-Dienst, und der zugleich damit entstandene Gebrauch der Amulette, Talismane, Zaubersteine der fr"uheste gewesen,\footnote{Zoega de orig. et usu Obelisc. p. 241.} mit allm"ahliger Zunahme dieses Aberglaubens wurden bei immer vermehrtem Verkehr der entferntesten V"olker, diese Steine endlich ein Gegenstand des Handels, der vorz"uglich von Chald"aern und Persern, am meisten aber von arabischen Idum"aern oder Edomiten in die fernsten L"ander getrieben ward, dessen Hauptsitz (wenigstens Stapelplatz) Heliopolis\footnote{Von dem Handelsverkehr dieser Stadt mit Ph"onizien, Afrika, und dem s"udlichen Asien, s. Heeren Ideen "uber den Handel der alten V"olker 1. Th. S. 759.} war, nebst den umliegenden St"adten am Libanon (wo der Sonnendienst herrschte, und woselbst, wie wir gesehen haben, sich ein so gro"ser Vorrat B"atylien befand) was umso nat"urlicher ist, als durch diese Gegend eine noch heut zu Tage von den Caravanen besuchte Haupthandelsstra"se geht; aber nicht blo"s ph"onizische Stoffe, arabische Gew"urze, Gold, Perlen, Edelgesteine und andere Handelswaren wurden dort vertauscht, auch religi"ose Gegenst"ande teils zum Priester- und Tempeldienst geh"orig,\footnote{Ein Beispiel finden wir noch im christlichen Zeitalter an jenen silbernen Dianen-Tempelchen, die von dortigen K"unstlern verfertigt weit und breit verschickt wurden.} teils durch Aberglaube dem Volke unentbehrlich geworden, wurden auf dieser Stra"se in alle L"ander verf"uhrt, die ein gemeinsamer Cultus verband. Wir wissen aber, dass Sabaeism oder Verehrung der Gestirne die fr"uheste Religion war, deren Gebr"auche und Symbolik man unter mannichfachen Modifikationen von Hinterasien an zum Nil nach Europa her"uber bis in den n"ordlichen Kaukasus verbreitet findet; und da, wie wir fr"uher gezeigt haben, Sterne, dieser uralten Lehre gem"a"s, f"ur g"ottliche Wesen angesehen wurden, deren Einfl"ussen alles Geschaffene unterworfen w"are, so ist der daher entspringende Glaube an geheime, den Pflanzen, Metallen und Steinen in wohnenden (durch Einfluss der Konstellationen) wirkenden Kr"afte, wie aller mit solchen Steinen getriebene Missbrauch aus dieser Quelle herzuleiten.

Mehrere Altertumsforscher\footnote{Unter andern Montfancon ant. expliquée T. 2. art. 3. der zu den Abbildungen in Chifflet-Traktat on Talismanen und Abraxas noch viele andre gesammelt hat.} haben die gnostischen Valentinianer und Basilidianer als Urheber des gro"ser Verkehrs angegeben, der bis ins vierte und f"unfte christliche Jahrhundert mit B"atylien getrieben wurde, ja sie selbst als Erfinder er unter dem Namen Talismane, Amuletten, Skarab"aen, Abraxas, bekannten Zaubersteine angegeben; allein diesem widerspricht schon die Tatsache; dass die Gnosis (wie gleichfalls die Lehre des Manes) von der die meisten theurgischen Sekten und philosophischen Lehrsysteme jener Zeit entstanden, aus dem Petsismus oder Zerduschts Lichtlehre hervorgegangen,\footnote{Man lese hier"uber die Quellen im Zend-Avesta Upnekhat, und was Kleuker dar"uber gesammelt hat' im kl. Teil seines deutschen Zend-Avesta an mehreren Stellen. Auch Beausobre Hist. crit. du Manichéisme in mehreren Stellen.} und gleichsam nur ein tr"uber Spiegel ist, in dem man gleichwohl die obschon entstellten Z"uge des ersten beinen Bildes erkennt. Christliche Ideen mischten diese Schw"armer mit Chald"aisch-persisch-"agyptischen\footnote{Von "alteren und sp"ateren "agyptischen Talismanen s. Kircheri Oedipus T. 2. p. 459.}; und wurden durch den Handel, den sie mit solchen Steinen trieben, in der sp"ateren Zeit das, was f"ur fr"uhere V"olker die Chald"aer und Araber waren; "uberschwemmt haben sie freilich mit ihren geweihten Talismanen das ganze Morgen- und Abendland bis nach Spanien und Gallien; aber Caylus hat an mehreren Stellen seines sch"atzbaren Recueil d'antiquites gezeigt, dass der Umsprung und Gebrauch solcher Zaubersteine aus den fr"uhesten Zeiten herr"uhre. Da es nun einmal erwiesen ist, dass B"atylien zu den vom Himmel gefallenen Steinen geh"oren, wir zugleich aber aus Plinius\footnote{Nat. Gesch. 2. Buch, Kap. 25. 6. 7. Heradian 1. Buch, 138. Livius in mehreren Stellen.} und anderen Quellen wissen, dass auch andere K"orper: Balken, Lanzen, Spiese, leuchtende, wie "Ahren geflochtene Kr"anze, Feuers"aulen, Sternschnuppen u. a. zur Erde fielen, so erhellet daraus, dass unter die Klasse der Διιπετρα, die deswegen heilige Steine hie"sen, weil sie dem Himmel zugeh"orten, und himmlischen Ursprungs sind, man nicht blo"s Aerolithen, sondern: "uberhaupt alle Steine, denen man "ubernat"urliche Kr"afte zuschrieb, und die als Gemmen gebraucht wurden, rechnen k"onne.\footnote{S. Passeri Gemmae astriferae, Kircheri Oedipus T. 2. P. 2. und die Schriften "uber Talismane und Abraxas, besonders Traite des Talismans ou Figures astrales. Paris 1668.} Dass ferner aller Aberglaube, den man im Altertume mit solchen Steinen trieb, aus dieser Quelle floss, woher auch die Talismane entstanden. Diese Talismane waren von verschiedener Form und Gr"o"se: "`Erant, sagt Kircher in "Odipus T. 2. p. 445 duplicis generis. Majora et Minora. Majora, et immobili positae solidata, in publicis locis Urbium; Templorum, coemeteriorum, tum regionum claustris ad hostium arcendum insultus et αντιτεχνιας δαιμονων κακουργων eludendos ponebantur. Minora et portatilia in domibus, in Collo, pectore, manibus hominum, animaliumque ad malorum Averruncationem portata serviebant. F"ur einen solchen kann auch der "agyptische Canopus gelten, der dem geheimen religi"osen Sinne nach, ein Symbol des Wassers als wohlt"atigen befruchtenden Elements war, wie dies eine Stelle des Abenephius, eines arabischen Schriftstellers zeigt, den Kircher im "Odipus T. 1. p. 211 anf"uhrt; habent ipsi, sagt er, idolum quoddam Canopis nomine, et est in modum Vasis tumidum, et quando Aquis plenum fuerit, Aqua per Ubera, quae in eo effinxerunt, refunditur, et indicatur efluxu, processus naturae omnia nutrientis. s. auch Suidas. --- Ruffin hist. eccl. 1. 11. --- Porphyr apud Euseb. eccl. Hist.

Ein Talisman von der gr"o"seren Gattung w"are jener neun Ellen hohe smaragdene Koloss des Serapis im "agyptischen Labyrinthe, dessen Appian erw"ahnt, s. Zoega de usu et orig. Obelisc. p. 8 auch andere Bilds"aulen von Isis und Osiris, Horus Hermes, die an die Eing"ange der Tempel, Grabm"aler, an Grenzorte u. s. f. als Schutzgenien gesetzt wurden. Ihr Gebrauch verliert sich in "Agypten, wie in allen L"andern in die fr"uhe Zeit des rohen Fetischen-Dienstes. Solche Schutzg"otter von aller Form und Gr"o"se finden sich nach dem Zeugnis der Reisenden noch heut zu Tage bei allen wilden V"olkern, und wie viele derselben ihren Toten selbst Talismane und Amulette mit ins jenseitige Leben geben, so wickelten die "Agypter aus Furcht vor den Nachstellungen Typhons und anderer b"osen schadenden Wesen zwischen die Bandagen ihrer Verstorbenen, Amulette, kleine Osiris Idolen, Skarab"aen, Nilpferdchen, kleine Knuphschlangen, geschnittene Steine, u. dgl. m. Hier l"asst sich vordersamst die Frage aufwerfen, ob eigentliche Aerolithen, d. h. jene Steine, die man nach der in neuerer Zeit mit ihnen vorgenommenen Analyse, f"ur "achte Meteorsteine erkennt, zu jenen geh"oren, die zu Gemmen bearbeitet und graviert werden konnten? --- Nachrichten in Theophrast, Plinius und anderen Schriftstellern "uber Steine\footnote{Von neuern vorz"uglich Paracelsus, Albertus magnus, Lud. Dulcis, von Boot Hist. Gemmarum et Lapidum, nach ihrem Resultat dargestellt in Br"uckmanns Abhandl. von Edelsteinen.} sind keine dar"uber vorhanden, indessen scheint aus ihrer Natur und der lockeren, leicht oft gleichsam durchsichtig verbundenen Textur derselben (welche dem Rade und der Bearbeitung des K"unstlers kaum widerstehen w"urde) die Schwierigkeit zu erhellen, sie zu diesem Zwecke zu gebrauchen, wodurch indessen nicht geleugnet wird, dass h"artere, mehr und inniger verbundene Meteorsteine vor Alters nicht sollten bearbeitet worden sein?\footnote{Schon Theophrast von den Steinarten Kap. 6. erw"ahnt drei verschiedene den Alten bekannte Steinarbeiten, λιθοτομια, lapicidaria, Steinmetzkunst, τορευτικη Steinmetzkunst, und τλυφη Steinschneidekunst; die Lithotomen gruben Inschriften mit eisernen Griffeln in Marmor u. d. gl., Toreuten drehten Gef"a"se aus Marmor, Alabaster u. s. w., Skalptoren arbeiteten in alle Steinarten, die Eisen nicht angreift, mit diesen gruben sie in die vorhergespaltenen Steine entweder vertiefte Figuren (jetzt Intaglio, Incisura genannt), oder erhabene Figuren, caelatura (jetzt Kameen), oder sie gaben Edelsteinen eine beliebte Form, z. B. Oval. In Kap. 42. sagt er ferner: das Eisen schneidet auch in festere und h"artere K"orper, weil es mehr Zusammenbang hat. Vergl. damit Plinius Nat. Gesch. 37. 12.} umso mehr, als aus den neuen Untersuchungen der HHr. Scherer und Schreiber "uber die m"ahrischen Meteorsteine (in Gilberts Annalen der Physik 1809, 1s St"uck) sich ergibt, dass diese Aerolithen nicht allein eine sch"one Politur annehmen, sondern auch zu Vasen und andere Formen sich leicht bearbeiten lassen; wie denn jener des Arztes Isidorus zu Emesa, nach der Beschreibung, die Photius von ihm gibt, mit zwei Sternen bezeichnet war, und wahrscheinlich geh"ort der persische Zylinder, den Millin in seine Monuments inedits nouvellement expliqués Tome I aufgenommen hat, wie die meisten, deren Caylus in seiner Sammlung erw"ahnt, unter die Aerolithen; aber in jenen, die zum Steinschnitte und k"unstlerischen Gebrauche nicht dienlich waren, hatte man gleichwohl R"ucksicht auf die nat"urlichen Striche und Zeichen, so sich auf ihnen befanden, die man f"ur heilige Zahlen und Zauberzeichen ausgab; woher auch die besondere Achtung entstanden sein mag, so man jenen unter dem Namen Echimten, Hysteriolithen bekannten Muschel und Versteinerungsarten bezeichnete. Was die Erkl"arung der auf ihnen befindlichen Zeichen betrift, bleiben dieselben nur so lange dunkel, als man nicht auf die Hauptquelle zur"uck geht, aus der sie entstehen, und worin der einzig wahre Schl"ussel zu ihrer Entzifferung zu finden ist, n"amlich das System der Emanation, verm"oge welchem es ein ewig einziges Wesen gibt, das alle andere schuf, und regiert, aber nicht unmittelbar, sondern durch mehrere ihm untergeordnete Mittelwesen, die den verschiedenen Teilen der Welt vorstehen, und sie als Boten des ewig unerschaffenen leiten; organische und anorganische Sch"opfung, Menschen, Pflanzen, Steine und Metalle stehen unter ihrer Gewalt. Diese leitende Mittelwesen aber, die den Gestirnen vorstehen, wurden bald mit ihnen verwechselt, und selbst g"ottlich verehrt. Sonne und Mond waren ihre Herrscher, ihnen untergeordnet eine Hierarchie von Planeten, Fixsternen, und das ganze Heer der leuchtenden Himmelsschaaren, denen man ihre Bahn, ihre Wohnungen, ihren Einfluss auf Ver"anderung der Jahreszeiten, Witterung, ja selbst den verschiedenen Konstellationen gem"a"s, auf das Lebensprinzip das Schicksal, Wohl und Weh aller Wesen zuschrieb, woher auch der Glaube des astralischen Einflusses der h"oheren auf die niedere Elementarwelt, und der daraus gleich anf"anglich damit verbundene Gebrauch der Magie, Divination, und astrologischen K"unste. Diesem Glauben an Einfluss der Gestirne und der ihnen vorstehenden Unterg"otter (eben derselbe, vor dem Moses das erw"ahlte Volk warnte,\footnote{Deuteron. 4. v. 16.} waren am fr"uhesten die Babylonier und Araber ergeben. Um die reinen Intelligenzen sich geneigt zu machen,\footnote{Pocock Specimen Hist. Arab. p. 139 seqq.} verehrten sie die Planeten in ihrem Heiligtume, daher schnitzten und pr"agten sie dieselben in Bildnisse aus, und wiesen jeder Gegend, jeder Pflanze, jedem Steine ein ihm entsprechendes Gestirn an, teilten unter sie die Jahreszeiten, Monate, Wochen, Tage und Stunden, beobachteten ihren Lauf, ihre Behausung, ihren Standort, ihr Auf- und Niedergehen, ihre Ann"aherung und Gegens"atze, ihre Phasen, Anschauungen, ihr Verschwinden, und was daraus erfolgte.\footnote{Ein arabisches Gedicht, welches Ebn-Khaldoun, ein Schriftsteller des 8ten christlichen Jahrhunderts in seinen historischen Prolegomenen anf"uhrt (S. Abd-Alatif Relation de l'Egypte traduit de Silvestre de Sacy pag. 512.) beschreibt folgender Ma"sen die dabei "ubliche Zauberformel:\\
\hspace*{0.5cm} "`Toi qui desire apprendre le secret de faire absorber les eaux, écoute les paroles de vérité que t'enseigne un homme bien instruit: laisse là toutes les recettes mensongères et les doctrines trompeuses dont les autres ont rempli leurs livres, et prête l'oreille à mes discours et aux conseils que je te donne, si tu ès du nombre de ceux qui ue suivent point le mensonge. Lors donc que tu voudras faire absorber les eaux d'un puits qui inspire l'effroi à l'imagination embarrassée et incertaine sur les moyens d'executer une telle entreprise, tu auras recours au talisman suivant. Fais la figure d'un homme dont les deux mains tiennent la corde qui sert à tirer le seau du fond du puits. Sur sa poitrine, trace la figure de la lettre ha, comme tu la vois ici; trace la autant de fois, que le divorce peut avoir lieu, et non davantage; qu'il foule aux pieds les figures de la lettre ta, sans cependant les toucher tout-à-fait, imitant la marche d'un homme prudent, fin et adroit. Qu'une ligne entoure tout cela; la forme carrée vaut mieux que la forme circulaire. Immole un oiseau sur ce talisman, que tu frotteras avec le sang de cette victime, apres quoi tu procéderas aux fumigations de sandaraque, d'encens, de stacté et de costus. Ensuite tu le couvriras d'une étoffe de soie, rouge, jaune ou bleue, où il n'y ait ni couleur verte, ni taches. Tu le lieras de deux brins de laine blanche ou rouge, d'un rouge pur. Il faut que cela se fasse quand le signe du lion monte sur l'horizon, ainsi qu'on l'a bien expliqué, dans le temps que la lune de ce mois n'éclaire point; la lune doit être jointe à la Fortune de Mercure, un jour de samedi, à l'heure où tu feras cette opération."'\\
\hspace*{0.5cm} Was vom Monde erw"ahnt wird, soll sa viel hei"sen, dass derselbe in demselben Zeichen mit dem Mercur sich befinden, und dieser in einer g"unstigen Gl"uckbringenden Station sein m"usse; denn die unmittelbaren Einfl"usse der Planeten sind nach Bewandtnis ihres Standortes, und der Aspekten gegen andere Planeten gro"sen "Anderungen unterworfen. "`Sunt -- sagt Albacit -- ad magisterium judiciorum astrorum isagoge Paris 1521. planetis loca in quibus confortantur, et loca in quibus fiunt fortunae, in quibus fiunt malae."' So deutet die Vereinigung des Monds und Mercurs in derselben Behausung, nach astrologischen Gesetzen, auf gl"uckliche erw"unschte Zukunst, und der Haupteinflu"s des Mercurs geht nach Ptolomaeus Opus quadri part. Buch 1. Kap. 4. 5. auf D"urre und Austrocknung.} Wollten sie hiernach sich z. B. den Saturn geneigt machen, und durch ihn etwas erwirken, so w"ahlten sie hierzu die erste Stunde des ihm geweihten Samstags; und indem sie eigene mit diesem Planeten sympathisierende Gew"ander umtaten, verrichteten sie dem geschnitzten Sinnbild des Gestirns ihre Gebetsformeln, mit vollem Glauben auf die Erf"ullung ihrer W"unsche.\footnote{Kircheri Oedipus, Artikel magia hieroglyphica aeptiorum. T. 2. p. 437.} Dasselbe hofften sie auch von denen eigends dazu geweihten mit dem oder jenem Gestirne in Sympathie stehenden Steinen, dem ein guter oder b"oser Daimon in wohnte, der jedoch erst durch die bei seinem Gebrauche ausgesprochenen Zauberformeln, womit die auf dem Steine gepr"agten Zeichen in Bezug stunden, wirksam wurde.

Gleichfr"uhzeitig war dieser Astraldienst in Persien einheimisch; unter den geistigen Mittelwesen aber, die gem"a"s des Zerduschtischen Licht-Systems die Gebote des Allerh"ochsten Unerschaffenen Zeruane Akherene in der Sch"opfung verrichten, war nach den Amschaspands (den reinsten Intelligenzen) keines in gr"o"serem Ansehen, als Mythra.\footnote{Dessen Cultus zwar in Persien eiheimisch ist, der aber, wie wir aus Plutarch in Pompejo und Firmicus de Errore profan. relig. 100. 5. lernen, sp"aterhin mit andern Modifikationen sich in Phrygien wieder erneuerte, und von Rom aus, wo er vorz"uglich im Jahr 687 herrschend war, sich im ganzen Occident verbreitete, bis er im Jahr 378 nach Christo g"anzlich vertilgt wurde.} Der erste vornehmste der Ized (Genien des Himmols, und Personifikationen der guten Sch"opfung). Mythra der Erhalter und Begleiter aller geschaffenen Wesen, Geber des Lichts, der W"arme, des de fruchtenden Regens und aller Lebenskr"afte; dem b"osen Einfluss der Dews und Daroudis entgegen gesetzt; der Besch"utzer alles Reinen, nicht zwar selbst die Sonne (d. h. jene des h"oheren Himmels) sondern ein Mittler zwischen den zwei Urelementen Feuer und Wasser, oder in elementarischer Beziehung zwischen Sonne unb Mond, dem m"annlichen und weiblichen Sch"opfungs-Prinzip\footnote{S. Zend à Vesta T. 2. vendidad. p. 209 seq. nach der franz"osischen "Uberziehung von Anquetil.}; daher er urspr"unglich wie alle Gottheiten hermaphroditisch, sp"uter aber mit getrennten Geschlechtern, m"annlich und weiblich vorgestellt ward, woher auch die unterscheidende Benennung von Mythras und Mythra, sp"aterhin μιθρισ Zeus und Astarte, Pater magnus und Dea magna, μιθρα die himmlische Venus genannt, denn Herodot\footnote{Herodot 50. 1. 100. 131.} sagt ausdr"ucklich: die Perser opferten nebst dem Jupiter auf hohen Gebirgen, der uranischen Venus, die sie M"uthra, die Assyrer Myllita, die Araber Alitta nennen.

Auch zu diesem Mythos gab wahrscheinlich ein gefallener Aerolithe Anlass, denn, eine alte Sage, wie wir aus den Kirchenv"atern lernen, erz"ahlt,\footnote{St. Justin Dialog. contra Tripho, p. 296. Julius Firmicus Error profan. relig. Cap. 5. Commodian Intr. 13. St. Hieronymus adv. Jovian 50. 1. T. 4. p. 2.} Mythras sei von einem Stein geboren worden: θεοσ εκ πετρασ und setzt die Ursache hinzu; weil man aus Steinen Feuer schlage; ein nicht unwichtiger Umstand, der trefflich den geheimen Sinn aufschlie"st, welchen der Mythos aller Stein-Gottheiten hatte, n"amlich: dass das Element des Feuers als Symbol des Lebens im Stein, wie in jedem Geschaffenen verborgen sei. Die Abbildung eines solchen Mythras findet sich noch in der Justinianischen Sammlung, und zwar nach der "altesten Form, ein aus rohem Felsen hervorgehender Kopf, ihm zur Seite zwei junge Mythras seine S"ohne, denn die Fabel l"asst ihn aus der Verbindung mit einem andern Steine, zwei Kinder erzeugen, die man deswegen Diorphi hie"s.\footnote{Diesen Mythos erz"ahlt Plutarch in der Abhandl. von den Fl"ussen, Art. Araxes.}

Aus dieser, wie aus allen ihr "ahnlichen Mythen, geht das Resultat hervor: dass B"atylien und heilige Steine als Symbole h"oherer Kr"afte, das ist der Elemente angesehen wurden, indem jedes Wesen des oberen Himmels (Amschaspands nach der persischen Lehre) deren es 7. gab, n"amlich: Mensch, Tier, Feuer, Metalle, Erde, Wasser, B"aume, in der niederen Welt --- der Erde, eine ihm entsprechende Form hat, der sie sich freundschaftlich zuwendet, und sie gern bewohnt, um aus und durch sie zu wirken; daher auch Abraxas, Talisman, Amulette, ihre Beziehung auf diese h"oheren Kr"afte haben, und gleichsam deren Orakel sind.

Ihre anf"anglich rohe Bezeichnung, wie an jedem Aerolithen, wurde immer zusammengesetzter und geh"aufter mit Hieroglyphen, je sp"ater man dieselbe von denen aus dem urspr"unglichen Licht- und Natur-Kultus ausgegangenen Sekten angewandt findet; indessen zeigt selbst die Signatur der gnostischen Abraxas mit 385. ihre Beziehung auf Zelt- und Jahres-Wechsel. "`Basilides,"' sagt der heilige Hieronymus,\footnote{In dessen Kommentar "uber Amos.} "`gab Gott dem Allm"achtigen den Nahmen Abraxas und behauptet: dass nach der Bedeutung der griechischen Buchstaben und der Tagszahl des Sonnenlaufs, Abraxas sich in seinem Kreise eingeschlossen befinde."' Diese Stelle wird, wie Macarius bemerkt, durch eine andere des heiligen Augustins\footnote{Montfaucon Ant. expl. Tom. II p. 356.} erl"autert, der von Basilides sagt: er behaupte, es g"abe 365 G"otter, weshalb er die Abraxas f"ur heilighalte, weil diese Tagszahl sich im Jahr befinde. Es sind n"amlich die griechischen Buchstaben: α, β, ρ, α, ξ, α, σ, analog den Zahlen: 1. 2. 100. 1. 60. 1. 200. die zusammen verbunden die Zahl 365 bilden; daher der N"ahme Abraxas gleichbedeutend mit Mithras, (beide die Sonne und ihren Umlauf symbolisierend) genommen ward. Aus welcher Quelle auch der Missbrauch entstand, den die Gnostiker mit diesen Nahmen trieben, die sie selbst mit dem g"ottlichen Lehrer Christus, als Bild der Sonne, vermischten, dasselbe gilt von der Benennung Οφισ (dem Symbol der Ophiten) und ιαω --- Sabao, Sabazos, Sabaoth, Herr der Heerschaaren, Adonai und anderen Benennungen des Allerh"ochsten, zu denen als Untergattungen noch die Nahmen der vollziehenden Himmelsboten, Kr"afte, Potenzen zu z"ahlen sind, deren man in Montfaucon\footnote{Ant. expl. loc. cit.} mehr als hundert gesammelt findet; ja die zartf"uhlenden Hindu, die alle geistigen Kr"afte symbolisierten, und in der Natur von den Sternen bis zum Grashalm, alles belebten, z"ahlen deren viele tausende.\footnote{Aoditja, (sagt das Sanskritische W"orterbuch Amarasinha, herausgegeben von P. Paulino, Rom 17s. p. 4.) ist der allgemeine Nahmen dieser Devatas: Aoditja in plurali duodecim Deos sunt, qui praesunt anni, mensibus, ac proinde allegorice duodecim. Stationis puncta, in quibus sol versari videatur, 1. P. Ildefonsus Mission. in Cod. ms. de sectis et Relig. Indorum. --- Item: in secundo Choro numerant (Indi) et adorant triginta et tres milliones Deorum, quos vocant Deos coelestes, inter quos numerant deum solem, deam Lunam, Deas planetas, et Deas Stellas; insuper in hocce Choro computant Elementa pro Diis; so sagt auch P. Marcus à Tumba, (von Fra. Bartholom"as im Amarasinha angef"uhrt) gli elementi, li pianeti, li venti sono Die; und zwar nach demselben Amarasinha von m"annlicher (Pullinga) oder weiblicher (Sri Devata) Natur oder geschlechtlos Clibè. --- s. auch F"orsters Bemerkungen zur Sokontala p. 256.} Unbekannt jedoch, wenigstens minder geachtet, blieben ihnen die Abraxas, deren wahre Heimat Chald"aa und Persien ist. Die oben erw"ahnte nahmen des h"ochsten Gottes und der Heerschaaren nun, verbunden mit den Zeichen der Gestirne, und der ihnen entsprechenden Konstellationen bildeten die auf diesen magischen Steinen gepr"agte Zauberformeln, mittelst welcher man in der Gestalt eines umgest"urzten Kegels, den man aus den Buchstaben ΑΒΡΑΚΑΔΑΒΡΑ zusammensetzte, Beschw"orungen vornahm, und dieselbe teils als Heilmittel bei Krankheiten, teils als Schutz und Rettungs-Werkzeuge wider b"ose D"amonen gebrauchte.\footnote{Diesen Gegenstand hat ausf"uhrlich erl"autert Kreuzer in seinem trefflichen Werke: Symbolik und Mythologie der alten V"olker. 1. Band. S. 286. -- Im Ganzen k"onnen alle Lokal-G"otter von St"adten, Gegenden und L"andern zu den Schutzgottheiten gerechnet werden; die meisten G"otter, die in "Agypten vor den Tempeln und Pyramiden standen, waren solche Talismane, wie dies auch aus dem Zeugnis der Arabischen Schriftsteller hervorgeht, die Langles in Nordens Reise angef"uhrt.}

Was die auf Talismanen vorkommenden Sinnbilder betrifft, gibt Montfaucon, der eine Menge derselben gesammelt hat, folgende Hauptgattungen an: den Engel mit 4 oder 6 Fl"ugeln, den Mensch-L"owen --- die Schlange mit dem L"owenkopf, den Hahn, den K"ufer, den Sphinx und Affen, alles Attribute, die bald so, bald anders modifiziert, sich auf die 7 Hauptkr"afte der Sch"opfung beziehen, und in den "altesten Religionen unter diesen Symbolen vorgestellt werden. Weitl"aufiger uns in die mannigfachen Gattungen der Talismane und in die Verschiedenheit ihrer Bezeichnungen einzulassen, w"are au"ser dem Zwecke dieser Abhandlung. 

Zur n"aheren Charakteristik der B"atylien wird es hingegen nicht undienlich sein, die Beschreibung vom Herabfalle eines der neuesten Aerolithen, und die Analyse seiner Bestandteile anzuf"uhren, indem sie uns einigen Aufschluss ihres magischen Gebrauches im Altertum geben kann. Vor anderen erw"ahnen wir den im Jahr 1773 unfern Sigena in Arragonien herabgefallenen merkw"urdigen Stein, von dem Proust, Professor in Madrid, der ihn untersuchte, folgendes sagt\footnote{S. dessen Abhandlungen in Gilberts Annalen, 24. Band, S. 261. verglichen mit der Analyse des Aerolithen, der im Jahre 1806. im ehemaligen Languedoc herabfiel; ebendaselbst p. 189.}:

Der Stein, 6 Pfund 10 Unzen schwer, war innerlich und "au"serlich mit P"unktchen von Rost durchs"aet, die h"ochst wahrscheinlich daher r"uhren, dass man ihn ins Wasser gelegt, um zu sehen, ob er sich darin ver"andern werde; er ist unregelm"a"sig eif"ormig, hat so zu sagen nur zwei Seiten, davon die eine abgeplattet, an den R"andern etwas abgestumpft, und in der Mitte etwas eingedr"uckt; die andere ist eine dreiseitig stumpfe Pyramide von ungleichen Seiten, deren Spitzen und Kanten ebenfalls stark abgerundet sind. Auch ihn umgab eine schwarze glasige Rinde, so dass beim ersten Anblick man ihn mit Pech "uberzogen glaubte, allein die Zerbrechlichkeit dieser Rinde, die St"o"se, welche der Stein ausgehalten hat, die vielen H"ande, durch die er gegangen ist, haben den gr"o"sten Teil desselben dieser Rinde beraubt, so dass sie sich jetzt nur in den Vertiefungen und auf den Seitenfl"achen der Pyramide zeigt. Die Grundfarbe des Steins, wie aller andern Meteorsteinen, ist ein einf"ormiges bl"auliches Grau, die Farbe eines schwarzen K"orpers, welchen ein wei"ser erhellt, oder vielmehr eine Verbindung von Erden, welche durch Eisen im Minimo der Oxidierung gef"arbt ist. --- Die Rinde dieser Steine ist "ubrigens zuf"allig, eine fremdartige Ursache hat offenbar ihre Oberst"ache ver"andert, gerade so wie in einem Kalkofen ein St"uck Sandstein oder Granit sich mit einer glasigen Kruste umgeben w"urde. Diese Ursache hat auf den Stein nur eine momentane Wirkung "au"sern k"onnen, wie daraus gewiss ist, dass wenn sie Zeit gehabt h"atte, ihre Wirkung "uber die Kruste hinaus fortzupflanzen, sie ein Aggregat von so schmelzbarer Art als diese Steine, notwendig ganz verglast haben m"usste.

Bey genauer Betrachtung dieser Rinde findet sich daher, dass sie die Wirkung eines Feuers gewesen sein muss, welches mit dem Ursprunge des Steins nichts zu tun hat; und die Hitze, die seine "au"sere Verglasung hervorbrachte, scheint gro"s genug gewesen zu sein, um seine Oberfl"ache zu schmelzen, aber nicht lange genug gedauert zu haben, um in das Innere einzudringen. Wenn auch nicht alle Steine dieser Gattung, wie viele Physiker behaupten, gl"uhend auf die Erde fallen, so kommen die meisten doch brennend, d. h. so warm herab, dass sie die Hand verletzen. Der Stein ist "ubrigens por"os wie Sand, der durch kein Zement verbunden ist; mit der gr"o"sten Leichtigkeit kann man durch ein St"uckchen hindurch blasen, wenn man es zwischen den Z"ahnen h"alt. Am Stahl schl"agt es kein Feuer. Seine Hauptbestandteile sind auf 103 Teile:
\begin{table}[H]
    \centering
    \begin{tabular}{l r l}
        Schwefel-Eisen in Minimo zu & 12 & Teilen \\
        Schwarzes Eisen-Oxid & 5 & Teilen \\
        Kieselerde & 66 & Teilen \\
        Magnesia & 20 & Teilen \\
        Manganes und Kalkerde & ~ & in geringer Menge \\ \hline
         ~ & 103 & ~ \\ 
    \end{tabular}
\end{table}
\paragraph{}
Das darin in ziemlicher Menge befindliche regulinische Eisen ist nur hineingemengt wie die gediegenen Metalle ihrer Gangart.\footnote{Bei dem im Jahr 1806. im Dep. du Gad gefallenen Meteorstein fand sich das Verh"altnis des Eisens als schwarzes Oxid zu den "ubrigen Bestandheilen, wie 40 : 100. Siehe Gilberts Annalen 24. Th. S. 202.}

Aus dieser Beschreibung, die im Ganzen mit allen anderen "uber "altere und neue Aerolithen gemachten Beobachtungen "ubereinstimmt, ergibt sich:
\begin{enumerate}
    \item dass sie insgesamt mehr oder weniger eisenhaltig sind, und
    \item dieser Eigenschaft gem"a"s, auch mehr oder minder auf die Magnet-Nadel wirken; --- dass
    \item ihre Form, besonders die der kleineren Steine, meist sph"arisch, teils ganz rund, teils oval ist.
    \item Gr"o"sere Steine hingegen trifte man oft viereckigt, pyramidalisch mit runder Basis, teils polygon, teils ganz unregelm"a"sig, und ebenso verschieden an Gewichte an.
    \item Die "au"sere Rinde ist zuf"allig, nur wenige Linien dicht, und f"ur eine leicht abgehende Kruste, die blo"s als eine Verglasung gelten kaum, anzusehen.
    \item Ihrer inneren Natur nach sind sie alle von hellgraulicher, mehr oder minder ins wei"se oder bl"auliche schie"sende Farbe; und
    \item das Gewebe, die Verkittung ihres inneren Korns ist so locker, dass sie dadurch eine Art Durchsichtigkeit erlangen, die v"ollig mit der Beschreibung "ubereintrifft, welche Plinius von den Astroiden und Sideriten gibt.
    \item Davon sind jene Steine doch ausgenommen, deren Masse fast ganz aus Eisen besteht (wie jene in Sibirien) und dadurch die Undurchsichtigkeit und Schwere dieses Metalls erhalten.
\end{enumerate}
\paragraph{}
Von diesen verschieden, n"amlich mit einer geringen Beimischung von Eisen und anderen Mineralteilen, meist aus Tonerde bestehend, sind jene im Jahr 1808. zu Stanneren in M"ahren niedergefallene Meteor-Steine, welche die Herrn Scherer und Schreiber in Wien im 1sten St"uck der Gilbertischen Annalen der Physik f"urs Jahr 1809, beschrieben haben. Ihre "uber diese Steinmassen gemachten Bemerkungen verdienen umso mehr beachtet zu werden, als sie "uber den Ursprung und die Natur der Aerolithen ganz neue Aufschl"usse gew"ahren. Mehrere Physiker sind der Meinung, dass diese K"orper beim Eintritt in unsere Atmosph"are in einen gl"uhenden Zustand geraten, und durch ihre Reibung gegen die Luft, darin unterhalten werden; andere glauben, sie k"amen durch den freien W"armestoff, der durch das Zusammenpressen aus der Luft ausgeschieden wird, in Fusion (welche letzte Meinung einige Erfahrung mehr, als jene erste blo"s hypothetische hat), Scherer hingegen macht wahrscheinlich, dass diese Massen weder in einem gl"uhenden, noch einem weichen Zustande teichiger Schmelzung herunterfallen, ihre Inkrustierung hingegen nicht w"ahrend dem Falle des Aerolithen durch die Atmosph"are allm"ahlig, sondern in einem blitzschnellen Momente durch eins elektrische Potenz (obschon nicht mit gleicher Intensit"at, und allseitig auf ihre Bruchseiten wirkend) erzeugt werde. Gleichen Ursprungs sind nach Scherer die Figuren auf der Rinde, und es findet in dem Akt der Inkrustierung s"amtlicher Meteorsteine ein gewisses Maas von Abstufung der Potenz statt, die auf die Steine gewirkt hat. In Hinsicht der "au"seren Rinde, haben diese m"ahrischen Aerolithen am meisten "Ahnlichkeit mit denen von Siena und Benares, wenig hingegen mit jenen von Eichst"adt, Aigle und Tabor. Ihre Bestandteile bestehen nach Vauquelins Analyse aus:
\begin{table}[H]
    \centering
    \begin{tabular}{l r l}
        Kieselerde & 50 & Teilen \\
        Kalkerde & 12 & Teilen \\
        Tonerde & 9 & Teilen \\
        Eisen-Oxid & 29 & Teilen \\
        Manganes-Oxid & 1 & Teilen \\
        Nickel-Oxid & 1 & Teilen \\
        Schwefel & ~ & Ein Atom \\ \hline
         ~ & 101 & ~ \\ 
    \end{tabular}
\end{table}
\paragraph{}
Sie enthalten (was die "ubrigen, bisher untersuchten, doch alle haben) weder Magnesia noch Chromium, sie sind leicht an Gewicht, zerreibbar und wirken nicht auf den Magnet. Obschon alle Meteorsteine darin "ubereinstimmen, dass sie sich fast immer ovalf"ormig oder prismatisch (vorz"uglich gern vierseitig) finden, so nehmen beide Naturkundiger gleichwohl als Ursache der verschiedenen Form dieser Massen an, dass sie zersprungene und schnell auseinandergerissene Teile eines gr"o"seren Meteors aus der h"oheren Luftregion sind.

An Farbe find diese m"ahrischen Aerolithen "au"serlich schwarz, zuweilen ins dunkelbraune ziehend, innerlich aschgrau, wohl auch bl"aulicht, man sieht darin dichtere dunklere K"orper, auch enthalten sie Schwefelkiesk"orner, doch wenige. Vom Basalt unterscheiden sie sich wesentlich durch den Bruch, die H"arte und den Strich. Sie f"uhlen sich sanft an, ritzen Glas nicht, und geben am Stahl keinen Funken, vor dem L"otrohre schmelzen sie zu einem dunklen Glase, welches der Magnet anziehet.

Wenn mit diesen Erfahrungen wir nun Patrins Hypothese verbinden, dass Luftsteine durch gleiche Ursachen, wie die Laven entstehen, n"amlich durch die feinen gasartigen Fl"ussigkeiten, die von der Atmosph"are in das innere der Erde, und von dieser in die Atmosph"are zirkulieren; dass diese Fl"ussigkeiten sowohl die Wirkungs-Mittel, als die Elemente zur Erzeugung der mineralischen K"orper, der Materie der Meteoren u. s. f. sind, welche durch Verbindung jener Fl"ussigkeiten mit einander nach den Gesetzen der Assimilation gebildet werden, so "offnet sich hierdurch ein neuer lichtvoller Weg zur Erkl"arung des Ph"anomens und des Ursprungs dieser Massen, die auf Meteorologie "uberhaupt vom wichtigsten Einfluss sein d"urften.

Den Ursprung dieser Steine betreffend, haben wir gleich anfangs die verschiedenen Meinungen der Physiker "uber ihre Entstehung gezeigt, worunter einer der neuesten, Proust, sie aus den unermesslichen noch unbekannten Polar-Gegenden herleitet, woraus sie, seiner Meinung nach, durch irgendeine m"achtig wirkende Ursache losgerissen, sich in die h"ohere Atmosph"are aufschwingen, und in die s"udlicheren Gegenden niederfallen.

Je mehr man aber die innere Natur dieser Steine untersucht, und je heller von der anderen Seite unsere Einsicht in die ersten einfachen Prinzipen des kosmischen Lebens unserer Erde, und der Weltk"orper "uberhaupt wird, je klarer scheint es: dass diese Steine weder von tellurischen Vulkanen, noch von Vulkanen des Mondes oder eines Himmelsk"orpers auf unsere Erde geschleudert werden, sondern: von Zeit zu Zeit eintretende starre Ausscheidungen aus dem Luftozean des Himmels sind; denn wenn nach Davy und den vorz"uglichsten neuern Chemikern, sich dartun l"asst, dass es nur eine einzige w"agbare Materie als Repr"asentant der Schwere, und Substrat der Schwerkraft in der Natur gibt, so wie gegenseitig zweierlei Licht, davon eines: das freie, merkbare, ungefesselte, unverschlungene; das andere hingegen, das verschlungene mit w"agbarem Stoff verbundene, gefesselte, l"asst sich aus derselben Ursache annehmen, dass diese ponderable Materie der niederen und h"oheren Lust vermittelst besonders verst"arkter elektrischer Wechselwirkung, die von Zeit zu Zeit zwischen zwei Weltk"orpern eintritt, nicht nur zu Wasser, sondern auch zu Stein und zu Metall werden. Zwar l"asst sich auch sagen, dass das beiderlei Licht (reines und verlarvtes oder gemeine und magnetische Elektrizit"at) die w"agbaren Grundstoffe von der Erde, der Sonne, dem Monde oder anderen Weltk"orpern aus der einen jedem derselben umgebenden Atmosph"are hinweg in den gro"sen Luftozean entf"uhren k"onne, von woher solche durch eintretende elektrische Wirkungen aus ihrem Luftf"ormigen Zustande wieder zur"uck in der Gestalt fester K"orper gebracht werden k"onnten; nicht leicht aber begreift sich dabei, wie diese verfl"uchtigten Teile dieselbe bleiben sollen, die sie vorher waren, und bei ihrer Ausscheidung wieder als dieselben erscheinen sollten?\footnote{Mehreres hier"uber in Dr. Haberles meteorologischem Jahrbuche f"urs Jahr 1810. Weimar 1816. ein Buch voll neuer trefflicher Ansichten.} Diese Luftsteine haben anf"anglich ein inneres elektrisches Leben, und verm"ogen sich, dem gem"a"s, so lange schwebend zu erhalten, als sie von neutralelektrischem Ether umgeben sind, indem sie eine polare elektrische Spannung hervorbringen, die der Gravitation entgegen wirkt; sobald sie in den Wirkungskreis eines anderen gr"o"seren und st"arkeren elektrisierten Himmelsk"orpers geraten, so wirkt ihr Gravitationsdruck allein. Als leuchtende Kugeln, teils in der h"oheren Luft, teils beim niederfallen zerplatzend, st"urzen sie dem Himmelsk"orper zu, in dessen elektrische kugelf"ormige Atmosph"are sie geraten sind.

Wenn nun, wie fr"uher gezeigt worden,\footnote{S. 31. u. f. dieser Schrift.} diese Himmelssteine als herabgefallene Sterne, und die Gestirne selbst als g"ottliche Wesen verehrt wurden, deren Einfluss alles Irrdische unterworfen ist, so ersieht man hieraus: dass, wie G"orres trefflich sagt, die Urzeit keine andere Geschichte hinter sich habe, als Naturgeschichte, und auch die Mythe in ihr ruhe: denn unter dem Bild der mannigfachen Genien, Intelligenzen, Mittelwesen u. s. f. verstand die erste "alteste Schriftsprache die Symbolik nichts anders, als die alles hervorbringenden allwaltenden Naturkr"afte; rings von ihnen umgeben, unter ihrer Macht stehend, und auf Erden wie am Himmel, ihre Wirkungen an sich sowohl, als an allen Gegenst"anden au"ser sich bemerkend; sagte dem Menschen in den fr"uhen Tagen seines Erdenlebens schon Ahnung und inneres Gef"uhl, was sp"ater heilige Priesterlehre und wissenschaftliche Forschung ihm offenbarten; dass der Standpunkt, worauf er lebt, die Erde im innigsten Zusammenhang, in steter Wechselwirkung mit den au"ser ihr bemerkten Weltk"orpern stehe, und die mannigfachen Naturerscheinungen, die Meteore Wirkungen h"oherer die niedere Welt beherrschender Kr"afte seien, die er aus Furcht und dem Gef"uhl eigener Schw"ache als G"otter verehrt. Aus dieser doppelten Ansicht h"oherer die niedere Welt beherrschender Kr"afte, und der Erscheinungen, die auf dieser Erde sowohl, als in der sie umgebenden Atmosph"are durch sie bewirkt werden, geht auch ein doppelter Ursprung des Polytheisms hervor. Jene Verehrung n"amlich, die ihren Blick himmelw"arts schwingend, die Gestirne und das Feuer als allgemeines Symbol ihrer allbelebenden Lichtnatur verehrt (der Sab"aism) dann die niedere, die mit der Erde sich nur befassend, alle mannigfachen, durch unsichtbare Kr"afte entstehenden Erscheinungen, in Bildern versinnlicht, (der grobe Fetischismus) der selbst im Holz, im rohen Steine, das "ubersinnliche Wesen, den Geist, den wundervollen D"amon, dem er diese Wirkungen zuschreibt, verehrt und anfleht.

Kehren wir auf die Erz"ahlung und ausf"uhrliche Beschreibung, welche Damascius\footnote{Siehe oben.} vom B"atylus des Eusebs, und den Zauberk"unsten gibt, die derselbe mit diesem Orakel-Steine trieb, so erkl"aren sich, wie mich d"unkt, diese Kunstst"ucke durch die innere Beschaffenheit des Steins und dessen Manipulation aus ganz nat"urlichen Ursachen, n"amlich: durch seinen Eisengehalt, und den Magnet; es ist n"amlich nicht blo"s wahrscheinlich, sondern durch Plinius und anderer ausdr"uckliche Zeugnisse, wie wir fr"uher gesehen haben, erwiesen, dass die Alten durch Anwendung des Magnets, viele an Wunder grenzende Erscheinungen hervorbrachten.

Erkl"arlich w"are nun leicht hierdurch, dass jener B"atylos, der nicht gern in des Arzt Eusebs Hand blieb, und dessen er weniger Herr war, als andere, die gleichfalls B"atylien besa"sen, wenn man annimmt, dass diese letzteren von der magnetischen Art waren, und jenen durch ihre Kraft an sich zogen. Denn derselbe Naturkundiger sagt:

Eisen wird vom Magnet angezogen, und h"angt gleichsam in einer Umarmung mit ihm selbst.\footnote{Buch 36. Cap. 25.}

Wenn es ferner in der Beschreibung hei"st: dass ehe der Stein zum Sprechen kam, er lange in den H"anden umher geworfen und bearbeitet werden musste, (ohne dass man ihn fallen lie"s) so darf man nur der Kunstst"ucke sich entsinnen, die man fertige Taschenspieler mittelst des Magnets hervorbringen sieht, um in diesen Gaukeleien, die das wunders"uchtige Volk jenes Zeitalters, als Wirkung einer d"amonischen Kraft anstaunte, f"ur nichts anders, als eine ganz physische Wirkung (des Magnets n"amlich) anzusehen; oder, welches ebenso m"oglich w"are, durch Galvanism, der, (wenn auch dem Nahmen, doch wahrscheinlich der Sache und Wirkungen nach) den Alten nicht ganz unbekannt blieb. Von Damascius wird fernes erw"ahnt, dass, wenn Eusebius sein Orakel befragen wollte, er es in eine Wand befestigte, und dann eine Antwort von demselben erhielt, die dem Zischen, oder weinerlichen Schrey eines Kindes "ahnlich war.\footnote{Falconet, in der Abhandlung von den B"atylien. Mém. de L'ac. des Inscr. Tome 6. p. 526. f"uhrt hier"uber eine merkw"urdige Stelle an, aus dem seltenen bisher noch Manuskript gebliebenen Buche: Hypopnesticum des Josephus (nicht des ber"uhmten Geschichtsschreibers, sondern eines Christen des 5ten Jahrhunderts). Nachdem die Rede von verschiedenen Bezauberungen war, f"ugt Josephus hinzu: "`Tempel-B"atylien, eine Art Divination, die mittelst gewisser in den Mauern befestigten Steinen gescheibt, welche Orakel aussprechen. Der Text hei"st: τα εν τοῖσ βαιτυλια δια λυθων εν τοῖσ στοιχεοῖσ προσραοστον θων, das Falconet durch ver"anderte Leseart also verbessert: δῖα λιθων εν τοισ τοιχοισ προσ χρησαντων wollte man εν τοισ στοιχεοισ beibehalten, so w"are es auszulegen, dass diese Orakel durch die Kraft der auf der Oberfl"ache der Steine eingepr"agten Schrift und Figuren wirksam w"urden.} Diese Befestigung in der Mauer scheint weniger (wie Dr. M"unter meint\footnote{S. dessen Abhandlung: Vergleichung der B"atylien der Alten mit den Steinen, welche in neuern Zeiten vom Himmel gefallen sind. In Gilberts Annalen der Physik, Th. 21.} daher zu r"uhren, dass Eusebius die Kunst, den Stein in der Hand zu bearbeiten, minder gut verstanden habe, wie andere Gaukler, sondern daraus sich erkl"aren zu lassen, dass diese Steine als Talismane angesehen w"urden, die man deswegen gern an die W"ande der Tempel "offentlicher und Privatgeb"aude befestigte, um dieselben vor sch"adlichen Geistern zu bewahren, oder jene, die sich ihnen n"aherten, daraus zu vertreiben, welcher Gebrauch (f"uhrte es uns hier nicht zu weit) aus h"aufigen Beispielen der chald"aisch-persischen, "agyptisch-ph"onizischen Astrologie und D"amonologie erwiesen werden k"onnte.\footnote{Nur eines sei hier erw"ahnt aus El-Makryzy Beschreibung von "Agypten, die Langles in seinen Bemerkungen zu Nordens Reise nach Nubien und "Agypten, Th. 3. S. 304. anf"uhrt: "`Als der Sultan Al Mahmouhn die Pyramide von Diyze "offnen und untersuchen lie"s, kam man nach langem Forschen auf einen Saal mit drei Th"uren, am Eingange von einer derselben waren drei S"aulen befindlich, von innen ausgeh"ohlt und in dieser H"ohlung befand sich das Bild eines Vogels. Die erste dieser S"aulen enthielt eine Taube von einer gr"unen Steinart; die zweite einen Falken von gelbem Steine; die dritte endlich einen Hahn vom Steine Kedan, einer Art H"amatit. Es waren, sagt El-Makryzy, Talismane, die bestimmt waren, den Eingang der Th"uren zu wahren, und b"ose Geister davon zu verscheuchen; zu demselben Gebrauch dienten wahrscheinlich auch jene mit Keilschrift bezeichneten babylonischen Backsteine, davon vor einigen Jahren mehrere nach London kamen (s. Dr. Hagers Abhandlung "uber die vor kurzem entdeckten babylonischen Innschriften im Asiatischen Magazin No. 3. 4. 6.) -- Entziffert ist bisher zwar keiner dieser Steine geworden; und wenn wir Plinius "Au"serung folgten (Nat. Gesch. Buch 7. Kap. 77.) so h"atten, nach Epigenes Zeugnis, die Babylonier astronomische Beobachtungen von 520 Jahren auf Backsteine verzeichnet, welches zu leugnen wir keineswegs berechtigt sind; allein jenen auf den ausgegrabenen Steinen befindlichen Zeichen nach zu urteilen (man sehe dieselben in obengemeldter Schrift nach) sind sie eher f"ur magische zu Talismanen dienende Zeichen zu halten; denn w"aren es astronomische Tafeln, so w"urde man sie wahrscheinlich nicht tief in Mauern vergraben haben.} Der Laut, den der B"atylos bei Mitteilung des Orakelspruches (der Sage nach) vernehmen lie"se, erkl"art sich am n"achsten wohl durch die T"auschung der Zuschauer, verbunden mit jenen Betrugsmitteln, deren sich die Beschw"orer "uberhaupt bei den Orakeln bedienten, worunter jenes des Bauchredens, welches Eusebius vielleicht in seiner Gewalt hatte, das n"achste und nat"urlichste scheint.

Durch die in dieser Abhandlung angef"uhrte Beispiele ergibt sich der Ursprung sowohl der religi"osen Verehrung dieser Steine, als die lange Dauer ihres Gebrauches, und indem uns das durch Eusebius\footnote{In Photii Bibl.} "uberlieferte Fragment des Sanchuniathon, worin es hei"st: der Gott Coelus habe diese Steine erfunden, den wahren Sinn aufkl"art, den die Alten dieser Mythe beilegten, n"amlich B"atylien als dem Himmel entfallene belebte Steine anzusehen, zeigt uns die merkw"urdige Stelle des Damascius und andere sie best"atigende in Priscian, Hesichius und dem Etymologicon, dass diese Steine als feurige Kugeln, stets von einem Meteor begleitet, herabfielen. So m"achtig ist der Hang zum Wunderbaren, dass der Glaube, den man an diese Steine hatte, verm"oge welchem man ihnen die gr"o"sten Wirkungen als Schutzg"otter, Amulette, Talismane, Zaubersteine zuschrieb, von den Zeiten des trojanischen Krieges (wie die Stelle aus dem orphischen Gedicht zeigt) und wahrscheinlich fr"uher im s"udlichen Asien, Indien, Persien, besonders Chald"aa, dem eigentlichen Vaterlande der d"amonologischen Theurgie, bis ins sechste christliche Jahrhundert, wo die gr"o"seren Orakel des Heidentums bereits schwiegen, allerw"arts erhielt, ja man kann sagen, nie ganz erlosch, selbst in neuerer Zeit nicht, wo der Volksglaube an die Wunder der Donnersteine noch stets lebend ist, so bei wilden V"olkern, und jenen, wo das Christentum die Blendwerke des Aberglaubens nicht g"anzlich gel"autert hat, oder wo Lokalit"at den Glauben an Visionen, Geister, h"ohere Zauberm"achte, geheime --- den Steinen und Pflanzen -- in wohnende Wunderkr"afte bef"ordert, wie bei allen Bergv"olkern oder Bewohnern neblichter Th"aler.

Der Glaube an das Wunderbare, und der Hang, f"ur unsere Bed"urfnisse, unsere Leiden Zuflucht und H"ulfe bei "uberirdischen Kr"aften zu suchen, ist im menschlichen Gem"uhte ebenso unausl"oschlich, als die Neigung, jedes Ereignis h"oheren M"achten zuzueignen, und dem Himmel entsteigen zu lassen.
\clearpage
\section*{Verzeichnis der Abbildungen.}
\paragraph{}
\begin{enumerate}
    \item[Fig. 1] Titelkupfer. Rhea, die ihrem Gemahle dem Saturn den in Windeln gelegten Stein, statt des verfolgten Jupiters, zu verschlingen gibt. --- Nach einer antiken Ara aus dem Mus. capitol.
    \item[Fig. 2] Eine Cyprische M"unze mit einem Konischen Idol oder dem Steingotte.
    \item[Fig. 3] Emesische M"unze auf den Dienst des Helagabolus oder der Sonne deutend.
    \item[Fig. 4] Emesische M"unze auf den Dienst des Helagabolus oder der Sonne deutend.
    \item[Fig. 5] M"unzen des ΖΕΥΣ ΚΕΡΑΥΝΙΟΣ aus Spanheim Dissert. de Numm. ant. Usu.
    \item[Fig. 6] M"unzen des ΖΕΥΣ ΚΕΡΑΥΝΙΟΣ aus Spanheim Dissert. de Numm. ant. Usu.
    \item[Fig. 7] M"unzen des ΖΕΥΣ ΚΕΡΑΥΝΙΟΣ aus Spanheim Dissert. de Numm. ant. Usu.
\end{enumerate}
\clearpage
\pagestyle{fancy}
\fancyhf{}
\rhead{}
\cfoot{\thepage}

\begin{figure}[ht]

\begin{subfigure}{0.5\textwidth}
\includegraphics[width=0.8\textwidth,height=0.8\textheight,keepaspectratio]{Fig1.png} 1
\end{subfigure}
\begin{subfigure}{0.5\textwidth}
\includegraphics[width=0.8\textwidth,height=0.8\textheight,keepaspectratio]{Fig2.png} 2
\end{subfigure}
\begin{subfigure}{0.5\textwidth}
\includegraphics[width=0.8\textwidth,height=0.8\textheight,keepaspectratio]{Fig3.png} 3
\end{subfigure}
\begin{subfigure}{0.5\textwidth}
\includegraphics[width=0.8\textwidth,height=0.8\textheight,keepaspectratio]{Fig4.png} 4
\end{subfigure}
\begin{subfigure}{0.5\textwidth}
\includegraphics[width=0.8\textwidth,height=0.8\textheight,keepaspectratio]{Fig5.png} 5
\end{subfigure}
\begin{subfigure}{0.5\textwidth}
\includegraphics[width=0.8\textwidth,height=0.8\textheight,keepaspectratio]{Fig6.png} 6
\end{subfigure}
\begin{subfigure}{0.5\textwidth}
\includegraphics[width=0.8\textwidth,height=0.8\textheight,keepaspectratio]{Fig7.png} 7
\end{subfigure}

\end{figure}
\clearpage
\end{document}
